\part*{Dritte Stunde:\\"`Welche die Bas zerschneidet"'}
\addcontentsline{toc}{part}{Dritte Stunde}

\chapter*{26. Dezember, Stephanstag}
\addcontentsline{toc}{chapter}{26. Dezember, Stephanstag}

\section*{1}

Amélie öffnete die Augen\dots

Sie lag im Bett. In ihrem warmen Bett im blauen Haus. -Ich lebe, noch! Dachte Amélie. Von Ferne hörte sie die Glocken des Münsters zum Mittag zwölfmal läuten. Je wacher Amélie wurde, umso mehr meldete ihr Körper die Blessuren und Quetschungen, die er auf seiner wilden, nächtlichen Flussfahrt davon getragen hatte. Das Flusswasser hatte einen modrigen, erdigen Geschmack mit einem Hauch Schiffsdiesel im Mund hinterlassen. Amélie setzte sich auf. Ihre Haare waren getrocknet und hingen in traurigen, schmierigen Strähnen an ihr herunter. Hoffentlich kommt Amset nicht wieder ins Zimmer gestürmt, dachte sie. Es klopfte an ihrer Zimmertür.

"`Jaaah?"' rief sie zaghaft. "`Guten morgen, Kind!"' Uff, es war Wibrandis mit einem Tablett auf dem der Kakao dampfte und ein frisches Schokoladencroissant lag und eine schlanke Vase mit einer Rose stand. "`Die Rose ist von Hans,"' erklärte Wibrandis "`der wilde Maa ist sehr stolz auf dich. Wie wir alle!"' Die ältere Frau stämmte ihre Fäuste in die Hüften und strahlte Amélie herzlich und stolz an.

Amélies Blick fiel auf die Schürze, die über Wibrandis Bauch und Busen spannte. Auf der Brust befanden sich die 'betenden Hände' von Albrecht Dürer, die restliche Platz der weissen Schürze war mit Friedenstauben und zarten, schwarzen Kreuzen bedruckt. Wibrandis bemerkte Amélies Blick und fragte besorgt: "`Gell, es ist nicht zu gewagt, oder? Die Herrschaften haben mich etwas angesteckt\dots"' Wibrandis errötet leicht und Amélie beeilte sich sie zu beschwichtigen: "`Nein, nein, Wibrandis, es sieht\dots toll aus, vielleicht etwas ungewohnt."' Wibrandis atmete hörbar aus und meinte: "`Genau, es ist etwas ungewohnt"'.

Während Amélie sich über ihr Frühstück her machte, bemerkte sie eine rötliche Katze, die mit Wibrandis ins Zimmer geschlüpft sein musste. Amélie kannte Katzen und wusste sofort, dass dies keine gewöhnliche Katze war. Sie war auf die Kommode gesprungen, um Amélie besser anstarren zu können. Instinktiv zog Amèlie die Decke höher, immerhin könnte es ein Kater sein.

Wibrandis schien die Katze nicht zu bemerken. Sie kramte für Amélie Kleider hervor. Als sie alles gefunden hatte, schaute sie zufrieden auf das geleerte Frühstückstablett. Die Rose von Hans stellte sie auf den Nachtisch. Amélie sog den Duft tief ein. Er war wie eine warme, sanfte Massage von Innen. Wie schön wäre die gleiche Behandlung auch aussen, dachte sie. "`So, Amélie! Husch, husch! Es steht viel auf dem Programm\dots"' Als Wibrandis Amélies Gesicht sah, lachte sie und sagte: "`Ja, ein Bad gehört auch dazu und zwar sofort."'

Sie reichte Amélie einen Bademantel und gemeinsam verliessen sie das Zimmer. Amélie versteckte sich hinter Wibrandis Rücken, für den Fall, dass Amset ganz zufällig vorbeikommen sollte. Als sie im Treppenhaus aus dem hohen Fenster schaute, bemerkte sie den Falken auf dem Dach, der rief und sich in die Luft hob. Er flog über das Dach aus ihrem Blick. Okay, dachte sie, Badezimmertür abschliessen, weil jetzt weiss er sicher, wohin ich gehe.

Sie kamen in das Erdgeschoss und gingen durch die grosse Eingangshalle in den rechten Flügel des blauen Hauses. Sie kamen in einen langen Gang. Das erste Zimmer wurde von Wibrandis bewohnt. Hinter ihrer Zimmertür war der Gang mit einem schweren Vorhang verhängt. "`In diesen Bereich dürfen nur die Frauen,"' berichtete Wibrandis begeistert. "`Hier ist das Bad."' Amélie betrat ein grossen Raum. In der Mitte befand sich eine riesige Badewanne. Sie war verschieden tief. Auf dem warmen  Wasser schwammen duftige, leichte Schaumberge. Blumen, Rosen, Ringelblumen, Orangenblüten und Passionsblumen trieben vorbei. An der Wand befanden sich Duschköpfe und Amélie stieg schnell aus dem Bademantel und duschte sich den Schlamm des Flusses und die Anspannung ab. 

Dann stieg sie langsam in das warme Wasser und liess sich an der tiefsten Stelle nieder. Die Arme entspannt auf den Rand gelegt, streckte sie die Beine aus. Ihre Zehen tauchten aus dem flaumigen Schaum auf. Amélie grinste und hatte zum ersten mal das Gefühl mit Göttern Ferien zu machen, könnte auch auf entspannende Weise cool sein\dots Sie legte den Kopfnach hinten auf den Beckenrand und schloss die Augen.

"`Du bist also die Liebe unseres Amsi!"' Amélie riss die Augen wieder auf. Ihr gegenüber sass, wie aus dem Nichts, eine Frau. Sie hatte feuerrote, lockige Haare, smaragdgrüne Augen und volle Lippen. "`Nein! Bin ich nicht!"' Amélie verschränkte wütend die Arme und schob die Unterlippe, vor. -Konnte man denn nicht fünf Minuten alleine sein? "`Ich bin neugierig auf Dich und wollte die Chance nutzten und dich kurz anschauen. Ich bin Tefnut!"' versöhnlich reichte sie Amélie ihre Hand. Die schlanken Finger fühlen sich warm an, wärmer als das Wasser. 

Amélie war eingeschüchtert. Tefnut war ein Kraftpacket. Ihr Körper, der sich erhob, um aus der Wanne zu klettern, war wunderschön. Die Haut glänzte wie Bronze. Sie hatte die Figur eine Pinup-Girls und bewegte sich geschmeidig, wie eine Katze. Ihren grünen Augen entging nichts und wenn sie sprach bewegten sich ihre zierlich Nase und ihre Ohren, wie die eines witternden Raubtieres. Sie ist die rote Katze! Durchfuhr es Amélie. Tefnut grinste. Kam es Amélie nur so vor, oder waren ihre Eckzähne lang und spitz? Amélie merkte, wie sie verschüchtert über den Schaum linste und wurde im gleichen Augenblick wütend. 

Was heisst, mal kurz gucken und Katzen? Ha! Sie hatte sich heute Nacht mit einem riesigen Krokodil geprügelt\dots Tefnut grinste noch immer, aber dann grollte sie tief aus der Kehle und brüllte. Löwengebrüll! Amélie hielt sich die Ohren zu und sah die Göttin trotzig an, während der Rest von ihr vor Schreck schlotterte! "`Tapfer, kleine Amélie, du gefällst mir und das ist gut, denn wir werden die Nacht miteinander verbringen."' Die Göttin hatte sich zu Amélie gebeugt und hielt ihr Kinn fest. Sie zwang sie, sie anzuschauen. Amélie bekam eine Gänsehaut in der warmen Wanne. "`Wer bist du?"' "`Ich bin die Wahrheit!"' Sagte die Göttin und gab Amélie einen sanften Kuss auf den Mund.

\section*{2}
\addcontentsline{toc}{section}{2}


Amélie schluckte den Ärger, ohnehin vermutete sie, die Wut könnte nur ein Ersatz sein, für andere Gefühle für die sie keinen Platz hatte, haben wollte. Angst stand ganz oben auf der Liste, oder Traurigkeit\dots
Sie konzentrierte sich fest auf den Blumenduft, den sie in ihren frisch gewaschenen, feuchten Haaren mit sich durch das Haus trug. 

Als sie in ihr Zimmer kam, fiel ihr Blick auf die Kommode. Auf einem roten Samttuch hatte Wibrandis das Schwert und den Kelch drapiert und die Rose von Hans. An der Vase lehnte ein Zettel. Der Stab ragte auf beiden Seiten über die Kommode hinaus.

Amélie nahm als erstes den Zettel, sie hatte da einen Verdacht. Tatsächlich war er von Hans geschrieben, wie sie vermutet hatte. "`Liebe Amélie"', stand dort in ungelenker Handschrift: "`Die Münze ist seltsamerweise noch nicht wieder aufgetaucht. Ich bringe sie Dir sobald ich sie habe. Liebe Grüessli, Hans"' Nun war es an Amélie aus vollem Hals zu lachen. Hans hatte also den Scherz des Gnomes nicht wirklich kapiert! Amélie war froh, es musste schliesslich niemand wissen, wo genau sich die silberne Münze zur Zeit aufhielt. Dennoch nahm sie sich vor so schnell wie möglich Wibrandis um Rat zu fragen und Hans zu sagen, dass er sie nicht länger suchen musste.

Dann bestaunte sie Scheide und Schwert. Die Scheide war aus Samt und Leder. Das Leder war gelblich. An der Spitze und am oberen Rand war das Leder auch Aussen als Verstärkung angebracht. Auf dem oberen Lederring schimmerte ein blauer Kristall.Der Samt war Königsblau und überzog die restliche Scheide. So fühlte sie sich steif, aber warm und weich an. Mit einem einfachen Gürtel, der aus dem selben gelben Leder gefertigt war, konnte die Scheide an die Hüfte gebunden werden. 

Die Scheide war neu gefertigt und das Leder wirkte völlig unberührt, als Amélie mit den Fingern darüber fuhr. Nur an der unteren Spitze war ein breiter Ritz zu spüren. -Der muss von dem Krokodil stammen, dachte sie. Dann bemerkte sie weitere, aber kleinere Zahnabdrücke, die auf dem sonst makellosen Samt zu sehen waren. -Duamutef? Sie konnte nicht anders und sog eine tiefe Nase von dem Lederduft ein, sie liebte den Geruch, dann schnallte sie sich den Gürtel um.

Amélie zog das Schwert aus der Scheide. Der Griff war mit dem gelben Leder frisch umwickelt. Sonst war das Schwert einfach. Die Klinge aus Eisen, keine Schnörkel, keine Zierden, keine Schmucksteine. Ausserdem hatte die Klinge Kerben und war offensichtlich alt. Amélie war etwas enttäuscht. Sie lies es trotzdem durch die Luft sausen und machte einen Ausfallschritt. "`Ha!"' Sie fuhr durch die Luft. "`Nimm das!"' Es gefiel ihr. Der Griff fühlte sich kühl an? Sobald sie die Hiebe ausführte, schien die Klinge zum Leben zu erwachen, vibrierte und summte wie ein klarer, scharfer Windhauch. Vorsichtig steckte sie sie in die Scheide zurück. Amélie wackelte zweimal mit der Hüfte. Das Gewicht des Schwertes fühlte sich- gut an!

Sie nahm den Stab. Er reichte ihr bis an die Schultern. Er war aus Holz. Er besass noch die Rinde, in die feine Muster eingeritzt waren, die wie Flammen aussahen. Die Muster bildeten sich aus lauter Dreiecken, deren Spitze nach oben zeigte. Am oberen Ende das leicht gekrümmt war und sich natürlich in ihre Hand schmiegte, sah sie die verschiedenen Farben des Holzes. Aussen unter der Rinde war es hell, gelblich und im Inneren befand sich ein rötlicher Kern.

-Cool, ich hab' jetzt einen Zauberstab! Amélie fuchtelte mit dem Stab durch die Luft. "`Abraaa, Kadabraaa!"' intonierte sie. Leider blitzte es nicht und sprühte auch keine Funken, wie sie gehofft hatte. Jedoch, der Stab war warm, sogar heiss geworden. Amélie war sich nicht sicher, aber das Muster der Flammen, hatte es sich verändert? Sie stand etwas ratlos und stellte ihn schliesslich an die Wand, um den letzten Gegenstand, den Kelch zu betrachten.

Wieso war er wieder heil? Sie konnte sich nur zu gut erinnern, wie sie in der Nacht die Scherbe mit dem Blut daran genommen und in Ihr Herz gestossen hatte.  Sie nahm den Kelch in die Hände. Er war schwer und kühl. Er war ein fein geschliffener Diamant, der an immer neuen Stellen das Licht ein funkelnde Regenbögen brach und sie im Zimmer verteilte.

Sie schauderte, denn auf dem Grund des Kelchs befand sich wieder Blut. Als sie genauer hinsah, begriff sie, dass es kein Blut, sondern ein blutroter Edelstein war, der an eine Goldkette befestigt war. Vorsichtig hob Amélie den Rubin aus dem Kelch und legte sich die Kette um den Hals. Durfte sie das? -Aber wozu waren die Sachen sonst in ihr Zimmer gebracht worden\dots Amélies Gefühl schwankte zwischen Stolz und Unsicherheit hin und her. Dann breitete sich Sicherheit aus. Sie kam aus dem Stein, der wie ein Strom von Gewissheit ihre Brust füllte. Ohne es zu bemerken, stand Amélie aufrechter, kerzengerade und dennoch entspannt.

Leider gab es keinen Spiegel in ihrem Zimmer. Sie überlegt, ob sie ins Bad schlüpfen sollte, da klopfte es an der Tür und im selben Augenblick stand Wibrandis im Zimmer. "`Schnell, schnell, die Herrschaften erwarten dich!"' 

Sie schob Amélie vor sich her. Als Wibrandis Amélie vor der Tür zu Thots Bureau ansah, machte sie verdutzt einen Schritt zurück: "`Jesses, Meidli, wie siehst du denn aus\dots!"' Bevor Amélie etwas erwidern konnte, öffnete sich die Tür und Thot erschien. Er sprach mit einer Person, die in seinem Bureau war, griff nach Amélie und zog sie in den Raum. Er wandte sich erst zu ihr um, als Wibrandis "`Herrjeh!"' rief und aus seinem Zimmer hinter ihm lautes Gelächter erklang.

Dem obersten Gerichtsschreiber der ägyptischen Götterwelt aus nächster Nähe dabei zuzusehen wie ihm die Kinnlade vor Überraschung runterklappte, wäre vermutlich für Amélie ohne das Lachen in ihren Ohren lustiger gewesen.

Amélie bemerkte entsetzt, dass jeder der drei Ohrensessel mit Göttern besetzt war. In den einen hatte sich ausgerechnet diese Tefnut niedergelassen. Amélie hatte gehofft, sie nicht so schnell und sicher nicht nochmal im Bademantel zu treffen. Vor allem nicht im dicken, rot-weiss gestreiften Frotteebademantel mit einem Zauberschwert umgürtet, barfuss und einem ansehnlich grossen Rubin um den Hals. 

Die Göttin, die vor Lachen weiterhin gluckste, hatte sich, nach ihrem Auftritt im Bad, einen kurzen, roten Bademantel aus weichfliessender Seide übergestreift. Seltsamer Weise hatte sie eine Schürze an. Keine grosse, wie die anderen Göttinnen, die Amélie mit Schürze gesehen hatte, sondern eine kleine weisse Schürze mit Rüschenrand und einer kleinen Tasche. Auf die Tasche waren Dreiecke in rötlichen Farben gestickt, angeordnet wie eine Flamme.

 Diese Schürze ist -heiss! Fand Amélie und merkte, wie sie noch röter wurde. Es musste eine Art Zeichen der Anwesenheit Tefnuts sein, Scham zu empfinden. Tefnut trug nur diese zwei Kleidungsstücke und rotseidene Pantöffelchen an den Füssen.  Sie hatte ein Bein übergeschlagen und wippte mit ihrem Pantoffel an der Zehenspitze. Nun fixierte sie Amélie, lächelte aber. Dann warf sie einen schmachtenden Blick zu dem zweiten Ohrensessel hinüber und Amélie folgte neugierig ihrem Blick und erstarrte\dots

-Mach', dass das nicht wahr ist! Im zweiten Sessel sass der schönste Mann, den Amélie je gesehen hatte. "`Das ist Schu. Der Gott der Lüfte und des Lichtes. Der Gott des Lebens und Ehegatte und Bruder der hinreissenden Tefnut\footnote{Erklärung siehe vorheriges Kapitel.}."' Stellte Thot den mächtigen  Gott vor. Sein ebenmässiges, fein geschnittenes Gesicht war leicht gebräunt. Die Augen leuchteten kornblumenblau und kristallklar. Der Lippen waren sanft geschwungen und zu einem spöttischen, freundlichen Grinsen verzogen. Im Mund trug er eine Selbstgedrehte. Sein Gesicht war von einer wilden, lockigen goldig schimmernden Mähne umrahmt, die ihm neckisch ins Gesicht und auf die Schultern fiel. Und um es auf die Spitze zu treiben, trug er einen unauffällig getrimmten Bart\dots

Er war ausgesprochen muskulös, aber schlank, was von der verboten engen Jeans und dem T-Shirt noch betont wurde. Auf dem schwarzen T-Shirt befand sich ein weisses Dreieck mit der Spitzte nach oben. Das Dreieck wurde von einer weissen Linie von links durchfahren. Auf der rechten Seite brach die Linie, als sie das Dreieck verliess in einen Regenbogen auseinander. Amélie war sicher das Bild auf einem Plattencover schon gesehen zu haben.
"`Hey, \dots"' nuschelte das göttliche, \dots männliche, göttliche Wesen und lächelte breit. Dabei entblösste es die kräftigen, grossen Eckzähne einer Raubkatze. "`Hii!"' quitschte Amélie. Warum kann ich jetzt nicht tot umfallen, oder besser, tot umfallen und im Erdboden versinken? 

Aber Thot hatte ein Einsehen und meinte: "`Last but not least, haben wir heute die Ehre den Sonnengott höchstpersönlich in unserer Morgenrunde zu begrüssen. Amélie, dies ist der Vater aller hier anwesenden Götter, Re!"' "`Hach, Thot. Wie schaffst du es nur? Jetzt fühle ich mich plötzlich alt!"' Lachte der Göttervater. Amélie grinste schüchtern. Sie hatte den Vater aller Götter sofort in ihr Herz geschlossen. Sie hatte ihn schon während des gemeinsamen Mittagessens beobachtete, wo er, meist vergnügt und laut-fröhlich, mit einer Sonnenbrille auf der Nase am Kopfende des Tisches sass. Er überragte alle, selbst Horus. Man musste kein Götterexperte sein, um seine Stellung und Autorität als Göttervater sofort zu spüren.

Auch jetzt trug er die Sonnenbrille. Amélie war ihm bisher nie so nahe gekommen und konnte nun den Glanz erkennen, den die Augen auch durch die dunklen Gläser abstrahlten. Das Haupt des Sonnengottes war von einer goldenen Carona umgeben. Die golden und rot schimmernden Locken bewegten sich, als gäbe es in dem stillen Raum einen Luftzug. 

"`Wie ich sehe, hast du die Insignien der Elemente schon entdeckt"' meinte Thot trocken. "`Entschuldigung!"' Murmelte Amèlie, ehrlich zerknirscht. Im Angesicht der Götter wurde ihr bewusst, wie kindisch sie war. "`Nun, ich finde es- prima!"' strahlte Re, der Göttervater. "`Ein Adept kann sich nicht früh genug mit den Insignien beschäftigen. Ausserdem hatten wir noch keinen, der das auf so schwungvolle Weise machte."' Thot lächelte -endlich, dachte Amélie. "`Wir haben einiges zu besprechen für heute Nacht, Amélie!"' Meinte er dann: "`Dies wird deine erste Reise ohne deinen Körper sein. Tefnut und Schu werden dich begleiten."' "`Und ich und mein Auge, passt gut darauf auf!"' Sagte Re mit erhobenem Zeigefinger. "`Dein Auge?'' fragte Amélie.

\section*{3}
\addcontentsline{toc}{section}{3}

-'Was soll ich denn machen, Hapi?' Duamutef liess betrübt den Kopf hängen. Er hatte sich mit seinem Pavianbruder vor dem Teich zwischen den Bäumen und Büschen des Innenhofes getroffen. Er musste schnell mehr über den Fluch rausfinden. Alles was er gestern gesehen und gehört hatte, bevor sein Vater ihn geschnappt und wieder in die Gegenwart gebracht hatte, verstärkte seine Gewissheit. Der Fluch hatte mit ihnen, den Horussöhnen zu tun, allem voran mit ihm, weil Amélie ihn im Traum mit den Kanopen gesehen hatte.

Sie waren Wächter, die Horussöhne. Eine ihrer wichtigsten Eigenschaften war ihre Unbestechlichkeit! Niemand konnte sie bestechen und sie liessen niemals in ihrer Achtsamkeit nach\dots und doch, und doch musste es jemandem gelungen sein vor ihrer Nase ein Herz zu verfluchen! Das war unmöglich! Es sollte unmöglich sein! Wer das bewerkstelligt hatte, war kein daher gelaufener Einbalsamierer, so etwas konnte nur ein Priester schaffen. Ein besonderer und mächtiger, in Magie bewanderter Priester. Aber welcher?

Das Herz galt bei den Ägyptern als Zentrum von Lebenskraft und Verstand. Im Gegensatz zu den Organen, die die vier Horussöhne in den Kanopen beschützten, blieb das Herz im Körper des Toten. Es war auch nach dem Tod lebenswichtig. Lebenswichtig für das Leben von dem die Ägypter glaubten, es beginne erst nach dem Tod. Deshalb wurde das Herz von den Priestern und von ihnen, den Göttern, die die Toten begleiteten, besonders geschützt. Mit Magie, mit Gebeten, mit Amuletten. Das Herz bildete den Kern und das alles entscheidende Werkzeug für den Ba, die Seele, nach dem Tod.

Nicht jeder Tote durfte nach Sechet-iaru, dem Gefilde, in dem die geprüften und für würdig und rein befundenen Toten für immer zufrieden und glücklich sein konnten. Die Toten mussten viele Abenteuer bestehen und das Grösste davon war das Totengericht. Thot und Anubis wogen dabei das Herz des Toten mit der Feder der Maat auf. Recht sprachen schliesslich die 42 strengen Totenrichter. Ihren Vorsitz hatte Osiris, der Herrscher der Duat, des Jenseits, persönlich. Jede Nacht. Wehe dem, dessen Herz schwerer wog als die Feder der Göttin! Dessen Ba, dessen Seele, wurde von dem schrecklichen Monster Ammit verschlungen und für immer ausgelöscht.

Damit jeder Tote seine Chance ergreifen konnte auch nach dem Tot ein schönes Leben zu führen, beschützten die Horussöhne und mit ihnen auch Anubis die Körper der Toten, solange sie sie brauchten. Sie überwachten den Vorgang der Einbalsamierung, die die höchsten Priester mit magischen Ritualen durchführen mussten.

Und genau dabei musste etwas passiert sein!

Aber wer würde so etwas tun? Duamutef schauderte, wer hasste Amélie so sehr? Wie konnte jemand, ein Priester, der über all die Seelendinge Bescheid wusste, ein Herz zerstören wollen und es damit für Ewig aus dem Strom der Zeit reissen? Aus dem Gefüge der Ordnung reissen. Das Herz verstecken, damit es nicht von der Maat gewogen werden konnte? 

Der Priester musste wahnsinnig sein vor Hass, denn er hatte mit dem Fluch, den er auf Amélies Herz gelegt hatte, auch sein eigenes Herz schwer belastet. Eigentlich müsste Ammit ihn gefressen haben! Dachte Duamutef. Aber war es so einfach? War der Übeltäter damals vor tausenden Jahren, nachdem er gestorben war und sein Herz gewogen wurde, selbst Ammit zum Opfer gefallen, weil er einen Fluch auf sein Herz geladen hatte? 

Mit Schaudern wurde Duamutef bewusst, dass genau das auch mit Amélies Herz passieren musste, wenn sie es vom Fluch befreien konnten. Es würde gewogen werden. Wie würde das Ergebnis aussehen. War Amélies Herz rein genug? Wie sollten sie das prüfen, nach all den tausend Jahren? Waren in den letzten Lebenskreisen durch den Fluch neue Verfehlungen entstanden, die das Herz zu schwer werden liessen?

All diese Fragen muss ich beantworten! Bevor Amélies Herz gewogen wird. Was, wenn es zu schwer ist? Was wird dann aus Osiris? Verzweifelt heulte Duamutef auf. Alles drehte sich in seinem Kopf, kaum war ein Gedanke gedacht, tauchten dazu hundert Fragen und neue Gedanken auf!

Hapi kratzte sich sein Kinn, Paviane haben recht viel davon. '-Tef! Du kannst das nicht allein durchziehen! Und wir anderen drei können auch nicht mehr ausrichten als du. Du musst den Chef, du musst Anubis um Rat fragen!' Duamutef verzog das Maul. Es war ihm nicht recht. Ausserdem wusste er viel zu wenig. Aber er konnte sich bei seinen Ermittlungen keinen Patzer mehr leisten, die Zeit drängte. Bis zum Gerichtstermin waren es nur noch vier Tage!

Anubis trat aus dem Gebüsch auf die kleine Lichtung vor dem Teich, auf dem sich Hapi und Duamutef getroffen hatten. Er blickte die beiden durchdringend an: '-Was ist los? Ich habe ja nur das Ende deiner Reise mit deinem Vater erlebt, Duamutef, aber das war seltsam genug.' 

Duamutef berichtete seinem Meister und Freund von dem Verdacht den die Brüder wegen den Kanopen hatten. Von seinem Reisebericht durch die Zeit war Anubis alles andere als begeistert. '-Noch auffälliger sein, wäre jetzt echt schwierig geworden', bemerkte er. Die beiden Horussöhne zuckten zusammen, normalerweise neigte ihr Chef nicht zu ironischen Bemerkungen.

Anubis schwieg eine Zeit. Dann bemerkte er: '-Wir können die Stunden und ihren vorgezeichneten Weg nicht verlassen, ohne endgültig aus der Zeit zu fallen, also müssen wir sie nutzen. Duamutef, du hast Glück. Auf der heutigen nächtlichen Fahrt wirst du auf die richtigen Seelen von Toten, die Bas, und auch Götter treffen. Frage sie aus! Sie sind seit Urzeiten in der Duat, sie können sich erinnern, dann musst du nicht in die Zeit pfuschen und Verwirrung stiften!' Duamutef wedelte mit dem Schwanz, wie immer hatte ihr Meister den einfachsten und logischen Weg sofort erkannt.

'-Ein Letztes,' Anubis wurde eindringlich, '-So lange wir hier in der Gegenwart sind: Tue das nie wieder!' Anubis war dicht an Duamutef getreten und knurrte '-Betrete nie wieder die Zeit!' Duamutef leckte ihm fragend über das Maul '-Warum Meister. Sonst reisen wir doch auch in der Zeit.' '-Ich hatte dich für klüger gehalten' bemerkte Anubis tadelnd. '-Stell dir vor, was passiert, wenn der Übeltäter, der sehr, sehr klug zu sein scheint, dich bei der Recherche erwischt? 

Was dann? Wenn er die Vergangenheit ändert und sein Verbrechen nicht begeht? Dann ist Amélie in unserer Gegenwart verflucht und es gibt keinen Täter mehr und keine Tat, die dafür verantwortlich sind! Wie willst du dann den Fluch lösen? Einen Fluch, den es plötzlich in der Vergangenheit nicht mehr gibt?' Duamutef war bleich geworden, was bei ihm nur an der dunklen Nase zu erkennen war. Hapi kreischte. '-Also, meine Horussöhne, auf, auf! Macht eure Arbeit und etwas mehr!'

\section*{4}
\addcontentsline{toc}{section}{4}

Nachdem Thot Amélie auf die nächtlichen Abenteuer vorbereitet hatte, verliess er sein Bureau und bog links in den langen Gang ein. Am Ende des Ganges öffnete sich die Tür und Isis kam heraus, gefolgt von einem Sonnenstrahl, der den dunklen Gang in warmes Licht tauchte. Thot bemerkte eine grosse, mit vielen samtenen, rosa Blättern gefüllte Pfingstrose in Isis Haaren und ein friedliches Lächeln auf ihren Lippen. "`Hallo, Thot!"' Strahlte sie. Sie trug eine Schürze, grün mit bunten Blumen darauf, ab und zu konnte Thot eine Schlange zwischen den Blumen auftauchen sehen, die jedoch wieder im Grün verschwanden. Osiris und Isis schien es besser zu gehen.

"`Wie geht es ihm?"' fragte Thot. "`Viel besser! Als Anubis und ich ihm heute Morgen aus dem Imiut heraus und ins Bett brachten, hatte er endlich wieder etwas Grün im Gesicht!"' sagte sie. "`Er hat Besuch von Geb, der hilft ihm die heutige Nacht vorzubereiten. Sie müssen die Route mit der von Amélie und Re abstimmen, damit sie viel Kraft des kosmischen Eies aufnehmen können."' Meinte sie und sogleich legte sich ihre Stirn in Falten. Dadurch wurde die goldene Schlange, die ihr als Haarband diente, geweckt, sie zischelte und blickte vorwurfsvoll auf Thot. "`Also, bis später"', Isis eilte an Thot vorbei den Gang runter, die Schlange wippte auf ihrem Haupt, sie behielt Thot im Auge und züngelte, bis die Göttin um die Ecke bog.

Thot blieb vor der Tür stehen. Das kosmische Ei. Er staunte, wie gut Berta und Hans den Ort für sie ausgesucht hatten. Nicht jede Stadt verfügte über einen solchen starken Schutz wie Basel. Die Form des Eies wurde von einem Ring aus Kirchen gebildet: Dem Münster, der Barfüsserkirche, der Leonhardskirche, der Peterskirche und der Martinskirche. Sie alle standen auf Plätzen, die stark mit den Kräften zusammen hingen, die zwischen Erde und Himmel ausgetauscht wurden. Thot schaute in seiner Vorstellung von oben auf die Stadt und die Kirchen schienen untereinander mit Adern verbunden. Das Blut dieser Adern kam mal aus der Erde und mal vom Himmel. Sie verflochten sich miteinander. Das Adergeflecht bildete schliesslich die Form eines Eies, dessen eine Hälfte in der Erde verschwand und die andere eine Kuppel über dem alten Teil der Stadt bildete.

Dabei waren es nicht die Kirchen selbst, die dieses Kunstwerk vollbrachten, nein, sie waren lediglich auf diesen  kraftvollen Plätzen, die es hier immer gegeben hatte, gebaut worden. Einige von ihnen hatten Tempel verdrängt oder ersetzt, die vor Christigeburt  an diesen heiligen Orten gestanden hatten.

Im Zentrum des Eies, im Zentrum des geschützten Raumes lag das Herz der Stadt tief in der Erde verborgen. Es war durch einen Brunnen gekennzeichnet. Ein Brunnen, um den sich viele Legenden rankten. Der Gerberbrunnen.

Früher wussten die Menschen das alles noch, dachte der Gerichtsschreiber wehmütig. Obwohl Thot wusste, wie notwendig das Vergessen all dieser Dinge für die Menschheit war, packte ihn die Wehmut. Schliesslich hatte er massgeblich an dem alten Wissen mitgewirkt, welches den Menschen tausende Jahre nützlich gewesen war. In bestimmten Momenten schmerzte es zu sehen, wie viel seines hermetischen Weges die Menschen vergessen mussten, damit sie zur Freiheit wachsen konnten.

Eijeijei! Thot mahnte sich selbst, stehst da und träumst\dots Er klopfte an Osiris Tür und betrat das geräumige Schlafgemach. Osiris sah in der Tat besser aus. Er trug die Farbe von jungem Roggen im Gesicht und er musste sich gelangweilt haben, denn an einigen Stellen wuchsen aus den Wänden und Möbeln verschiedene Pflanzen und Blumen. Es gab überall im Zimmer Fussabdrücke aus reifen Weizenähren. Das war seine schönste Liebeserklärung an seine Gemahlin im Diesseits. Überall wohin sie ihre Füsse setzte, spross Weizen in die Höhe, der dann sofort reifte und golden dastand.

Geb, der Erdgott, Vater des Osiris und direkter Erbe der Herrschaft über die Erde nach seinem Grossvater Re und Vater Schu, hatte sich im Zimmer auf dem Boden ausgestreckt. Da er wie sein Sohn eine grünlich Hautfarbe hatte, war er kaum zwischen den Pflanzen zu erkennen, die überall wuchsen. Geb lebte in der Erde und dort brauchte er keine Kleidung. Er wirkte wie ein Baumstamm, der längerer Zeit auf dem Waldboden verwittert war. Seine Haut war grünlich-hölzern, glatt und glänzend. Darunter war jeder Muskel zu erkennen. Am erdigen, durchdringenden Geruch des Fruchtbarkeitsgottes, der an ein wildes Tier erinnerte, konnte man unschwer erkennen wie wenig Wert er im Gegensatz zu seinem Vater Schu, auf sein Äusseres legte. Geb könnte ein grosser Bruder von Hans sein, dachte Thot: Handwerker, Tüftler, Gärtner. Seine wilde Mähne und sein Bart hatten eine braune Färbung und standen verfilzt und zottelig ab. Er sieht viel älter aus, als sein Vater Schu mit seinem luftig, lichtem Wesen, dachte Thot.

Seine Nilgans, Geb verliess seine Gefilde nie ohne seine Nilgans, die gewöhnlich auf seinem Kopf thronte, hatte es sich auf seinem Bauch gemütlich gemacht und schlief. Den Schnabel hatte sie oder er unter den Flügel geschoben. Ihretwegen, oder seinetwegen, die Götter wussten es nicht genau, wurde Geb auch 'grosser Schnatterer' genannt. Um den grossen Schnatterer nicht zu wecken, unterhielten sich auch Geb und Thot gedanklich. So war es still im Zimmer, obwohl sie miteinander sprachen. Nur das leise Schnarchen des Schnatterers und das eine oder andere 'Plobb!' eines Blattes, das sich ans Licht der Welt schob und entfaltete, waren zu hören.

'-Ich finde die Idee sehr gut' bemerkte Osiris. '-Vater meinte, er könnte mir die Gestalt des Benu geben.' Thot runzelte verwundert die Stirn. Der göttlich Vogel Benu, ein grosser Reiher, erschien nur alle 1461 Jahre. Der Benu war die Verkörperung von Osiris Ba-Seele.\footnote{ Später sollte er als Fabeltier in der Legende des Phönix weiterleben. Denn, wenn er auftauchte, baute er auf dem Tempel des Sonnengottes in Heliopolis sein Nest. Sobald die Morgenröte des nächsten Tages kam, verbrannte er, um verjüngt wieder geboren zu werden.}-'Das wäre gut, denn diese nächtliche Reise wirst du von den Ibissen bewacht, da passt die  Vogelgestalt.' Geb blickte mit hellbraunen Augen zu Thot hoch: -'Es sollte möglich sein, weil der Ort, den wir heute Nacht betreten ein Ei ist! Das kosmische Ei von Basel hat die Funktion eine Eies und deshalb kann sich Osiris in seiner Vogelgestalt darin bewegen. Es ist als wenn das Vogelkind in einem solch grossen Ei ist, dass es da drinnen herumspazieren kann!' -'Geb, das ist genial! Osiris kann sich innerhalb des Eies frei bewegen!' Geb schmunzelte, -'Genau!' 

-Wir bräuchten jemanden aus dem Jenseits, der uns im Ei begleitet. Wir wollen ja nicht so wirken, als würden wir alles in Beschlag nehmen, gab Geb zu Bedenken. Osiris antwortete, -ich dachte an Hans! Er weiss alles über die Stadt. Thot überlegte, -logisch! zumal er auch in der jenseitigen Welt zuhause ist. Ich werde ihn gleich fragen, aber es sollte kein Problem geben, er ist ganz wild, haha! sein Basel zu zeigen.

Das bedeutete jedoch einige Arbeit für Vater und Sohn. Geb setzte sich auf, nicht ohne vorher vorsichtig, vorsichtig den Schnatterer sanft auf den Arm zu nehmen und ihn an Osiris Fussende auf das Bett zu setzten. Der Schnatterer schnorchelte und steckte den Schnabel unter den anderen rotbraunen Flügel mit der schönen dunkelgrün schillernden Spitze. Geb schaute ihn liebevoll und zärtlich an. Thot grinste wie hielt Nut, die Himmelsgöttin, es aus mit ihrem gansverrückten Göttergatten? 

\section*{5}
\addcontentsline{toc}{section}{5}

Nachdem Anubis gegangen war, kamen auch Kebi und Amset in den Garten. Amset sass auf dem Boden, neben der kleinen Höhle aus Zweigen und Blättern, die er für Hapis Frau und seine Nichte und die Neffen gebaut hatte und zupfte daran und stopfte Löcher in der Wand. Die drei kleinen Paviankinder machten sich einen Spass daraus alle Blättchen wieder aus der Höhle zu ziehen. 

Duamutef lag unter einem Busch. Kebi hatte sich auf einen Ast über seinem Kopf niedergelassen. Hapi meinte -'wir müssen die Liste anschauen und und uns aufteilen. So haben wir mehr Zeit, um die Götter und Toten der Stunde zu befragen.'

-'Amset, du übernimmst die Klagefrauen.' meinte Duamutef. 'Dich haben sie am liebsten. Amset?' Duamutef stupste seinen Bruder ans Bein. Amset fiel ein Blatt aus den Fingern und das verklärte Lächeln auf seinem Gesicht wurde von einem verwirrten Blick abgelöst. -'Hä?' Amset blickte Duamutef ratlos an. -'Amsi! Améliehie retten!' prustete Kebi und flatterte mit den Flügeln. "`Amélie!"' hauchte Amset. -'Oh, meine Güte!' schnaufte Duamutef und vergrub seine Schnauze zwischen den Vorderläufen. 

Obwohl Amset sich dann sehr konzentrierte, musste er sich den restlichen Nachmittag mit dem Gekicher und den Kommentaren seiner Brüder zufrieden geben. Nur Hapi hielt sich etwas zurück, schliesslich war er selbst verheiratet. 

Als Amélie die Brüder zum Abendessen rief, waren sie für die Nacht bereit. Dachten sie\dots

\section*{6}
\addcontentsline{toc}{section}{6}

"`Nochmal!"' bellte Tefnuts Stimme durch Amélies Zimmer. Während die Horussöhne sich im Garten berieten, musste Amélie unter der strengen Aufsicht von Tefnut für die Nacht trainieren. Sie lag auf dem Bett und übte ihren Körper zu verlassen und mit ihrer Seele, eingehüllt in einen Teil der frisch gewonnenen Lebenskraft im Zimmer zu spazieren. 

Tefnut hatte sich für das Training umgezogen. Amélie musste zugeben, dass die Löwengöttin auch im Catsuit mit Tigermuster hervorragend aussah. Die rote Mähne hatte sie mit einem Frottéstirenband gebändigt und die Füsse steckten in Gymnastikschuhen.

Zu Beginn hatten sie Dehnungsübungen gemacht, wobei Tefnut Amélie anwies nicht nur auf ihren Körper und die richtige Haltung zu achten, sondern auch auf den Raum, um sich. "` He, ich spüre etwas!"' Hatte Amélie verwundert gerufen. Es hatte sich angefühlt, als ob die Luft um Amélie in Schwingung geraten wäre. Etwas bewegte sich hauchzart, nachdem Amélie still stand. 

"`Was für Übungen waren das?"' "`Die ersten sind von einem alten Freund aus China. Die anderen sind recht neu. Der Doktor hat sie vor kurzem erfunden."'  "`Doktor?"' fragte Amélie. Sie hüpfte durch das Zimmer und kreiste mit den Armen. "`Schluss jetzt, mit dem Geplausche!"' Meinte Tefnut,"`jetzt geht es los, legt dich auf dein Bett."'

Aber es gab Schwierigkeiten. Amélie schaffte es ihre Füsse und Beine zu lösen und ihren Kopf. Jedoch im Herzbereich blieb die Seele wie festgeklebt an ihrem Körper hängen. Tefnut hatte vorsichtig an der Seele gezupft, aber sie rührte sich nicht von der Stelle. "`Das ist doch wie verhext!"' rief die Göttin endlich. Amélie zuckte zusammen und wurde bleich. "`Verhext?"' "`Tschuldingung, ist mir so rausgerutscht!"' "`Aber,"' murmelte Amélie, "`du bist die Wahrheit!"` "`Stimmt!"' Tefnuts grüne Augen schillerten. Sie dachte nach. Plötzlich klopfte es an der Tür.

Tefnut beeilte sich sie zu öffnen. "`Komm' rein, Vater!"' Der Sonnengott betrat Amélies Zimmer. Tefnut musste ihn gerufen haben. "`Wir haben es versucht. Aber Amélies Seele löst sich nicht vom Herzbereich. Und irgendetwas sagt mir, dass es nicht gut wäre es mit Gewalt zu versuchen."'

Re setzte sich neben Amélie auf das Bett. Er nahm ihre Hand in die seine und lächelte ihr aufmunternd zu. "`Ich kenne da einen Trick!"' meinte er. "`Wenn sich die Seele nicht von Amélies Körper lösen will, dann muss sie den Körper halt mitnehmen. Oder einen Teil davon."' "`Wie"' Amélie fuhr auf. "`Wie willst du einen Teil mitnehmen?"' "`So!"' Der Sonnengott packte Amélies Handgelenk fester, bevor sie die Hand wegziehen konnte. Er zog ein Messer mit einer silbernen Klinge aus der Tasche und bevor jemand etwas sagen konnte, stach er in Amélies Zeigefinger. "`Schnell, Tochter!"' Re balancierte den Finger, damit der Blutstropfen nicht auf das Bett fiel. Amélie starrte wie ein hypnotisiertes Kaninchen auf den Blutstropfen und ihre Fingerspitze. Amélie wurde schwarz vor Augen. Sie plumpste zurück in ihr Kissen. 

Als sie aufwachte trug Tefnut ein breites Grinsen im Gesicht und die Kette mit dem Rubin um den Hals. Re war verschwunden. Amélie blinzelte. "`Wo ist Re?"' "`Er ist wieder gegangen, nachdem er das gemacht hat."' Tefnut hielt Amélie den Rubin vor die Nase. Im Inneren des Steines sah Amélie einen dunklen Fleck, der sich träge bewegen konnte. "`Was ist das?"' "`Das ist dein Blut! Auf diese Weise können wir einen Teil deines Körpers mitnehmen."' Tefnut lächelte.

"`Komm, versuch es gleich!"' Amélie konzentrierte sich wieder, diesmal versuchte sie sich vorzustellen, sie wäre in dem Rubin mit dem Blutstropfen eingeschlossen. Es klappte. Ihre Seele löste sich vom Körper und sie konnte im Zimmer damit herumgehen. Auf Tefnuts Geheiss, spazierte sie sogar wie eine Fliege an der Decke. Der Rubin war eine Zwischenstation geworden. Amélies Körper auf dem Bett hatte eine blaurote Verbindung zu dem Rubin und von dort führte eine grünrosige zu Amélies Seele.

Sie wurde ausgelassen und purzelte saltoschlagend durch das Zimmer. "`Perfekt!"' Tefnut strahlte. 

Amélie flatterte mit den Armen und stieg in die Luft, wo sie sich drehte. "`Warum strampelst du so?"' fragte Tefnut. "`In der ätherischen Welt funktionieren die Dinge mit Gedanken, Schätzchen!"' Amélie beobachtete Tefnut, die in Sekundenschnelle in allen vier Zimmerecken auftauchte und wieder verschwand. Die Göttin flackerte. Ihr Tigercatsuit hinterliess dabei gelb-schwarze Schlieren in der Luft, ansonsten konnte Amélie keine Regung entdecken. "`Wahnsinn!"' Entfuhr es ihr. "`Probier es aus! Je besser du es kannst, umso besser für dich heute Nacht!"'

Amélie übte und übte. Am Anfang, so hatte Tefnut gesagt, müsste sie sich bewegen, wie sie es mit ihrem Körper auch machte. Dies wäre so, damit die Seele, die es gewöhnt war mit dem gesamten Körper auf der Erde zu spazieren, nicht einen Körperteil des ätherischen Körpers vergass. "`Sonst stehst du dann plötzlich ohne Hände da. Und das wäre dumm, denn die brauchst du."' Hatte Tefnut erklärt.

Also spazierte Amélie im Zimmer auf und ab. -Es ist schwer. Wunderte sie sich. Es ist als ob meinem Gehirn die Muskeln fehlen, in Gedanken die Seele, oder wie heisst es, den ätherischen Körper zu bewegen\dots Amélie bewegte sich schneller. Sobald sie jedoch einen Körperteil wie Arme oder Beine vergass, löste sich der ätherische Körper auf und sie wurde zu einem Teil der Luft und konnte sich nicht mehr bewegen.

Ihr Körper aus Fleisch und Blut lag die ganze Zeit auf dem Bett und wirkte als ob er schliefe. "`Es ist fast wie schlafen"' hatte Tefnut gesagt. "`Aber es ist etwas anderes, den normalerweise bleibt dein ätherischer Körper bei dem aus Fleisch und Blut zurück, wenn du schläfst. Deshalb müssen wir sehr vorsichtig sein. Die Verbindung darf nicht abreissen, sonst bleibt in dem Bett nur ein Haufen Fleisch und Knochen liegen, der nicht mehr lebt."' 

Nachdem sie es geschafft hatte in der einen  Ecke zu verschwinden, um in der anderen wieder aufzutauchen, musste Amélie trainieren die Insignien der Elemente zu benutzten. Sie griff nach dem Schwert und hielt in ihrer ätherischen Hand, die sich wie aus Wasser bewegte, das Schwert, das ebenfalls aus Wasser zu bestehen schien und durchsichtig schimmerte. "`Sieh' doch!"' Amélie schwang es und durchtrennte die Luft. Sie fuchtelte begeistert und schnitt ein Blatt der Rose ab, die Hans ihr am morgen gepflückt hatte.

Das Blatt fiel auf die Kommode und als es sie berührte löste es sich in Staub auf. "`Oh!"' Amélie starrte auf das schillernde, kaum sichtbare Schwert. Mit zitternder Hand legte Amélie das Schwert zurück. Sobald es in der Nähe des echten Schwertes kam, verschmolz es damit. Nur wenn Amélie die Hand über das Schwert hielt, konnte sie die Konturen des ähterischen Schwertes fühlen. "`Wenn du mit diesem Schwert etwas abtrennst, trennst du dasjenige vom Leben!"' Tefnut sprach sanft.

"`Aber mein richtiger Körper?"' fragte Amélie. "`Bleibt der einfach allein hier in der Kammer?"' "`Anubis wird darüber wachen. Er ist der Wächter der Schwelle. Er kennt sich am besten aus, was passiert, wenn sich die Verbindung löst und ein Körper stirbt."' Antwortete Tefnut. "`Ich will aber nicht sterben!"' sagte Amélie. "`Alle reden dauernd vom Sterben!"' Tefnut dreht sich zu ihr um und kniff sie in die Wange und zog sie leicht daran: "Tja, Mädel, dann bleib einfach am Leben!"' Mit diesen Worten verliess die Göttin das Zimmer, wobei sie schrumpfte und schliesslich in ihrer Katzengestalt mit aufgerichteten Schwanz laut schnurrend aus dem Zimmer spazierte.

Amélie schlüpfte in ihren Körper zurück, der immer noch auf dem Bett lag. Die Muskeln waren völlig entspannt. Bevor Amélie die Augen öffnen konnte, war sie richtig eingeschlafen.

\section*{7}
\addcontentsline{toc}{section}{7}

In Thots Bureau fand die letzte Besprechung statt. Anubis lag in seinem Korb Der Erdgott Geb sass auf dem Teppich, der neben seinem persischen Muster schon diverse andere in der kurzen Zeit dazu bekommen hatte. Geb hatte sich gegen den Schreibtisch gelehnt und neben ihm war der dunkle Tintenfleck. Unter ihm zeigten sich die ersten schüchternen Grashalme, die sich in ihrer Höhe und Grüntönen dem Teppichmuster anpassten.

Tefnut und Schu teilten sich einen Sessel. Tefnut hatte sich in ihrer Katzengestalt auf dem Schoss ihres Gatten eingerollt. Schu trank einen Martini. Während sein Vater Geb an einem Bier nippte.

Thot hatte in dem andern Sessel Platz genommen. Auf dem kleinen Tischchen stand ein Teller mit Schokolade. Diesmal mit Pistazienkernen. Und eine kleine Tasse Espresso.  

Der Sonnengott betrat als letzter das achteckige Zimmer und setzte sich in den letzten leeren Sessel. Er legte die Beine übereinander und schaute vergnügt in die Runde. Er hatte nichts lieber, als mit seinen Kindern, Enkeln und Urenkeln zu reisen. "`Thot, was sieht das Reiseprogramm vor?"' "`Heute Nacht werden wir das erste mal an Land verbringen. Basel verfügt über einige 'Barken, die in der Erde sind', allerdings heissen sie hier Kirchenschiffe."'\footnote{Amduat: In der offiziellen Reiseleitung der dritten Nacht ist der Sonnengott in seinen vielfältigen Gestalten in mehreren Barken, 'die in der Erde sind', unterwegs.}

"`Was hat es mit diesem Ei auf sich?"' fragte Schu. "`Das Ei schützt das Zentrum dieses Ortes. Und es gibt Osiris die Gelegenheit sich frei darin zu bewegen. Er reist wie Amélie mit seinem Ba, seiner Seele."' "`Du meinst als Benu? Ist es nicht zu gefährlich für Osiris jetzt schon in die Stadt zu gehen?"' "`Natürlich"', antwortete Thot: "`Aber er muss sich mit seinem Sternenbild, dem Orion verbinden. Er muss sich, um es mit modernen Worten zu sagen, seinen Landeplatz abstecken\dots"' -'Ich hoffe nur, wir kommen damit durch' gab Anubis zu Bedenken. -'Seth wartet nur auf seine Gelegenheit\dots' 

Thot nahm einen Schluck aus seiner Tasse und sagte: "`Wir werden ein Ablenkungsmanöver starten. Die Barken der Nacht werden auf dem Rhein starten und Horus wird sein bestes tun, um seinen Urgrossvater zu vertreten. Isis kennt ihren Bruder und kann ihn ablenken, wenn er auftaucht. Auf diese Weise sollte es möglich sein, dass Re an Land bleiben und mit uns kommen kann."'

"`Es ist so aufregend!"' Rief der Sonnengott und klatschte in die Hände. "`Kinder, Schu und Tefnut ihr bekommt eines meiner Augen und das andere nimmst du mit Geb. Mit den Augen wird euch nichts entgehen."' Der Sonnengott nahm die Sonnenbrille ab. Die anderen Götter hielten sich die Augen zu, den selbst für göttliche Augen, strahlen die des Sonnengottes zu hell. Sobald Re sie herausgenommen hatte, wurde es dunkler hinter den Augenlidern der anderen Götter. Sie öffneten ihre Augen wieder. Re hatte die Sonnenbrille wieder aufgesetzt. In seiner Hand lagen zwei leuchtend gelbe, durchsichtige Edelsteine mit zahlreichen Facetten. Ein sanftes leuchten ging von ihnen aus. Schu und Geb nahmen je einen Stein und verstauten ihn in einem kleinen seidenen Beutel, den jeder von den Zweien um den Hals trug. Als Gott wusste man nie, in welcher Gestalt man sich bewegen würde und mit einem Beutelchen und einer Halskette konnte man mit seinen Accessoires nichts falsch machen, auch wenn man sich zwischendurch wie Schu in einen Löwen verwandeln sollte.

 Danach trennten sie sich und jeder ging in sein Zimmer, um sich für die Nacht auszurüsten.
 
  \section*{8}
\addcontentsline{toc}{section}{8}
 
 "`Bäh!"' Amélie hatte die Arme weit von sich gestreckt und gab sich alle Mühe nicht zu tief Luft zu holen. In jeder Hand hielt sie einen stinkenden, grossen und sehr toten Karpfen. Sie war auf dem Weg zum Teich.
 
 "`Schätzchen, ich habe da was für dich. Du musst dich schliesslich bei Sobek, deinem Lehrmeister der letzten Nacht bedanken!"' Mit diesen Worten hatte Hathor Amélie in der Küche empfangen. Nachdem Amélie aufgewacht war, hatte sie erschrocken bemerkt wie spät es war und war in die Küche geeilt. Sie meinte, die anderen müssten schon das Nachtessen verputzen.
 
Hathor war in die Vorratskammer verschwunden und hatte aus einem Holzfass die beiden Karpfen gezogen. Sobald sie den Deckel gehoben hatte, schlug ein bestialischer Gestank aus dem Fass. "`Das ist meine Vorratsbox für die Aasfresser."' Hatte Hathor erklärt. Sie strahlte, wie nur eine Grossmutter und Mutter strahlen kann, die an alle ihre Schützlinge beim Picknick gedacht hatte.

Amélie war vorsichtig nach draussen getappt und hatte Glück, denn die maroden Fische blieben ganz bis sie am Teich ankam.

-Hallo! Brutales Kind! Sobek hatte sie scheinbar erwartet, denn sobald sie angekommen war, kletterte er mit seinem Vorderteil an Land. Er war riesig. -Wenn es gestern Nacht nicht so dunkel gewesen wäre, wäre ich vor Angst abgesoffen! Dachte Amélie. Wenn ich wollte, könnte ich mich bequem in das Maul legen\dots -'Machs doch! Wie war es möglich mit einer Krokodilschnauze zu grinsen? Sobek konnte es. Er klappte sein Maul auf und Amélie sah die fingergrossen Zähne. -Wuäh! -'Tja, Mädchen, du darfst mir gerne mal die Zähne putzen! "`Witzig!"' Amélie hatte endlich die Sprache wieder gefunden. "`Ich habe dir was mitgebracht!"' Amélie hielt die Karpfen hoch. Sobek hob witternd die Nüstern. Amélie bemerkte die grosse Wunde in der Wange die das Schwert gerissen haben musst. Verschämt liess sie die Arme sinken. - Wirds bald? Sobek schielte auf den Fisch. Amélie warf die gammeligen Fische einen nach dem anderen in seinen Rachen.

Das gewaltige Maul klappte zu. -'Mmmh! Jamjam!' "`Danke! Danke, dass du mir geholfen hast"' -'Danke, dass du mich fressen wolltest! Meinst du wohl? Sobek schaute Amélie mit seinen kleinen Reptilienaugen an. -Das hast du doch gedacht, oder? "`Nein! Äh, ja! Also\dots"' Amélie wurde rot. -Recht hast du! Natürlich hätte ich dich gefressen, wenn du dich dumm angestellt hättest! Aber, Hut ab, hast es gut gemacht! Das mit dem Schwert war gemein! "`Tschuldigung!"' Murmelte Amélie. -Mache deine Sache gut, Mädchen! Und wenn du Hilfe brauchst, kannst du auf mich zählen. Auch jetzt schien das Krokodil zu grinsen.

Amélie ging wieder ins Haus in die Küche, während Sobek sich langsam zurück in den kleinen Teich schob. feine Ringe bildeten sich auf dem Wasser. Zur gleichen Zeit bildeten sich unterhalb des Blauen Hauses bei der Anlegestelle der Götterbarke Ringe im Fluss. Für kurze Zeit war ein schuppiger Rücken zu sehen. -Zwei klitzekleine Fische\dots Wird Zeit für eine richtige Mahlzeit, dachte das Krokodil und schwamm den Fluss bergauf. 

\chapter*{3. Nacht}
\addcontentsline{toc}{chapter}{3. Nacht}


\begin{quotation}

\emph{III Et sicut omnes res fuerunt ab uno, meditatione unius, sic omnes res natae fuerunt ab hac una re, adaptatione.\\Und so wie alle Dinge aus dem Einen stammen, durch einen Gedanken des Einen, so sind alle Dinge aus dieser einen Ursache durch Anpassung entstanden.  \\Tabula Smaragdina}

\end{quotation}



\section*{1}
\addcontentsline{toc}{section}{1}

Sie hatten sich nach dem Abendessen vor dem Haus getroffen. Anubis war in Amélies Zimmer vor ihrem Bett geblieben und hütete ihren Körper. Amélie hatte vor lauter Aufregung einige Zeit gebraucht, bis sie ihren Körper verlassen konnte. Erst als Tefnut als Katze in ihr Zimmer kam, den Rubin an einem Halsband um den Hals,  schaffte Amélie es, sich ganz von ihrem Körper zu lösen. -'Was ist mit den Insignien?' fragte sie. -'Du rufst sie, wenn du sie brauchst. Es ist viel zu anstrengend sie in Gedanken die ganze Zeit mitzunehmen.'

Amélie verliess das erste mal ohne Körper ihr Zimmer. Sie schwebte hinter Tefnut her. Ab und zu sah sie die Lebensschnur, die zu dem Rubin führte, aufblitzen.

Vor der Haustür waren sie dann zu Re und Thot gestossen. Re trug auch in der Dämmerung seine Sonnenbrille. Er trug eine bretonische Kabanjacke und eine Kapitänsmütze, als Attribut an den Basler Winter. Thot hatte sich in einen dicken grauen Wollmantel mit Pelzkragen gehüllt. Er trug eine Ballonmütze, die er tief ins Gesicht gezogen hatte und Fellhandschuhe. Amélie, die nur für die Götter sichtbar war, schlüpfte zusammen mit Tefnut dem Luftgott Schu hinterher durch die Haustür.

Der Luftgott hatte sich nicht umgezogen und stand lässig mit Jeans, T-Shirt und barfuss in der frostigen Winterluft. Sie schauten über die Terassenmauer auf den Rhein. Unter ihnen raschelte es, als die übrigen Götter der Reihe nach aus dem Unterirdischen Gang kamen, der das blaue Haus mit dem Garten verband. Feierlich schritten sie zur Barke. Re kicherte, als er seinen Urenkel Horus mit den Widderhörnern sah. "`Kinder, es ist toll nach den tausenden Jahren mal ohne die Hörner unterwegs zu sein!"' Er hob die Nase und sog die Abendluft laut ein. "`Kommt! Auf zu den Schiffen!"' -Schiffe? fragte Amélie in Gedanken. "`Die Kirchenschiffe!"' -Hä?!

"`Ach, Kinder! So ein freier Abend in den Ferien ist unbezahlbar!"' Schwärmte Re. Sie standen auf dem Marktplatz, den sie durch das schmale, nach Pissoir stinken Martinsgässlein erreicht hatten. Amélie war froh, ohne Nase konnte sie den Gestank nicht riechen. Sie konnte ihn jedoch wie eine popelgrüne und schwefelgelbe Wolke sehen und spürte ihn wie kleine Nadeln an ihren Lebensleib piksen. Amélie schmunzelte: Hätte nie geglaubt, dass ich Pipigeruch spannend finden könnte, dachte sie. Die Götter achteten nicht darauf.

Sie gerieten dafür auf dem Marktplatz aus dem Häuschen, als sie den offiziellen, alljährlichen Weihnachtsbaum mit den grossen glänzenden, roten Kugeln und den vielen Lichtern erblickt hatten. "`Hey, Babe! Das 's was!"' Staunte Schu. Der Luftgott lächelte und schob den Joint vom einen Mundwinkel in den anderen. Sie standen still und staunten mit offenem Mund. Tefnut hatte sich in ihre menschliche Gestalt begeben. Ihr Gatte hatte ihr sanft einen Arm um die schlanke Taille gelegt und sie ihren Kopf an seine Schulter geschmiegt. Auch Amélie staunte. In der geistigen Welt war die grosse Tanne viel beeindruckender. Sie strahlte Wärme ab. Während der Baum selbst eine violette leuchtende Silhouette hatte und die Glaskugeln silbern leuchteten, strahlte er gelbes, warmes Licht aus. Der violette Baum pulsierte zart, wie ein schlagendes Herz. Die Bewegung setzte sich wie Wasserringe in alle Richtungen fort über den ganzen Platz.

Amélie wunderte sich. Die dunklen Kabel glühten in der Götterwelt rot-gelb. Es wirkte, als ob die Tanne dort wo die Kabel verliefen, von einem kleinen zähen, rot-gelben Lavastrom umkreist wurde. Amélie bemerkte nach einiger Zeit ein Stöhnen und Ächzen, als würde sich jemand sehr anstrengen. Die Lichter selbst, waren dunkel. Eigentlich waren sie das einzige an dem Weihnachtsbaum, was nicht leuchtete.

"`Kinder, das wir das erleben dürfen!"' Schniefte Re und wischte sich die Augen. Er legte Thot schwer eine Hand auf die Schulter. Auch Thot schien aufgewühlt zu sein. Er legte die Hände aneinander und verneigte sich! Amélie traute ihren Augen nicht! Was war denn mit denen los, schliesslich war dies nur eine Tanne mit glitzernden Kugeln! Sie wendete sich dem Baum zu. Zugegeben, er war beeindruckend! 

"`Du wirst es später verstehen, Amélie! Aber wir müssen jetzt weiter!"' Sagte Thot mit bebender Stimme. -'Du wirst es später verstehen, du wirst es später verstehen, äffte Amélie Thot nach. Padautz! bekam sie einen Klaps von Tefnut an den Kopf. "`Pass' auf, was du denkst! Grade in dieser Nacht!"' zischte die Göttin. Sie verwandelte sich in die rötliche Katze zurück und lief hinter den drei Göttern her, die sich schon dem Totengässlein zugewendet hatten, das sie direkt vor die Peterskirche brachte.

Amélie huschte hinter den Göttern her, nachdem sie ihren Lebenskörper und ihre Seele sortiert hatte. Natürlich hatte der Klaps Tefnuts nicht weh getan, aber er hatte die unzähligen kleinen Energiebällchen durcheinander gewirbelt. Amélie musste sie wieder in Ordnung bringen, indem sie sich vorstellt, wo jedes der Teile zu sein hatte. 

"`Schönes Schiff!"' Meinte Re, als sie auf dem kleinen Platz vor der Peterskirche angekommen waren. "`Es wirkt allerdings etwas angestaubt, oder Thot?"' Thot antwortete seinem Chef: "`In der Tat, die Menschen haben andere Interessen."' "`Was haben die denn für Priester?"' fragte Schu erstaunt. "`Ich meine, so'n richtiger Priester, der mit Herzblut bei der Sache ist, der sollte doch das Volk auf Trapp halten!"'

Sie hörten eine ärgerliche Stimme. Sie gehörte einem jungen Mann, der eine Aktentasche trug und laut mit jemandem stritt, obwohl er allein unterwegs war. "`Ein Besessener?"' maunzte Tefnut erfreut und lief auf Katzenart mit erhobenem Schwanz auf den Mann zu, der ihnen von der Universität aus entgegenkam. Er blieb wieder und wieder stehen. "`Nein Schatz! Ich bin schon unterwegs! Ich hatte Seminar!\dots Was kapierst du daran nicht?"' Die Götter schauten interessiert. Sie konnten in dem Studenten lesen wie in einem offenen Buch. 

Auch Amélie bemerkte mit der Zeit Farben. Es waren aber nicht nur Farben. Mit ihrem ganzen Lebensleib konnte sie den Lebensleib der anderen Person wahrnehmen. Neben dem Lebensleib, den Amélie schon etwas kannte, umgab ihre Seele ein weiterer Körper. Dieser war es, der die Farben produzierte. Die Farben hüllten die Seele ein und durchdrangen sie. Sie veränderten sich ständig und dennoch blieben einige 'Kleckse' wie ein Grundton an bestimmten Stellen stehen. -Das muss der Sternenleib sein, den Thot erwähnt hat, dachte Amélie. -Sieht lustig aus! Ob das die berühmte Aura ist?

Amélie lachte. Es fühlte sich merkwürdig an. Der Mann war wütend und deshalb leuchtete seine Bauchregion rot. Das Rot wirbelte in seinem Körper, vorallem im Bauch. Allerdings trat die Bewegung auch nach aussen. Der Wuttornado bewirkte eine Bewegung in der Umgebung, die Amélie spürte, auch wenn die Energibällchen sich nicht stören liessen und munter wie ein Insektenschwarm gemütlich durcheinander flogen.

"`Was willst du denn? Ich bin ja auf dem Weg! Was hat denn die Julia jetzt damit zu tun? Nein! Die war überhaupt nicht da! Nerv' mich nicht!"' Der junge Mann war lauter geworden. Was Amélie jedoch viel mehr interessierte war das Wesen, das bei den letzten Worten aus dem Mund des Mannes geschlüpft war. Gleichzeitig hatte sich eine grau-grüne, matschige Farbe in das Rot gemischt. -Also ob er wie die Pechmarie im Märchen mit Pech beworfen wurde, bemerkte Amélie.

"`Schon ein recht prächtiges Dämonchen, nicht wahr?"' Thot war neben Amélie getreten und wies auf die hässliche, durchscheinende Gestalt, die sich um den Mann tummelte. Das Biest war eklig. Es schien schleimig zu sein und war nackt. Der Kopf war klein, das grösste daran der breite, wullstige Mund aus dem spitzte Zähne ragten. Arme und Beine waren dünn und verkümmert. Der Bauch war dick  und unter dem Lendenschurz, den das Wesen trug konnte Amélie ahnen wie gross das Geschlechtsteil sein musste. 'Bäh! Das Ding ist ja pervers!' Entfuhr es ihr. "`Es muss obzön sein!"' antwortete Thot. "`Wieso?"' fragte Amélie. "`Schau, es geht weiter."' 

"`Sophie, jetzt hör doch mal! Das ist nichts! Ich lieb dich doch!\dots"' rief der Mann in das Telephon. Er war inzwischen stehen geblieben und ruderte mit den Hände zu seinen Worten. "`Süsse, komm! Ich will nur dich\dots Ich bin gleich bei dir! Ja, dann reden wir! Ja, Schatz\dots"' In der Zwischenzeit hatte sich eine Frau dem Mann genähert, die ebenfalls aus der Universität kam. Sie trug einen Mantel mit Fell und Stiefel mit hohen Absätzen. Als sie den Mann erreichte, schmiegte sie sich von hinten an ihn. Er erschrak furchtbar und stiess sie weg. "`Ich, ich muss Schluss machen\dots"' rief er und beendete das Gespräch. "`Spinnst du, Julia? Das war knapp!" brüllte er die Frau an. "`Wieso, du wolltest doch mit der fetten Kröte eh Schluss machen!"' antwortete die schnippisch. Der Mann hob beschwichtigend die Hände: "`Ja, klar! Gib mir etwas Zeit! Hei! Süsse!"' Er nahm die Frau in den Arm und säuselte:"`Ich lieb' dich doch!"' Amélie konnte es nicht fassen, der benutzte sogar dieselben Worte! 

"`Schau hin, Amélie! Du wolltest wissen, woher der Dämon kommt. Du solltest es selbst rausfinden können. Dieser Mann ist eine gute Fundgrube!"' Amélie hatte entsetzt gesehen wie bei jedem weiteren Wort, weitere Dämonen aus dem Mund des Mannes kletterten. Sie huschte dichter an ihn ran, obwohl es sie gruselte, denn im Gegensatz zu dem Mann, konnten die Dämonen sie sehen. Sie machten mit ihren aufgeblasenen Lippen Küssmünder und einer griff sich unter den Lendenschurz und wackelte mit dem riesigen Penis. Dann kreischten sie vor Lachen, als Amélie schnell wieder zu Thot flüchtete. Je näher sie dem Mann gekommen war, umso mehr von den Dämonen konnte sie sehen. Es waren viel mehr, als die, die in dieser Nacht aus seinem Mund geschlüpft waren. Um den Mann wimmelte es von ihnen. In allen Grössen und in verschiedensten Formen. Alle waren hässlich und erstaunlich laut.

"`Tschau, mein Schatz!"' der Mann gab Julia einen Kuss auf den Mund und sie eilte weiter. Als sie an ihnen vorüber ging, konnte Amélie auch bei ihr einen ganzen Schwarm Dämonen entdecken, die ihr folgten. Sie zerrten an den Farben, die Amélie um Julia erkennen konnte und rissen Stücke davon ab und stopften sie in ihre riesigen Mäuler. 

Amélie wendete sich zu Thot um. Aber plötzlich schwirrte vor ihrer Nase ein weiterer Dämon. Er sah sie an und kreischte erschrocken. Dann flüchtete er hinter ihren Rücken. Dort begann er dann wie irre zu kichern. "`Tja, sie sind nicht die hellsten!"' Schmunzelte Thot. "`Das ist allerdings dein Dämon, Amélie!"' -'Wie, mein Dämon? Ich hab keine Dämonen!' Entrüstete sie sich und im gleichen Moment schlüpfte ein kleines, geflügelter Winzling aus ihrem Mund. Er hielt sich riesige Hände vor die kleinen Augen. 

Als ob er wie ein kleines Kind 'Guck-guck' spielen wollte, nahm er die Hände langsam vom Gesicht und klatschte sie dann, nachdem er Amélies Blick aufgefangen hatte, wieder vor die Augen. Dabei schien er sich hervorragend zu amüsieren und quietschte vor Vergnügen. Mit der Zeit sammelten sich mehr und mehr Dämonen aus Amélies Umfeld und alle hielten sich die Hände vor die Augen und spielten 'Guck-Guck'! Das Gekreische und Geschnatter wurde Ohrenbetäubend. Amélie nahm die Hände vor das Gesicht und wendete sich ab, was nichts half, denn ihre Hände waren durchsichtig und die Dämonen wohnten in ihrem Seelenraum und scherten sich im wahrsten Sinne einen Teufel darum, ob Amélie sie haben wollte oder nicht. Sie flatterten aufgeregt um sie herum und schnappten nach ihren Gefühlen und nach ihrer Seele. Amélie ging in die Knie und versuchte sich zusammen zu rollen. Je ängstlicher sie wurde, umso grössere Stücke konnten die kleinen Teufel aus ihr heraus beissen.

-'Ihr kleinen Miststücke! zischte Amélie. Am Ende der Verwirrung und Angst spürte sie ein Fünkchen Hass. -'Ihr Miststüüüücke!' Amélie sprang auf und streckte wie im Schlaf ihren Arm und die Hand aus. Wie der Stab in ihre Hand kam wusste sie nicht. Aber mit dem grössten Vergnügen schlug sie damit auf die kleinen Gestalten ein. Mit kräftigem Schwung fuhr sie in die Menge. Die kleinen Dämonen kreischten, diesmal vor Schmerz und zogen sich zurück. Leider konnten sie aber nicht ausweichen, weil sie zu Amélies Umfeld gehörten.

"`He, Mädchen!"' nach einiger Zeit begriff sie, dass Schu mit ihr sprach. -'Was!- schrie sie wütend und hieb ein weiteres mal durch die Luft. "`Hör' auf!"' Schu kam und packte ihre Hand wie ein Schraubstock. "Mädel, du hast diese kleinen Kerle selbst gemacht! Sie können nicht verschwinden!"' Amélie sah in die klaren, blauen Augen des Luftgottes. Die Berührung seiner Hand beruhigte sie. Amélies Gedanken beendeten ihre Achterbahnfahrt und bewegten sich rasant auf ein Zentrum zu. Sie stellte den Stab vor sich auf den Boden. "`Die kleinen Racker sind nicht nett! Und die grossen sind gefährlich! Aber du hast sie gemacht. Du bist für sie verantwortlich!"'

Ein Schauer lief durch Amélies Seele. Der Luftgott hatte sie los gelassen und stand still vor ihr und betrachtete sie, während die kleinen Dämonen aufgebracht und nervös um ihn und Amélie flogen und zappelten. Amélie versuchte sich nach Innen zu wenden, einen Ort zu finden, an dem sie nicht von dem Gebrabbel und Gezappel abgelenkt werden würde. Sie fand den Ort in ihrer Mitte, im Herzen. Von dort aus konnte sie ihre Gedanken fassen. "`Gut, Mädchen!"' Schu lächelte jetzt. Amélies nicht vorhandenen Knie fingen an zu zittern, er hat mich gelobt\dots "`Du darfst sie sogar benutzten! Deine Gedanken!"' Amélie schluckte die ironische Bemerkung, die ihr auf der Zunge lag runter. -Warum kann sich so ein schöner Machoa*** alles erlauben?! Dachte sie bitter. "`Gut! Aber vielleicht konzentrierst du dich nun weniger auf meinen wohlgeformten Hintern!" 

Schu grinste, aber freundlich. "`Alles ist miteinander verbunden! Du kannst die Nervensägen nur loswerden, wenn du sie akzeptierst. Du solltest sie lieben, denn du hast sie gemacht."' -'He! Das sind kleinen, hässliche, fiese Biester!' "`Ja! Ich habe auch nicht gesagt, du sollst sie zum Knuddeln mit ins Bett nehmen, was du ohnehin tust. Du kannst dich aber nicht von ihnen trennen, also kannst du sie nur akzeptieren."' Amélie blickte den Luftgott zweifelnd an. Sie bemerkte ihr Herz, das schneller schlug und spürte den Stab in ihrer Hand, dessen Spitze anfing sanft zu glühen.

Amélie schaute auf den Boden. Sie presste ihre Lippen zusammen. -,Sag` mal, Schu, läuft bei euch Göttern alles auf dieses Liebe-alle-und-Blablubb raus? Ich finde es gruselig, wenn ich alles lieben soll, egal wie eklig es ist\dots Ich komme mir vor wie ein Idiot!' Schu war der einzige der Götter, den Amélie sich getraute das zu fragen. Wer sich als Gott bekiffte, sollte von solchen Fragen nicht schockiert sein. Schu dachte nach, während er einen tiefen Zug nahm. ,,Schätze schon\dots'' -,Äh, was?' Fragte Amélie irritiert. ,,Schätze schon, dass es darauf hinausläuft\dots'' -,Was für ein Sch***!' Amélie hieb mit dem Stock in die Luft. Nochmal und nochmal. Die Spitze des Stabes sprühte Feuer und Rauch, die Dämonen quietschten und kreischten vor Vergnügen. Dann schrie sie, tonlos.

"`Peace! Amélie, Peace!"' der Gott hob die Hand mit ausgestreckten Zeige- und Mittelfinger. Dann hauchte er einen Rauchring von seinem nie endenden Joint, der sich um Amélies Nase kringelte. Sie lachte. ,,Weisste, Mädchen! Es ist viel schwerer zu den Guten zu gehören. Und es macht weniger Spass und bringt kein Ansehen\dots Du entscheidest dich\dots Und wenn du zu den Guten gehörst, dann ist es dein Job, alles und alle zu lieben und mehr is` da nich` bei!'' Amélie merkte, wie sie sich beruhigte. -Seltsam, dachte sie, es ist mir vorher nicht aufgefallen, aber ich habe mich entschieden. Schon vor langer Zeit\dots Und ja, verdammt, ich bin bei den Guten! -,Verdammt!' Entfuhr es ihr, als sie mit erkennendem Blick den Gott anstarrte. Schu grinste wieder, sein unwiderstehliches Grinsen, mit seinen unwiderstehlich geschwungenen Lippen. Amélie seufzte:-,Mannmannmann!' Murmelte sie und machte sich wieder an die Arbeit, die die Guten zu machen pflegen.

 Sie griff sich einen von den Dämonen und schaute ihn an. Das war dem kleinen Kerl nicht recht. Er strampelte in Amélies Hand und gab ärgerliche Grunzen von sich. Er hatte einen riesigen Mund, der mit Schokolade verschmiert war. Er hielt sich unter Amélies strengen Blick wie ein kleines Kind, das nicht gesehen werden wollte, die Augen zu. Zwischendurch schielte er zwischen den dicken, mit Schokolade verschmierten Fingern durch und grinste frech. Amélie erinnerte sich plötzlich: Sie stand in der Küche und hatte die Hände hinter dem Rücken verborgen. Berta kam vom Garten durch die Küchentür herein und lächelte freundlich. "`Amélie? Was machst du denn in der Küche?"' "`Och, nichts."' hatte sie gesagt. "`Möchtest du eine Karotte? Ganz frisch!"' "`Ja, gern. Ich hab solchen Hunger,"' hatte Amélie geantwortet. Sie hatte die Karotte genommen, war in den Garten gelaufen und hatte sie den Kaninchen gegeben und die geklaute Schokolade in den Mund gestopft.

-Da kommst du also her! Amélie schaute streng. Der Dämon streckte seine Zunge raus und verschwand aus ihrer Hand und stürmte davon zu den anderen. "`Lass es gut sein!"' -'Was macht denn der Mann mit Tefnut?' rief Amélie  dann fasziniert. Der Mann war nach dem er die Frauen los geworden war, weiter gelaufen und hatte dann Tefnut entdeckt. "`Hallo, Miez! Ja komm, komm, pusspuss!"' Er kniete sich hin und streckte seine Hand nach der stattlich roten Katze aus. Die schnurrte und trabte auf die Hand zu und stiess ihren Kopf hinein. Sobald der Katzenkopf die Hand berührt hatte, zuckte der Mann als ob er einen feinen, elektrischen Schlag bekommen hatte. Er streichelte zwar der Katze weiter über den Rücken, aber sackte gleichzeitig zusammen. Seine Schultern begannen zu beben und dann schluchzte er laut. Er schlug die Hände vor das Gesicht.

Tefnut schmiegte sich nach Katzenart laut schnurrend an die Beine und den Rücken des hockenden Mannes. Der schrie auf und viel in Ohnmacht. -'Tefnut!' rief Amélie entsetzt. -'Was hast du gemacht?' Tefnut kam mit erhobenem Schwanz zu Amélie gelaufen. Amélie begann ebenfalls sie zu streicheln, ganz automatisch, wie es halt passiert, wenn eine Katze kommt. -'Schon vergessen Schätzchen?' schnurrte die Göttin, -'ich bin die Wahrheit!'

Amélie richtete sich auf, denn sie fühlte sich schwindelig. Dann bemerkte sie wie ihre Dämonen, die kleinen und die grossen sich alle hinter ihr vor Tefnut versteckten. Während die Katzengöttin sie umkreiste, umkreisten die Dämonen sie ebenfalls. Dadurch hatte Amélie den Eindruck ihre Seele und der Sternenkörper, in dem sich die Dämonen befanden, würde gleich zerreissen. Sie schnappte nach Luft. Tefnut hört nicht auf, um sie herum zu gehen. Amélie war froh, dass die Göttin sich etwas von ihr entfernt hatte und einen grösseren Kreis machte.

-'Tefnut, das ist super unangenehm. Lass' dass!' "`Ich mach nichts, Schätzchen! Ich bin nur so da!"' schnurrte die Göttin. "`Du hast beschlossen der Wahrheit ins Gesicht zu sehen und da muss man bestimmte Regeln halt befolgen."' -'Was soll ich denn machen?' Amélie merkte wie ihr schlecht wurde. "`Kleine! Ich habe es dir doch erklärt! Wenn du die kleinen Stinker akzeptierst ist alles okay!"' Schu stand wieder neben ihr. Er hatte seine Frau auf den Arm genommen und knuddelte sie. Amélie hatte noch nie eine solche Katzenschnurren gehört! Aber Schu nahm dann die Katze und warf sie Amélie zu.

Sie schrie und die Katze schrie. Instinktiv fing Amélie die Göttin auf. Und tatsächlich konnte sie die Göttin halten. Die Katze war wie eine richtige Katze weich und flauschig, sie schnurrte. Amélies Herz machte einen Glückshüpfer und die kleinen Dämonen schüttelte es durcheinander. Ein kleiner Schwups Liebe traf einen der Dämonen am Kopf. Dieser schaute erstaunt. Dann grinste er. Er streckte seinen dicken Stummelfinger aus und stupste damit die Katzennase. -Hahaha! lachte er, zugegeben etwas irre. Dann flatterte er fröhlich herum. Amélie musste so lachen, dass alle Dämonen von dem Freudelicht besprüht wurden und sich beherzt auf die Katze stürzten. Tefnut fauchte und die Dämonen bremsten sich im letzten Moment. Wie eine Welle aufgetürmter Monster blieben sie kurz in der Luft stehen, bevor sie dann alle übereinander purzelnd zu Boden plumpsten. Tefnut schnurrte wieder auf Amélies Arm.

Die Dämonen suchten sich wieder ihren Platz in Amélies Seele. "`Amélie\dots"', -er hat meinen Namen gesagt durch fuhr es sie. "`Willste mal n' Zug?"' Bevor Amélie hysterische anfangen konnte zu kreischen, kam Re dazwischen: "`Junge, ich glaub du solltest es mit dem Lob nicht übertreiben! Und mit deinem Charme auch nicht"'. Der Sonnengott legte Amélie die Hände auf die Schultern. "`Wenn du dich von Schus Charmeoffensive beruhigt hast, will ich dir jemanden vorstellen."' 

Er drehte Amélie herum und sie stand einem kleinen, dicken, aber nicht fetten Mann gegenüber. Er trug eine braune Cordsamthose und Filzpantoffeln an den Füssen. Oben hatte er ein helles Hemd mit zarten Karomuster an, das unter einer tannengrünen Wolljacke über den Bauch spannte. Amélie blickte in ein freundliches, bärtiges Gesicht. ,,Das ist Peter! Peter hat direkte Conection zum obersten Chef."' "`Ach, Re! Übertreib es nicht. Hör' nicht auf ihn, Mädchen. Ich bin auch nur ein armer Sünder! Und wer bist du?"' -'Amélie.' sagte sie und nahm die Hand die Peter ihr entgegenstreckte. Sie war etwas rau, er schien einen Handwerkerberuf zu haben. Der Handteller fühlte sich seltsam an. Als ob er eine Geschwulst darin hatte. Amélie warf verstohlen einen Blick auf die Hand und entdeckte die wulstigen Narben einer alten, kreisrunden Wunde.

"`Sehr beeindrucken für dein Alter wie du die Wahrheit anpackst."' Schmunzelte Peter. Sein grauer Bart wackelte als er leise kicherte. Er trug das graue Haar kurz und hatte am Hinterkopf eine kleine Glatze, sie sah aus wie die Tonsur eines Mönches. "`Ja, die Jugend ist zu beneiden!"' seufzte Re. "`Nicht doch, mein Herr, ihr seit doch jung geblieben"' murmelte Peter. "`Weisst du Peter, das macht die gute Seeluft. Du weisst es ja selbst, wer auf dem Wasser arbeitet, der bleibt automatisch jung."' -'Bist du auch Kapitän' fragte Amélie neugierig. "`Nö. Ich bin ein einfacher Fischer."' meinte Peter.

Die Turmuhr der Peterskirche läutete: "`Ich will nicht drängeln, aber wir haben ein grosses Programm. Peter willst uns nicht dein Schiff zeigen?"' mischte sich Thot ein. "`Na, klar! Kommt! Ich habe schon alles vorbereitet!"' Peter stieg die wenigen Stufen zum Eingang hoch und öffnete ihnen die grosse Holztür: "`Seit willkommen in meinem bescheidenen Reich!"'

\sterne

Der junge Mann rappelte sich benommen hoch. Er hatte keinen blassen Schimmer wieso er am Boden gelegen hatte. Als er jedoch seiner Frau die nächste Lüge erzählen wollte, begann es ihn schrecklich zu würgen. Es dauerte ungefähr eine Woche bis er es verstanden hatte und darauf verzichtete zu lügen. Auf diese Weise verlor er beide, Julia und Sophie. Er verlor auch seinen hervorragenden Ruf bei seinen Professoren.

Er machte eine Ausbildung zum Gärtner und lebte den Rest seines Lebens zufrieden. Später fand er eine Frau, die ihm wirklich wichtig war, heiratete sie und wurde Vater von zwei Kindern. Solange er lebte, hatte er immer eine rötliche Katze.

\sterne

Neugierig betrat die kleine Gruppe die Peterskirche. Die roten Sandsteinsäulen, die das Mittelschiff von den Seitenschiffen rechts und links trennte, ragten dunkel auf. Durch die runden Fenster an den Seiten, die wie Bullaugen eines Ozeandampfers aussahen, kam der schwacher Schein der Strassenbeleuchtung. Die hölzerne Kanzel, die  vorne an einer Säule klebte, schien lebendig. Hinter dem Lettner und seinen dunklen Bögen ragte das hohe Fenster des Chores auf. 

Peter schlurfte ihnen voraus und hielt auf den linken Bogen zu. Sie kamen in eine kleine Seitenkapelle. Re nahm die Sonnenbrille ab und sofort leuchteten seine Augen. -Er hat seine Taschenlampe immer dabei, dachte Amélie. Der Raum erschien in einem warmen Licht und ein zusätzlicher, heller Lichtstrahl beleuchtete die Stelle, auf die der Gott schaute. Dadurch konnten alle die bunten Wandgemälde betrachten. 

Dann setzte Re seine Brille jedoch wieder auf: ,,Du hast da recht viel Verkehr an Bord, Peter!'' bemerkte er. Auch Thot sah sich aufmerksam um. Tefnut strich dem Peter um die Beine und schnurrte. Auch sie hatte die Katzenohren gespitzt und witterte in die Luft.

,,Siehst du es, Amélie?'' fragte Thot. -,Nöh!' Amélie drehte sich, ihr gefielen die kräftigen roten und türkisen Farben. Im Dämmerlicht, das Res Augen durch die Sonnenbrille weiterhin erzeugten, schien der Raum unter Wasser zu sein. Amélie drehte sich übermütiger. Plötzlich wurde sie von einem Luftzug erfasst. Ein kleiner Wirbelwind in der stillen Kirche. Der Wirbel saugte sie an wie ein Staubsauger. Schwups, war sie im Boden verschwunden. ,,He, halt!'' rief Thot erschrocken. Re jedoch gluckste vor lachen und Tefnut fauchte und verwandelte sich schnell in ihre menschliche Gestalt. ,,Jesses!'' rief Peter und schnappte nach Luft. Während Thot Tefnut schnell seinen Mantel umwarf, sie hatte natürlich keine Kleidung mitgenommen, plumpste Peter auf den Stuhl. Er holte ein Taschentuch aus der Hosentasche und wischte sich die Stirn. Er hatte einen hochroten Kopf. ,,Heilige Mutter Gottes\dots!'' stammelte er.

Re war derweil durch den Türbogen in den Chor verschwunden. Sie hörten ihn lachen. Tefnut, Thot und Peter kamen genau in dem Moment in den Chorraum gestolpert, indem Amélie mit einem stimmlosen Schrei von der Decke fiel. Re fing sie auf. ,,Hui, was für ein Spass!'' rief er. ,,Gleich nochmal!'' er lief mit Amélies Lebensleib zurück in die Kapelle und liess sie über dem Wirbel fallen. Prompt verschwand sie wieder im Boden -,Haaaaaaalt!'. Re sprang mit seinen langen, dünnen Beinen zurück in den Kirchenchor und breitete in der Mitte die Arme wieder aus. Diesmal kam Amélie jedoch eleganter an der Decke zum Vorschein. Wie eine Elfe schwebte sie auf Re zu. Schliesslich liess sie sich fallen.

,,Eine tolle Sause, Peter! Das gibt es nicht in jedem Kirchenschiff!'' ,,Danke, mein Herr, zu freundlich.'' Obwohl er sich weiterhin die Stirne wischte und ab und zu einen unsicheren Blick in Tefnuts Richtung warf, hatte er sich von seinem Schreck erholt. ,,Gell, Peter! Diese groteske Idee mit dem Zölibat, die ist sehr neu, oder?'' fragte Re, dabei grinste er und knuffte Peter mit dem Ellenbogen. Peter kratzte sich am Kopf: ,,Du hast recht, Herr! Die Zeiten ändern sich und werden nicht in allem Besser!'' Er straffte die Schultern und richtete sich auf, was in seinem Fall bedeutete, er wurde nicht unbedingt grösser, aber sein runder Bauch rutschte höher.

Sie begaben sich zurück hinter den Lettner in das Hauptschiff. Schu schnarchte in einer der Bänke, die Füsse lässig auf der Vorderbank. Amélie kicherte, als Thot entrüstet seufzte. Tefnut verwandelte sich zurück in ihre Katzengestalt und sprang dem Luftgott auf den Bauch. Er erwachte erschrocken und schubste die Katze runter. Die miaute beleidigt und verschwand zwischen den Bänken. ,,Hey, Bab!'' rief er ihr nach.

Im Hauptschiff war es dunkler und als Amélie sich zu Peter umwandte, der unter dem Lettner stand, sah sie, dass auch Peter von verschiedenen Dämonen umschwirrt wurde. Es gab drei riesige unter ihnen. Amélie hatte selten hässlichere Gestalten gesehen. 

Die Dämonen hatten lederne Flügel, wie Fledermäuse. Sie waren blässliche, schlotternde Gestalten mit langen, spitzen Zähnen und dürren, spinnenartigen Finger, die unentwegt tasteten. Die grossen, gespenstisch-weissen Gesichter blickten ängstlich und sobald sie sich beobachtet fühlten, versuchten sie sich hinter Peter zu verstecken.

Aber es gab auch andere Dämonen um Peter, die zu Amélies Überraschung schön aussahen. -Sie sehen aus wie kleine Engel! wunderte sich Amélie. Sie waren ebenmässig von Gestalt und hell und viel durchsichtiger als die anderen, hässlichen Dämonen. 

Peter bemerkte Amélies Blick und sah zu Thot, der wiederum nickte. ,,Du hast sie entdeckt! Meine Dämonen!'' Amélie fühlte sich ertappt und wurde rot. ,,Du musst dich nicht schämen, schliesslich kannst du sie in deinem jetzigen Zustand, ohne Körper nicht übersehen, schliesslich sind sie in der gleichen Ebene wie du.'' -,Entschuldige. Aber es ist so, als würde man jemanden bei etwas erwischen, wofür er sich schämen müsste.' murmelte Amélie. ,,Du meinst, es ist so, als würde ich nackt sein.' Amélie schaute unglücklich drein -,Genau! Es ist mir peinlich sie zu sehen!' ,,Schau, Kleine, deswegen will niemand die Wahrheit so genau wissen!'' Tefnut war unbemerkt hinter Amélie getreten und hatte ihr in menschlicher Gestalt die Hand auf die Schulter gelegt. 

Diesmal liess sich Peter nicht aus der Ruhe bringen. Er lächelte Amélie und Tefnut zu. ,,Sieh genau hin, was siehst du?'' fragte er und breitete die Arme aus. Amélie betrachtete den Heiligen und die Wesen, die ihn umgaben. Dann sah sie es.

-,Es sind mehr schöne Wesen als hässliche! Von den hässlichen sehe ich nur drei und von den anderen gibt es viel mehr, sie sind aber viel kleiner.' ,,Gut, sehr gut!'' bemerkte Thot, der seinen Mantel zwischen den Bänken wiedergefunden hatte und zu ihnen kam. ,,Und was schliesst du daraus?'' -,Ich weiss nicht? Ich finde die kleinen, hellen Dämonen hübsch.' Amélie sah Thot zweifelnd an. ,,Und?'' -,Ich meine, ist das okay, sind sie gut?' ,,Eine gute Frage, Amélie!'' bemerkte Thot. ,,Da stellt sich als erstes die Frage: Was ist gut und was ist schlecht? Aber dafür langt diese Nacht nicht aus. Deshalb, erlaube ich mir, es abzukürzen: Ja, die hellen, schönen Dämonen sind gut. Und woher kommen sie, was meinst du?'' 

Amélie starrte zu Peter hinüber. Der unterhielt sich mit Tefnut. Und während er mit ihr sprach und er angestrengt die Stirne runzelte, erschien ein kleines Tröpfchen. Wie eine winzige Seifenblase. -,Gedanken?' fragte Amélie. ,,He, du bist gut!'' meinte Thot. ,,Genau! Sie beginnen als klitzekleine Gedanken!'' Er schwieg erwartungsvoll und auch Re kam dazu und schien gespannt auf ihre Antwort zu sein. Schu schlenderte näher.

,,Da hat es tatsächlich ein Einhorn auf der Miniorgel.'' murmelte er. Amélie blinzelte irritiert und dann, plopp, schwebte ein winziges Einhorn durchsichtig und in einen Regenbogen gehüllt vor ihren Augen. Aber dann verschwand es. -,Was?' Amélie blickte zu Thot. ,,Ein flüchtiger Gedanke verschwindet so rasch wie er entsteht. Die Dämonen die du siehst, sind häufig gedachte Gedanken. Stell dir vor der Gedanke wird jedesmal, wenn du ihn denkst, stärker und stärker.'' -,Wie stärker?' fragte Amélie skeptisch. -,Das hört sich an, als würden Gedanken lebendig sein!' ,,Hör` sich einer die kleine Braut an,\dots'' nuschelte Schu und paffte an seinem Joint eine grosse, einhornförmige Rauchwolke in die Kirche. -,Moment! Nein, neh! Das wollt ihr mir jetzt aber nicht unterjubeln?! Ich glaub kein Wort!'

Re legte seinen Arm um Amélie und führte sie langsam Richtung Ausgang. ,,Was  wäre denn daran so schlimm?'' fragte er unschuldig. Amélie schauderte und blieb ihm eine Antwort schuldig.

Am Ende der Bankreihen drehten sie sich um. Peter stand mitten im Hauptschiff. Tefnut, Schu und Thot kamen durch das Seitenschiff zu ihnen geschlendert, auch sie drehten sich um. ,,Tschau, Peter! Grüss deinen Meister!''

Peter hob die Hand und winkte. Und jetzt sah Amélie erst, dass eine weitere Gestalt hinter Peter stand. Sie war klar, durchscheinend und licht und so riesig, dass ihr Kopf an die Decke des Kirchenschiffes reichte. Auch die Gestalt winkte und zu Amélies erstaunen, verneigten sich alle vier ägyptischen Götter ehrfürchtig. Bevor sie reagieren konnte, drückte Tefnut ihren Kopf nach unten zu einer Zwangsverbeugung. Hätte Amélie in ihrem Körper gesteckt, wäre sie mit ihrem Kopf auf die letzte Kirchenbank geknallt\dots 

Als sie die Kirche verliessen, hörten sie Peters tiefes, freundliches, glucksendes Lachen.

\section*{2}
\addcontentsline{toc}{section}{2}

Die Horussöhne machten sich nach dem Abendessen auf den Weg. Alle vier, denn diesmal wollte Hapi nicht zurückbleiben. Sie hatten sich mit den anderen Göttern, die ebenfalls diese Nacht auf der Barke fahren wollten, im Treppenhaus des blauen Hauses vor der Kellertür versammelt. Damit kein Verdacht erregt wurde, waren alle Hausbewohner für die Barkenfahrt präpariert worden.

Auch Wibrandis hatte sich zur Verfügung gestellt. Horus hatte derweil ein Widdergehörn auf den Kopf geschnallt. Horus war nicht besonders gut verkleidet. Aber es ging um den ersten Eindruck. Wenn der zwiespältige Gott und Osiris Bruder Seth auftauchen sollte, um Unheil zu stiften, dann würde er es bei einem flüchtigen Blick nicht erkennen. Aber sie wussten auch, Seth war ein mächtiger Gott und sie konnten ihn nur täuschen, wenn er sich täuschen lassen wollte und andere Pläne hatte, als seinen Bruder Osiris zu verfolgen.

Die Horussöhne kletterten die steile Treppe eingezwängt zwischen den Ruderern abwärts. Durch die Fensterbögen des Treppenhauses konnten sie in das mit Fackeln erleuchteten, hohe Kellergewölbe schauen. Amset sah, wie Wibrandis staunend schauderte und dann ein Kreuzzeichen vor der Brust machte. Für den jungen Gott hinterliess ihre Hand in der Luft einen lichtes Feuerkreuz vor ihrer Gestalt, das sie wie ein Schild weiter vor sich trug.

Sie sogen tief die Abendluft ein, als sie den modrigen, feuchten Gang zwischen Gewölbe und Garten verliessen. Über ihnen kreiste Kebi, der es vorgezogen hatte, vom Küchenfenster aus zum nächtlichen Abenteuer zu starten. Er rief ihnen einen Falkengruss zu. Hapi winkte und Tef heulte eine Antwort. 

Amset schaut sich um und liess den Blick an der steilen Maurer der Terrasse am Rheinsprung hinauf wandern. Er sah Amélie. Wäre er ein Mensch gewesen, hätte er sie nur sehen können, wenn er Adept der geheimen Künste gewesen wäre. Als Gott, der in der geistigen Sphäre lebte, sah er Amélie viel klarer, als wenn sie in ihrem schweren, irdischen Körper steckte. Sie leuchtet viel heller und schöner. Amset lächelte und wollte ihr winken, als er eine kleine, schrecklich hässliche, verkümmerte Gestalt in der von Amélie bemerkte. Bevor er etwas denken konnte, hatte Hapi seine Hand genommen und zog an ihr. Es war schon eine Lücke in der Prozession entstanden.

Grübelnd folgte Amset seinen beiden Brüdern, Hapi und Tef auf die Barke. Sie gingen bis an das Heck. Von dort aus hielten sie Ausschau nach den 'Zeugen', die in der dritten Stunde, hier in Basel in der dritten Nacht dabei waren und die sie dringend unauffällig verhören mussten\footnote{Thot ist ein Meister der Zeit. Er war vom atlantischen, aber menschlichen Weisen in den Stand des ägyptischen Götterpantheons erhoben worden und hatte so als einziger ägyptischer Gott die nötige 'Zeiterfahrung'. Und deshalb war er in der Lage die Zeit des Amduat zu dehnen. Freilich nur, weil er von Berta, dem Geist der Rauhnächte, Unterstützung erhielt. Das Amduat teilte die Nacht in 12 Stunden. Weil 12 Stunden für das Abenteuer bei weitem nicht reichten, hatte Thot die Zeit von 12 Stunden auf 12 Nächte wie ein Gummiband lang gezogen. Wie jeder aus der Schule weiss, können lang gezogene Gummibänder viel Ärger anrichten. Das es dann auch 12 Tage hatte, konnte man als 'Bonus-Angebot' der Reiseleitung ansehen.}.

Duamutef fand seine Zeugen als erster. Es war eine Gruppe von vier Göttern, deren Aufgabe es war in der dritten Stunde den Toten behilflich zu sein. Diese durften sich mit ihren Seelen und Schatten vereinigen, wenn der Sonnengott vorüber zog. Die Toten jubelten und klagten, sobald die Barke vorbei gezogen war. Ihre zusätzliche Aufgabe, die der Götter und der Toten in diesem Zeitabschnitt war es auch 'mit lauter Stimme den Widersacher zu zermalmen'

Das hatte nun den Vorteil, dass die Brüder nicht lange suchen mussten, den in allen Ecken und auf allen Bänken in der Barke tönte es. Sie mussten die Gunst der Stunde nutzten, denn Seth ging das Geschrei furchtbar auf die Nerven und mit etwas Glück würde er diese Nacht nicht auftauchen\dots

-'Guten Abend!' bemerkte Tef. {\Large"`Möge unsere laute Stimme den Widersacher zu Boden zwingen!"'} Brüllten die vier glücklich und aus vollem Hals. -{\LARGE'Guten Abend!'} heulte Tef. {\Large"`Möge unser Geschrei\dots}, äh, was?"' Irritiert schauten sich die Götter an, dann bemerkten sie den Schakal. Sobald sie Duamutef erblickt hatten, kreuzten alle vier die Arme vor der Brust und verneigten sich wiegend vor und zurück, während sie ihre Lobpreisung begannen:{\Large"`Guten Abend Duamutef, Oh, Sohn des mächtigen Horus\dots"'} -'{\Large Psssst!}' Die vier schauten pikiert. "`Aber edler Duamutef, es ist doch unsere Aufgabe Euch zu ehren und den Widersacher mit lauter Stimm\dots"' -'\dots den Widersacher mit lauter Stimme zu zermalmen! Schon klar!' 

Es lief nicht gut. Zwei von den Göttern hatten abweisend die Arme verschränkt, der Dritte schob die Unterlippe vor und der Vierte hatte sich abgewendet. -' Tschuldigung! Ihr Jungs seit die lautesten Brüller, die wir haben!' "`Wirklich?"' fragte der Dritte. -'Ja! Und nun brauche ich Eure Hilfe! Ihr seit halt die Besten!' "`{\Large Danke, Oh, Duamutef, edler Sohn des Horus!'} -'Pssst! Jungs. Es ist ein Geheimnis!' Duamutef rollte mit den Augen, die hellsten Köpfe waren sie nicht\dots {\Large "`Geheimmm! Oh, edler Duamutef, wir sind}\dots {\scriptsize wir sind gerührt über Euer vertrauen!"'}

Duamutef winkte die vier in eine ruhige Ecke der Barke und begann sie auszufragen, ob sie in der Zeit des Pharao Haremhab etwas aussergewöhnliches erlebt hätten. "`Haremhab? Guter Mann! Ja, guter Mann! Hat damals versucht zu retten, was zu retten war!"' Meinte der Erste. "`Er hatte viel mehr Macht als der kleine Tut-Anch-Amun. War so zu sagen ein 'alter Hase', was die Regierung angeht. Der hat nach dem Atumkult und der Stadt Amarna wieder Ordnung geschaffen. Das konnte der junge Tut-Anch-Amun als Sohn Echnatons und halbes Kind natürlich nicht."' Duamutef spitzte die Ohren. Was der Zweite da über die Ordnung sagte, war interessant. Denn vielleicht wusste Maat mehr, als er bisher angenommen hatte\dots 

"`Er hat es nicht einfach gehabt. Haremhab musste die Priester unter Kontrolle bringen. Die waren ausser Rand und Band, weil sie ihre Kulte nicht durchführen durften. Die einen machten heimlich weiter und andere gingen unter die Magier."' Ergänzte der Dritte. Der Vierte fügte hinzu: "`Es ist nicht gut, wenn die mächtigen Priester nicht im öffentlichen Dienst angestellt sind. Wenn sie sich selbst überlassen werden, dann neigen sie dazu ihre Macht zu missbrauchen und rutschen schnell in die schwarze Magie."' "`Und der Haremhab war stark und klug und hat die Priester soweit gebracht ihre Ämter wieder zu bekleiden."' "`Ja,"' ergänzte der Zweite den Ersten: "`Leider war einiges verloren gegangen von dem wichtigen Wissen. Die Ausbildungskette der jungen Adepten war unterbrochen worden und die Priester die das thotsche Wissen direkt aus dem alten Reich überliefern konnten, starben und nahmen ihr Wissen mit!"'

 "`Ach, dieser fürchterliche Banause Echnaton! Nachdem er es vermasselt hatte, gab es keine anständige Barkenfahrt mehr! Es ist ein einziger Murks aus den verschiedensten Ritualen und Litaneien entstanden."' Klagte der Dritte. "`Du sagst es, Kollege, wir könnten praktisch jede Nacht ein anderen Reiseführer benutzten, um durch die Duat zu fahren und würden jedes mal woanders durchkommen. Dabei ist doch das gute alte Amduat der beste und einzig wahre Reisebegleiter!"' seufzte Nummer Vier.  
 
"`Und natürlich ein Handtuch\dots"' fügte der Erste abwesend hinzu, die anderen drei Götter und der Schakal schauten ihn an. "`Äh, das ist mir glatt so rausgerutscht! Muss an den modernen Zeiten liegen, die wir durchkreuzen\dots"' der Erste wischte sich den Schweiss von der Stirn. Der Zweite schüttelte den Kopf und  murmelte: "`Handtuch? Hand-Tuch? Was soll das sein? Ein Tuch für eine Hand? Ist ja eklig! Was ist mit dem Rest von der Leiche\dots?"'\footnote{Im Neuen Reich setzte man vermehrt auf Quantität, als auf Qualität. Es entstanden zahlreiche neue Totenbücher, die verschiedene Stellen des Amduat mehr ausschmückten oder ausliessen, je nach Geschmack des priesterlichen Schreibers oder den Vorlieben seines Auftraggebers dem Pharao. Später konnten auch hohe Staatsangestellte Totenbücher bestellen, bis irgendwann Kreti und Pleti eines hatten. Was dem Bürger heute sein neuer Kleinwagen, war schliesslich dem alten Ägypter sein Grab mit Grabmalerei und ein exklusives Totenbuch, das ihn und seine Familie sicher nach Sechet-iaru brachte. \dots mit einer grosszügigen Interpretation könnte man sagen, waren sie damals auch mit 'Handtuch' unterwegs. Besser gesagt mit mehreren\dots}

-'Danke, Jungs! Ihr habt mir wirklich sehr geholfen.' Beendete Duamutef ihr Gespräch. Er wollte die Götter nicht zu lange von ihren nächtlichen Aufgaben abhalten, damit sie nicht vermisst würden.

-'Ich sage Hathor, sie soll euch ein extra Bier bringen!' "`{\Large Danke, oh Duamutef, Sohn des edlen\dots} -'Psssst!!!- Duamutef hatte sich vor Schreck unter die Bank geflüchtet. Ärgerlich kam er wieder hervor. Nein, diese Heimlichtuerei war nichts für ihn. Vor allem, wenn die Zeugen solche Deppen waren! "`{\scriptsize Entschuldigung, oh, Duamutef, Sohn des edlen Horus, wir danken Euch!"'} Die vier verbeugten sich ruckartig vor dem Schakal, der es nicht verhindern konnte reflexartig zusammen zu zucken. Die Götter verschwanden in die Mitte der Barke: "`{\LARGE Du, Widersacher, wirst von unserer lauten Stimme zu Boden gezwungen! \dots"'}

Duamutef schob sich mit dem Hinterteil voran unter der Bank raus. Eine Schakalgestalt war toll, allerdings konnte auch ein Gott von den tierischen Instinkten, die damit verbunden waren, überrascht werden. Er schüttelte sich kurz. -Ich muss Hathor sagen, dass die Ruderer schlampig putzten. Unter den Bänken könnten Archäologen Scherben von Bierkrügen aus den Jahrtausenden vor Christi finden. Grummelte er, während er nach seinen Brüdern Ausschau hielt.

\section*{3}
\addcontentsline{toc}{section}{3}

Hans und Geb verliessen durch das hohe, eiserne Tor den Garten des blauen Hauses. Sie trugen keine Kleider. Ihre grün-braune Haut glänzte im Licht der Strassenlaternen. Geb trug auf seinem Rücken den Imiut, den er von Anubis ausgeliehen hatte. Im Imiut befand sich Osiris in einem Ei. Damit sein Sohn in das Ei gelangen konnte, hatte Geb es am Nachmittag gelegt. Osiris reiste wie Amélie mit seiner Seele. Diese würde im richtigen Augenblick aus dem Ei schlüpfen können und mit den neu gewonnenen Lebenskräften zum heiligen Benuvogel werden.\footnote{Ägyptische Götter sind sehr kreativ und pragmatisch, was die Fortpflanzung angeht. Wie es dazu kam, dass Geb das Ei für den Benu legen kann, der die Ba-Seele seines Sohnes Osiris darstellt und gleichzeitig seinen Sohn mit seiner Gattin, der Himmelsgöttin Nut zeugte, wurde nie ganz geklärt. Auch Götter wollen im Bett Privatsphäre haben!

Der Benu ist eine Phönixart, weshalb nur ein Kurzaufenthalt in dieser praktischen Vogelgestalt möglich ist. Die Seele hat im alten Ägypten verschiedene Seelenteile, ähnlich wie wir heute Körperteile unterscheiden. Die Ba-Seele ist der Teil, der die unverwechselbare Persönlichkeit trägt, die Empfindungsseele.}

Geb pfiff zufrieden leise vor sich hin. Er war immer zufrieden und leise, solange es der 'Schnatterer' auch war\dots wer kann schon entspannt sein, wenn er eine kreischende Nilgans auf dem Kopf trägt?
 
Hans stapfte vergnügt nebenher, den schwarzen Hahn auf der Schulter. Der Erdgott war recht nach seinem Geschmack. Sie sahen nicht nur äusserlich aus wie Zwillinge, sondern hatten beide ihr Gemüt am gleichen Fleck. Hans kaute gemütlich und still (was wiederum Geb sehr gefiel) an seiner Pfeife. Sie brauchten zur Zeit nicht viele Worte, während sie durch den Rheinsprung zur Schifflände schlenderten. Hinter ihnen her tabbelte Hansens Dackel, der es sich nicht nehmen liess ausgiebig in den Hunde-News zu lesen und selbst zu kommentieren.\footnote{Er schnüffelte an den feuchten Ecken und pinkelte an die prominentesten Stellen.}
 
\sterne

Geb  und Hans kamen unten an der Schifflände an. Und blickten versonnen auf die Mittlere Brugg, wie die Basler sie nannten. Sie war eine der ältesten Rheinbrücken überhaupt. Eine lange Zeit hatte die eine Hälfte aus Holz bestanden, um die Brücke bei Gefahr schnell abbrennen zu können und den Feinden den Weg abzuschneiden. Bei der alten, bis in die heutigen Tage gepflegte Feindschaft von Gross- und Kleinbaslern, konnte der Feind gleich 'dääne dr Brugg' zu finden sein\dots

"`Eh, du Löli! Prolet!"' quäckte eine Stimme hinter ihnen über die Strasse. "`Hesch du dir e Bruedr us dr Wäbstube klaut? HäHä! Iiiiik!Klack!"' "`Du arrogants Grossmaul, du muesch grad dumm tue, was?"' zischte Hans, der plötzlich alles andere als friedlich wirkte. Der Hahn reckte seinen Hals und schlug mit den Flügeln. ,,Gooock?'' ,,Psssst!'' machte Geb und schaute sich suchend um. Es war niemand zu sehen. "`Prolet, Prolet! Iiiik!sKlack! HAHAHA!!!"' "`Ich stopf' dir s Maul, du Schnuuri! Du Deigaff!"' rief Hans und schüttelte die mächtige Faust gegen das Haus an der Ecke: ,,GooOOCK!!'' ,,Haaans? Könntest du etwas leiser sein, der Schn\dots"' raunte Geb. \begin{Large}
,,KiiieekerIiieeekIhh!''\end{Large} Zu spät! Ein weiteres Iiiiik! gefolgt von einem Klack! ertönte und der Schnatterer legte los. Er war aus einem wunderschönen Traum vom warmen Nilufer ins kalte Basel gerissen worden und sehr ungehalten. Geb hielt sich verzweifelt die Ohren zu. Hans war erschrocken zusammengezuckt und presste sich ebenso die Hände auf die Ohren. Der Hahn klappte beleidigt den Schnabel zu, gegen den Schnatterer hatte er keine Chance.  "`Hahaha! So öppis biireweichs hani no niäne gse: Zwei blutti Wäbstübler, wo sich d Ohre heebe! Und ihri Vögeli ufm Kopf ummenand drääge!!!! Hahahaha!!!!"' kreischte es vom Haus her.\footnote{Alemannisch Dictionary: blutt, adj. = nackt

Wäbstübler = Jemand, der in der Spinnerei schafft = Ein psychisch erkrankter Mensch = Kurz: Ein Verrückter

Weiterer WICHTIGER Hinweis: Mehr als drei Ausrufezeichen sind, wie ja alle wissen, ein Zeichen von Wahnsinn!!\dots !} Der arme Dackel vom Lärm umzingelt, raste, jaulend und fiepend um Hans und Geb.

Es schnatterte, gackerte, kicherte, heulte und kläffte bis aus den umliegenden Restaurants vorsichtig die eine oder andere vornehme gekleidete Herrengestalt aus der Eingangstür schielte. Der Pförtner des besten Hauses am Platz, des 'Troi Roi' kam gelaufen. Er fuchtelte mit dem behandschuhten Zeigefinger: "`He, Sie! Ich habe die Polizei alarmiert! Was geht hier vor? Auu! Auua! Das Hundevieh hat mich gebissen!"' Der Dackel wild vor Ohrenpein hatte seinen ohnmächtigen Zorn an der ersten Wade ausgelassen, die erreichbar und nicht göttlich war. "`Komm, Geb, mr sööte do vrschwinde\dots!'' Hans fasste Geb am Arm und zog ihn mit sich zum Marktplatz. Ihre Füsse klatschten über die nasse Strasse. Von weitem hörten sie das Martinshorn des Polizeiwagens rasch näher kommen. Sie versteckten sich  hinter dem Weihnachtsbaum, als der Wagen an ihnen mit Blaulicht vorbei zur Schifflände raste. Augenblicklich verstummten die Tiere. Nur die Tatzen des Dackels, die eilig über das Kopfsteinpflaster kratzten, waren zu hören.

Hans klopfte Geb auf die Schulter und nahm den Dackel auf den Arm: "`Duet mr ächt leid, Kollege! Abr die Missgstalt got mr eifach uf de Geischt!"' "`Jo, aber Hans, wer war'n das?"' fragte Geb. "`Lass es z erschtmool guet si! Ich zeigs dir denn scho, abr vorher muess ich mir öppis iifalle lo, wie ich's dem Löli ka heizaale!"' Geb war nicht wohl, er war sich nicht sicher, wer wem etwas heimzahlen würde. Aber in einem hatte Hans recht, sie mussten weiter. Sie mussten die Grenzen des kosmischen Eies prüfen, ob es den Schutz bieten konnte, den sie für Osiris brauchten.

-,Hans?' -,Jo?' -,Hast du eine Ahnung wieso uns die Menschen auf der Strasse sehen konnten?' -,Wie meinsch?' -,Wir sind Götter\dots! Sie sollten uns nur sehen, wenn wir es wollen! Und wir wollen nicht, weil wir keine Hosen tragen! Wir wollten inkognitum reisen!' -,Jetzt, wo du s seisch! Kei Aanig! Viilicht liegts an däne Raunächt. Leidr kann ich d Berta nit froge, di wüssts, denk ich!'

\sterne

Oberwachtmeister Moser stöhnte. Das Protokoll der letzten Nacht war wieder an ihm hängen geblieben. Dabei war er der ältesten Mitarbeiter (noch 134 Tage bis zur Pensionierung) und bevorzugte nach wie vor eine Schreibmaschine. Aus diesem Grund wurde er von seinen Kollegen 'Dino' genannt. 

Die junge Meyer hatte verzückt 'voll retro!' gerufen, als sie die erste Schreibmaschine ihres jungen Lebens sah. Dann hatte sie laut gelacht, nachdem sie hörte, dass es keine Löschtaste gab. Bei dem Gedanken an die junge Frau Meyer wurde ihm gleich versöhnlicher zu Mute. 

Er verschränkte die Finger, drehte die Handflächen nach Aussen und streckte sich. Die Finger knackten wie eine Maschinengewehrsalve. Er hielt sie vor sich in die Luft und liess sie auf und ab zappeln. 

Er rückte den Stuhl zurecht und die letzte Nacht zog an an seinem inneren Auge vorbei. Wer hätte das gedacht? Er hatte geglaubt, er hätte alles gesehen, was es in seinem Beruf als Streifenpolizist zu sehen gäbe. Aber, er hatte sich getäuscht. Seine Finger sanken auf die Tasten:

{\tt Protokoll vom 26.12.20XX
 
21:05: Meldung in der Leitstelle: Der Portier des Hotels 'Troi Roi', Herr Häsler, meldet, dass sich zwei auffällige Personen vor dem Restaurant 'Lälle König' aufhalten. Sie seien nackt und würden einen grossen Vogel, vermutlich eine Gans, auf dem Kopf mit sich führen, die laut schrie. Sowie einen schwarzen Hahn und einen Hund, der sich wie tollwütig benahm und dem Herr Häsler, als er nach dem rechten hatte sehen wollen, in die Wade gezwickt hatte. Vielleicht solle man den Tierschutz auch informieren, so der Herr Häsler.

21:15: Streifenwagen mit den Polizisten Moser und Meyer trifft an der Schifflände ein. Von den beiden Ruhestörern ist nichts zu sehen. Es befinden sich viele Bürger auf der Strasse. Der Portier Häsler (siehe oben), Gäste aus dem 'Club de Bale, Passanten. Da viele Zeugen die Aussage von Herrn Häsler bestätigen (Liste der Zeugen im Anhang) beschliessen die Polizisten Moser und Meyer die Umgebung genauer zu prüfen.}

Wachtmeister Moser kratze sich nachdenklich am Kopf und fuhr mit seinem Protokoll fort:

{\tt 21:30 Die Polizisten Moser und Meyer fahren langsam den Marktplatz Richtung Barfüsserplatze hinauf. Sie können keine verdächtigen Personen sehen. Die Beamten beschliessen per Pedes die Gerbergasse zu inspizieren.

In der Nähe des Gerbergässleins auf dem kleinen Platz treffen die Beamten auf die beiden Verdächtigen. Sie haben sich direkt vor dem Gerberbrunnen eingefunden. Sie liegen, verdeckt vom Mauervorsprung und dem Geländer am Boden und reden laut und aufgeregt miteinander. Die beiden Männer sind tatsächlich unbekleidet. Die Gans, oder der Ganter, watschelt auf dem Platz herum und betreibt Ruhestörung. In den Wohnung über dem Bastelladen ist ein Fenster aufgegangen und ein Bürger beschwert sich über den Lärm. Er beruhigt sich, als er sieht, dass die Beamten bereits vor Ort sind. 

Sobald sich die Beamten jedoch den Verdächtigen zuwenden, greift der Gänsevogel an. Er schnappt und schlägt mit den Flügeln. Auch ein Dackel ist anwesend, der die Beamten mit fletschenden Zähnen angeht, sobald sie sich den Verdächtigen nähern wollen. Ferner ist auch ein schwarzer Hahn vor Ort, der für einen Hühnervogel, nach Meinung des Beamten Moser, sehr gross ist. Der Hahn verletzt die Beamtin Meyer am Arm, als diese versucht an ihm vorbei die Männer über das Geländer anzusprechen.

Den Beamten bleibt nichts anderes übrig als die beiden Männer aus sicherer Distanz anzurufen. Diese reagieren in keiner Weise auf die Ansprache der Beamten.In der Zwischenzeit haben sich weitere Fenster geöffnet. Die Anwohner beschweren sich, die Polizei solle endlich eingreifen und zwar ein bisschen plötzlich und gefälligst leise!}

Polizist Moser schnaubte empört! Die Anwohner hatten gut Lachen gehabt! Die konnten von oben zusehen wie die tapferen Polizisten von den Bestien gemetzelt wurden!

{\tt 21:40: Die Beamten Meyer und Moser haben zwei weitere Kollegen informiert und gerufen, da sie sich den beiden Männern nicht nähern können, ohne von der Gans, dem Hahn und dem Dackel angegriffen zu werden.

Der Beamte Moser hat tiefe Kratzer im Gesicht erhalten, die von den Sporen des Hahnes stammen, während die Kollegin Meyer mehrfach von dem Dackel in die Wade gebissen wurde. Die Gans greift abwechselnd beide Polizisten an.

21:46: Ein Streifenwagen mit Martinshorn trifft mit den Beamten Schmied und Hofer ein. Die Beamten haben, wie von Moser angefordert Fangnetze für die Tiere mitgebracht.

Die Beamten Meyer, Hofer und Moser versuchen die Tiere mit den Netzen einzufangen. Die Gans ist zu gross für das Netz, das für streunende Katzen gebraucht wird. Die Gans kann  sich daraus befreien, sobald sie mit den Flügeln schlägt.
Während die Gans wild mit den Flügeln schlägt, um sich aus dem Netz zu befreien, trifft sie den Beamten Hofer am Schienbein, worauf dieser zu Boden geht und von den Beamten Meyer und Moser unter weiteren heftigen Schlägen der Gans aus der Gefahrenzone befreit werden muss. Dabei hat sich der Dackel bei Meyer am Hosenbein festgebissen, wodurch die Beamtin den Kollegen Hofer loslässt und stürzt. Daraufhin werden beide am Boden liegenden Beamte von Hund und Gans massiv bedrängt.

Währenddessen hat sich der Hahn bei Kollege Moser auf dem Kopf festgesetzt und diesen mit heftigen Flügelschlägen bei der Rettung der Kollegen behindert. Erst nach erheblicher Anstrengung gelingt es dem Beamten Moser unter Verlust der Brille und mit heftigem Nasenbluten, das durch Flügelschlag verursacht wurde, den Hahn von seinem Kopf zu vertreiben.

Dann gelingt es dem Beamten Moser den Beamten Hofer, der durch den Flügelschlag mit einem Bein nicht mehr auftreten kann, zum Streifenwagen zu bringen und erste Hilfe zu leisten. Die Beamtin Meyer kann den Dackel mit ihrem Stock abwehren und sich selbstständig in den Streifenwagen flüchten.}

\sterne

Moser war nicht sicher, ob das alles in dieser Ausführlichkeit in das Protokoll gehörte. Aber seine Frau hat ihm den ganzen Kopf dick mit Salbe eingeschmiert  und mit einer guten Lage Binden umwickelt. Kollegin Meyer, deren Verband an den Waden man nicht sehen konnte, hatte ihn heute morgen ausgelacht, als sie mit Hofer in sein Bureau geschaut hatte. ,Von hinten siehste aus wie ein mumifiziertes Ei!' hatte sie gesagt. Er hatte gezwungenermassen mitgelacht, er wollte nicht vor dem lauthals lachenden Hofer wie eine beleidigte Leberwurst dastehen. Und der Hofer? Er war wohl ihr Held! Obwohl er an allem Schuld war! Schliesslich hatte die Gans ihn umgeworfen, und sie mussten ihm zur Hilfe kommen\dots Nein, es war nicht gerecht! Aber der Hofer, der war halt jung\dots, muskulös, durchtrainiert\dots und ein Depp! Bei dem Gedanken an den Kollegen, der nun eine Schiene trug und an Krücken laufen musste, konnte der Moser wieder etwas schmunzeln. (Vermutlich ist meine Ersatzbrille mit dem billigen, dicken Gestell auch nicht unbedingt vorteilhaft, dachte er.)

Ob er es im Bericht erwähnen musste, dass der Dackel der Kollegin Meyer den Stock entwendet hatte? Der blöde Köter war fröhlich kläffend mit dem Stock davon gehopst -und keiner von den Polizisten hatte sich getraut das staatliche Eigentum zurück zu holen.


{\tt 22:15: Dank dem wagemutigen und intensiven Einsatz der zwei Polizisten Meyer und Moser,}

Polizistin Meyer jagt hinter dem Dackel her, der sich ein Spass daraus macht mit ihrem ,Stöckchen' wegzurennen und Moser schlägt wild mit dem Netz nach Gans und Hahn, um sich die beiden vom Leib zu halten, wobei er von den beiden Vogelviechern im Kreis gejagt wird zur Freude der Passanten\dots

{\tt kann sich der Beamte Schmied endlich den beiden Männern nähern. Es wird höchste Zeit, den es hat sich inzwischen eine wachsende Menschenmenge aus den Anwohnern und Passanten gebildet. Diese beginnen mit ihren Smartphones Fotos zu schiessen und Filmchen zu drehen. 

Die beiden Herren reagieren unfreundlich auf den Beamten Schmied, der sie aufgefordert hat ihre Papiere zu zeigen. Während der eine der beiden den toten Hundebalg vom Rücken nimmt, zeigt er keinerlei Reaktion. Der andere fragt auf alemannisch, ob der Polizist Schmied ihm vielleicht sagen könne, wo er, zum Henker, ohne Taschen Papiere unterbringen soll. Ferner beleidigt er den Beamten mit den Worten: ,Ob er zu dusselig ist den Wappenherold der Ehrengesellschaft zur Hären zu erkennen? Typisch Grossbasler!'

Der Beamte Schmied ruft daraufhin über Funk zwei weitere Streifen und eine Sanität in die Gerbergasse. Da sich die Menschenmenge weiter vergrössert, brauchte es Beamten, die diese auf Abstand halten. Durch die grosse Anzahl an Zivilpersonen, die sich im Kreis um den Platz angesammelt haben und die Beamten bei der Bändigung der wilden Tiere behindern, ist der Beamte Schmied von den übrigen Kollegen isoliert worden. Er ist mit den beiden Verdächtigen in der Mauernische vom Gerberbrunnen eingeschlossen.}

{\tt 22:50: Die zwei weiteren Streifen treffen ein. Die Polizeiwagen versperren die Gerbergasse, um die Verdächtigen an einer Flucht zu hindern und weitere Personen auf Abstand zu halten.} 

Dummerweise können alle Passanten und Polizisten, die sich schon auf dem Platz befinden auch nicht mehr weg, dachte Moser. Was hatten sich die Kollegen bloss gedacht?! Der arme Schmiedi war komplett eingekeilt zwischen der Mauer und den beiden riesenhaften, nackten Männern!\dots 

{\tt Der eine Nackte nimmt den Tierbalg von seinem Rücken, in diesem bewegt sich etwas. Der zweite nackte Mann, der sich als Wappenherold bezeichnet hat, versucht nun den Beamten Schmied in ein Gespräch zu verwickeln und vom seinem Partner fern zuhalten. Er hebt die Arme und ruft, es sei eine christliche Mission im Rahmen der Weihnachtsfeierlichkeit und sie dürften nicht unterbrochen werden. Der Mann rudert und wedelt mit den Armen und Händen und versucht Platz zu machen. Er schnauzt den Beamten Schmied an, er solle dafür zu sorgen, dass nicht so viele Passanten herumstünden. Als der Beamte Schmied seine  Pistole hebt, entschuldigt der Mann sich für die Umtriebe, das wäre nicht geplant gewesen, normalerweise würden sie beide, der andere Herr Geb und er unsichtbar operieren und nicht soviel Volk anlocken. Andererseits fände er es schon toll, dass die Basler offensichtlich ein Gespür für die Rauhnächte hätten.

Der Mann wollte sich trotz mehrmaliger Aufforderung des Polizisten Schmied weder abführen lassen, noch den Ort verlassen. Polizist Schmied forderte den Mann weiterhin auf zu gehen. Da das nicht fruchtete wollte er ihn in Gewahrsam nehmen und die beiden auf die Wache bringen. Sobald er jedoch den nackten Mann berühren wollte, flog plötzlich und unerwartet ein riesiger Reiher aus dem Tierbalg des anderen Mannes.}

Nun musste Moser eine Pause machen, denn ab da war es kompliziert geworden. Als der riesige Vogel in dem engen Mauervorsprung des Gerberbrunnens auftauchte, geriet der Kollege Schmied, der sehr unter Strom stand, völlig aus der Fassung. Er richtete seine Pistole auf den Vogel und die beiden Männer abwechselnd und brüllte: ,Ergeben Sie sich! das ist Ihre letzte Chance!' Leider hatten die Zuschauer, die auf dem Platz waren und nicht weggehen konnten sich in der Zwischenzeit dem neuen Geschehen bei dem Gerberbrunnen zugewendet. Als Schmiedi die Pistole gezogen hatte, gab es von allen Seiten ein Blitzlichtgewitter. Schmied, völlig eingeklemmt zwischen dem plötzlich aufgetauchten riesigen Vogel, den nackten Verrückten und dem Pöbel,  brüllte dann, es sollten sich alle verpissen!\dots -Leider waren unter den Gaffern einige, die filmten. 

Dann war ein Schuss gefallen. Gott sei Dank, hatte Schmied jedoch niemanden getroffen. Daraufhin hatte der nackte, der mit Schmied gesprochen hatte schrill gepfiffen und der Dackel war zwischen den vielen Füssen zu ihm gesaust. Die Der grosse Vogel flog auf\dots

-Das glaubt uns doch kein Schwein, dachte Moser. Er machte sich betrübt auf den Weg zurück in sein Bureau. Ihm fielen die vielen Passanten ein, deren Filmchen sicher die Eine-Million-Klick-Marke geknackt hatten. Er kratzte sich am Kopf. Aber es war unbefriedigend, der Verband war im Weg.

Er setzte sich auf seinen Drehstuhl und streckte die Beine aus und verschränkte die Hände über dem Bauch: 

Der grosse Vogel stolzierte auf seinen langen Beinen aus der Nische des Gerberbrunnens. Er war wunderschön anzusehen. Er hatte einen langen langen Schnabel, der orange leuchtete. Seine Federn waren in allen Rot, Orange- und Gelbtönungen gefärbt. Seine Beine waren schwarz. Er hatte grosse Schwingen, die er erst ausbreitete als er mitten in dem Kreis von Polizisten und Passanten stand, die alle laut staunten und Platz machten. Obwohl er sehr vorsichtige war, schlug er Moser seine Schwinge kräftige an die Nase, die prompt wieder zu bluten begann. Die Federn de Flügels waren heiss gewesen, wie ein Feuerhauch. Zuhause hatte er eine Brandblase auf der Nase entdeckt\dots

Bevor sich jemand rühren konnte, erhob sich der gewaltige Vogel in die Luft, die beiden Männer ergriffen seine Beine und er flog mit ihnen und dem kläffenden Dackel über aller Köpfe in Richtung Lohnhof davon. Keine 30 Sekunden später standen und lagen die Polizisten alleine auf dem Platz, die Passanten waren dem Vogel nach gestürmt. 

Das leise Schluchzen von Schmiedi in der folgenden Stille\dots, das würde Moser lange verfolgen\dots

\section*{4}
\addcontentsline{toc}{section}{4}

Amélie, schwebte hinter den drei Göttern und der Katze her. Sie hatte den Eindruck diesen Peter sollte sie kennen. Aber ihr wollte nicht einfallen woher. Peter, Fischer, Verleugnen\dots Während sie grübelte und schwebten ihre Gedanken wie kleine Lichtpunkte und geomerische Laternen um sie. Das Schiff des Fischers hatte zu beginn des Gedankens die Form eines Piratenschiffes, änderte sich in verschiedene Schiffsformen bis es schliesslich wie ein Ruderboot aussah aus dem ein Netz ins Wasser ragte. Die Buchstaben von Peters Namen tanzten um sie. Sie wechselten die Reihenfolge und es kamen weitere Buchstaben dazu: Peter, Petra, Peer, Peterus, Petrus,\dots Petrus? Petrus! Oh, mein Gott, dachte sie: Doch nicht DER Petrus?

 Die anderen drei waren die Strasse weitergegangen und verschwanden hinter einer Biegung. Schnell, dachte Amélie und bemerkte zu ihrem Schrecken, dass sie nicht allein war!

Sie war an einem hohe Eisenzaun angekommen. Er gehörte zu einem alten, grossen roten Haus, das mit seinen vielen Verzierungen prächtig aussah. Der Innenhof des Hauses lag tiefer als die Strasse und der Zaun trennte den tiefen Hof von der Strasse zu der die oberen Stockwerke auf gleicher Höhe waren. Im Hof unten bewegte sich etwas. Ein tiefes Knurren drang an Amélies Ohr. Die Schatten zweier riesiger, schwarzer Hunde huschten hin und her. Unter dem Eingang trat eine Gestalt vor: Der Mann mit den roten Haaren, den Amélie in der ersten Nacht gesehen hatte! Er winkte ihr zu und hob zum Gruss\dots seinen Kopf an den Haaren hoch!

Amélie stürmte die Strasse hinter den Göttern her. Vor Schreck purzelte sie über ihre nicht vorhandenen Beine und fiel. Und landete im gleichen Augenblick bei der Kirche vor der die Götter auf sie warteten.

,,Ah, du hast endlich herausgefunden, wie du dich in diesem Zustand beamen kannst? Faszinierend!'' lobte Thot, wobei er unverschämt grinsen musste. -,Da war dieser Mann\dots' schnaufte Amélie. ,AH, hast du den Gastgeber dieser Kirche schon vor seinem Haus getroffen?' Re schob Amélie durch die Tür, die natürlich nicht geöffnet werden musste\dots In dem Moment fiel Amélie auch ein, dass sie nicht ausser Atmen geraten konnte\dots Sie hörte auf zu schnaufen, woraufhin Thot wieder kichern musste. -,Woher weisst du, was beamen ist?' raunzte Amélie. Thot hob seine Hand wie zum Gruss und spreizte Zeige- und Mittelfinger auf die eine und Ring- und Kleinen Finger auf die andere Seite. Er schaute Amélie an: ,,Faszinierend! Wo das alte Wissen seine Spuren hinterlässt, findest du nicht auch?''

Sie betraten die Kirche. Es war still. Der grosse, hohe Raum mit den roten Sandsteinsäulen schien zu schlafen. Amélie atmete auf. Sie liess sich auf einem der Stühle nieder, die an Stelle von Kirchenbänken den Leihenteil der Kirche ausfüllten. Es war reine Gewohnheit, sitzten ohne Körper ist viel anstrengender als einfach in der Luft rumhängen\dots

Tefnut fauchte plötzlich und kletterte an der Holzsäule der Kanzel empor und sprang von dort auf den Baldachin aus Holz über der Kanzel. Ihr fauchen ging in ein tiefes kehliges Knurren über und mit phosphoreszierenden, grün-rot leuchtenden Augen starrte sie gegenüber zur Tür. Amélie war vor Schreck aufgesprungen und versteckte sich hinter Re, der nach einer kleinen Schlenderei im Mittelgang zwischen den Stühlen stand. Thot und Schu, die unter dem Lettern standen, schauten neugierig auf einen Nebel, der langsam unter der Eingangstüre in den Kirchenraum wabberte.

Amélie erstarrte. -,In  einem Horrorfilm würde die Hauptdarstellerin jetzt infernalisch schreien!' dachte sie bebend -,In der Tat, das wäre der Szene angemessen', antwortete Re stumm. Offensichtlich hielt selbst der Sonnengott es für angebracht seinen Kommentar nur zu denken.

Der Nebel verdichtete sich zu einer Wolke, keiner der Reisenden muckste, Tefnut sass wie eine Statue auf dem Holzbaldachin. Der Nebel verdichtete sich und wurde zu der Gestalt eines schwarzen Hundes. Eines grossen, knurrenden, Zähne fletschenden Hundes und das einzige, was noch grässlicher war, waren zwei weitere Nebelfäden, die auf der Türschwelle auftauchten.

Mit Grauen beobachtete Amélie wie sich der zweite schwarze Hund materialisierte und schliesslich der Mann mit dem gewaltigen roten Bart und dem mittelalterlichen Mantel wenige Meter vor ihr aus dem Nebel wuchs.

Thot bewegte sich als erster. Er ging mit ausgestreckter Hand auf den Mann zu, der leicht durchscheinend blieb. ,,Sei gegrüsst, Johann! Oder soll ich David sagen?'' -,Sei gegrüsst Herr! Es spielt keine Rolle mehr.' Re kam interessiert auf den Fremden zu und weil Amélie ihre Deckung nicht verlieren wollte, musste sie notgedrungen  folgen.

Sobald der Sonnengott sich regte, setzten sich die Hunde, die aufgeregt um die Kanzel geschnüffelt hatten und immer wieder zu der Katze hochstarrten, folgsam. Der Fremde verbeugte sich ehrfürchtig vor Re und sein Kopf rutschte vom Hals und polterte zu Boden. -,Verzeiht, Hoheit!' murmelte er erschrocken und hob den Kopf wieder auf seine Schultern. -Mir ist schlecht! dachte Amélie.

-,Ist das der Neophyt?' fragte er Thot und zeigte auf Amélie. ,,Ja, genau! Sie ist etwas scheu.'' Er lächelte und Re schob Amélie vor sich, behielt aber eine wärmende, schützende Hand auf ihrer Schulter, die sie aber vermutlich auch festhielt, sollte sie weghuschen wollen. (Nur die Götter waren in der Lage, Amélie in ihrem Sternenkörper festzuhalten.)

Ein peinliches Schweigen entstand, denn Amélie traute sich nicht den Mann, das Gespenst anzuschauen. Amélies Blick wanderte schliesslich über den Boden zu der Gestalt und sie schauderte. Aus den alten spitzen Lederstiefeln ragten die Schienbeinknochen. Sehnen und Haut, die rötlich und grau schimmerten, hingen daran. 

Das Gespenst war in ein Totenkleid aus Leinen und einen Rock aus Kamelhaar gekleidet. Die Kleidung war damals aus teurem, gutem Stoff genäht, hing aber nun schmutzig, grau, zerrissen in modrigen Lumpen um den halbverwesten Körper.

Amélie gab sich einen Ruck und blickte dem Gespenst ins Gesicht. Dieses war von einem langen, fast gepflegten roten Bart umrahmt, der viel verdeckte. Dadurch wirkte das Gesicht viel lebendiger als der restliche Körper. Die rote Seidenkappe, die mit den roten Haaren um die Aufmerksamkeit stritt, gab der elenden Gestalt etwas lächerliches. Um den Hals hing ein Reif aus vertrockneten und zerbrochenem Rosmarin. 

Die Augen waren geschlossen, schauten dennoch auf Amélie. Das Gespenst streckte den knochigen mit sehnigen Fetzen behangenen Arm vor. Amélie war doppelt froh, die Hand des Gespenstes steckte in einem Handschuh! Und sie hatte ihre Hand zuhause gelassen! Trotzdem spürte sie das vertrocknete, rissige Leder, das sich selbst wie vergilbte, Mumienhaut anfühlte. -,Ich bin Amélie!' Zu ihren Entsetzten merkte Amélie wie ihre Empfindungen des Ekels und ablehnenden Gedanken dem Gespenst wie dichter, dunkler, fädenziehender Dunst entgegenfloss. 

-,Ich bin David Joriszoon, bekannter unter dem Namen David Joris, der Erzkätzer.' Amélies schwarzer Dunst lichtete sich, denn die Worte lösten ihr Interesse aus. Die kurze Aufmerksamkeit, die kleine Hinbewegung ermöglichten Amélie wieder klarer zu sehen. Sie spürte die Hand Res, die sie sanft beschwerte und in die Kirche zurückholte. Sie blickte sich um und begegnete Thots Blick. Der wies stumm mit dem Kopf in Joris Richtung.

Der tonlose Schrei den Amélie ausstiess, halte durch die Kirche. Die Fenster klirrten leise, ein Hauch streifte das Entsetzten. Die schwarzen Hunde flüchteten erschrocken zu ihrem Herren. Und Tefnut, die auf dem Baldachin der Kanzel eingeschlafen war, rutschte vor Schreck und Überraschung ab. Sie verwandelte sich instinktsicher in ihre Löwengestalt und machte so der Leonhardkirche alle Ehre, an deren Kanzel bisher noch kein Löwe gehangen hatte. 

Hatte Amélie geglaubt die Gestalt des Gespenstes sei gruselig, so hatte sie sich getäuscht: Der Schrecken befand sich hinter der mickrigen Gestalt des Gespenstes. 

Ein riesenhaftes, dunkles Gebilde streckte sich bis unter das Dach der Kirche empor. Im inneren drehte sich die Gestalt eines Mannes mit rotem Bart, die mit Händen und Füssen an einen Fünfstern gekettet war. Der Stern, der von einem Ring umgeben war, schien aus schwarzen Metall zu bestehen und drehte sich unaufhörlich. Dabei wurde die Gestalt in alle Richtungen geschleudert. Hing Kopfüber, kopfunter. Bewegt wurde der Ring mit der hilflosen Gestalt von verschiedenen Gestalten. Darunter riesige, Monster, die ebenfalls aus schwarzen, dünnen Metallstangen zu bestehen schienen. Die Monster wurden von Wesen in Richterkutten gelenkt und angestachelt das Rad fleissig zu drehen. Menschliche Wesen, die höhnisch zu lachen schienen und aus deren Brust unaufhörlich schwarzes Pech hervorquoll, standen um den jämmerlich Gefangenen. Sklette, die an der rotierenden Gestalt zerrten, sobald sie sie packen konnten. Und im Hintergrund auf einem Thron mit Dreizack in der Hand und etlichen kleinen Teufeln umgeben, der Höllenfürst. Er ragte hoch auf, sein Unterleib aus dem zottigem, stinkendem Haar eines Ziegenbockes und Hufen, sein Oberkörper ausgemergelt, hungrig, das Gesicht mit dem spitzen, langen Kinn zu einer lachenden Grimasse verzogen\dots

-,Dort, das ist meine Frau Dirkgen, die treue Seele findet keine Ruhe.' seufzte David. Dirkgen, eine mollige, gross gewachsene Frau mit hellem Haar kniete rechts neben der schaurigen Szene, sie weinte händeringend und schien zu beten. Doch ihr Licht verschwand wie in einem schwarzen Loch in dem Leid. Mit ihr waren einige andere helle Seelen, doch es waren zu wenige.

Amélie sah unsicher zu Thot. Sie spürte wieder die Kraft von Res Hand. Sie vibrierte leicht nun bemerkte Amélie wie der Sog des dunklen Geschehens an ihrem Seelenlicht zerrte und tatsächlich kleine Streifen davon abriss und aufsaugte. ,,Das nenne ich mal raue See!'' bemerkte Re. ,,Da gehts ja zu wie auf Brüderchens Sandbank!'' ,,Nun Übertreibst du aber, Vater!'' Schu war unbemerkt hinter sie getreten. Auf seinen starken Armen trug er die löwenhafte Tefnut, die dort zappelte und fauchte und den Kopf hin und her warf.

-,Was ist da los?' fragte Amélie. Thot anwortete: ,,Das geschieht, wenn Menschen schlechte Gesetze machen und danach richten.'' Dann blickte er zu Joris: ,,Allerdings haben den guten David nicht nur schlechte Gesetze, sondern auch schlechte, sehr schlechte Gedankenmuster gefangen genommen.'' Joris, der bis dahin stumme den Kopf gesenkt hatte, richtete sich auf und sagte zu Amélie: -,Ja, es hat mich einiges erwischt, aber unschuldig bin ich nicht. Thot ist zu freundlich, um es zu sagen, aber ein Neophyt sollte gewissen Regeln einhalten, sobald er den Zugang zur geistigen Welt gefunden hat\dots'

Amélie schnaubte. -,Gell, das ist wieder so eine Das-erkläre-ich-dir-später-Angelegenheit?' Thot lachte:,,Jepp! Wobei du lernst so schnell, ich muss bald eh nichts mehr erklären!'' Dann zeigte er auf die Monster, die wie aus Metallstangen zu bestehen schienen: ,,Sie sind die Herren schlechter Gesetzte und ungerechter Richter.'' -,Wie kann man sie loswerden?' fragte Amélie. ,,Garnicht!'' war Thots schlichte Antwort. ,,Am besten ist es, wenn wir sie kennenlernen, ihnen Beachtung schenken. Sie können nicht verschwinden! Allerdings verändern sie sich, wenn sie gesehen werden. Am schlimmste ist es, wenn sie ignoriert werden, denn sie haben Macht. Wenn sie können, dann versuchen sie die Menschen dazu zu verleiten mehr schlechte Gesetzte zu machen und mehr falsche Urteile zu fällen.''

Amélie überlegte, -,aber\dots Wir haben Tefnut!' PLING! Tönte es, wie wenn ein feiner Gong angeschlagen würde und tatsächlich, Re hatte einen winzigen Gong in der einen und einen winzigen Schlägel in der anderen Hand und strahlte: ,,Die Kandidatin hat 10 Punkte! Tataaa!'' Es folgte verblüfftes Schweigen und dann lachte Joris lauthals bis ihm vor Lachen der Kopf runterfiel und dann lachten alle.

Der Sonnengott trat auf die Seite und machte eine einladende Geste: ,,Darf ich bitten? Wenn du nichts dagegen hast, dann übernehme ich kurzfristig das Ruder?'' fragte er Joris. -,Ich bitte darum,' antwortete das Gespenst, während es seine Seidenkappe ausklopfte, aus der Jahrhunderte alter Staub aufwirbelte und sie aufsetzte.

Sie folgten Re, der Amélie weiterhin umarmte, in den Altarraum. Tefnut tappte mit ihren gewaltigen Löwentatzen fauchend hinter ihrem Gatten her. Sie liess Joris nie aus den Augen. Die schwarzen Gespensterhunde schlichen mit eingeklemmten Schweif hinter ihrem Herren her.

,,Cool!'' War das erste, was sie von Schu hörten. An den Stufen zum Altarraum blieben sie stehen. Toth hob die Augenbrauen und Re gluckste vor Freude. Amélie spürte einen Hauch, von dem sie nicht wusste, ob es Luft war. Sie fühlte wie sich eine Säule hauchzart von unten nach oben an die Decke empor kreiste. Schu formte mit den Lippen ein O und ein Rauchring löste sich. Er schwebte träge auf die Säule zu. Als er sie berührte, zerfaserte der Rauch in mehrere Bänder, die in einem Zeitlupentornade erfasst nach oben flossen. 

Als die Bänder an der Decke ankamen, formten sie sich wieder zu einem Ring. Dieser sank langsam wieder nach unten und machte dabei in seinem Inneren eine Säule aus Licht und zarten Gestalten sichtbar, die durch das Dach der Kirche nach unten und oben pulsierten. ,,Gut gemacht, mein Junge.'' meinte Re anerkennend -,Wie schön!' entfuhr es Amélie, als sie die lichten Gestalten gewahr wurde, die sich gegen den Strom von der Decke im Wirbel abwärts bewegten. ,,Ja, das ist schön.'' meinte Thot. Joris lief eine Träne über die pergamentene Wange.

,,Hier hat deine Leidenszeit begonnen, als die Richter dein Grab in dieser Kirche entweihten, hier kann sie jetzt enden.'' sagte Re freundlich zu Joris. Thot trat dazu: ,,Du weisst, was das bedeutet?'' Joris schaute auf die dunkle Wolke und die gefesselte Gestalt und die Teufelsfratze. Er schluckte: -,Ich werde dem Schöpfer und der Waage der Maat ein zweites mal begegnen. Aber ich fürchte mich nicht. Ich will für mein Tun und Lassen gerade stehen.'

,,So sei es!'' sprach Re. Er trat in die Lichtsäule und verwandelte sich in die prächtige Gestalt einer Mumie in goldstrahlendem Sarkophag. Joris trat rasch an die Lichtsäule. Sein zerstörter Gespensterkörper begann golden zu schimmern und von den Füssen an löste sich Teile und verbanden sich mit dem Sarg des Gottes. Auch die Hunde lösten sich auf zum Schluss fiel die rote Seidenkappe auf den Boden. Als der letzte ätherische Rest des Gespenstes verschwunden war, verwandelte sich der Sonnengott in seine mächtige Abendgestalt mit dem Widderkopf. Der Säule bewegte sich schneller und schneller. Die riesige schwarze Wolke, die Joris gefangen gehalten hatte wurde angesaugt. 

Die dunklen Gestalten brüllten und kreischten. Sie wehrten sich. Die kleinen verschwanden im Licht des Re, die grossen klammerten sich an die Stufen und Säulen der Kirche. Je mehr von den teuflischen und menschlichen Gestalten verschwanden, desto unruhiger wurden die Spektren, die Geschöpfe der schlechten Gesetzte. Sie verlangten heulend Respekt und Opfer. Tefnut sprang auf sie los. Amélie hielt den Atem an. Hatte Thot nicht gesagt, sie würden sich nicht auflösen? Die Göttin der Wahrheit war aber nicht zimperlich. Sie schnappte mit ihren gewaltigen Zähnen die Beine der Wesen und hielt sie fest. Als welche zu entkommen drohten wuchsen der Göttin ihre menschlichen Arme und packten die Flüchtigen. Wo immer sie die Göttin berührte, schrumpften die falschen, die schlechten Gesetzte, wo sie sichtbar wurden und die Wahrheit berührten, verloren sie Grösse, Wichtigkeit und Substanz. Schliesslich verkrochen sie sich wie kleine Strichmännchen in den dunkelsten Winkel im Laienraum. ,,Bravo, Tefnut!'' rief Thot durch das Tosen und Toben. Die Göttin liess sich auf einen Stuhl sinken. Die Löwenohren noch immer gespitzt.

Re hatte sich in seine letzte Gestalt verwandelt und stand im priesterlichen, weissen, knielangen Rock und Kappe, ruhig mit dem Was-Szepter, dem Hirtenstab Symbol und Schlüssel von Macht und Glück in der Hand. Die finsteren Gestalten waren im Wirbel verschwunden, der sich wieder beruhigt hatte und wieder zart und fein, jetzt um den Sonnengott, rotierte. Für einen kurzen Moment schimmerte eine zweite Gestalt auf, die sich sanft um den Sonnengott drehte und dann langsam empor stieg. Amélie meinte zu sehen, wie sie ihr zuwinkte. Sie wollte zurückwinken. Erst jetzt bemerkte sie neben sich Thot und Schu, die sie beide festhielten. Sie schaute sie verwundert an. ,,Schätzchen, wir wollten sichergehen, dass du nicht plötzlich davonfliegst. Bei dir weiss man ja nie!'' nuschelte Schu und grinste erleichtert. Amélie winkte Joris zu. Ein kleiner Goldfunke schwebte von ihr hinter dem ehemaligen Gespenst her und verschwand in der Mitte des sternenförmigen Säulenkranzes.

Re hatte sich inzwischen wieder hergerichtet. Und wirkte so entspannt wie nach einem erfrischenden Bad. ,,Ein schönes Schiff, Kinder! Ein sehr schönes Schiff. Die Verbindung in unsere Welt ist wunderbar ausgebaut. Etwas verkalkt in den letzten Jahrhunderten, aber ursprünglich solide ausgeführt und gut ausgerichtet.'' Er verschränkte die Finger und streckte die Arme: ,,Doch, doch, so macht arbeiten auch in den Ferien Spass!'' 

Er stieg die vier Stufen aus dem Altarraum runter und nahm seine Tochter, die sich inzwischen in eine Katze zurückverwandelt hatte und schlief, vorsichtig auf den Arm. Sie schnurrte wohlig, während ihr Vater ihr den Hals und das hingestreckte Kinn kraulte. ,,Bist du meine brave, braave, grosse Miezekatze? Hast du die bösen, bösen Buben vertrieben! Jajaja!'' Während Re die rote Katze kraulte und diese stupste, murrte und purrte, fiel ihr blick auf Amélie. Im Auge der Göttin blitzte es rot-grün auf und sie streckte die Pfote in ihre Richtung und liess jede Kralle einzeln vorspringen. KLING, KLING, KLING, KLING,\dots KLING!

Amélie konnte sich nicht satt sehen an dem spiralenden Licht und den Wesen, die Ruhe und Frieden ausstrahlend von oben in den Raum eintauchten, sich nieder sinken liessen und wieder aufstiegen. Aber die Zeit in der Kirche war zu Ende. ,,Du kannst am Tag in die Kirche gehen! Die lichten Kerlchen sind immer da, sobald sich jemand im Raum befindet, der nach ihnen Ausschau hält.'' meinte Thot. -Vielleicht hat Amsi morgen Lust\dots, dachte Amélie. Im Gehen bemerkte sie die rote Seidenkappe. Ohne darüber nachzudenken, hob sie sie auf und steckte sie in ihre Tasche.

Kaum hatten die fünf die Kirche verlassen, trafen sie auf das nächste Abenteuer.

Als sie auf den kleinen Platz vor der Kirche traten, wimmelte es von Polizisten und Schaulustigen. Wieder und wieder wurde der Platz von den Blitzlichtern der in die Höhe gereckten Smartphones erhellt. 

Die Polizisten schienen kopflos zwischen den Schaulustigen hin und her zu eilen, wobei sie sich aus den Augen verloren. 'Es sieht aus, als würden sie Verstecken spielen', dachte Amélie und musste grinsen. Die Götter fanden es nicht lustig, denn sie hatten auf der Mauer den Osiris-Benuvogel entdeckt, der von den Zuschauern und den Polizisten umzingelt war. 

Re stiess kräftig die Luft aus und reckte die Schultern, er griff nach seiner Sonnenbrille. Doch bevor er sie abnehmen konnte, legte ihm Thot die Hand auf den Arm. 'Nicht! Oh, Herr! Wir müssen ihnen anders helfen, sonst gefährden wir womöglich die Menschen!' Re liess ein tiefes Grollen hören, Amélie stellt überrascht fest, dass sie, obwohl sie Unbehagen spürte, keine Angst hatte. Lag das am Lebensleib? Konnte der keine Angst empfinden? Sie hatte keine Zeit darüber nachzudenken.

'Wir müssen etwas tun!' Selbst in Gedanken war die Wucht der Worte des Sonnengottes und seine Aufregung spürbar. 'Tefnut!' rief Amélie. Die rötlich Katze maunzte. Einer Katze sollte es nicht möglich sein so laut zu miauen, das eine Ansammlung von Hundert Personen auf sie aufmerksam würde. Doch Tefnut konnte es. Das Miau-miau drang jedem einzelnen ins Ohr und entfaltete dort die wohltuende, beruhigende Wirkung von Katzenlauten.

Synchron bewegten sich die Köpfe der Menschenmenge, Zuschauer und Polizisten gleichermassen, wie in einem gigantischen Ballett. Die rötliche Katze tapte langsam und schnurrend mit erhobenem Schwanz auf die Menschen zu. Als sie in der Mitte stand und sich aller Blicke sicher sein konnte, verwandelte sie sich in eine Frau.

Zuerst schwieg die Menge. Dann brach ein ohrenbetäubender Lärm aus Rufen und Pfiffen los und ein Blitzlichtgewitter. Amélie spürte wie Thot sie berührte: 'Schnell! Bevor die Show vorbei ist!' Er schob Amélie hinter Schu und Re her, die sich am Rand der Menge vorbei schoben und eine treppe hinabstürmten. 'Was wird aus Tefnut!' Amélie huschte neben Thot her, der es tatsächlich schaffte auch im schnelle Lauf Haltung und Stil zu bewahren. 'Sie kennt den Weg!' antwortete er nur.

Sie waren am Ende der Treppe angekommen und rannten weiter auf einen Platz mit Tramschienen und einer riesigen Kirche, die sich auf einem Podest über dem Platz erhob und hoch in den Nachthimmel aufragte, da wurde Amélie von einem rötlichen Katzenschemen überholt.

Vor dem Eingang zur Kirche, der vollgehängt war ,mit Plakaten zu einer Museumsausstellung drehten sie sich um und starrten zusammen auf den Dachgiebel hinter dem sich die Mauer des Lohnhofes befand.

Ein aufgeregter Schrei tönte über den Platz und der mächtige Benu tauchte über dem Dach auf. An seinen Beinen hingen Geb und Hans. Es krachte als Hans mit einem Bein den First streifte. Der Barfüsserplatz auf dem sie standen war erfüllt vom Geschrei der Menschen und dem Gebrüll der beiden Erdgötter, die es beide scheinbar nicht sehr flugtauglich waren. 

Der Vogel verschwand über ihren Köpfen Richtung Münster. Es folgte ein hektisches Flappflappflapp und sie folgten mit ihren Köpfen dem schwarzen Hahn der sich bemühte dem grossen Vogel zu folgen. Sie wendeten sich der Kirche zu, als sie ein weiteres scharrendes Geräusch hörten. Aus dem schmalen Gang, den sie benutzt hatten, kam der Dackel geschlittert. Er wackelte über das Pflaster, was seine Stummelbeine hergeben mochten. Den kurzen Moment der Stille, der aus der Trägheit einer grösseren Ansammlung menschlicher Gehirne entstanden war, nutzten Amélie und die Götter, um unbemerkt in die Barfüsserkirche zu schlüpfen.

'Da stimmt was nicht' zischte Schu in die Dunkelheit. 'Nein, da stimmt etwas ganz und garnicht!' sagte Thot und lauschte aufmerksam. 'Verflixt!' schnaufte Tefnut, die sich in ihrer Frauengestalt, an die nächste kühle Steinmauer gelehnt hatte. es stieg Dampf von ihr auf. 'Thot! Wieso konnten die Menschen uns sehen?' Thot schaute die Göttin ernst an: 'Das, meine liebe tefnut, sollten wir sobald als mödlich herausfinden.' ' Zumiendest heisst es nichts gutes, das ist sicher!' bemerkte Re und zu Amélies erstaunen wirkte der sonst so unerschütterliche Sonnengott besorgt.

\section*{5}
\addcontentsline{toc}{section}{5}

Hapi hing an dem Baldachin der Barke und hielt Ausschau nach den vier Zeugen die er aushorchen wollte. Auf beiden Seiten der Barke sassen die Wesen, Götter und Tote, auf den Bänken. Unter den Toten hatte sich eine grosse Anzahl von Baslern dazugesellt. 

Einziges Kriterium, das es für die Toten gab, damit sie auf der Barke mitfahren konnten, war, dass sie geläutert waren. Dabei spielte es keine Rolle, welcher Religion sie angehörten. Ihre Herzen, besser gesagt ihre Taten mussten gewogen und für gut befunden worden sein. Damit hatten sie Anrecht auf die paradiesischen Gefilde, die es bekanntlich nicht nur bei den alten Ägyptern in der Form des Sechet-iaru gab, sondern in allen anständigen Religionen.

Für die Toten war es eine schöne Abwechslung und einige freuten sich sehr Basel und 'ihren Rhii' wieder zusehen. Da sie die Litanei der ägyptischen Reisenden nicht kannten, hatten sie ohne gross darüber nachzudenken, aus voller Kehle (die sie eigentlich nicht hatten) angefangen ein Lied zu singen:

Z'Basel an mym Rhii,\\
jo, dört möcht i sii!\\
Wäiht nit d'Luft so mild und lau,\\
und der Himmel isch so blau,\\
an mym liebe,\\
an mym liebe Rhii.\\\footnote{Vom guten alten Hebbel stammt der Text, Melodie steuerte Franz Abt bei}

Hapi schnippste den Takt mit den Fingern mit. Er überlegte, ob er Re und Horus vorschlagen sollte, statt des elenden Gebrülls jede Nacht, auch in der Duat die Götter und Toten ein schwungvolles Lied singen zu lassen. 

Dann hatte er seine Kandidaten entdeckt. Es waren vier ehemalige Priester. Sie hatten ihre Mumiengewänder an. Sie sassen daher nicht auf der Bank, sondern standen in einer Reihe nebeneinander vor dem Baldachin. Ihr Körper steckten in den weissen Binden, nur ihre Köpfe ragten oben raus. Um ihre frühere Funktion anzuzeigen, hatten sie verschiedenen Kopfschmuck auf, deshalb nannte Hapi sie auch Hörnchen (er trug Kuhhörner), Schlange (weil er eine Schlange auf dem Haupt hatte), Tasse (weil er eine Schale auf dem Kopf balancierte) und Glatze (Glatze hatte nur ein Tuch auf dem Kopf).

Hapi schwang sich über den Baldachin und liess sich auf Glatzes Schulter nieder. Es war ungemütlich und heikel, aber da die vier unbeweglich waren und es so laut war, konnte Hapi die Priestertoten nur so verstehen. Glatze schwangte gefährlich, als sich der Pavian vom Baldachin auf ihm niederliess.

"`He! Edler Sohn des Horus! Das ist gefährlich! Möge er sich woanders niederlassen!"' rief Glatze ängstlich und empört. -'Pssst! Ich muss dich etwas fragen. Vermutlich bist du der einzige, der mir helfen kann\dots' Glatzes Ohr schien riesig gross zu werden. -'Was denn?' Natürlich konnte er als Toter und Priester Gedanken lesen. -'Kannst du dich an Haremhab erinnern?' -'Guter Mann, guter Pharao. Hat uns wieder in Amt und Würde gebracht.' -'Was redet ihr da?' mischte sich Hörnchen ein. -'Das geht dich nichts an, der edle Hapi hat mich gefragt!' Schnaubte Glatze.

-'Der edle Hapi hat sicher nichts dagegen, wenn wir dich nötigenfalls verbessern.' Hörnchen schaffte es, selbst seinen Gedanken einen arroganten Tonfall zu geben. -'Oder ergänzen!' Auch Schlange konnte ausgesprochen nasal denken. -'Was für ein Dunkel, was für ein grausiges Dunkel\dots' bei Tasse schienen ausser der Tasse auf seinem Kopf, einige andere in seinem Schrank zu fehlen, dachte Hapi. Die Gedanken waren wie mehrere Knäule aus bunter Wolle mit denen eine siebenköpfige Bande Kätzchen gespielt hatte. -Wenn man genauer hinsieht, dachte Hapi, sind Wolle und Kätzchen das Letzte, was zu dem schwarzen, abgrundtiefen Geflächt an Erinnerungen und Verwirrung passt\dots

Die drei anderen ignorierten, was Tasse beitrug und fuhren fort: -'Er hatte es nicht einfach. Viele von uns waren eigene Wege gegangen.' meinte Glatze. -'Welch Wunder! schliesslich mussten wir Priester die Götter praktisch im Untergrund bei Laune halten.' Ereiferte sich Hörnchen. -'Ich habe ihm gesagt, es ist gefährlich. Er darf sich ihr nicht nähern!' murmelte es in Tasses Kopf.

-'Haremhab hat die Götter wieder gnädig gestimmt!' Meinte Glatze. -'Aber dennoch waren sie nicht besänftigt. Der arme Pharao hat es zu spüren bekommen,' meinte Hörnchen theatralisch. -'Er ist Kinderlos geblieben! Er ist bis nach Karkemisch gekommen und hat sich am Bau der grossen Tempeln in Luxor und Karnak unsterblich gemacht, aber mit den Frauen hatte er kein Glück.' Schlange machte ein betrübtes Gesicht, das seine tragischen Worte unterstreichen sollte.

Tasse wachte kurz aus dem Gedankengewirr auf: -'Es war nicht des Pharaos Schuld. Es war die Priesterin\dots' dann rollten seine Augen wieder in zwei verschiedene Richtungen davon.

Hapi hatte genug gehört. Viel mehr würde er aus den Vieren nicht herausbekommen. Im Gegensatz zu seinen drei Kollegen, nahm Hapi Tasses Worte sehr ernst. Auch deshalb, weil es merkwürdig war, wenn eine solche verrückte Gestalt wie Tasse, laut den Totenriten berechtigt war in der Barke zu fahren. Tasse musste etwas erlebt haben, was seinen Geist zerrüttet hatte, obwohl er selbst unschuldig an der Sache sein musste, die ihn beschäftigte. Hapi hatte Angst weiter in Tasse einzudringen.

-'Danke die Herren Priester, wie erwartet ward Ihr mir eine grosse Hilfe. Ich werde Hathor bitten, Euch eine Extrarunde Bier bringen zu lassen!' Drei der Köpfe neigten sich huldvoll: "`Wir danken Dir, oh, edler Hapi, Sohn des grossen Horus!"'

"`Bier her, Bier her, oder ich fall' um!!!!"' hörte Hapi Tasses dünnes Stimmchen trällern und ein kalter Schauer lief ihm über den Rücken. Er hangelte sich über den Baldachin zurück. Er musste schleunigst Amset finden. Hoffentlich hatte er seine Zeugen noch nicht befragt! -'Amsi! Amsi! Warte! Amset hatte seine Zeugen schon ins Auge gefasst. Sie waren in der Mitte der Barke. -'Was denn?' fragte Amset. -'Frage sie nach einer Priesterin.' gab Hapi durch. -'Priesterin?' -'Mehr weiss ich auch nicht, aber es ist eine heisse Spur, glaubs mir!' antwortete Hapi und dachte an Tasses rollende Augen.

 \section*{6}
\addcontentsline{toc}{section}{6}

Das letzte, was die Polizisten, die alle mit offenem Mund staunend dem riesigen Vogel mit den beiden Männern im Schlepptau hinterhergesehen  hatten, bemerkten, war der der Schwanz des Dackels, der eilig in das Gerbergässlein verschwand. Während Schmied und Hofer ins Auto sprangen und mit Blaulicht die Passanten auseinander trieben und zum Barfüsserplatz davon fuhren, rief Meyer: ,,komm, Mosi, die kriegen wir!'' Warum immer ich, schnaufte Moser als sie steilen Treppen zum Heuberg hinter den flüchtenden herrannten. 

Als die beiden Polizisten Gas gaben, rannten ihnen gleich mehrere Schaulustige, weiterhin hinterher filmten und johlten. Eine Bierdose drehte sich auf dem Pflaster, die einer der Schaulustigen hastig fallen gelassen hatte.

Als Moser und Meyer auf dem Platz vor der Lohnhof angekommen waren trafen sie auf die Kollegen mit dem Streifenwagen. Die ersten Schaulustigen streckten vorsichtig ihre Nasen um die Ecke. Der Vogel war auf die Terrasse gelandet. Die beiden nackten Männer lagen zerknittert vor der Mauer und rieben sich die Köpfe. Der Dackel drängelte sich durch die Beine der Zuschauer und hüpfte freudig auf seinen Herren zu und schleckte ihn im Gesicht. Der andere Nackte redete in einer fremden Sprachen auf den Vogel ein. Der schwarze Hahn stolzierte über den Platz und schaute die Zuschauer, die sich dichter und dichter an die Götter pirschten, misstrauisch und mit schiefem Kopf aus den kleinen, gelben Augen an.

Zum Schrecken der Polizisten, die sich auf die Festnahme eingestellt hatten, nachdem sie weitere Kollegen auf der Treppe vom Barfüsseplatz gehört hatten, waren die beiden nackten Männer auf die Begrenzungsmauer geklettert. Der Dackel bellte wild und winselte. Der grosse Vogel jedoch erhob sich in die Lüfte und als er über die Umfriedung flog griff jeder der Männer wieder nach einem Bein des Vogels. Mühelos stieg der Vogel weiter auf. Dann flog er mehrere weite Kreise über die Altstadt. 


\section*{7}
\addcontentsline{toc}{section}{7}

'Vielleicht kann Franz uns weiterhelfen,' meinte Re. 'Ich hoffe es.' Thot blickte sich nachdenklich um. Sie standen immer noch im Eingangbereich der grossen Kirche. Amélie staunte. Diese Kirche schien ihr grösser als die beiden anderen. Sie wunderte sich, denn die Kirche war mit verschiedenen Trennwänden und einem Kassentresen ausgestattet. Thot hatte Amélies Blick bemerkt. 'Die Barfüsserkirche ist offiziell keine Kirche mehr, sondern das historische Museum.' 'Man war sogar besonders gründlich mit den Barfüssern und hat die Kirche entweiht.' 'Entweiht? Was bedeutet das?' fragte Amélie. 'Damit eine Kirche, eine Kirche sein kann, muss sie von einem oder mehreren Priestern geweiht werden.' Amélie blickte mit skeptischen Blick zu Thot auf. 'Stell es dir vor, wie eine Telefonverbindung. Die Kirche ist das Telefon, aber damit du telefonieren kannst, braucht es eine Nummer und eine Verbindung.' Thot strahlte und die drei anderen Götter nickten anerkennend mit den Köpfen. 'Da hast du aber'n gutes Gleichnis gefunden', meinte Schu und schob seinen Cigarettenstummel in den anderen Mundwinkel und lies ein Wölkchen aufsteigen.

'Ja, also nein!' sagte eine nörgelnde Stimme. 'Mir sagt man ja nichts, aber das Rauchen ist in der Kirche nicht gestattet! Das weiss ich. Ganz sicher ist das Rauchen nicht erlaubt!' Unter dem Lettner tauchte eine kleine, breite Gestalt auf, die in Unmengen von Stoff gehüllt schien. ,,Äh, Franz?'' rief Thot unsicher. 'Hier gibt es keinen Franz!' sagte die kleine, runde Gestalt und kam auf sie zu. 'Welcher Franz? Ich kenne keinen Franz! Ich weiss nicht, wo der Franz hin ist.' Die Stimme wurde unsicher.

Durch die hohem Fenster, die auf der rechten Seite mit grossen Tücher verhängt waren, fiel das spärliche Licht der umliegenden Gebäude. Erst als die Gestalt sie beinah erreicht hatte, konnten sie erkennen, dass es sich um eine winzige Frau handelte. Sie war sogar kleiner als Amélie. Was ihr an Höhe fehlte, machte diese Dame jedoch in die Breite wett. Das ausladende Kleid aus weissem Stoff mit zarten rosa Blümchen liessen sie wie ein Ballon aussehen. Obwohl sie ein eng geschnürtes Mieder trug, das an allen Ecken und Enden knarzte, sobald sie sich bewegte, konnte keiner bestreiten, dass sie fett war.

Ihr Gesicht, geprägt von den kräftigen, hängenden Pausbacken und den tiefen Falten, die von den Nasenflügeln zu ihren Mundwinkeln führten, liessen sie wie eine traurige Bulldogge aussehen. Ihre dünnen, dunklen Haare waren unter einer weissen Haube verborgen.

Als sie die Gruppe erreicht hatte, fiel ihr Blick auf Tefnut, die sich vorsichtshalber in ihre Katzengestalt verwandelt hatte. 'Tiere?' rümpfte die Dame die Nase, 'Tiere sind in der Kirche nicht erlaubt!' Sie entdeckte Schu, der plötzlich unruhig von einem Fuss auf den anderen trat und versuchte keine Rauchwölkchen zu machen. 'verlassen Sie dieses heilige Haus!'

'Was soll das alles nur? Es bleibt immer, immer an mir hängen! ich verstehe nicht, wieso diese Leute hier einfach in die Kirche kommen.' Die Dame murmelte vor sich hin und ging auf nervös auf und ab. Sie bemerkte, dass sich keiner von den Besuchern von der Stelle gerührt hatte. Sie rang die Hände und schaute immer wieder flehentlich nach oben. 'Was soll ich denn noch tun?'

'Äh, Gnädige Frau, wir möchten zum Franz. Er wollte uns hier heute Nacht empfangen. Sie wissen nicht\dots?' 'Nein!' rief sie empört. 'Hier gibt es keinen Franz! Nichts ist es mit Franz! Gehen Sie und nehmen Sie ihr Vieh mit und den Bettler! ich kann sie nicht mehr sehen, diese Brut!' Thot wurde schneeweiss im Gesicht und schaute sich vorsichtig nach Schu und Tefnut um.

Die Katze war Schu um die beine gestrichen. Nun nahm sie Kurs auf die Frau. 'Tefnut!' zischte Thot. Doch die Katze ging langsam Schritt für Schritt auf die runde Dame zu.

Amélie bemerkte ein silbriges schillern auf der blassen Gespensterhaut der Frau. Sie war wieder in ihre wirren Worte gefangen und beachtete die Katze nicht. Der silbrige glanz wurde stärker und die Haut wurde flüssig. Wie flüssiges Silber. es bildeten sich Tröpfchen, die dann hinunterglitten. Stück für Stück verlor die Dame auf diese Art ihr Gesicht. 

Amélie war schockgefroren. Während die Frau nichts von der Veränderung ihres Äusseren zu bemerken schien, aber wirrer und wirrer, schliesslich taumelnd im Kreis irrte, nachdem die Augäpfel verschwunden waren und bis auf einieg Fetzten am Unterkiefer nur der Schädelknochen übrig geblieben war. Einge Haarsträhnen hingen daran und am Hinterkopf wurden dunkle, poröse Stellen am Knochen sichtbar.

'Arme Seele!' Maunzte Tefnut der Gespensterfrau nach, die sich schliesslich blind und völlig verwirrt humpelnd und wankend aif den Weg zum Lettner gemacht hatte.

'Hast du es gesehen, Thot?' fragte Tefnut 'Ja, das habe ich!' seufzte er. Amélie bemerkte eine Träne im Auge der Göttin. 'Aber, Tefnut, nimm es nicht so schwer, die alte Schachtel wusste einfach nicht, das du eine Göttin bist!'versuchte Amélie zu trösten. 'Wie?' Tefnut hob verwundert das Katzenköpfchen. 'Ach, das meinst du! Das ist nicht schlimm. Was sie mit der armen Frau passiert ist, dass ist allerdings etwas anderes!' Die Göttin schlug wütend mit dem Schwanz.

'Kinder!' sagte Re und richtete sich auf. 'Schnell! Uns bleibt nicht viel zeit fürchte ich!' Der Sonnengott lotste sie aus der Kirche. Es schien nicht viel Zeit vergangen zu sein, denn der Barfüsserplatz war voll von Menschen, die eilig dem Münsterhügel zustrebten, um dem Benu zu folgen.

'Es gibt eine Abkürzung. Gut das Hans mir die auf der Karte gezeigt hat.' Inzwischen waren sie bei den letzten drei Stufen, die zu den Tramhaltestellen führten stehen geblieben. Da machte der Sonnengott eine stampfende Bewegung mit seinem Fuss und sie sanken durch das Strassenpflaster . Sie landeten unsanft auf dem Boden eines Steintunnels. Ausser Thot und Tefnut, die beide unglücklicherweise direkt in den Birsig gefallen waren.

Als Katze hasste Tefnut Wasser (im Gegensatz zu ihrer menschlichen Gestalt, in der sie ein heisses Bad zu schätzen wusste). Die vier anderen blickten dem fauchenden Pelzknäuel nach. Zuerst sprühte es Tropfen von der Katzengestalt, dann sahen sie nur noch einen Streifen aus Dampf, der sich im Tunnel Richtung Rhein verlor.

Thot rappelte sich erschrocken auf. Er klopfte angewidert seine Kleider aus. 'Sorry, alter Knabe!' Lachte Re und schlug seinem alten Gefährten auf die Schulter. Dann überprüfte der Sonnengott die nassen Stelle am Hintern des Freundes. Thot blieb auf dem Weg stehen und sah konzentriert drein. Im Nu begann es zu dampfen und der Mantel und Hose waren wieder trocken.

Damit der kleine Fluss Birsig den Verkehr nicht weiter störte, hatten die Basler ihn kurzerhand in der Innenstadt in einen Tunnel verband. Dieser bot ihnen nun einen menschenfreien, direkten Weg an den Rhein. In der Mitte plätscherte der Flüsschen in einer Rinne und an den beiden Seiten war ein bequemer breiter Weg. 'So ein schnelles Miezekätzchen!' sagte Re und lachte aus vollem Hals. Amélie staunte über das sonnige Gemüt des Göttervaters. 

Sie kamen an die Mündung des kleinen Flusses. Der Torbogen an der Schifflände der Fluss und Wanderer an das Rheinufer entliess, war schwach von den Lichtern des gegenüberliegenden Ufers erhellt.

Tefnut war die kurze Metallleiter mit der man an den Quai des Rheines hinaufsteigen konnte, schon hochgeklettert und putzte sich ihren dicken Katzenwinterpelz. Die drei Götter und Amélie folgten ihr. Sie liessen ihren Blick schweifen. Der Blick auf der rechten Seite war von der Mittleren Brücke gänzlich eingenommen.

Sie konnten die Pfalz und die Barke auf der anderen Seite der Brücke nicht sehen, aber sie hörten den Lärm von aufgeregten Stimmen herüber wehen. Die Martinshörner der Polizeiwagen bildeten einen kakophonischen unheilvollen Grundton. 


\section*{8}
\addcontentsline{toc}{section}{8}


Amset schaute Hapi an. -'Wie: Eine Priesterin?' -Ich weiss es nicht! Von den alten, vertrockneten Priestern weiss ich nur das. Alles andere über die Zeiten des Haremhab wissen wir schon\dots' Amset überlegte, dann meinte er -'Gut dann gehe ich zu ihnen. Wenn jemand den Klatsch und Tratsch von damals kennt, dann sie.' -'Nun schau nicht so leidend!' Hapi kicherte. Wiedereinmal freute er sich ein Pavian in einer klaren und instinktsicheren Pavianwelt zu sein. Er blickte Amset mitfühlend nach, als dieser sich zu den vier jammernden und zeternden Gestalten in der Mitte der Barke aufmachte.

Es dauerte eine Weile, bis die beherzten Klagefrauen Amset bemerkten. Es konnte daran gelegen haben, dass sie ihren Job sehr ernst nahmen und sich unentwegt die Haare rauften, sich auf die entblössten Brüste schlugen und heulten und kreischten. Sie bewarfen sich gegenseitig mit Asche, die sie in einem grossen Tontopf in ihrer Mitte stehen hatten und aus ihren Klagen waren Gebetsfetzen zu hören. 

Oder es kam daher, weil Amset die Frauen fasziniert beobachtet. Bisher hatt er sie kaum beachtet, obwohl er Nacht für Nacht mit ihnen auf der Barke fuhr. Sie waren professionelle Heulerinnen. Sie wurden dafür bezahlt zu klagen. Sie kamen mit ihren weissen, schmucklosen Gewändern und nahmen an der Trauerprozession teil und an der Beerdigung. -Niemand kommt an ihnen vorbei, dachte Amset und beim Leichenschmaus, da folgen dann die Geschichten und Gerüchte. 

Zwei waren betagte Grossmütter, die es zu einer wahren Meisterschaft gebracht hatten. Es klatschte und quickte und grunzte, die Asche wirbelte um ihre wirren, grauen, verfilzten Haare. Und ihre grossen Busen wogten mehr, als Amset lieb war. Die beiden anderen waren jünger und sie bemühten sich auf attraktive Art und Weise zu leiden und zu klagen. Ihre Haare waren kunstvoll zerzaust und irgendwie liess die Asche ihre jugendlichen Wangen zwar bleich, aber auch hilflos und auf düstere Art schön schimmern.

 Wie jede Nacht hatte Hathor den Damen einen Teil des Lohnes in Naturalien und im Voraus, in Form von Bier bezahlt. "`Sie kommen einfach besser in Schwung mit einem Schlückchen Bier!"' meinte die praktische Hathor nur. Damit die Heulerinnen 'in Schwung' blieben, hatten sie spezielle Trinkschalen von Hathor bekommen. Die einfachen Schalen aus Ton hatten unten drunter zwei winzige Füsse. Je mehr sich die Frauen ins Zeug legten, umso schneller brachten die Schalen weiteres Bier.
 
Die beiden jungen hatten Amset längst erspäht. Und zu ihrem grossen Erstaunen, zeigte der Horussohn plötzlich Interesse. Da er einer der wenigen Götter war, die jung und von menschlicher Gestalt waren, hatten die Mädels ihn längst im Visier gehabt. Nur hatte der Gott bisher keinerlei Reaktion gezeigt, wenn sie sich Nacht für Nacht anstrengten, besonders weiblich und sexy zu leiden.

"`Edler Amset! Können wir etwas für dich tun?"' fragte die eine. Amset schauderte bei dem Gedanken, an den Job, den die beiden hoffentlich nicht für ihn ausüben mussten, oder womöglich für Amélie\dots "`Amsi, alles in Maat?\footnote{Götterwortspiel: Maat = Ordnungsprinzip}"' fragte die Zweite, weil der Gott plötzlich blass geworden war.

"`Ich brauche Eure Hilfe! Ich muss Euch einige Fragen stellen."' "`Waaahicks kriegen wir dafür?\dots Hicks?"' Eine der Alten, eine Hand voll mit Asche und in der anderen die Bierschale, sah Amset scharf zwischen ihren verfilzten, grauen Haarbüscheln an. Er überlegte fieberhaft. "`Ich werde den anmutigen Damen morgen Nacht von dem Getränk namens Obstler mitbringen\dots"' "`Abgemacht! Wenn du es vergisst, wissen wir wo dein Zimmer ist, Börps! Tschuligung, isch alles so traurig hier?"' die Zweite rollte mit den Augen. 

Amset hatte sich zu den Alten gekniet, die auf zwei kistenartigen Hockern sassen, 'wer will, dass wir am Boden hocken, soll uns gefälligst helfen aufzustehen' hatten sie gesagt, nachdem sich ein oberpenibler, toter Expriester beschwert hatte. Danach sagte keiner mehr etwas, es fanden sich sogar zwei kleine Götter, die den Damen ihre Hocker jede Nacht auf die Barke trugen.

"Waaahicksss, willu wissnn?"' fragte die Erste. "`Wir sagen dir alles, was du nur willst!"' säuselte eine der jungen und hängte sich begeistert an Amset Hals. Nur mit Mühe konnte er sich befreien, die junge Frau fuhr schmollend ihre mit Asche dekorativ schwarz gefärbte Unterlippe aus. Die zweite junge sah ihre Stunde gekommen und versuchte sich auf Amsets Schoss zu schlängeln. "`Alles, alles, was du willst!"' flötete sie, es ergab sich nicht jede Nacht die Chance von einer Klagefrau zur Muse eines wichtigen Gottes aufzusteigen\dots

"`Börrrps! Tschuligung?"' die zweite Alte hielt sich die Hand vor den Mund. Dabei fiel ihr Blick auf eine der Schalen, die mit frischem Bier angewatschelt kam. "`Duuh, Schäähtzchen, Duu!"' zärtlich nahm sie die Schale und sah Amset über den Rand an.

"`Ich muss etwas über die Priesterin wissen, die zu Haremhabs Zeiten Unheil angerichtet hat!"' sagte Amset auf gut Glück. Die Zweite Alte, die einen Schluck Bier genommen hatte, verschluckte sich und sprühte Asmet und der jungen Klagefrau, die sich auf seinen Schoss gezwängt hatte alles um die Ohren. \begin{LARGE},,Börp!?!''\end{LARGE} Die andere Alte blickte Amset wachsam an und schien vom Vollrausch auf Stocknüchtern und Glasklar in Null Komma Nix beschleunigt zu haben, ohne über 'Los' zu gehen und einen verkaterten Morgen abzuwarten. 

"`Was! Willst! Du!"' zischte sie. Die zweite packte die beleidigt da hockende Junge am Arm: "`Ihr Junggemüse könnt Euch für heute Nacht frei nehmen! Wir kommen ohne Euch zurecht!"' "`Aber! Grossmutter!"' riefen die beiden unisono empört. Die Alte nahm ihren Handbesen, den sie unter ihrer Kiste verstaut hatte und scheuchte die Jungen weg, indem sie auf sie einschlug, wo sie sie erwischen konnte. "`Weg! Wegwegweg! Das ist nichts für Euch!"' Die zweite Alte half nach, indem sie den Mädchen auf die Beine schlug. Erschrocken rappelten sie sich hoch und flüchteten in den vorderen Bereich der Barke, wo sie sich damit begnügten, den hübschen, toten Baslern schöne Augen zu machen.

Die beiden Alten beugten sich zu Amset. Er hatte vom Besen einiges abbekommen und sein Gesicht schien von einer missgelaunten Katze als Kratzbaum missbraucht worden zu sein: "`Was weisst du!"' mit brennenden Augen blickten sie ihn misstrauisch durch die traurigen Strähnen an. "`Woher weisst du es?"'

Amset wusste nichts. Besser gesagt, jetzt wusste er mehr. Er hatte mit seiner Frage hoch gepokert und offensichtlich einen Royal Flash gezogen. Nun hiess es geschickt vorgehen und sich von den beiden harten, alten Nüssen nicht unterkriegen lassen. Aber Anubis hatte seine Lehrlinge nicht im Stich gelassen. "`Meine Damen, das sollten wir in Ruhe besprechen\dots "' antwortete er ihnen und zog eine Flasche Baslelländlichen Kirschschnaps aus dem Umhang\dots -'Wenn du es richtig anpackst, werden sie dir eine ganze Oper singen, die Schnapsdrosseln' hatte Anubis prophezeit.

Als die übrigen drei Brüder Amset nach einigen Stunden fanden, lag er laut schnarchend in einer Ecke. Duamutef und Hapi rümpften die Nase, weder Schakal noch Pavian liebten Alkoholgeruch. Kebi hatte nichts, was er rümpfen konnte\dots Eine der Schalen mit Füssen sass auf Amsets Bauch und schnarchte auch. Eine andere lehnte an seiner Seite und regte sich nicht einmal als Tef an ihr schnupperte. -'Hoffentlich hat es etwas genutzt', meinte er. -'Er scheint Anubis gründlich missverstanden zu haben, was den Schnaps angeht', fügte Kebi an. -'Die Weiber sind ihm eben über. Die alten Vetteln lassen sich von Amsi  doch nicht unter den Tisch trinken', meinte Hapi altklug.

"`Brüder?"' Amset öffnete die Augen: "`Amélie! Ihr müsst sie bewachen! Kebi! Du musst Thot und Re warnen\dots"' Duamutef winselte und schleckte Amset unwirsch im Gesicht. -'Sag schon, was ist los?' "`Oooh! Mir, mir isch scho schlecht\dots"' Amset stöhnte und drehte sich auf die Seite und würgte. Eine dritte Schale flüchtete verschreckt aus ihrem Versteck und lief davon.

-'Aus dem kriegen wir heute Nacht nichts raus!' Hapi kratze sich am Kinn. -'Aber, welche Gefahr?' fragte Kebi, -'meint ihr ich sollte seine Worte befolgen?' -'Ja!' Antwortete Duamutef und auch Hapi nickte. -'Gut! Ich werde die anderen warnen und dann über die Stadt gleiten und schauen, was Sache ist.' Kebi wollte losfliegen, aber dann fiel ihm etwas ein. -'Meine Zeugen! Ich habe sie noch nicht befragt!' -'Flieg Bruder, wir regeln das\dots' Kebi erhob sich und verschwand in der Dunkelheit.

\section*{9}
\addcontentsline{toc}{section}{9}

Sein Gefieder bemerkte Moser schien tatsächlich aus Flammen zu bestehen, denn der Vogel leuchtet wie eine Fackel und hinterliess sogar eine Lichtspur. Die drei Runden die er über die Stadt drehte konnten sie deutlich in dem dunklem Winterhimmel sehen. nach der dritten Runde bewegte sich der Vogel Richtung Rhein. Die Polizisten sprangen in den Wagen und kamen genau in dem Moment auf der Pfalz des Münsters an, um zuzusehen, wie die beiden Männer die Beine des Vogels losliessen und in den Rhein stürzten. 


Der Vogel blieb in der Luft stehen und wurde dabei grösser und grösser bis er schliesslich, sehr leise explodierte. In tausend Sternen und Fünkchen regneten die Flammen, die zuvor Vogel gewesen waren über den Rhein nieder. Moser schluckte. Er konnte sich nicht losreissen. im Gegensatz zu Hofer und Schmied, die aufgeregt auf das erleuchte Wasser deuteten. Es waren die Silhouetten der beiden Flüchtigen zu sehen, die in den Garten der Universität anzuladen schienen. Seufzend stapfte Moser hinter den jungen Kollegen her. Natürlich war von den beiden nichts mehr zu sehen.

\sterne

'Hans, Hans! Osiris ist gelandet!' Geb schwamm mit kräftigen Zügen in der Strömung des Flusses, neben ihm war auch Hans aufgetaucht. 'Isch r in dm Rücksägg akho meinsch?' schnaufte Hans. Auch er teilte mit seinem gewaltigen Schultern das eiskalte Wasser. 'Ich kann ihn spüren! Osiris! Du bist fett geworden, mein Junge!' Geb lachte auf und prustete einen grossen Schluck Wasser aus. Er war froh! Hans blieb hinter ihm und liess, wie abgesprochen den Imiut nicht aus den Augen.

Dies war der heikelste Moment ihrer Mission, den Osiris war erst sicher, wenn sie den Schutz aller Götter, die Barke erreicht hatten. Hans bemerkte eine Bewegung über ihren Köpfen und erschrak. Dann hörte er die Stimme des jungen Falken.

'Kebi!' rief Geb schon. 'Grossvater!' hörten sie den Falken. 'Bringt euch schnell in Sicherheit! Beeilt euch!' Der junge Falke drehte ab. Ein AUgenblick später war er verschwunden.

Die Barke hatte sich quer in die Strömung gestellt. Die beiden Erdgötter rumpelten von der Strömung gezogen gegen die Bordwand. Viele Hände streckten sich ihnen entgegen. Ein goldener Widderkopf voran. ,,Kommt, kommt! Grossväterchen! Vater! Hans!'' tönte es freudig und dumpf unter der Maske.

Geb und Hans packten jeder einen Arm des Horus und dieser hievte die beiden Götter auf die Barke. Die Göttinnen eilten mit Decken dazu. Isis blickte Geb an. Der drückte sie an seine Brust und sagte: ,,Wir haben es geschafft! Enkelin, du kannst stolz sein auf deinen Mann!'' Meinte Geb es nur, hatte er einen Schluchtzer gehört. Als er in Isis gesicht sah, dachte er, er müsse sich getäuscht haben, die sie lächelte und reckte das Kinn.

\section*{10}
\addcontentsline{toc}{section}{10}

Die fünf hatten mit offenen Mündern die Verwandlung des Benus beobachtet. Re seufzte:,, Ich glaube sie haben es geschafft!'' Amélie war sich nicht sicher, es hatte nicht danach ausgesehen, als wenn von dem Vogel etwas übrig geblieben wäre.

Sie hörten einen Falkenschrei. 'Kebi!' Tefnuts feine Katzenaugen hatten den Falken zuerst entdeckt. Der Falke landete auf der Schulter seines Urgrossvaters Re. 

'Osiris ist in Sicherheit. Sie sind jetzt auf der Barke!' dachte Kebi. ,,Das ist mal eine gute Nachricht'', sagte Thot. 'Leider gibt es auch eine schlechte!' antwortete Kebi. ,,Ah, noch eine! Welche genau meinst du?'' sagte Re und schmunzelte. Kebi raschelte mit den Flügeln. 'Amset sagt, ich soll euch warnen. Es scheint so, als ob unser Aufenthalt hier in Basel nicht nur Freunde hat.' Re runzelte die Stirn und Tot strich sich nachdenklich über das Kinn. 

Schu, der unruhig auf dem Steg hin und her gewandert war und dabei heftig an seiner Zigarette sog, blieb stehen: ,,Das ist doch zum Ammit füttern! Was ist da los, Kebi?'' Der Falke schaute seinen Lieblingsurgrossvater\footnote{Kebi hatte eine riesige Anzahl an Grossvätern. Im Laufe der Dynastien waren ganze Familienzweige neu dazu gekommen und wieder verschwunden. Daher erlaubte er sich den Luftgott Lieblingsgrossvater zu nennen, er fand als Falke könne er sich das erlauben, ohne die anderen Grossväter zu kränken. Re war natürlich eine Ausnahme. Ganz klar. Wer wollte es sich schon mit dem obersten Göttervater verscherzen?}  besorgt an. 'Ich weiss es nicht, Urgrossvater. Ich bin auf dem Weg nach Norden, um mehr rauszufinden.' ,,Warum nach Norden?'' fragte Thot. Seine gebogene Nase befand sich plötzlich dicht vor Kebis Schnabel, der verzweifelt daran entlang schielte. 

,,Lass' gut sein Thot!''sagte Schu, nachdem er den Falken, der ihm von der Schulter gerutscht war, behutsam aufgefangen hatte. Er warf ihn in die Luft und Kebi nahm dankbar den Windhauch, den der Grossvater ihm unter die Flügel wehte und flog dem Rhein folgend in die Nacht.

,,Warum hast du ihn so leicht davon kommen lassen?'' fragte Thot. ,,Er weiss offensichtlich eine ganze Menge!'' Schu schaute den Rhein runter. ,,Mein Gefühl sagt mir, dass wir uns beeilen, statt reden sollten. Wir müssen noch ein letztes Schiff besuchen. Und wenn du mich fragst, ich bin froh, wenn ich wieder im Schutz des Hauses bin.'' Re und auch Thot nickten nur. Sie machten sich auf den Weg und schlängelten sich über den schmalen Steg unterhalb des troi Roi entlang und kamen ungesehen an die Tür der Predigerkirche.

Amélie fand, die Kirche sähe aus, als würde das Krankenhausgebäude neben ihr sie gleich mit einem letzten Schubs über die Strasse in den Rhein befördern. Die Kirche wirkte kleiner als ihr Kolleginnen.

Sie betraten die Predigerkirche. Die sieht am meisten wie ein Schiff aus, fand Amélie. Der vordere Teil war ganz schlicht, weiss und hoch. Das Mittelschiff hatte hoch oben eine Reihe kleine, runde Fenster. Die Seitenschiffe hatten unterschiedliche Fenster, die rechte, südliche Seite hatte höhere Fenster. Von dort fiel Licht von der Strasse herein. Die linke Seite der Kirche lag im Dunkeln.

Im weissen Lettner befand sich in der Mitte ein schmaler Durchgang. Der Altarraum dahinter schien eine Ewigkeit weit weg zu sein, unerreichbar für die Laien.

Es gab keine Kirchenbänke, sondern mehrere Reihen Stühle. Dadurch wirkte der vordere Bereich noch schlichter und leerer. Amélie drehte sich und schaute zu der imposanten Orgel hinauf, die über dem Eingang auf einer hölzernen Empore thronte. ,,Ohlala, ein echter Silbermann!'' sagte Thot. 

Sie blickten sich um, die Kirche schien leer zu sein. Schu und Tefnut strichen durch das Seitenschiff, während Re, Thot und Amélie langsam den Mittelgang vorwärtsgingen. In der Mitte des Laienteils blieben sie stehen und lauschten. ein schwaches, kratzendes und schabendes Geräusch war zu hören. Unregelmässig.  

-'Ich glaube im ALtarraum ist ein Licht!' meinte Amélie. Instinktiv wurden sie noch leiser. Schritt für Schritt bewegte sich die Gruppe bis unter den Lettner zu der Pforte aus Eisengitter, die den Altarraum absperrte.

Tefnut war die erste, die auf leisen Pfoten durch das Gitter glitt. Die anderen folgten ihr. Die Götter bewegen sich einfach durch das Gitter, als wäre es nicht da, dachte Amélie. Sie hatte sich, trotz der langen Nacht, noch nicht daran gewöhnt. 

Hier unter dem dunklen Lettern fiel ihr zum ersten mal ins Auge, dass die Götter auch eine Art Lebensleib hatten. Dieser leuchtete zart und schien aus einer Myriade von winzigen Lichtpünktchen zu bestehen. Thot ging direkt vor Amélie durch das Gitter. Plötzlich schien für ein Moment die Zeit still zu stehen\dots

Amélie stutzte, als ihr bewusst wurde, was sie sah. Es waren nicht nur Lichtpunkte um Thot herum, sondern Thot selbst, seine Kleider, aber auch die eisenstäbe des Gitters waren aus Punkten zusammen gesetzt! Als Thot sich weiter bewegte, glitten die Teilchen, die zu Thot gehörten mühelos an den Teilen des Gitters vorbei. Für einen kurzen Augenblick, waren Thot und das Eisengitter eine gemeinsame Anzahl von Teilchen. Es war unmöglich auseinander zu halten. Thot und Gitter mischten sich und trennten sich. Der einzige Unterschied zwischen Thot und dem Eisengitter war der, dass Thot sich bewegte und sich dadurch wieder vom Gitter trennte und das Gitterteilchen an ihrem Platz blieben.

Amélie war fasziniert und klappte ihren Mund erst zu, als Thot auf der anderen Seite des Gitters ihr zuzwinkerte und sich dann rechts in den Altarraum bewegte. Amélie bekam spürte wie Schu durch an ihrem Lebensleib streifte. -'Wie?' Schu blickte sich mit seinen blauen Augen um und begann ,,All you need is love, dadamdada daa!'' zu singen, während er Thot folgte. Amélie hörte Re, der gleich nach Tefnut durch das Gitter geschlüpft war, fröhlich kichern.

-,What to\dots !' Es war plötzlich totenstill. Kein schbenden Geräusch, kein Mucks. Amélie huschte ohne weiter auf das Gitter zu achten durch den Durchgang. 

Die Götter waren stehengeblieben. Rechts, gleich neben dem Lettner, ganz versteckt stand ein Tisch. An diesem sass ein hochgewachsener Mönch. Er trug eine weisse Kutte und einen schwarzen Umhang. Er sah aus wie ein Pinguin, ein ernster Pinguin. Amélie bemerkte die gespitzte Feder, die der Mönch in der rechten Hand in der Luft hielt. -'Das kratzende Geräusch!' 

Der Pinguin-Mönch musterte seine Besucher mit zusammengekniffenen Augen. Er erhob sich steif und blieb hinter seinem Schreibtisch stehen. Die Hände hatte er in die Ärmel seiner Kutte geschoben. Er verneigte sich leicht und erhob sich dann wieder zu seiner vollen Grösse.
,, Salve Magi!'' sagte er leise. ,,Dei!'' antwortete Thot. Der Dominikaner bekreuzigte sich und schien gehofft zu haben, sie hätten sich damit allesamt in Luft aufgelöst. Amélie fühlte ein prickeln, die Kälte des Mönches setzte sich an ihr fest. Tefnut liess miaute. Re stand gelassen, aber wachsam. ,,Was kann ich für Euch tun?'' fragte der Dominikaner.

,,Wir möchten einen Blick in Eure schöne Kirche werfen,'' sagte Re. Thot berührte Amélie leicht. Sie war müde und feindselige Mönch ging ihr auf die Nerven. Es war still, keiner rührte sich.

Der Mönch hatte sich wieder gesetzt und schrieb wieder. Die weisse Feder in seiner Hand kratzte und schabte. Die Spitze der Feder schob sich über das Papier und hinterliess schwarze Buchstaben, während das andere Ende der Feder in der Lift neben dem Gesicht des Mönches kreiste. Wie auf dem Papier hinterliess die Feder in der Luft Formen. Es waren kleine Figuren umgeben von Mustern, Strukturen. Die Figuren stiegen auf und dehnten sich dabei aus. 

-Wie Rauch, dachte Amélie. Sie erschrak. Sie konnte in den Kopf des Mönches hinein sehen. Er hatte sich, wie Thot vorher, in seine kleinsten Teilchen verwandelt, ein Bild aus Milliarden Pixel. Im Inneren des Kopfes waren die gleichen Gestalten zu sehen, die begonnen hatten in der Kirche zu schweben. Je eifriger der Dominikaner schrieb, umso mehr Gestalten wurden aus seinem Kopf durch den Arm in die Feder gesogen und durch die Buchstaben befreit.

,,So ein Heuchler!'' sagte Schu. ,,Geh hin, sehr es Dir an!'' Amélie näherte sich von hinten dem Mönch, der seine nächtlichen Besucher stoisch ignorierte. Amélie erkannte plötzlich, um was für Gestalten es sich handelte. Es waren alles Frauen! Sie musste kichern. Der Kopf des Mönches war zum bersten angefüllt mit verschiedensten Frauen.

Viele waren nackt. Es gab junge Frauen und ältere mit breiten Hüften und üppigen Formen. Einige trugen lange Kleider, wie es im Mittelalter üblich war, einige hatten nur ihren Rock an. 

Amélie wurde magisch angesogen von den Frauen. Einige beteten, andere brauten in einem Kessel über dem Feuer und rührten mit einem grossen Holzlöffel in dem Sud. Einige schauten sich misstrauischum, als ob sie etwas im Schilde führten. 

Am gruseligsten war die alten Frauen. Ihnen hingen die grauen Zotteln wirr im Gesicht. Die Augen schielten in verschiedene Richtungen und aus den aufgerissenen Mündern ragte ein einzelner Zahn. Sie hatten alle krumme Nasen und lallten und kreischten wirres Zeug. 

-'Hexen!' Amélie blickte zu Thot. -'Sicher?' fragte der zurück. Amélie konzentrierte sich wieder auf die Gestalten im Kopf. Und dann auf die, welche oben aus der Schreibfeder schwebten.

Der Altarraum war angefüllt mit den Frauen. Je mehr es wurden, desto wilder wurden sie, Amélie konnte auch Teufel zwischen ihnen erkennen. Die einen versuchten holde, unschuldige, betende Jungfrauen zu bezirzen, sobald sie mit dem Beten aufhörten, die anderen Teufel tanzten mit den reiferen Frauen herum. -'Die Frauen \dots mit den Teufeln! Wuuääh!' 

Der dämmrige Kirchenraum hatte sich in einen Hexensabbat verwandelt. Es tobte, schrie, lamentiete und kreischte. Mittendrin standen die ägyptischen Götter und blickten sich um. Allein im Lichtschein der Kerze zwischen dem Gesicht des Mönches und dem Papier war Leere. Der grosse Mann war tiefgebeugt, seine Zungenspitze klemmte zwischen seinen Lippen, er würdigte dem Geschehen um ihn herum keines Blickes.

Das Licht veränderte sich. Die Reisenden brauchten einen Augenblick, bis sie begriffen, woher der neue Lichtschein kam. Am Saum der Kutte und an den Schuhspitzen, die darunter hervor schauten, züngelten Flammen. Sie störten den Mönch nicht. Einzig mit der Zeit rötete sich die Haut an seinem Hals. Die Röte stieg auf bis sein Kopf zu glühen schien. 

Als das Licht im Raum immer flackerte, sah Amélie über die tobende Menge hinweg, inzwischen flog die eine, oder andere der Frauen johlend auf einem Besen durch die Luft, durch die Fenster einen hellen Schein. dann war sie wieder abgelenkt, denn Amélie konnte auch Frauen entdecken, die in ihren Töpfen durch den Raum flogen. Vor allem die älteren schienen diesen Fluggerät zu bevorzugen. 

-'Feuer!' Amélie deutete auf die Fensterscheiben der Apsis. Wie eine Walze durchbrachen die Flammen die Kirchenmauer, die Frauen schrien. Jede einzelne von ihnen fing Feuer. Zu Amélies entsetzten, brannten alle Gestalten, bis auf die Götter. Auch der Dominikaner. Der schrieb weiter, Wort für Wort. Im Gegensatz zu den Frauen verbrannte er nicht. Die Frauen wanden sich in den Flammen, schrien, rangen die Hände, krümmten sich, wenn die Hitze genug von ihnen gefressen hatte und verendeten.

Die einen schwebten geläutert zur Decke und weiter durch das Dach, die anderen wurden von Klauen und Krallen in den Boden gezerrt. Die Flammen um den Mönch wurden kleiner.

Eine Frau war übrig geblieben. Sie hatte ein weisses Gewand und blondes, langes Haar, das in zwei Zöpfen über die Brust bis auf ihre Hüften fiel. Sie war in einen weissen Umhang mit Kapuze gehüllt und glich einer weissen Madonna. Der Mönch hob verwundert den Kopf. Die Feder schwebte über dem Papier. 

Sie schwebte in der Luft und drehte sich langsam um sich selbst. Als sie nach der ersten halben Drehung wieder von der Seite sichtbar wurde, konnten alle sehen, dass sie ihr weisses Gewand gehoben hatte. -'?' Amélie wandte den Kopf kurz zu Thot um zu sehen, wie er auf die Frau reagierte. Er lächelte. 

Sie war wunderschön. Wieder kehrte sie ihnen den Rücken zu, als sie  ihr Gesicht sichtbar wurde, war sie gealtert. Ihre Haare waren offen und feuerrot. Der Schleier strahlte in kräftigen Zinnober, ihr Gewand war verschwunden. Ihre Füsse, die auf dem langen Saum des zinnoberfarbenen Umhanges standen, schienen in Blut zu baden.

Der Mönch starrte sie mit offenem Mund an. Die Feder kratzte kurz, als sie auf das Papier aufsetzte. 

Die Frau drehte sich weiter, der Umhang wurde schwarz. Als sie sich ihnen wieder zuwandte war ihr Gesicht mit dem schwarzen Umhang bedeckt. Ihre Hände hielten das Gewandt hoch, doch darunter war es tief schwarz.

Der Mönch stiess einen Schrei aus, die Feder knackte als sie zerbrach. Die Frau wurde grösser und grösser, bis sie die Apsis ausfüllte. Sie hatte angefangen zu brennen und zeigte drohend auf den Mönch, der zu Boden fiel und panisch Schrie. Er zuckte und wand sich. 

,,Hinweg, Satan!'' rief der Mönch. Er hob abwehrend den Arm. Die Alte drehte sich ein letztes mal und erstrahlte im Licht. -'Maria!' dachte Amélie erstaunt. Die Frau lächelte mild, sie trug das rote Gewand der Maria und den blauen Sternenmantel. Sie lächelte sanft. Re und Thot klatschten begeistert. Wo war Schu geblieben? Und Tefnut?

,,Heilige Mutter Gottes!'' entfuhr es dem Mönch. er faltete die Hände und begann zu beten. Es gab einen Knall, der von den Kirchenmauern widerhallte. Die riesige Maria verschwand und eine nackte, gut gebaute, rothaarige Tefnut plumpste von weit oben herunter.  Unten stand Schu und fing seine Gemahlin auf. Sie schlang ihre Arme um seinen Hals und die beiden küssten sich wie beim ersten Date. Es platschte neben ihnen. Der Dominikaner war in Ohnmacht gefallen und lag ausgestreckt auf dem Rücken. Tefnut schwang sich von Schus Armen. Sie verwandelte sich in ihre Katzengestalt und tabte hoch erhobenen Schwanzes aus dem Altarraum.

Die vier anderen blickten ihr hinterher. Schu nahm einen tiefen Zug und blies den Rauch auf den bewusstlosen Mönch. Der begann zu husten und erwachte. ,,Was, was?'' ,,Soviel Leid, Bruder Nider, unschuldiges Blut.'' Re seufzte und folgte seinem Sohn durch den Torbogen. Der Mönch sass benommen am Boden. ,,Ich wollte die Menschen beschützten. Auch die Frauen!'' sagte er. ,,Ich weiss, Bruder.'' antwortete Thot. ,,Vielleicht hätte es funktioniert, wenn dein Formicarius nicht in falsche Hände geraten wäre.''  Der Dominikaner liess den Kopf hängen. Er hob die Hände und liess sie in den Schoss fallen. ,,Was soll ich denn machen!'' fragte er. ,,Weinen!'' antwortete Thot und drückte ihn leicht an der Schulter bevor er Amélie vor sich her zum Torbogen schob.

-'Die ganzen Frauen?' fragte sie. ,,Wer hat es erfunden?'' fragte Thot zurück. Amélie überlegte. -'Der Mönch?' Thot schwenkte die Arme: ,,Bravo!'' und klatschte. Amélie schien nicht zufrieden. -'Aber die Hexenverbrennungen! Die hat es doch in echt gegeben!' dachte sie. ,,Ja!'' antwortete Thot. ,,Und?'', -'Was? Und?' fragte Amélie. ,,Uuuund!'' Thot wirkte kurz müde. -'Du meinst, die Frauen sind gestorben, weil der Mönch da sein Buch erfunden hat?' Thot sagte: ,,Im weiten Sinne, ja!''

Sie marschierten durch den Laienraum auf die Empore mit der gewaltigen Orgel zu unter der sich der Ausgang befand. Die anderen warteten. ,,Kinder, es war eine lange Nacht! Jetzt bin ich froh, wenn ich in meinem Sessel sitzen und meinen Tee geniessen kann!'' sagte Re. Doch es sollte noch eine Weile dauern, bis dem Sonnengott der Wunsch erfüllt wurde. 

Als sie vor die Tür der Predigerkirche traten, sahen sie über eine Mauer auf den Rhein! Aber bevor Amélie genau begriff, wo sie war wurde sie mit den anderen in einen durchsichtigen Schlauch gezogen. So muss es sich in einem Staubsaugerrohr anfühlen, dachte Amélie, bevor ihr hören und sehen verging. Sie schwebten nicht in der Luft, sondern waren dazu gezwungen wie Marionetten den Weg vom Blauen Haus den Rheinsprung hinunter, über die Schifflände und durch den Blumenrain am Les troi roi vorbei in rasender Geschwindigkeit bis vor die Tür der Predigerkirche zu laufen. 

Amélie und Schu prallten mit dem Rücken an die Kirchentür. Sie drückten sich dagegen und glitten durch die Tür. -'Wo ist Re?' fragte Amélie. -'Stimmt!' antwortete Schu -'Und wo ist Thot?' Sie blickten sich um. Amélie bemerkte den Schein der Kerze im ALtarraum, den sie nur erkannte, weil sie ihn dort vermutete.

Tefnut schleckte sich die Pfote. Sie war dreckig von der erzwungenen Wanderung. -'Ich denke Re hatte kein Problem mit der verfluchten Zeit und Thot\dots ?' Eindrucksvoll zuckte die Katze mit den pelzigen Schultern. ,,Schätze, wir warten hier in der Kirche mal ab\dots '' Schu setzte sich auf einen der Stühle und legte die Füsse hoch auf einen zweiten. Tefnut sprang auf seinen Schoss und schlief ein. Dann hänge ich halt mal ab, dachte Amélie. Für sie war es anstrengender auf einem Stuhl zu sitzen, als den Lebensleib frei in der Luft zu halten. ,'Schade!' dachte Amélie laut, bei Peter in der Kirche wäre es jetzt lustiger!' Schu liess einen Rauchring aus seinem Mund gleiten -'Jep!'  

Amélie bemerkte den Blick, den Tefnut und Schu wechselten. Sie sahen besorgt aus\dots

\section*{11}
\addcontentsline{toc}{section}{11}

Thot hatte sich an der Tür zum blauen Haus festgehalten. Seit sie Kebi getroffen hatten, war er auf der Hut gewesen. Die Zeitschleife hatte ihn daher nicht überrascht. Er schlüpfte ins Haus. In der Eingangshalle atmete er tief durch.

Bilanz: Kirchenschiffe hatten sie alle besucht. Das war gut. Re war draussen, das war höchstens schlecht für alle die der Mission schaden wollten. Thot vermutete, der Sonnengott würde als erstes zum Anleger gehen und die Barke und Osiris in Empfang nehmen. Auch das war gut. Amélie war bei Schu und Tefnut. Das war gut. Aber Schu und Tefnut waren nicht da, sondern durch den Zeitfluch wahrscheinlich in die Predigerkirche geschleudert worden und das war schlecht!

Amélie blieb nicht viel Zeit! Sie musste sich vor Sonnenaufgang wieder mit ihrem Körper verbinden\dots 

Thot eilte durch den Gang zur Kellertür und hetzte durch das hohe Kellergewölbe. Er schlüpfte in den Geheimgang und traf, wie er es erwartete Re am Bootssteg.

,,Amélie und Osiris!'' sagte Thot. ,,Ich weiss, mein Sohn!'' Re legte Thot die Hand auf die Schulter. Sie blickten auf den trägen Fluss und die majestätische Barke, die aus der Flussmitte langsam auf sie zuglitt. 

\sterne

Wenn ein später oder in diesem Fall sogar eher früher Passant an der Mauer der Terrasse vor dem Blauen Haus stehengeblieben wäre und auf den Rhein geschaut hätte, hätte er ein interessantes Schauspiel beobachten können.

Zwei Männer in normaler Strassenkleidung standen auf einem Steg. Und auf dem Rhein schwamm eine ägyptische, grosse, goldene Barke. Die Barke war mit eine grossen Schar Gestalten beladen, die nicht alle menschlich aussahen. Es war eine Besatzung aus blassdurchsichtigen Menschen, die Kleidung aus den verschiedensten Epochen trugen und alemannische Lieder sangen und Mumien, und Menschen die halb Mumie, halb Mensch waren. Dazwischen gab es Wesen, die die Menschen überragten und deren menschliche Gestalt sich mit denen von Tieren mischte. 

Eine dieser Gestalten, die grösste, befand sich unter einem Baldachin. Sie trug einen Widderkopf aus Gold. Diese Gestalt, die von zwei grossen und kräftigen Männern und einer kleinen, zarten Frau begleitet wurde, schlängelte sich durch die bunte Mannschaft und wartete am Bug.

Die Barke drehte bei und der Mann am Ufer, der eine Kapitänsjoope trug stieg auf das Schiff. Der widderköpfige umarmte ihn. Als sich die beiden voneinander lösten, trug der Mann mit der Kapitänsjoope den Widderkopf und wirkte viel grösser als vorher. Der andere hatte nun eine  menschliche Gestalt. Er trug einen Rucksack. Mit dem Rucksack, den beiden nackten und der Frau stieg der Mann an Land, begrüsste den zweiten, der am Ufer gewartet hatte und dann huschte die Gruppe durch eine verborgene Tür unterhalb der Terrasse aus dem Blick des Passanten. Bevor dieser sich wundern könnte, käme jedoch der Mann im Mantel wieder zum Vorschein mit dem Rucksack, der nun deutlich leer war und schlaff.

Der Mann sprang an Deck der Barke und diese legte wieder ab. Es schien eilig zu sein, denn die Mannschaft ruderte, obwohl die Barke flussabwärts glitt. Der Mann mit dem goldenen Widderkopf war, nachdem ihn eine runde Frau umarmt und geherzt hatte, unter den Baldachin gestanden.

Die Barke verschwand so schnell unter der Mittlere Brücke, dass der Passant sich die Augen reiben würde und glauben würde, er hätte geträumt.

\sterne

Während die Barke Thot mit dem leeren Imiut an die Schifflände brachte, um Amélie zu retten, Begaben sich Horus, Isis, Geb und Hans durch den Keller schnurstraks in Osiris Zimmer. Der Herr der Unterwelt lag auf den Armen seines Sohnes, der ihn vorsichtig und schnell die Treppen hinauf trug. Während Geb vorausging und Isis ihm mit einer Taschenlampe folgte, sicherte Hans den Rücke des Falkengottes. Der Hahn, der auf Hansens Schulter gesessen hatte, faltete seine Flügel auf und flog auf das hohe Dach. Er liess sich auf einem der vielen Schornsteine nieder.

,,Mir ist nicht ganz wohl, so allein in dem grossen Haus,'' sagte Isis. ,,Muescht dir khei Sorge mache,'' sagte Hans. ,,Min Hund, dr passt scho uf, dass niemens iine kha!'' Isis verzog den Mund, sagte aber nichts. Sie stellte sich einen Seth vor, der durch die Gänge des Hauses lief und an dessen Wade ein Dackel hing und knurrte. Hans lächelte: ,,Weisch, dr Waldi, dr kha au anderscht!'' Isis sagte nichts, als sie in den schmalen Gang im linken Seitenflügel einbogen, legte sie ihr Schlangendiadem auf den Boden. Kaum berührte es diesen wurde die goldene Schlange mit den smaragdgrünen Augen lebendig und glitt unter eine Kommode. Sie zischte. ,,Nit schlecht!'' sagte Hans und grinste. Er blieb als einziger vor der Tür. Dort machte er es sich auf dem Boden gemütlich und steckte sich seine Pfeife an. ,,Joooh, du kleens Ding, pspsps'' flüsterte er. Nach einiger Zeit kam die Schlange und rollte sich neben ihm zusammen.

\section*{12}
\addcontentsline{toc}{section}{12}

,,sind die Kinder in Sicherheit?'' fragte Hathor besorgt. Die goldenen Kuhhörner mit der rotgoldenen Sonnenscheibe auf ihrem Kopf zitterten vor Anspannung, als sie sich an Thot wendete. Der legte beruhigend seine Hand auf ihre Schulter. ,,Ja, edle Mutter, Osiris kann zufrieden sein. Jetzt müssen wir schleunigts noch Amélie retten,'' antwortete er. Hathor blickte zu dem hageren Gesicht auf und erschrack, denn Thot war ausnehmend blass. ,,Was ist passiert?'' fragte die Göttermutter. Thot seufzte, ihr konnte man natürlich nichts verheimlichen\footnote{schliesslich war Hathor nicht eine Mutter, sondern DIE Mutter. Es gab nichts, was sie nicht schon gesehen hätte. Und nicht zu vergessen, sie war auch Oma, Uroma und Ururoma!} ,,Es hat eine Zeitumkehrung gegeben. Amélie ist mit Schu und Tefnut in die Predigerkirche zurückgeschleudert worden!'' hathor sah an den Himmel, so wie andere auf ihren Arm schauten. ,,Verflixt! Es ist knapp! Die Sonne geht jeden Moment auf!'' rief sie.

In diesem Moment rammte die Barke den Anleger unterhalb des Hotels hinter der Mittleren Brücke. Thot wollte sofort losstürzen. ,,Halt!'' hathor hielt ihn am Mantel zurück:,, Du nimmst Isfet mit!'' ,,Was?'' noch ehe der entsetzte Thot etwas sagen konnte, hatte Hathor ein ringelstrumpfiges Bein unter der Bank vorgezogen. ,,He, Du! Lass mich!'' wie eine Furie kam der Rest der jungen Chaosgöttin auf die Füsse. ,,Oh, Mama!'' Isfet schaute unschuldig. ,,Junge Dame, du hilfst deinem Onkel Thot Amélie zu retten!'' sagte Hathor. ,,Ja, Lauft zu!'' herrschte Hathor die beiden verduzten Gesichter an und automatisch machten sich die Beine auf den Weg, bevor Thot und Isfet ,Amen!' sagen konnten\dots

,,Warum denn ich?'' fragte Isfet, während sie neben Thot die schmalen Rheinweg zum Totentanz hinaufhetzte. ,,Schnell! Ich weiss es nicht!'' rief Thot. Plötzlich blieben sie stehen. Sie hatten den kleinen Platz vom Totentanz erreicht und im ersten zarten Dämmerschein sahen sie den Schatten.

Als würde eine riesige Wolke über den Himmel streifen, zog ein Schatten über den Platz. Er kam aus dem Norden und überzog Häuser uns Strassen. Ein Schrei, direkt über ihnen, liess sie zusammenfahren. Sie hörten ein Rauschen. ;,Kebi!''schrie Isfet, den Kopf in den Nacken gelegt. Thot breitete seinen Mantel aus und der Falke stürzte hinein.

,,Oh, nein! Kebi?'' Isfet strich behutsam über die Federn. Der Falke hatte die Augen geschlossen, ein winziger Blutstropfen bildete sich an der Schnabelspitze. Das Herz hämmerte wild unter den Brustflaum. Thot öffnete seinen Mantel und schob den Falken vorsichtig in eine grosse Innentasche. ,,Schnell!'' Kurz bevor der wandernde Schatten die Kirchenstufen erreicht hatte, stürmten sie durch die -geschlossene Tür.

Die drei Wartenden in der Kirche erschraken. Tefnut machte einen riesigen Buckel und fauchte, ihre scharfen Krallen schabten über den Boden. Die rote Katze tänzelte rückwärts, ihr Fell und den Schwanz auf dreifache Grösse aufgebläht.Amélie war vor Angst zu einem winzigen Ball zusammengeschrumpft und Schu hatte sie blitzschnell mit seinen Händen eingefangen.

Zwischen den stühlen stand jetzt eine riesige Löwin, sie brüllte. Im Durchgang des Lettners erschien das schlotternde Gespenst des Niger. Es bekreuzigte sich immer wieder, sank auf die Knie und betete laut und inbrünstig.

Sie starrten sich an. Und für einen Moment war es absolut still. Hinter den Fenstern, die von der einsetzenden Dämmerung leicht erhellt gewesen waren, wurde es abrupt dunkler. Obwohl es mucksmäuschenstill war, glaubten sie einen Orkan zu hören, der mit dem Schatten über die Stadt zog.

,,Wir müssen hier raus!'' rief Thot schliesslich. Schu war auf ihn zugegangen und zeigte ihm die Kugel aus Licht in seinen Händen, alles, was von der verängstigten Amélie übrig geblieben war. Tefnut trat mit gewaltigen Pranken an seine Seite und öffnete das Maul. Thot musterte sie scharf. Tefnut starrte aus ihren gelben Löwenaugen zurück. Schu blickte von einem zum anderen. ,,Wir haben keine andere Wahl!'' meinte er und legte die Kugel in Tefnuts Maul. Thots Augen weiteten sich. Die Göttin schloss den gewaltigen Kiefer, während sie den Herrn des Westens fixierte.

Isfet war inzwischen durch die Leihenteil der Kirche geschlendert. Sie hatte das Gespenst des Niger begrüsst, das dies jedoch nicht zu schätzen wusste und schluchzend zusammengebrochen war.

,,Ihr solltet zur Barke zurück!'' sagte Isfet, deren Schwarzen Augen, die wie verschluckte Galaxien schimmerten zu funkeln begannen. ,,Ich denke, ich kann das Ding eine kurze Zeit ablenken.'' Sie grinste, oder bleckte die Zähne.

Thot schauderte, ihm blieb keine andere Wahl\dots, er musste sowohl Tefnut vertrauen, die in ihrer Löwengestalt kein Pardon machte und er musste Isfet \dots , es ging nicht, unmöglich! Er konnte, er durfte Isfet nicht trauen! -Hathor!' rief er in Gedanken verzweifelt. -,Vergiess niemals, Thot! Sie ist auch meine Tochter! Und die Tochter des Re!' hörte er die Stimme der Mutterkuh durch die Kirche schallen.

Ein zarter Lichtkegel erschien auf dem Boden und Tefnut sprang. Mit gewaltigen Sätzen, direkt durch die Seitenwand der Kirche, die dem Rhein am nächsten war. ,,Hu-Rah!'' brüllten die beiden Götter und stürzten der Löwin hinterher. 

Isfet blieb in der Kirche zurück. Sie schlenderte auf die Gestalt unter dem Lettner zu. ,,Tue mir nichts! Geh weg, du Teufelin, Du Satansweib!'' kreischte das Gespenst panisch. Über ihm erhob sich eine Stichflamme. 

Aus den Flammen ragten die Pfähle, an denen die Leiber derer zuckten und brannten, die zum Tod auf dem Scheiterhaufen verurteilt waren. ,,Weiter so, mein Schatz!'' In den schwarzen Augen der Göttin glitzerte es gierig. Ihr Gesicht war zu einer Fratze verzerrt. Sie grinste und fletschte gleichzeitig die Zähne. Ihre Haut war aschfahl geworden, wie die des Gespenstes.

Aus dem Kopf des Niger-Geistes stiegen unablässig Buchstaben, die wie Ameisen emsig aufeinander folgten. Die Ameisenbuchstaben strebten alle auf ein dickes Buch zu. Bevor sie das Buch erreichten begannen sie vereinzelt zu brennen. Die brennenden  Ameisen fütterten die Flammen, die aus dem Buch schlugen und zu den Scheiterhaufen wurden weiter und weiter.

Die Flammen wuchsen, wuchsen, das brausen und tosen, knacken und krachen der Flammen zerschnitten das Kirchendach und die Schreie der sterbenden und gemarterten Seelen brach in den Himmel. Isfet kreischte und jauchzte. Die Kleider waren von ihr abgefallen. Ihre Haut war grau und fleckig. von dem kecken, punkigen Mädchen war nichts übrig, ausser ein geringelter Strumpf\dots

Sobald das Kirchendach durchstossen war, hielt was immer da draussen war, den Atem an. Der stille Orkan, der geballte Schatten, der sich bis zur Mittleren Brücke vorgefressen hatte, hielt inne und drehte sich wie ein gigantischer Wirbel um. Für einen kurzen Moment zog sich die Dunkelheit über der Apsis der Predigerkirche um die daraus schiessenden Flammen zusammen. Wie ein schwarzer Panther, der der geruch einer saftigen Beute wittert. Dreimal umkreiste der Schatten die Flammen und die Göttin der Zerstörung, die auf dem Dach der Kirche stand und brüllte.

\sterne

Thot fiel ausser Atem auf die Barke, neben ihm krachte Schu auf den Boden. Tefnuts landung war eleganter, aber auch die Löwin lag gänzlich erschöpft auf den Planken. Behutsam packte Hathor den mächtigen Unterkiefer ihrer Enkelin, die vor Erschöpfung und Aufregung ein leichtes Grollen aus der Kehle aufsteigen liess. ,,Bist doch meine Beste, meine Stärkste\dots!'' sanft kraulte sie die Kehle und die gewaltige Löwin öffnete die Maul und liess den kleinen Lichtball, der weiter geschrumpft war, in die Hand ihrer Oma fallen.

Wie aus dem nichts war Horus erschienen. Der Falke liess sich auf der Schulter seiner Ururgrossmutter nieder und diese hängte ihm ein kleines Tässchen mit Amélies Lichtball um den Hals.  Die Barke, die den Rückweg eingeschlagen hatte, glitt unter der Mittleren Brücke durch. Sie küsste dem Falken den Schnabel und warf ihn in die Luft.
In dem Moment hörten sie Isfets Schrei: ,,Hu-Rah!'' Der Falke flog so schnell er konnte in die Höhe zu den über dem Rheinsprung thronenden Häuser hinauf. 

Auf der Barke hielten alle den Atem an. Nur der Sonnengott, der wie es sich für den Sonnenaufgang gehörte in seiner mächtigen Skarabengestalt schwarz-bläulich schillerte, hielt den Sonnenball mit seinen mächtigen Hinterbeinen am Horizont fest. Der riesige Käfer zitterte, denn die Sonnenscheibe, die halb über den Horizont aufgestiegen war, bebte unter ihm. 

Im ersten Stock des blauen Hauses öffnete sich ein Fenster und Hans erschien darin. Er streckte seine mächtigen Pranken aus und fing den Falken auf. Gleichzeitig schwang der schwarze Schatten wie eine Peitsche vom Dach der Predigerkirche in die Luft und warf sich mit einem gewaltigen Brausen in die Luft und dann auf den Rheinsprung. Das letzte was Hans spürte, war sein Hahn, der wie eine schwarze Rakete vom Dach zu dem einzigen offenen Fenster geflüchtet war und seinem Meister zu ihrer beider Bedauern mit den Sporen voran im Gesicht gelandet war. Hans taumelte zurück und Maat, die dem wilden Mann assistiert hatte schlug mit glänzenden Augen das Fenster zu.

Ein kleines Stück von dem Schatten fiel auf die Fensterbank. Maat sah es sich an. Sie berührte den dunklen Fleck, der wie ein Funken Dunkelheit aussah mit der Fingerspitze. ,,Nei! Nitt!'' rief Hans, er hatte mehrere tiefe Risse im Gesicht. Der Schatten verschwand\dots Der wilde Mann beugte sich drohend über die Göttin der Ordnung, die in ihrer Feriengestalt gemütlich unter des Wilden Mannes Arm spazieren konnte, ohne ihre Frisur zu zerstören. Sie beugte sich zurück und so standen sie Nase an Nase. Maat war sich im Nachhinein nicht sicher, was gefährlicher war, der  seltsame Schatten, oder der erschöpfte, besorgte wilde Mann, dem das Blut über die Stirn lief.

\sterne

Dieser Morgen sollte dem einen oder anderen Basler, wegen seiner eindrücklichen Dämmerung in Erinnerung bleiben. Die blutrote Sonne schien für eine Zeit halb über dem Horizont hängen zu bleiben ohne sich zu bewegen und dann plötzlich stand sie schon ein kleines Stück darüber\dots Als ob sie an einem Gummiband befestigt gewesen war\dots 

Die alten Ägypter, die das Schauspiel aus mythischer Sicht verfolgt hätten, wären vermutlich vor Angst gestorben, als ihr Cheprie, der sonnengöttliche Skarabäus nämlich, nachdem er die Sonnen für ein Äon am Horizont festgehalten hatte, plötzlich von der bockenden Sonne abgeworfen wurde und hinter den Horizont purzelte. ,,Kinder, ich bin froh, dass meine Gläubigen in alle Winde verstreut sind,'' erklärte Re später: ,,Wenn mich jemand so zerzaust unter dem Horizont vorklettern gesehen hätte, dann hätte der Thot tagelang neue Mythen erfinden müssen...''

\sterne

Müde stieg Thot die Treppe zu seinem achteckigen Zimmer hinauf. Sie hatten Glück gehabt, verdammt Glück\dots 

,Wer hätte gedacht, dass wir Götter eines Tages darauf zurückgreifen müssen', dachte er. Er spürte den Luftzug schon bevor er die Tür zu seinem Zimmer öffnete. ,,Beim Fluch des Pharao!'' rief Thot und riss die Tür auf. 

Ein scharfer Wind blies ihm entgegen, das Fenster war sperrangelweit offen. ,,Was ist da los?'' Thot stürmte zum Fenster und stolperte über etwas grosses, das verborgen hinter dem Schreibtisch am Boden lag. Er konnte sich mit den Händen an der Fensterbank auffangen, bevor sein Kinn damit Bekanntschaft machen wollte.

Er zog sich hoch und knallte das Fenster zu. Dann drehte er sich zu dem grossen Bündel unter seinem Schreibtisch um: ,,Berta! Bei Osiris, was ist geschehen?''

Er berührte ihre Wange:,, Eiskalt!'' Er rannte durch das Zimmer und riss die Tür auf: ,,Hathor! Wibrandis! Hans! Re!'' Dann eilte er zum Regal und füllte ein Glas mit dem besten, schottischen Whisky, den er finden konnte.

Er kniete neben der Göttin nieder und hielt ihr das Glas unter die blasse Nase. ,,Berta!'' flüsterte er. Die Nüstern der Göttin begannen sachte zu zittern. Sie schlug die Augen auf und setzte sich ruckartig auf, dabei nahm sie Thot mit einer fliessenden Bewegung das Glas aus der Hand und kippte die goldige, ölige Flüssigkeit in den Rachen.

Jeder andere hätte sich den Kopf angeschlagen, dachte Thot als er auf die kleine Person blickte, die mit mit gegrätschten Beinen aufrecht unter dem Schreibtisch sass, umbauscht von ihren vielen Reiseröcken. Nicht einmal der Knoten aus schneeweissen Haaren berührte die Tischplatte.

\begin{Large}
,,Bööörps!''
\end{Large} Berta wischte sich mit dem Handrücken über den Mund. Dann flüsterte sie:,, Ich hab es geahnt! Aber ich konnte sie nicht aufhalten!'' ,,Wen? Wen konntest Du nicht aufhalten?'' fragte Thot. ,,Amélies Eltern!'' antwortete sie.
