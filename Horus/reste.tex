%\Restecke
\chapter*{Reste}
\addcontentsline{toc}{chapter}{Reste}

\section*{prolog}
\addcontentsline{toc}{section}{prolog}

Dies ist die Geschichte einer Reisegruppe. Da diese Reisegruppe aus Göttern der alten ägyptischen Pharaonenzeit besteht, die das schöne Basel in der Schweiz des 21. Jahrhunderts besuchen, handelt es sich um eine ungewöhnliche Reise und um ungewöhnliche Ereignisse und Abenteuer. 

Alle Götter lieben es kreativ und innovativ zu sein, deshalb beginnen sie stets mit dem Anfang, der Schöpfung. 

Das All!  Nicht das All, das wir kennen mit all den Lichtern und Blinkesternen, sondern das All vor Urzeiten, jung und neu. Wir sitzen in einem Raumschiff und beobachten, was passiert. Es ist dunkel, wie in Mutters Schoss. Die Panoramascheibe unseres Raumschiffes ist auf Nachsicht eingestellt. Alter Astronautentrick, um die Geburt von Planeten am Anfang allen Daseins und vor der Erschaffung des Lichtes zu beobachten.

Von der Kaffeemaschine tropft ein einzelner Tropfen herab, wir angespannt, zucken zusammen\dots und vor uns taucht majestätisch und langsam ein riesiger Planet auf. Er befindet sich an der Stelle, wo heute unsere Sonne ist und füllt den gesamten Platz zwischen Sonne und Saturn, so gross ist er. 

Langsam dreht er sich und wir bemerken unterschiedlich stark strahlende Felder. Wir sehen ein Zentrum in der Mitte, von einem helleren umgeben und eine abschliessende Hülle darum. Eiförmige Gebilde  bewegen sich in dem Planeten. Wir schauen auf den Zeitmesser, der sich rasend durch die Zeit bewegt und 1000 Jahre in einer Millisekunde vorbeiziehen lässt. Die Zeit wird langsamer und wir schauen wieder raus. Es wird dunkler, es scheint als verlösche der riesige Planet. Nachdem er verschwunden ist, beschleunigt der Zeitmesser erneut. 

Und plötzlich erscheint vor unserem Panoramafenster ein neuer Planet an Stelle des alten. Dieser Planet ist kleiner und gasförmig. Wir können die Wärmesicht abschalten, wir sehen ihn wunderschön durch die Scheibe leuchten, wir tragen Sonnenbrillen, so hell leuchtet der Planet. Er strahlt glühend, leuchtend weiss, bis auf einige rauchig-dunkle Stellen, die aus Gas bestehen und dichter sind als der Rest. An einigen Stellen ballt sich Wärme, wie wir sie auf dem vorherigen Planeten gesehen haben.

Dieser Planet scheint zu atmen. Eine Zeitlang strömt das Licht weit in das All hinein. Angesogen von einer unsichtbaren Kraft und wird feiner und heller. Dann zieht es sich in das Zentrum zurück, dabei wird es dunkler, dichter, wie rauchige Luft. Kleine Gebilde kristallisieren sich darin, noch können wir sie nicht erkennen. Später werden sie die Sterne des Tierkreises bilden.

Wir haben die Zeit vergessen. Ein Blick auf unsere Zeitanzeige verrät uns, es sind Millionen Jahre vergangen. Wir schauen auf den Planeten und dieser ist wiederum in der Dunkelheit verschwunden.

Wir haben kurz Zeit uns einen Kaffee zu kochen und ein Schnittchen zu machen, da taucht an der Stelle ein neuer Planet auf. Er reicht von der Sonne bis zum Mars. Er ist fester als sein Vorgänger und schillert in allen Regenbogenfarben. Durch den Farbschleier sehen wir Nebel und durch den Nebel sehen wir zum ersten mal so etwas wie Leben. Eine grünliche Masse, wie gekochter Spinat, die lebendig ist und ab und zu ein "`Blubb"' von sich gibt\dots

Wir haben nicht viel Zeit dieses farbenprächtige Bild zu geniessen, denn die Masse des Planeten und die ganze Umgebung drumherum geraten in Aufruhr. Unsichtbare Kräfte reissen den Planeten auseinander. Während gleichzeitig das Universum von unsichtbaren Riesenhänden zusammengepresst wird. Zeitschlaufen öffnen sich und verschlucken Materie in dunklen Röhren, um sie an anderer Stelle auszuspucken. Unsichtbare, unglaublich starke Kräfte kämpfen gegeneinander. Trümmer fliegen herum, Feuerstürme toben, Luftlawinen wälzen sich durch den Raum.

Unser kleines Raumschiff ist zu dicht an dem Geschehen dran, es wird gepackt und geschüttelt. Wir purzeln durcheinander, es gibt eine riesige Kaffeesauerei und die lecker Schnittchen kleben an den Wänden. Draussen sehen wir Blitze, Gaswolken. Druckwellen schütteln uns. Ein riesiges Durcheinander findet statt. Der ursprüngliche Planet ist nicht mehr zu sehen.

Würden wir die alten Mysterien kennen, die ganze Ewigkeiten und durch viele Kulturen hindurch weitergegeben wurden, z.B. die Bhagavat Gita, würden wir wissen, in was wir hineingeraten sind: Den berühmten "`Streit im Himmel"'. Berühmt und berüchtigt, weil er die Geburt des Bösen beschreibt, das zu diesem Zeitpunkt nicht böse ist, sondern getrennt und allein.

- Gespenstische Stille-

Wir klopfen uns die Kleider ab, der Lehrling wird geschickt den Wischmop zu holen und die Kaffeepfützen zu entfernen und eine traurige Schinkenschnitte fällt von der Decke herunter.

Draussen hat es sich beruhigt, aber, was für ein Anblick!

An der Stelle des Planeten sehen wir eine Sonne, sie reicht nicht mehr bis zum Mars, dafür befindet sich auf der Marsumlaufbahn ein Mond. Er umkreist die Sonne und. Was auch immer die Turbulenzen verursacht hat, hat einen riesigen Ring an Trümmern hinterlassen, den Asteroidengürtel, den wir später zwischen Mars und Jupiter finden!

Die Sonne leuchtet hell und ist leicht, während sich auf dem Mond die Materie zu Luft und Wasser verdichtet. Auch hier schillern wieder die Regenbögen und der lebendige Spinat macht "`Blubb"'\dots

Ein letzter "Neustart" findet statt, Licht aus, Licht an und vor uns taucht die Erde auf. Sie ist viel grösser, denn es werden sich aus ihr Uranus und Saturn herauslösen, bevor sie sich endgültig verfestigt hat. Als nächster trennt sich der Jupiter und dann der Mars.

Wir können es nun wagen auf das Raumschiff zu verzichten, auch wenn wir bei der jetzigen Erdatmosphäre vermutlich einen Raumanzug nötig hätten, mit unserer heutigen Gestalt.
Ein letzter Blick auf den Zeitmesser, dort ist an Stelle der Zahlen eine Schrift erschienen: Genesis, Tag 1

Die Sonne verlässt die Erde und später, als letztes der Mond. Nachdem es sich die Planeten alle für diese Runde gemütlich gemacht haben, tritt das erste mal gross auf die Bühne in physischer Gestalt: Der Mensch.

Natürlich waren wir Menschen schon vorher da, wir waren auch dabei, als der Wärmeplanet sich im All begann zusammenzuballen, aber wir hatten kein Bewusstsein und nichts womit wir eines hätten haben können. Dieses Bewusstsein brauchte all die vielen Zeitrotationen auf unserem Zeitmesser, Millionen von Jahre und mehrere Äonen von Heimatplaneten, bevor es soweit war.

Und dann kam Atlantis. (Womit unsere Geschichte, der Leser möge den schmählichen, aber einkalkulierten Fauxpas verzeihen, einen Gènréwechsel von Science-Fiction zu Fantasy durchmacht, kommt nicht mehr vor, versprochen.)

\section*{2.2.5}
\addcontentsline{toc}{section}{2.2.5}

Das Krokodil schlug wild um sich. Der mächtige Schwanz platschte auf das Wasser und es drehte sich, die Strömung, die sich auf die engste Stelle des Rheinknies zu bewegte, wurde schneller. Schon trieben sie unter der Wettsteinbrücke durch. Amélie klammerte sich fest und das Krokodil versuchte sie abzuschütteln.

Sie befanden sich unterhalb der Pfalz. Das Krokodil hatte die Orientierung wieder gefunden und tauchte ab. Amélie schwamm und strampelte. Sie bekam plötzlich einen Stock zu fassen. Er ragte von der Leuen-Fähre ins Wasser. Amélie gelang es sich am Stock hochzuziehen, um sich an die Bordwand zu klammern. Sie hatte genug. Sie schaffte es sich zu halten und einen Fuss über die Bordkante zu schieben. 

Bevor sie sich ganz in die Fähre hieven konnte, wurde sie von einer mächtigen Pranke ins Wasser zurück geschubst. Obwohl sie untertauchte hörte sie ein lautes Gebrüll und als sie auftauchte, ragte ein Löwe, die Tatzen auf die Fährenplanken gestemmt vor ihr auf und funkelte sie an. Er hob den Kopf in den Nacken und brüllte ein zweites mal. Dann nahm er den Stock, an dem Amélie sich noch immer mit einer Hand hielt ins Maul und löste ihn vom Schiff. Amélie stürzte mit dem Stock in der Hand zurück in den Fluss.

Sofort rauschte das Krokodil heran, aber sie hatte damit gerechnet und stach so gut sie konnte mit dem Stock nach dem Auge des Ungetüms. Das Krokodil tauchte ab, aber Amélie war schnell und konnte sich an dem rauen Schuppenkleid festhalten. Mit dem freien Arm und den Beinen klammerte sie sich an das Reptil, das tiefer und tiefer tauchte. Sie rammte ihm den Stock überall hin, wo sie konnte, bis das Krokodil auftauchte. 

Sie schnappte nach Luft."`Ich schlag dich zu Brei, wenn du nochmal abtauchst!"' japste sie und haute zur Bekräftigung den Stock tüchtig auf den Schuppenkopf. In dem Augenblick kamen sie unter der Mittleren Brücke durch. Ein Schiff kam auf der linken Seite den Rhein hinauf gefahren. Die Wellen brachen sich an den Brückenpfeilern und schwappten über das Reptil und seine Reiterin. Der Strudel hinter der Brücke verstärkte sich und das Krokodil nutzte die Gelegenheit Amélie abzuschütteln. 

Sie kämpfte verzweifelt gegen den Sog der Walze an, doch es war zu spät sie ging unter. Aber sie war vorbereitet, statt nach oben, tauchte sie so tief sie konnte und die Strömung der Walze aufhörte. Sie spürte den sandig, schlammigen Boden und stiess sich ab. An der Oberfläche atmete sie tief durch. Verdammtes Schiff! Sie hatte den Stock verloren! 

\section*{3.1.2}
\addcontentsline{toc}{section}{3.1.2}

Wibrandis hatte sie zu Thots achteckigen Bureau geschickt.

Sie wollte klopfen, hielt aber inne, als sie das Stimmengewirr hinter der Tür hörte. Ob Tefnut in dem Zimmer war? Und wer noch? Amset? Wieso Amset, dachte Amélies Trotzstimme im Hinterkopf.

"`Herein, nur herein"' Thot öffnete die Tür, bevor Amélie geklopft hatte. Wie hatte sie denken können, sie bliebe unbemerkt? Eingeschüchtert betrat Amélie den Raum. Die drei Ohrensessel waren besetzt. In dem einen erkannte sie Tefnut, die Göttin hatte sich einen kurzen, roten Bademantel aus weichfliessender Seide übergestreift. Seltsamer Weise hatte sie eine Schürze an. Keine grosse, wie die anderen Göttinnen, die Amélie mit Schürze gesehen hatte, sondern eine kleine weisse Schürze mit Rüschenrand und einer kleinen Tasche. Auf die Tasche waren Dreiecke in rötlichen Farben gestickt, angeordnet wie eine Flamme. Diese Schürze ist heiss! Fand Amélie und merkte, wie sie Rot wurde. Es schien ihr ein Zeichen von Tefnuts Anwesenheit zu sein, Scham zu empfinden. Deswegen hatte Wibrandis wohl einen roten Kopf bekommen am Morgen, als Amélie ihre Schürze ansah. Allerdings musst sie sich bei ihrem Modell keine Sorgen wegen der Temperatur machen\dots Tefnut schien nur diese zwei Kleidungsstücke und rotseidenen Pantöffelchen an den Füssen zu tragen.  Sie hatte ein Bein übergeschlagen und wippte mit ihrem Pantoffel an der Zehenspitze. Sie fixierte Amélie, lächelte aber. Dann warf sie einen schmachtenden Blick zu dem zweiten Ohrensessel hinüber und Amélie schien für immer vergessen\dots

"`Das ist Schu. Der Gott der Lüfte und des Lichtes. Der Gott des Lebens und Ehegatte und Bruder der hinreissenden Tefnut\footnote{Erklärung siehe weiter vorn.}."' Stellte Thot den mächtigen  Gott vor. Er trug ein schwarzes T-Shirt, auf dem ein Dreieck aus weissen Linien, hinter dem sich ein weisser Streifen in einen Regenbogen auffächerte, abgebildet war. Amélie hatte das Bild schon gesehen. War es nicht ein Plattencover? Dazu eine verschlissene, verboten enge Jeans. Der Gott schien jung. Er war muskulös und schlank. Die Verwandtschaft mit dem Sonnengott war deutlich zu sehen. Eine goldblonde Löwenmähne umrahmte seine Gesicht und fiel auf seinen Rücken. Er lächelte Amélie verschmitzt an. In seinem Mundwinkel hing eine verdächtig lange, selbst gedrehte Zigarette, die nicht nach Tabak duftete. "`Hi, Amélie"'. Bevor Amélie in die Verlegenheit geriet wie ein Groupie zu quietschen, lenkte Thot ihre Aufmerksamkeit auf den nächsten Gast. 

"`Last but not least, haben wir heute die Ehre den Sonnengott höchstpersönlich in unserer Morgenrunde zu begrüssen. Amélie, dies ist der Vater aller hier anwesenden Götter, Re!"' "`Hach, Thot. Wie schaffst du es nur? Jetzt fühle ich mich plötzlich alt"' lachte der Gott. Amélie grinste schüchtern. Sie hatte den Vater aller Götter sofort in ihr Herz geschlossen. Sie hatte ihn schon während des gemeinsamen Mittagessens beobachtete, wo er, meist vergnügt und laut-fröhlich, mit einer Sonnenbrille auf der Nase am Kopfende des Tisches sass. Er überragte alle, selbst Horus. Man musste kein Götterexperte sein, um seine Stellung und Autorität als Göttervater sofort zu spüren. Auch jetzt trug er die Sonnenbrille. Amélie war ihm bisher nie so nahe gekommen und konnte nun den Glanz erkennen, den die Augen auch durch die dunklen Gläser abstrahlten. Das Haupt des Sonnengottes war von einer goldenen Carona umgeben. Die golden und rot schimmernden Locken bewegten sich, als gäbe es in dem stillen Raum einen Luftzug. 

"`Amélie, heute Nacht warst du grossartig!"' Begann Thot die Runde. Alle Götter murmelten zustimmend, Re hielt den Daumen in die Höhe und Schu zwinkerte mit einem strahlenden Himmelblauen Augen. Amélies glühte, -Hilfe, ich seh' aus wie eine Tomate. "`Ich denke, Sobek dein Lehrmeister wird der gleichen Ansicht sein, wenn du ihn nachher besuchen gehst!"' Amélie schluckte, sie hatte nicht viel Lust das Krokodil wieder zu treffen. 

In diesem Augenblick betrat Hans das Bureau. Er hatte sich eine Rose an die Latzhose gesteckt, welche er als einziges Kleidungsstück akzeptieren wollte, obwohl es Winter war. -Anscheinend tragen Götter Kleidung als Statement, oder aus Spass, statt als Schutz für den Körper, dachte Amélie. "`Jetzt sind alle Vertreter der Elemente zusammen."' Bemerkte Thot. Schu und Tefnut hatten sich erhoben und neben Hans gestellt. Sie wirkten plötzlich sehr ernst und feierlich. 

Hans trat als erster an Amélie heran. "`Meidli!  Du hesch de Prüfung bestande. Du hesch de Lebenskraft könne ufnä. Daher bisch du brechtig die Insigne des Erdvolkes die silberne Münze z trage und z dinem Schutz z verwende."' Hans lächelte auf Amélie runter und schüttelte begeistert ihre Hand. Amélie war derweilen blass geworden. "`S isch nur seltsam, dänn de Münze hätte hüt am morge in meiner Hosentasche sölle erschiene. Isch s abr nich\dots nu weiss ich auch nich!"' Der wilde Maa zuckte mit den breiten Schultern und liess den Kopf hängen. "`Ähm"', Amélie räusperte sich "`könnte es daran liegen, dass mir die Münze gestern im Fluss in den Hals gerutscht ist und ich sie verschluckt habe?"' "`Joh, Meidli, du packscht s Züüg grad richtig a, hä?"' Hans lachte. Dann drückte er mit dem Zeigefinger in Amélies Bauch. Er dachte einen Moment nach und meinte dann: "`Joh, nich schlecht, dann hesch du de Münz noch s Momentli bei dir. Ich werd de Rest mit dr Wibrandis bspreche\dots. Wird wohl nich nötig si, dir de Buuch ufzschnide, gell?"' Amélie schluckte.

Dann trat Tefnut vor. "`Ich bin das Feuer. Ich übergebe dir den heiligen Feuerstab. Es ist derselbe, den dir der Leu der Leuenfähre gestern gab und den du Sobi über den schuppigen Schädel gezogen hast."' Tefnut lachte und reichte Amélie graziöse den Stab. Amélie nahm ihn. Er war aus Holz. Er besass noch die Rinde, in die feine Muster eingeritzt waren, die wie Flammen aussahen. Die Muster bildeten sich aus lauter Dreiecken, deren Spitze nach oben zeigte. Am oberen Ende das leicht gekrümmt war und sich natürlich in ihre Hand schmiegte, sah sie die verschiedenen Farben des Holzes. Aussen unter der Rinde war es hell, gelblich und im Inneren befand sich ein rötlicher Kern.

 "`Es ist ein Wachholderstab"' bemerkte Thot zu Amélies Blick. Eine kleine Erinnerung tauchte auf, Berta hatte ihr vom Wacholder, oder auch Manchandelboom erzählt. -Cool, ich hab' jetzt einen Zauberstab! Amélie strahlte. "`Du wirst den Stab heute noch brauchen!"' Meinte Tefnut. "`Wir sehen uns heute Nacht, Schätzchen!"' "`Okay!"' stotterte Amélie. Tefnut verliess das Bureau, nachdem sie Schu einen Kuss auf die Wange gehaucht hatte. "`Bis gleich, Liebster!"' flüsterte sie und Schu blies ihr sanft den Rauch seiner Zigarette ins Gesicht. Thot räusperte sich. Zu Amélies Vergnügen, bemerkte sie den leichten rosa Schimmer auf den Wangen des sonst so ernsten Totenrichters und Lehrers.

Amélie wurde von Thot zu Schu geschoben. Der hielt plötzlich die Scheide und das Schwert in den grossen Händen. "`Dies ist das Luftschwert. Es darf von denen benutzt werden, die die Kunst der Elemente beherrschen. Es gibt vieles, was du noch lernen musst. Aber die Herren der Luft, zu denen auch der Vogel Gryff auf der Fähre und ich gehören, meinen, du bist es wert."' Schu reichte Amélie die beiden Gegenstände. "`War' übrigens nicht leicht, das Schwert aus Sobis Rachen zu holen. Da kannst du dich bei Duamutef bedanken."'

Die Scheide war aus Samt und Leder. Das Leder war gelblich. An der Spitze und am oberen Rand war das Leder auch Aussen als Verstärkung angebracht. Auf dem oberen Lederring schimmerte ein blauer Kristall.Der Samt war Königsblau und überzog die restliche Scheide. So fühlte sie sich steif, aber warm und weich an. Mit einem einfachen Gürtel, der aus dem selben gelben Leder gefertigt war, konnte die Scheide an die Hüfte gebunden werden. 

Amélie zog das Schwert aus der Scheide. Der Griff war mit dem gelben Leder umwickelt. Das Schwert war einfach. Die Klinge aus Eisen, keine Schnörkel, keine Zierden, keine Schmucksteine. Amélie war etwas enttäuscht. Sie lies sich, wie sie hoffte, nichts anmerken, sie wollte den Luftgott nicht kränken. "`Danke!"' "`Heute Nacht wirst du kannst du es dann ausprobieren!"' Amélie blickte Schu erstaunt an. Der lies einen Rauchring aus seinem Mund, der sich um Amélies Nase kringelte. Sie nieste. Schu hob lässig die Hand und ging aus dem Zimmer.

"`Ja, ja, die Kinder."' Murmelte Re. Er schaute erwartungsvoll zu Thot. "`Von mir bekommst du den letzten Gegenstand."' Amélie stutzte, als sie den Kelch in den Händen des Gottes sah: "` Die Schale, die ist völlig zersplittert? Ich hab doch sogar mit einem der Splitter zugestochen?"' Thot reichte Amélie den Kelch. Er war schwer und kühl. Fein geschliffener Kristall, der an immer neuen Stellen das Licht ein funkelnde Regenbögen brach und sie im Zimmer verteilte. "`Sehr hübsch"' fand Re. Er hob etwas die Sonnenbrille an und sofort befanden sie sich alle in einem Farbenmeer. Die Farben schillerten und schimmerten, Amélie konnte den Kelch nicht mehr sehen, obwohl sie ihn in den Händen hielt. Dann erblickte sie im Inneren des Kelchs eine rote Flüssigkeit. "`Ist das, das Blut? Von wem? Wieso?"' "`Später"' gab Thot zur Antwort, "`später wirst du es erfahren."' Er nahm den Kelch aus Amélies Händen. "`Die Gabe des Kelchs trägst du seit gestern Nacht in dir. Die anderen drei Insignien werden dir helfen, dich an die Macht des Blutes zu erinnern."' 

\section*{3.2.8}
\addcontentsline{toc}{section}{3.2.8}

Von den Beamten angesprochen, flog der Vogel auf und verschwand in einer Art Seifenblase. Die beiden Flüchtigen springen zum Entsetzten der beiden Beamten die Maurer runter in das Lohnhofgässlein. obwohl die beiden Beamten am Fuss der Mauer keine Personen sehen können, alarmieren sie die Sanität.

Moser wischt sich die nasse Stirn. Ein weiteres Rätsel dieser Nacht. Wie haben die beiden Männer den Sprung nicht nur unbeschadet überstanden, sondern konnten gleich darauf verschwinden? und wieso hatte es so ausgesehen, die beiden würden Mitten in der Luft in einer riesigen Seifenblase verschwinden, genauso wie der Vogel? Bedrückt dachte Moser an das Geschrei seines Chefs am Morgen. 

\section*{3.2.3}
\addcontentsline{toc}{section}{3.2.3}
 
 Auf diese Weise sorgten Geb und Hans in dieser Nacht nicht nur für Osiris, sondern auch für die Abstinenz von insgesamt 19,75 Personen, die ihnen in der Nacht begegneten und danach keinen Tropfen Alkohol mehr anrührten. Nur einer wurde jeweils am 2. Weihnachtstag regelmässig rückfällig. Und zwar immer, wenn er den Weg von seiner alten Mutter nach Hause geschafft hatte, ohne einem leise pfeiffenden, nackten Mann mit einem toten, schwarzen Hund auf dem Rücken und einer Gans auf dem Kopf zu begegnen. Damals hatte er vor Schreck geschrien und der riesige, nackte Mann hatte ihn gebeten leise zu sein, weil er für nichts garantieren könne, wenn der grosse Schnatterer aufwachte! Der tote Hund auf dem Rücken des gewaltigen Mannes hatte begonnen zu zappeln. Was geschehen war, als auch hinter ihm ein weiterer nackter Riese aufgetaucht war und gemeint hatte, man solle sich Gästen gegenüber höflich und zurückhaltend benehmen, konnte er nicht erinnern.

\sterne

Geb liess die Schultern hängen. Er hasste Streit mit dem Schnatterer mehr als mit seiner Frau Nut, die zugegeben, auch toleranter war als die Gans. 'Man soll es nicht für möglich halten, aber es ist so!' pflegte Geb zu sagen, ganz so, als würde es ihn selbst verwundern.

\sterne




















\section*{3.2.10}
\addcontentsline{toc}{section}{3.2.10}

Ein Mann, der in ein weisses, langes Chorhemd und eine samtene, blaue Dalmatica mit goldenen Streifen gehüllt war, kam ihnen eilig watschelnd und mit den Armen rudernd, entgegen geraschelt. Er war kräftig und sehr dick und strahlte Willenskraft und Stärke aus. Er trug einen Bart aus vollem, braunem Haar, das auf dem Kopf nur durch eine Tonsur gebändigt schien.

Sein Gesicht war gerötet. Er trat mit ausgebreiteten Armen auf Re zu. ,,Willkommen, willkommen, die edlen Götter!'' rief er. ,,Ich bin Burkhard von Fenis, ehemals Bischofs zu Basel und enger Vertrauter des Königs Heinrich IV!'' Er umfasste die Schulter des Sonnengottes und führte ihn in die Mitte der Kirche. ,,Tja, eine bescheidene Hütte!'' ,,Wie bitte?'' Re sah ihn irritiert an:,,Wieso Hütte?'' ,,Damals habe ich mir gesagt, Burkhard habe ich mir gesagt: Basel braucht eine Stadtmauer. Und weil auf dem Hügelsporn über dem Birsig noch Platz war, habe ich die Kirche praktisch mit der Stadtmauer zusammen bauen lassen. Später haben die Augustiner-Chorherren einen ganzen Klosterbetrieb hier eingerichtet.''

Stolz blickte Burkhard auf sein Werk. ,,Ich bin dafür die Dinge anzupacken, wissen Sie?'' Er streckte die Brust raus, dabei entfaltete sein dicker Bauch eine weitere Dimension wie ein Klapptisch. Das Chorhemd hob sich wie eine Tischdecke und gab den Blick auf die breiten Füsse in spitzen Lederschuhen und den Waden mit den roten Seidenstrümpfen frei. 

,,Wie hätte ich die Hasenburg, meine heimatliche Burg Fenis verlassen können, wenn ich mich auf die faule Haut gelegt hätte?'' Burkhard reckte sich, um erhobenen Hauptes auf seine Gäste zu schauen. ,,Immer auf Zack: Wissen was der König braucht, bevor der König überhaupt weiss, das er es braucht!'' Er machte ein listiges Gesicht und klopfte sich mit dem Zeigefinger an den Nasenflügel. Dann faltete er seine Hände im Rücken und lächelte. Er schwelgte in seinen Erinnerungen.

,,Tja, so konnten wir dem Gregor, dem falschen Papst, ein Schnippchen schlagen! Ha, er dachte schon, er hätte uns mit dem Kirchenbann mundtot gemacht! Aber nicht mit uns!'' Er stellte sich auf die Zehenspitzen und schüttelte wild den ausgestreckten Zeigefinger. ,,Nicht mit uns!'' Er blickte Re an. Auf den Zehenspitzten und hochgerecktem Finger, theatralisch machte er eine Pause: ,,Heinrich, habe ich damals gesagt. Heinrich, wir gehen nach Canossa! Wir gehen nach Canossa. Der König wollte nicht! Natürlich nicht! Krone oder Stolz! Habe ich ihn gefragt. Willst du König bleiben, oder ein stolzer Exkönig?'' Burkhard schnaufte hingebungsvoll. Sein Kopf war krebsrot und Schweiss ran an seiner Stirn und Schläfe runter in seinen Kragen. Was in seinem körperlosen Zustand nicht nötig gewesen wäre, die Dramatik jedoch gut zur Geltung brachte.

,,Auf mich hat er dann gehört, der Heinrich! Und dankbar ist er gewesen! Hat mich fürstlich belohnt, weil ich Mumm hatte, wie er gesagt hat, weil ich ihm gesagt habe, wie es aussieht!''

Die vier anderen hatten sich auf der Kirchenbank niedergelassen. Schu mit Tefnut, die es sich auf seinem Arm gemütlich gemacht hatte, Thot, kerzengerade, aber entspannt und Amélie, die sich nur hingesetzt hatte, weil sie es gewohnt war. Schliesslich war körperlos sitzen anstrengender, als in der Luft rumhängen!

,,Und wie fährt sich das Schiff so?'' fragte Re in die Stille. Schu brüllte vor Lachen und selbst Thot musste hinter vorgehaltener Hand kichern. Amélie hatte Glück, denn ihr Lachen blieb ungehört.

Burkhards Gesicht wandelte sich vom krebsrot zum Karminrot. Er hatte sich erstaunlich gut im Griff, denn seine Stimme klang ruhig:,,Tja, das müssen Ihre Heiligkeit andere fragen. Seit der Reformation haben andere das Steuern übernommen.'' Er schien darüber aber nicht unglücklich.

,,Was siehst du?'' fragte Thot und lehnte sich zu Amélie. Sie hatte Bewegung um Burkhard von Fenis gesehen. An den Anblick der Dämonen hatte sie sich gewöhnt. Aber um den Bischof gab es andere Wesen, die viel grösser waren. ,,Hast du etwas bemerkt, als Burkhard gesprochen hat?'' fragte Thot. ,,hast du an deinen Gefühlen etwas gemerkt?'' Amélie versuchte sich zu erinnern. ,,Es war sehr spannend! Das mit dem Papst und so. Der muss ein richtiger Mistkerl gewesen sein.'' ,,So? Weisst du das sicher?'' Amélie sah Thot an der Nasenspitze an, dass sie es sicher nicht sicher wusste. ,,Mach es nicht so spannend\dots'' meinte sie.

,,Gut, gut!'' schmunzelte Thot. ,,Schau dir die Wesen an. Die von denen ich rede sind gross, daher können wir nur Teile von ihnen sehen.''

Amélie entdeckte Schatten. Riesige dürre Beine und lange Finger, die um Burkhard herum staksten. ,,Sie sind die Herren schlechter Gesetzte und ungerechter Richter.'' -,Wie kann man sie loswerden?' fragte Amélie. ,,Garnicht!'' war Thots schlichte Antwort. ,,Am besten ist es, wenn wir sie kennenlernen, ihnen Beachtung schenken. Sie können nicht verschwinden! Allerdings verändern sie sich, wenn sie gesehen werden. Am schlimmste ist es, wenn sie ignoriert werden, denn sie haben Macht. Wenn sie können, dann versuchen sie die Menschen dazu zu verleiten mehr schlechte Gesetzte zu machen und mehr falsche Urteile zu fällen.''

Amélie dachte nach. -,Was hat Burkhard für Gesetzte gmacht?' Thot wiegte den kopf: ,,Er hat die Gesetzte nicht alleine gemacht, das kann niemand. Um Gesetzte zu machen, braucht es viele Menschen. Burkhard hatte viel Macht zu seinen Lebzeiten und hat, oft ohne es zu wissen, mit geholfen eine ganze Arme von Spektren und Gespenstern, denn das sind sie, zu schaffen.''

Amélie sah zu dem dicken Bischof hinüber, der offenbar wieder in Erinnerungen schwelgte und Re mit Händen und Füssen weitere Abenteuer aus seinem Leben erzählte. -,Gibt es denn Könige und Bischöfe, die keine Monster geschaffen haben?' ,,Das ist eine gute Frage!'' Amélie bemerkte erstaunt wie Thot angestrengt nachdachte. ,,Ich muss sagen, es mag welche gegeben haben, aber es fällt mir keiner ein. Es gibt Könige und Herrschende, Richter, die wenige Gespenster schufen und andere, die sehr viele hervorbrachten, aber keine?''

Re kam zu ihnen, er war gut gelaunt, während der Bischof erschöpft und enttäuscht aussah. Er hatte sich das Gespräch mit dem höchsten ägyptischen Sonnengott anders vorgestellt. So von Herrscher zu Herrscher. 

,,Amélie, komm, ich will dir etwas lustiges zeigen.'' Der Sonnengott schlüpfte, wie alle anderen, selbst der dicke Bischof, durch die geschmiedete Tür in den Chorraum.

An den Stufen zum Altarraum blieben sie stehen. Tefnut maunzte. Toth hob die Augenbrauen und Re gluckste vor Freude. Amélie hatte Mühe. Sie konnte einen Hauch spüren, von dem sie nicht wusste, ob es Luft war. Sie fühlte wie sich eine bewegte Säule hinter dem Altar an die Decke empor wand. Schu formte mit den Lippen ein O und ein Rauchring löste sich. Er schwebte träge auf die Säule zu. Als er sie berührte, drehte er sich wie von einem Zeitlupentornade erfasst nach oben. 

Als der Ring die Decke berührte und sich auflöste, kam an dieser Stelle eine Lichtsäule geführt durch die Turmspitze und breitete sich wie ein Spottlicht nach unten aus. ,,Gut gemacht, Junge.'' meinte Re anerkennend -,Wie schön!' entfuhr es Amélie, als sie die lichten Gestalten gewahr wurde, die sich gegen den Strom von der Decke im Wirbel abwärts bewegten. ,,Ja, das ist schön.'' meinte Re.

,,Ouuuhh.'' der Bischof, der überrascht einen Schritt in den Wirbel gemacht hatte, wurde sanft von den Füssen gehoben. Er schwebte mit den lichten Wesen bis an die Decke mit. Ein zartes Geräusch, pookh,  ertönte und der Bischof in der Mitte des Wirbels auf den Boden glitt. Er machte wieder: ,,Ouuh!'' dann sank er in Ohnmacht. Die Anwesenheit der lichten Wesen hatte ihn, der seit Jahrhunderten mit Gespenstern zusammen lebte, umgehauen. ,,Ich glaube unser Freund ist ein guter Bauherr, aber er hat keine Ahnung wie man ein Kirchenschiff steuert.'' scherzte Re.

Amélie konnte sich nicht satt sehen. Aber die Zeit in der Kirche war zu Ende. ,,Du kannst am Tag in die Kirche gehen! Die lichten Kerlchen sind immer da, sobald sich jemand im Raum befindet, der nach ihnen Ausschau hält.'' meinte Thot. -Vielleicht hat Amsi morgen Lust\dots, dachte Amélie.

Sie warfen einen letzten Blick auf den grossen Bauch, der wie eine Insel vom Boden des Altarraumes aufragte, dann wendeten sie sich dem Ausgang zu.

und Re auf den kleinen Platz trat, hörten sie einen schrillen Vogelschrei. Die anderen vier blieben erschrocken unter dem Vordach stehen, Re hob die Arme und ein Bündel sauste von oben hinein. ,,Kebi!'' rief der Sonnengott. Mit dem Bündel im Arm begann er plötzlich zu flackern. ,,Vater, was ist\dots'' rief Schu aufgeregt und die anderen eilten auf Re zu.
\begin{Large},,Uuuiih! Stop! Bleibt wo ihr seit!''\end{Large} Es war zu spät!

Bevor einer von ihnen ,,Piep!'' sagen konnte, wurden sie in einen dunklen Tunnel gesogen. Ein Augenaufschlag später fanden sie sich vor der Tür des Blauen Hauses wieder.

Amélie kicherte. Aber im Gegensatz zu den anderen, die laut fluchten und schnauften,  als sie begannen ihre Beine und Arme zu sortieren, schwebte sie mit ihrem Lebensleib über ihnen. Die arme Tefnut hatte es am schlimmsten getroffen. Die drei Götter waren auf sie drauf gefallen. Nur ihre schlanke, gelenkige Katzengestalt hatte sie vor dem schlimmsten bewahrt. ,,Aua, auuuu!'' rief Thot, der zu zweitunterst gelandet war auf. Dann konnte Amélie die Krallenbewehrten Tatzen der Göttin sehen, die sich unerbittlich in alles schlugen, was der Freiheit im Weg war. Dann schoss die Katzengöttin aus dem Haufen und machte einen empörten Buckel.

