\part*{Erste Stunde:\\,,Welche die Stirnen der Feinde des Re zerschmettert''}
\addcontentsline{toc}{part}{Erste Stunde}


\chapter*{24.Dezember, Adam- und Evatag}
\addcontentsline{toc}{chapter}{24.Dezember; Adam und Evatag}

\section*{1}

Amélie öffnete die Augen, was für ein Traum\dots

Aber, he, das war nicht ihr Zimmer! Wo in aller Welt war sie? Sie zog die weisse Bettdecke, die mit den Daunen von einer Million Gänsen gefüllt war, zur Nasenspitze und rutsche tiefer in das weisse, flaumige Kissen. Sie lauschte, während die Augen das Zimmer durchstreiften. 

Die Wände waren weiss. Eine Kommode aus Nussbaum stand an der einen Wand, daneben war ein Tür, die in ein anderes Zimmer führen musste. Das Bett stand gegnüber. Am Kopfende des Bettes fiel durch zwei grosse Sprossenfenster trübes, graues Winterlicht, in den quadratischen Raum. Durch eine zweite Tür, gegenüber den Fenstern, hörte sie eilige Schritte hin und her gehen. Auf der anderen Seite dieser Tür musste sich ein langer Gang befinden. Das Zimmer schien in einem grösseren Gebäude zu sein. Stimmen waren zu hören, die einen wie leises Murmeln, die anderen laut und nahe der Tür.

Das Zimmer roch nach Wachs mit dem der dunkle, alte Holzboden gepflegt worden war. Es roch nicht nur alt, sondern unpersönlich, wie ein Büro- oder Amtsgebäude. Die Geräusche und Stimmen schienen einer aufgeregten Grossfamilie zu gehören.

Amélie streckte sich. Und starrte an die Decke. An der Decke war ein schwarzer Punkt. Ein schwarzer Punkt, der sich bewegte! Eine Spinne? Amélie konnte ein Quietschen nicht unterdrücken und versteckte sich unter die Decke. Vorsichtig spähte sie hervor, das Tier war verschwunden! Abgestürzt? Wohin? Auf sie? Mit einem Schrei, sprang Amélie aus dem Bett. Schüttelte sich wild und hopste durch Zimmer. Wo war die verdammte Spinne?

 Da, da war sie! Über der Kommode! Amélie schnappte ihr dickes Kissen und hob es über den Kopf, bereit auf alles einzuschlagen, was mit einem Bein zuckte. Dann sah sie genauer hin, es war keine Spinne, es war eine Ameise. Überrascht lies Amélie das Kissen sinken und betrachtete das Tier. Eine Ameise? Im Winter? Amélie beugt sich vor, ihre langen, glatten schwarzen Haare rutschen über die Schultern nach vorne. Diese Ameise sah erschöpft aus!

Amélie setzte sich auf das Bett. Erst jetzt bemerkte sie, was für Kleider sie trug, sie hatte ein weisses Unterhemd mit breiten Spitzenträgern und eine weisse Unterhose an. Woher? 

Kalt war es, sie schlang die Arme um sich und zog die Beine an. Sie versuchte sich zu erinnern: War es ein Traum gewesen? Ein Traum mit Ameisen. Sie hatte geträumt, eine riesige Zahl Ameisen würde sie durch einen unterirdischen Gang schleppen. Die Ameisen wurden von einem grossen, kräftigen Mann mit einem dunkelbraunen, struppigen Vollbart und wildem Haar angetrieben, der mit grünem und braunem Laub bedeckt war und von \dots Berta?

 Amélie war betäubt worden. Sie konnte sich während des Transportes nicht bewegen und dämmerte auf dem Marsch dahin, der, wie ihr schien, Tage dauerte. Einzig geweckt wurde sie, wenn ihr Kopf oder ein anderer Körperteil unsanft gegen eine Baumwurzel schlug, die  in den erdigen Gang hineinragten. 
 
 Sie erinnerte sich an grässlichen Lärm quietschender Räder auf Schienen. Tunnel, dunkel, betoniert und mit Leitungen vollgestopft. Ein blendendes Licht, S-Bahnen, die direkt auf sie zu gebraust kamen und mitten durch sie, die Ameisen und den fluchenden, wilden Mann durchfuhren, ohne abzubremsen. Sie verschwanden im Tunnelgewirr, Berta schrie und trieb die Ameisen an, während diese Amélie mit dem Kopf gegen die Wand stiessen, bis die Wand nachgab, sich öffnete und sie sich wieder in einem Erdgang befanden. Amélie fasste auf ihren Scheitel. Tatsächlich fühlte sich die Kopfhaut wund an. Was, zum ***? So einen Quatsch konnte man nicht wirklich erleben, oder? Wie konnte sie von einem Haufen Ameisen entführt werden? 

Es klopfte an der Zimmertür.
"`Hallo? Amélie! Bist Du wach?"' Amélie schlüpfte unter die Decke. Panik machte sich breit, das war eindeutig die Stimme von einem Kerl. Einem jungen Kerl! Sie hörte angespannte Stille, dann schnaufte es vor der Tür. "`Ja! Nein! Stopp!"' rief Amèlie. Aber da stand der junge Mann bereits in der Tür und an ihm vorbei drängelte sich ein graziler, schlanker Hund mit dickem, goldbraunem Fell. Er trug einen schwarz-grauen Sattel aus längerem Haar auf dem Rücken und die Ohren waren spitz und gross. Der Hund war es, der aufgeregt schnüffelte. Er blieb in einiger Entfernung stehen und starrte Amèlie unhündisch an.

"`Ich hab keine Kleider,"' Amélie spürte, wie ihr die Röte ins Gesicht stieg und tauchte tiefer unter die Decke. "`Ah, oh! Ich glaube Isis hat dir etwas zum anziehen in die Kommode gelegt,"' erklärte der junge Mann eifrig der Bettdecke. "`Ich bin übrigens Amset und das ist mein Bruder Duamutef."' Er strahlte und zeigte auf den Hund\dots

Isis? Bruder? Ein Hund? Bin ich von Verrückten entführt worden? Amèlie reckte ihre Nase vor und rief empört: "`Verschwinde! Ich will mich anziehen!"' Amélie nahm das Kissen und warf es. Es prallte an der hastig geschlossenen Tür ab und plumpste zu Boden.

-'Haha, sie hat nach fünf Minuten schon den ersten Gegenstand nach dir geworfen, Amset!' -'Ach, halt dein Maul, Tef!' -'Hast du gewusst, dass sie dich für verrückt hält?' "`Maul halten, Tef!"' Amset war es lauter rausgerutscht, als er wollte -'Was schreist Du denn so?' Der Schakal kräuselte die Schnauze -'Haha!' -'Und dich hält sie für einen Hund,' konterte Amset. Duamutef zog unwirsch die Lefzen hoch.

Der Schakal und der junge Mann stiegen die grosszügige Treppe hinunter. Sie waren Götter und Brüder, der eine war ein Mann und der andere ein Schakal. Wie sie miteinander sprachen? Mit Gedanken. Sie waren vier Brüder, die berühmten Horussöhne, unbestechliche Wächter im Totenreich. Der dritte Bruder Kebechsenuef war ein Falke und der vierte, Hapi, ein Pavian. Und sie waren jung. Und sie hatten Glück, denn Götter sind ewig -z. B. jung.

-'Geh raus in den Garten deine Geschäfte machen, Tef!' Amset war echt sauer! Warum eigentlich, überlegte er und schickte versöhnlich ein: -'Und Grüss Hapi und Kebi von mir,' hinterher. -'Heb' mir was vom Frühstück auf, ja?' Der Schakal verschwand durch die Tür, die Amset ihm öffnete, in den Garten.

Amélie tappte auf Zehenspitzen zur Kommode. Aus dem Augenwinkel bemerkte sie eine Bewegung am Fenster. Es war nichts zu sehen, ausser dem grauen Himmel.

 "`Gib sie her!"' "`Hol sie dir doch! Komm doch! Kooomm!"' "`Isfet, gib sie her, ich bring dich um!"'"`Vergiss es, Maat, das darfst du nicht!"' Die andere Tür flog auf und zwei Mädchen fielen ins Zimmer. Sie wälzten sich am Boden. Die eine von ihnen hielt eine weisse Straussenfeder von sich gestreckt, nach der die andere hektisch griff. Sie bekam sie nicht zu fassen, weil ihre Gegnerin energisch zappelte. Amélie schnappte sich behutsam die Feder und zog sich ans Fenster zurück. "`Gib mir die Feder!"'

Die schlanke Gestalt, etwas kleiner als Amélie, streckte gebieterisch die Hand aus. Sie trug ein gerade geschnittenes, schwarzes Kleid mit weissem Spitzenkragen. Sie hatte weisse Söckchen ebenfalls mit Spitze an, im Winter! Und schwarze Spangenschuhe aus Lack. Die schwarzen Haaren trug sie zum Pagenkopf geschnitten, mit einer weissen Schleife gebunden. Die Haare waren zerzaust und die Schleife hing schief, ihre Wangen waren gerötet. Aber als sie der zarten Gestalt ins Gesicht sah, bekam Amélie eine Gänsehaut.

 Ihre Augen waren gelb. Goldgelb. Und sie strahlten eine machtvolle Würde und Weisheit aus. Amélie schluckte und gab dem Mädchen die Feder, stumm und erschrocken. "`Danke!"' sagte die kleine Person und ging erhobenen Hauptes aus dem Zimmer. 
 
 "`Isfet!"' sagte die andere und streckte Amélie die Hand entgegen. Diese hätte ein Zwilling sein können, allerdings einer, der aus dem Nest der Ordnung gefallen schien. Isfet hatte kurze, verwuschelte, schwarze Haare. Sie steckte in einem riesigen, bunten Sweatshirt mit Kapuze und einem Pinguin drauf, ihre Beine in pinken Leggings mit blauen Tupfen und die Füsse in dicken Stiefeln.
 
"`Amélie,"' sagte Amélie. "`Ich weiss!"' Isfet liess sich auf das Bett fallen. "`Alle scheinen das zu wissen"', maulte Amélie. "`Und ich weiss nicht mal, wo ich bin. Verdammt, wo bin ich? Und wer seit ihr? Ein Haufen Verrückte?"' Isfet grinste: "`Du bist schon in Ordnung, weiss du? Zieh dich an und komm runter, dann lernst du den restlichen Haufen kennen."'

Amélie zog die Kommodenschublade auf. Da, sie zuckte zusammen! Da war wieder eine Bewegung vor dem Fenster. Isfet sprang ans Fenster, riss es auf und schrie: "`Kebi, lass' das, sie ist ein Mädchen und ein Gast, n' bisschen Anstand, jaah!"' Sie schloss das Fenster wieder und drehte sich zu der bleichen Amélie. "`Diese Brüder. Wenn sie nicht arbeiten, sind sie immer zu Spässen aufgelegt, die Jungs!"' Isfet ging aus dem Raum und Amélie trat ans Fenster. Brüder? Schon wieder, wer waren diese 'Brüder'? Auf dem gegenüberliegenden Dach sass ein Raubvogel, ein Falke. Komisch, dachte Amélie, mitten in der Stadt\dots

Hatte jemand von der gegenüberliegenden Seite aus dem Fenster geschaut? Amélie betrachtete das Haus. Es erstreckte sich auf drei Seiten. Zwei Stockwerke war es hoch und im Dach waren viele kleine Gauben mit weiteren, kleinen Fenstern. Blau, weiss war das Haus. Im Innenhof war ein üppiger Garten angelegt mit einem grossen Teich. Seltsamerweise war der Garten grün, die Bäume trugen Laub. Die Bäume verdeckten den offenen Teil, so dass Amélie nicht sah, ob der Garten eingezäunt war und wie die Strasse aussah. Jedenfalls war sie in einer grösseren Stadt, denn es waren weitere Dächer zu sehen und Strassenlärm tönte herauf. Zwischendurch war ein Kreischen zu hören, wie Amélie es aus dem Affenhaus im Zoo kannte.
 
 Energisch zog Amélie die Vorhänge zu und machte sich über die Kleider in der Kommode her. Sie passten ihr wie angegossen und trafen genau ihren Geschmack, jedenfalls etwas, dachte sie und machte sich auf die Suche des Frühstücks. Ach, ja, fiel ihr ein, heute ist Heiligabend!


\section*{2}
\addcontentsline{toc}{section}{2}


Amélie sass im Garten auf einer Marmorbank. Wenn das eine 'harmlose Gruppe von Touristen aus Ägypten' ist, dann bin ich Kaiserin von China, dachte sie. Auch wenn alle betont entspannt und lässig taten. Der eine ältere Typ hatte sogar am Frühstückstisch seine Sonnenbrille anbehalten! Und diese Hathor wollte Amélie Würmer aus der Nase ziehen, dabei wusste sie selbst nicht, wie sie hierher gekommen war. Basel, hatten sie gesagt. Was soll ich bitte in Basel? Und dann war da dieser Typ im schwarzen Anzug, der sprach, als wenn er einen Einstein verschluckt hätte. Der wirkte am normalsten, wenn er nicht so ernsthaft mit seinem grossen, schwarzen Hund reden würde, als ob der alles verstünde, was er sagte. Anubis, hiess der Hund, den Namen hatte Amélie schon in einem Buch über Ägypten gelesen, ebenso wie den Namen Thot, so hiess der im schwarzen Anzug.

Ganz in der Nähe hörte Amélie ein hohes Kreischen und zuckte zusammen. Das musste der Raubvogel sein. Offenbar hatte der Falke einen Freund gefunden, es ertönten zwei Raubvogelstimmen. Erst jetzt blickte sie sich genauer um. Der Garten war nicht nur grün, sondern auch warm. Angenehme 20 Grad, leicht feucht, es roch sumpfig, was an dem grossen Teich mit Schilfgürtel lag. Die Bäume waren recht hoch und sahen von unten viel höher aus, als aus dem Fenster. Waren das Lianen? An einigen Stellen standen dichte Büsche, die die Sicht versperrten. Es raschelte dahinter.

Amélie seufzte, wenn Berta da wäre, die wüsste bestimmt, was das alles zu bedeuten hätte. Berta war viel mehr als nur ihre alte Amme, sie wusste alles und mit Berta an der Seite, konnte einem nichts passieren. -Abgesehen von den Abenteuern, in die man automatisch hineingeriet, wenn Berta es für richtig hielt. War das hier eine Berta-Sache?

Amélie fühlte sich beobachtet. Sie schaute sich um, sah aber niemanden. Sie hörte Schritte. Und dann kam Amset auf die kleine Lichtung. "`Hi!"' "`Hi"', er setzte sich neben Amélie. "`Warst du schon in Basel, Amélie?"' "`Ne, du?"', "`Nöh, aber es ist nett!"'"`Amset, warum bin ich hier?"' Amélie dreht sich um und schaute ihm direkt ins Gesicht. "`Also, äh\dots"' "`Raus mit der Sprache!"' Schrie sie. "`Warum wache ich am Heiligabend mitten in Basel bei einer Truppe ägyptischer Touristen, wie ihr euch nennt, auf?"'"`Amélie! Ich weiss nicht, ob ich es dir sagen darf"' "`Wenn du mir nicht sagst, Amset, was du weisst, dann schreie ich!"' "`Das tust du jetzt schon! Okay, okay,"' er hob beschwichtigend die Hände: "`Du träumst seltsame Sachen!"' "`Ich träum' seltsame Sachen? Wer sagt das?"' "`Berta und Thot!"' "`Berta! ich hab es mir gedacht."' Amélie sah Amset in die Augen. Sie waren goldbraun. Wenn sie nicht so wütend gewesen wäre, wäre ihr aufgefallen, dass er mit seinem schmalen Gesicht, der leicht gebräunten Haut und dem kräftigen, zu einem Pferdeschwanz gebundenen schwarzen Haar gut aussah. "`Welcher Traum?"' "`Du hast von Ägypten geträumt? Sagt Berta."' 

"`Von Ägypten?"' "`Berta hat es Thot erzählt. Du hast geträumt, dein Herz wäre gestohlen worden und ein Hund hat es dir wieder gebracht."' Amélie sagte erstaunt: "`Das stimmt! Ich träumte von einem dunklen Tunnel durch den ein Hund zu mir kam mit einem Bündel in dem mein Herz war. Und eine Stimme, die gruslig und beängstigend klang, sagte, das Herz wäre mir nur geliehen worden, ich müsste es mir erst verdienen. Es war ein schlimmer Traum. Seitdem habe ich das Gefühl, als würde jemand jeden meiner Schritte prüfen und mir das Herz ausreissen, sobald ich einen Fehler mache."' Amélie schluchzte auf, wischte sich mit dem Handrücken über die Nase.

"`War es ganz sicher ein Hund?"' fragte Amset. "`Was? Keine Ahnung, ja schon!"' "`Oder war es ein Schakal, wie Duamutef?"' Der schlanke Hund vom Morgen kam aus dem Gebüsch und blieb vor Amélie stehen. "`Das ist ein Schakal?"' fragte Amélie schwach. "`Du musst uns alles genau erzählen, Amélie!"' Sagte Amset aufgeregt und schüttelte Amélie an der Schulter. "`Ich muss nichts, Amset! Ich will nach Hause. Ich will zu meiner Familie und nach Hause."' Amélie sprang von der Bank auf und lief aus dem Gebüsch auf eine grosszügige Auffahrt aus Kies. Sie rannte auf das schwarze, verschnörkelte Eisentor zu. Im Augenwinkel sah sie einen grossen, kräftigen Mann mit wirrem Haar und langem, wildem Bart in einer grünen Latzhose, der den Kies harkte.

Sie riss an der grossen Tür, die sich im hohen, geschmiedeten Tor befand. Sie war offen. Amélie stürmte hinaus und schlug die Türe fest zu. Die Kälte traf sie wie ein Schock. Der Garten war verschwunden. Durch da Gitter des Tors sah sie einen kahlen Innenhof mit Kopfsteinpflaster, einem Brunnen und einer grosszügigen Treppe, auf der man von zwei Seiten das Haus betreten konnte. Sie probierte die Tür, sie war offen. Sie ging hindurch. Der Garten und Amset waren verschwunden.

\section*{3}
\addcontentsline{toc}{section}{3}


Duamutef stand in dem dunklen Gang und hob witternd die feine Nase. Mit dieser Amélie stimmte etwas nicht. Ganz und gar nicht. Er hatte das ägyptischen Grab sofort gefunden. Und bisher war der schlichte unterirdische Gang wie alle anderen. 

Schritt für Schritt tappte er vorwärts. Die Luft stand still, als Wächter und Horussohn konnte er jedes Grab finden und aufsuchen, in das er hinein wollte. Warum war er nicht in der Grabkammer gelandet, sondern im Gang? Als Gott der Kanopen hätte er mitten im Grab zum Vorschein kommen sollen. 

Er lauschte in die Dunkelheit und ein Hauch, eine winzige Bewegung, eine einzige Welle traf auf die feinen Härchen in seinen Ohren. Wie Radars drehten sie sich hin und her. Eine Disharmonie. Eine kleinste Unstimmigkeit. Er ging einen Schritt weiter und lauschte wieder, senkte die Nase und sog den abgestandenen Geruch von tausend Jahren ein\dots Und nun kam zu dem leisen Ton ein giftiger, metallisch-beissender Geruch hinzu. 

Duamutef blieb stehen. Und starrte angestrengt in die absolute Dunkelheit. Und dann sah er ihn, den Fluch. Er war wenige Zentimeter entfernt. Wie durchsichtiges Schillern einer Seifenblase, bildete er eine Membran, die dem Durchgang versperrte, zart, wie ein Spinnennetz.

Duamutef schob die Nase vorsichtig weiter. Er jaulte laut auf, als der Fluch das erste Tasthaar an der empfindlichen Schnauze erwischte und bis auf die Haut verbrannte. Er jaulte nochmal, diesmal aus Zorn. Wer hatte es gewagt in seinen, in ihren Schutzbereich, den der Horussöhne einzudringen und ihnen den Durchgang zu versperren?

\section*{4}
\addcontentsline{toc}{section}{4}


"`Osiris? Wie geht es?"' fragte Thot seinen Freund und Meister, denn, wenn auch einer der mächtigsten Götter, so war er der fragilste, zumindest im Diesseits. '-Es geht Thot!' Osiris war es nicht möglich im Diesseits zu sprechen. Er konnte sich mit seinen Gedanken an die anderen Götter wenden.  "`Es wird hart werden. Was ist mit dem Mädchen, ist sie bereit? Wir haben nicht viel Zeit und ich hoffe, sie weiss, welche Aufgabe sie hat?"'  fragte Isis. Sie sass auf der Bettkante von Osiris Bett. Thot bemerkte ihre Augenringe, eine grosse Last lag auf den Schultern der mächtigen Heilerin und Gemahlin des Osiris. Sie musste den Körper ihres Mannes in dieser Zwischenzeit der Rauhnächte stabil halten, einen Körper, der seit tausenden Jahren in die Unterwelt gehörte und im Diesseits jederzeit Schaden nehmen konnte. 

Der Herr der Unterwelt lag von mehreren Kissen gestützt in einem grossen Bett. Mehrere Federbetten waren um ihn herum drapiert. Auf dem Nachtisch lagen Amulette, Räucherwerk kräuselte sich zart in die Luft des Zimmers und Glas- und Fayenceflaschen standen bereit. In den Glasflaschen waren rote, grüne und goldene Tinkturen zu sehen. 

Sein Zimmer lag im ersten Stock, im linken Flügel des blauen Hauses am Ende des Ganges. Jeder der zu ihm wollte, musste an den unzähligen Türen der anderen Familienmitglieder vorbei. Reine Schutzmassnahme, denn der mächtige Herrscher der Unterwelt, hatte ebenso mächtige Feinde. Allen voran seinen Bruder Seth, der es sich nicht entgehen lassen würde Osiris ein weiteres mal zu töten, wenn er Gelegenheit dazu hätte. 

Im Moment brauchte es Seth vielleicht nicht mehr. Thot hatte Osiris seit seiner Ermordung vor vielen tausend Jahren nicht so geschwächt gesehen. Er lag still, wie ein Toter in seinen vielen Decken und Kissen ein Hauch von einem Gott. Seine Haut, die in der Unterwelt in sattem, kräftigen Grün schimmerte, war grau. Osiris war nicht nur der Herr der Unterwelt, der Duat, wie die Ägypter sagten, sondern auch ihr mächtigster Fruchtbarkeitsgott. Schliesslich war es seine Aufgabe aus dem Jenseits, dem unsichtbaren Bereich, dem Lebenskraftraum, die Pflanzen und Tiere nach der Nilschwämme zum Wachsen zu bringen. Er war die Kraftquelle, der für Befruchtung der Natur sorgte. Doch von all dem sah er nichts mehr, stellte Thot mit kummervoller Miene fest.

Sie hörten Hathor, die in der Küche im Erdgeschoss ein Freudenlied von einem stachligen Tiere trällerte\footnote{Eingeweihte wissen, es handelt von einem Igel, um dem englischen Meister kleiner, dicker, vergnügter Frauen in schwarzen Kleidern und mit spitzen Hüten zu huldigen.}, während sie das Mittagessen zubereitete. Aber Osiris liess sich nichts anmerken. Er war der mächtige Herr des Westens, der Unterwelt und er würde mit der Hilfe der Götter auch wieder am Leben der Erde teilhaben können, wenn Thot nicht zu viel versprochen hatte. Und seine Mutter bald eine Gesangspause machte. Osiris hatte grosses Vertrauen in Thot, schliesslich hatte er ihn alles gelehrt, was er wusste\dots
 
 "`Das Mädchen? Es tut mir leid, Isis, aber das Mädchen ist weggelaufen. Sie weiss nichts von ihrer Aufgabe!"' Thot senkte verlegen den Kopf. "`Was?"' rief Isis entsetzt und sprang auf. "`Ich höre wohl nicht richtig!"' Sie funkelte Thot aus ihren schwarzen Augen an. In was für ein Abenteuer hatten sich diese verrückten Götter, oder Männer, was in diesem Fall das selbe war, jetzt wieder gestürzt? "`Ich bring Euch um!"' zischte sie. '-Haha!' meinte Osiris matt, wurde aber sofort ernst, als er das Gesicht seiner Frau sah. 
 
'-Isis, Schatz, lass' es dir erklären!' Osiris blickte aus grossen Augen und wandt sich hilfesuchend zu seinem Freund um. "`Ja, liebe Isis, in der Tat, sollten wir dir wohl einiges erklären\dots"' meinte auch Thot. "`Ich geb' euch fünf Minuten, bevor ich Euch den Kopf abreisse, beiden!"' Thot schluckte und senkte den Blick und fand seinen Bauchnabel Aug` in Auge mit Isis. Diese hatte die schlanken Arme verschränkt, den Kopf im Nacken starrte sie zu ihm hoch, während sie mit den Zehen ungeduldig auf den Boden klopfte. Selbst wenn sie wütend war, blieb sie zierlich und klein. Was sie nicht daran hinderte neben ihrem Gemahl die mächtigste Göttin zu sein. Sie sieht so hübsch aus, wenn sie wütend ist, dachte Thot. Kleine Blitze stoben um ihr Haupt mit den kräftigen, langen schwarzen Haaren, die Luft knisterte. Ihre kohlrabenschwarzen Augen sprühten und ihre vollen Lippen waren zusammen gepresst. Ihr schlanker Körper steckte noch immer in einem weissen, ägyptischen Leinenkleid und dem bunten Perlenkragen. Sie hatte keine Zeit gehabt sich wärmer anzuziehen, weil sie seit ihrer Ankunft ihren Gatten umsorgt hatte.
 
"`Berta hat uns eingeladen, weil es mit Amélie, so heisst das Mädchen, ein Problem zu geben scheint. Berta hat das Mädchen die letzten Jahre betreut und beobachtet. Sie ist vielversprechend. Aber wie gesagt, es ist ein Problem aufgetaucht, dass Amélie sich in Ägypten zugezogen haben muss. Wir haben nicht herausgefunden, was es ist. Und weil wir aus der Entfernung nichts sehen konnten, hatten wir die Idee nach Basel zu gehen\dots"' "`So, hattet ihr?"' fauchte Isis. "`Weil eine vielversprechende Schülerin von Berta, die ich, wie ihr wisst, sehr schätze, ein Problem hat, schleppst du, Thot, meinen Mann aus dem Jenseits nach Basel? Kannst du mir einen Grund nennen, warum ich ihn nicht noch in dieser Stunde wieder in seinen Sarg stecke und zurück in die Duat bringe?"'

 '-Weil ich ihn darum bat, mich hierher zu bringen, Schatz!' Osiris Stirn glänzte vor Anstrengung und seine Gedanken waren schwach und leise, aber bestimmt: '-Dies könnte meine Chance sein, dem Fluch meines Bruders endlich etwas entgegenzusetzen, endlich wieder die Erde zu spüren!' Isis war still. Eine Träne lief langsam ihre Wange herunter. Sie ballte ihre Fäuste.
 
 "`Isis, \dots"' "`Still! Ich will nichts hören. Auch wenn ihr Götter seit, meine beiden Herren, könnt ihr trotzdem vorher fragen, ob ich einverstanden bin, denn ihr wisst selbst, ohne meine Hilfe kommt ihr nicht aus."' "`Isis, \dots"' "`Still! Kein Mucks."' Schrie sie. "`Schaff' die Göre wieder her und schau, Thot, wie du sie in den Griff bekommst, bevor unsere Zeit abgelaufen ist. Und ich"' sagte sie und blickte streng auf ihren Gatten "`werde versuchen dieses Häufchen Elend am Leben zu erhalten."' Sie wischte sich die Träne ab und griff nach einem grossen Löffel, füllte ihn mit einer blutroten Flüssigkeit aus einer Glas-Phiole und leerte ihn behutsam und gleichsam zornig in den Mund des ergebenen Osiris. Währenddessen schlich sich Thot aus dem Zimmer und ging in den Garten.



\section*{5}
\addcontentsline{toc}{section}{5}

Sie trafen am Teich bei der Marmorbank zusammen. Thot und Anubis, sein engster Weggefährte und göttlicher Bestatter. Im Gegensatz zu Thot, der mit seinem menschlichen Äusseren reiste, hatte sich der ruhige und besonnene Totenwächter für seine Hundegestalt entschieden. Kein Gepäck, ein robuster Magen, einen guten Riecher und keine Probleme mit den sanitären Anlagen, Anubis wusste seine Hundegestalt sehr zu schätzen. Ausserdem fand er seine Hundegestalt viel kleidsamer. Er war ein grosser, schwarzer, dem ägyptischen Ideal folgend schlanker und langbeiniger Hund mit grossen Ohren.

Wie die Horusbrüder, deren Gestalten unterschiedliche waren, teilten sich Anubis und Thot über Gedanken mit, was für sie nichts ungewöhnliches war. Im Moment hockte Anubis neben Thot, der auf der Mamorbank sass, am Boden und schaute so traurig drein, wie es seine Hundeschnauze zuliess. Sie seufzten beide. Anubis Ohren drehten sich in die Richtung des Falkenrufes, den nur seine feinen Ohren gehört hatten. 

"`Wir müssen Amélie wiederfinden, alter Freund. Alleine kann sie die Sphäre nicht betreten, die Alessandro und ich um die Häuser aufbauen mussten, damit die göttlichen Herrschaften hier urlauben können."' Thot war der Meinung, er könnte besser denken, wenn er die Dinge beim Namen nannte, deshalb beschränkte er sich nur auf die stille, gedankliche Zwiesprache, wenn es die Situation erforderte. "`Ich habe bei der Suche an Amset gedacht, die beiden scheinen sich etwas kennengelernt zu haben"' Anubis schnaufte. -'Lieber Thot, ich glaube, wegen Amset ist die Amélie weggelaufen. Wenn er auch keine Schuld hat, so ist er wohl ein Grund. Er bringt sie durcheinander. Seine Brüder noch dazu.' "`Ach, ja, Menschen können sehr schnell empfindlich werden, wenn die Gegensätze ins Spiel kommen\dots Dennoch, wir brauchen die Hilfe der Brüder."' -'Ich weiss, ich habe Kebechsenuef schon gerufen und Amset und Duamutef. Hapi fällt in der Stadt als Pavian zu sehr auf. Ausserdem ist er dabei seine Kinder und seine Frau, die von der Reise durcheinander sind, zu beruhigen.' 

In dem Moment ertönte, wie auf Bestellung, ein lautes Geschnatter hinter der Bank im Gebüsch und ein Pavianmännchen, verfolgt von einem Pavianweibchen, das einen Stock in der Pfote hielt mit dem es auf den Kopf des Männchens einschlug, stürmten hervor. Das Weibchen kreischte. Das Männchen versuchte vergeblich einerseits beschwichtigende Gesten zu machen und sich gleichzeitig die Pfoten schützend über den Kopf zu halten. Drei kleine Paviankinder stürzten, gleichfalls lärmend aus dem Gebüsch und tobten auf den Teich zu. Dessen Wasser kräuselte sich. Das Pavianmännchen und seine Frau packten die drei kleinen Äffchen, klemmten sie unter die Arme und verschwanden wieder im Gebüsch. Das Kreischen des Weibchens wurde wieder etwas leiser. Die Wasserringe im Teich verschwanden seicht.

Anubis, der sich flach auf den Boden gelegt hatte und von Thot die empfindsamen Ohren zugehalten bekommen hatte, blickte auf.-'Ja, ich denke, der Hapi wird wohl keine Zeit für die Suche haben, solange seine Frau mit dem Urlaubsort nicht einverstanden ist.' Anubis seufzte: 'Ich habs ihm gesagt, Hapi, habe ich zu ihm gesagt, deine Frau ist eine kluge, aber gewöhnliche Äffin, sie wird keine Freude am winterlichen Basel haben. Hab` ich ihm gesagt. Aber er wollte sie mit den drei Kleinen nicht alleine lassen.'

Einen kleinen Augenblick sassen, bzw. lagen, die beiden Freunde ganz still, einzig ein kleines Plätschern war aus dem Teich zu hören. Da landeten zwei Falken. Der ein von ihnen verwandelte sich in einen Mann. Einen kräftigen, muskulösen Mann mit braunem, kurzen Locken und lapislazuliblauen Augen. Der dunkle Wimpernkranz seiner Augen hinterlies einen Schatten, als ob die Augen geschminckt wären. Er war barfuss. Er trug eine beige-braune Jeans und ein sandbraunes T-Shirt mit schwarzgrauen Tupfen, passend zu dem Gefieder des anderen Falken, der sich auf seine Schulter gesetzt hatte und eine Maus im Schnabel trug. 

"`Na, ihr beiden Hübschen, seit ihr wieder am Welt erfinden?"' lachte er. "`Guten Morgen, Horus, wie ich sehe habt ihr beiden schon eine Rundflug über das Rheinknie gemacht,"' bemerkte Thot. "`Jep!"' Der Falke auf der Schulter verschluckte die Maus und liess einen Blutstropfen fallen, Horus wischte sich mit dem Handrücken versonnen über den Mund und hinterliess dort ebenfalls eine Blutspur. "`Es geht nichts über frische Luft unter den Flügeln am Morgen!"' -'Und eine leckere Maus, statt Müsli wie bei Muttern, gell, Papa?' "`Sei nicht so frech, Kebi,"' schmunzelte Horus und streichelte dem Falken sanft über die Kehle. 

"`Amélie ist weg!"' Unterbrach Thot Vater und Sohn "`Die Kleine gefällt mir, hat sich sofort auf die Feuerprobe gestürzt, was?"' -'Ja, aber Isis gefällt das nicht. Wir sollten sie schnell wieder finden,' antwortete Anubis. -'Kebi wir brauchen deine Hilfe und die deiner Brüder.' '-Alles klar, antwortete Kebi, ich rufe Amsi und Tef\dots ' 

Schritte ertönten aus dem Garten. "`Morgen Papa!"' Amset und Duamutef kamen aus dem Gebüsch, "`wir haben Hapi versucht zu helfen, seine Frau zu beruhigen und den Kleinen ein Winternest gebaut. Für Paviane ist es recht kalt."' "`Ich würde euch ja Suchen helfen, Jungs, aber ich muss das Haus und die Umgebung für Re und Osiris sichern und die Barkenfahrt planen."' liebevoll tätschelte Horus den Kopf des Schakals. Thot sagte: "`Ich habe noch einiges für Osiris vorzubereiten. Wenn ihr Hilfe braucht ruft mich, ich bin in meinem Labor."' Thot erhob sich und ging mit Horus in die Richtung des Hauses. "`Ich geh' wohl erstmal bei Isis vorbei"' hörten sie Horus sagen, "`manchmal braucht eine Mutter ihren Sohn\dots".

-'Also gut, ich werde euch begleiten. Ich denke, wir sollten Isfet mitnehmen,' meinte Anubis. -'Was, das Chaoskind?' Duamutef schlug unruhig mit seinem Schweif, -'die bringt alles durcheinander.' -'Ja,' antwortete Anubis, -'aber, wenn alles durcheinander ist, dann wirkt sie durchaus\dots ordnend.' "`Aber dann gehen wir mit ihr zusammen. Mir wäre es unheimlich, sie alleine in der Stadt zu wissen"' meinte Amset. "`Kebi kann von oben suchen, Tef Amélies Spur verfolgen."' -'Das ist ein guter Plan.' Sie trennten sich, der Schakal und der Falke verschwanden in die Stadt, Anubis und Amset suchten Isfet.

\section*{6}
\addcontentsline{toc}{section}{6}


"`Hey, Papa,"' Maat schlüpfte in das Zimmer ihres Vaters Re. Er hatte ein Zimmer zur Rheinseite mit einer schönen Aussicht über Kleinbasel und den Rhein gewählt. Re liebte Flüsse und Boote. Schliesslich war er selbst Kapitän einer Barke. Er hatte seine Sonnenbrille auf, es ging nicht anders, wenn er inkognito bleiben wollte, denn seine Augen leuchteten wie das hellste Sonnenlicht, was sie ja auch waren. Seine Haare waren lockig und dicht und wenn er sie nicht in einem Pferdeschwanz gebändigt hätte, stünden sie ab wie Sonnenstrahlen. So hatte sich aus feinen Haaren eine Art Corona um sein Haupt gebildet. Seine Kleidung hatte er lässiger als Thot gewählt, der stets in massgeschneiderten, schwarzen Anzügen steckte, und hatte eine Bluejeans an, Seglerschuhe und ein blaues Jacket unter dem er einen weissen Rollkragenpullover trug. Er sass in einem hohen, gemütlichen, ledernen Ohrensessel und lass die Basler Tageszeitung. Er schien sich zu amüsieren.

Maat brachte ihm einen Becher mit seinem geliebten Ceylon-Tea. Der Becher war offensichtlich mit der Hingabe eines Mädchens bemalt worden, das Rosa, Katzen und seinen Vater liebte, was z.B. an der Aufschrift "`Daddy is the best"' zu erkennen war. Re erkannte Stil, wenn er ihn sah. Er war der Meinung, dass die ehemaligen Besatzter Ägyptens, elende Räuber und Unterdrücker gewesen waren in gewissen Bereichen aber durchaus Stil besassen. Er nahm, wenn immer möglich um Fünf Uhr seinen Tee in eben diesem Becher\dots Im Urlaub, so meinte er, könnte, müsste man eine Ausnahme machen, daher nahm er den Tee heute früher, in der Hoffnung es bliebe Zeit für einen Zweiten.

"`Maat, meine Liebe, wie geht es, hast du dich an deine Feriengestalt gewöhnt?"' "`Nicht ganz Vater, der Körper einer 12 Jährigen ist nicht nur nützlich, sondern verwirrend. Ich habe das grosse Verlangen, dich jetzt fürchterlich anzuschreien, weil ich unbedingt mit in die Stadt gehen will, um diese Amélie zu suchen. Und es gefällt mir nicht, dass Isfet darf und ich nicht!"' Re seufzte "`Maat, wir müssen alle das eine oder andere Opfer bringen bei diesem Abenteuer, dennoch bin ich sicher, am Ende werden die guten Erinnerungen überwiegen. Du weisst, du kannst nicht einfach in der Stadt herumlaufen\dots "' "`Ich weiss,"' Maat schob die Unterlippe vor.

 "`Aber das Isfet darf, ist gemein!"' Sie stampfte mit dem Fuss auf. "`Nein, ist es nicht!"' Re sprach ruhig und gelassen, hinter dem Glas seiner Sonnenbrille blitzte es kurz auf. Maat setzte sich und nahm einen Schluck Tee, aus dem Katzenbecher. "`Deine Schwester ist oft genug der Störenfried, gönne ihr die gute Tat, die ihr hoffentlich gelingen möge, weil wir sonst ein grosses Problem hätten."' "`Du hast ja recht, Vater. Ich weiss nicht, wie die Menschen es ihr ganzes Leben aushalten mit all diesen Drüsen und Körperdingen. Kein Wunder benehmen sie sich merkwürdig und gegen jegliche Ordnung."' Maat hatte vorsichtig ihre Feder, die sie stets im Haar mit sich führte, hervorgeholt und sich gedankenvoll damit über die Wange gestrichen. "`Siehst du, meine Liebe, was bin ich froh, können wir diese aufregenden Ferien machen."' Re strahlte. Maat schaute verwundert auf die Feder und strich sich noch einmal damit über die Wange, sie strich mit den Fingern über ihr Gesicht, dann lächelte auch sie.
 
"` Vater?"' Maats Stimme war ernst und erwachsen:"` Als diejenige Kraft, die die Ordnung vertritt, muss ich es genau wissen. Ich muss genau wissen, wie wir uns an diesem Ort und in dieser Zeit aufhalten."' Re räusperte sich "` Du hast recht. Schliesslich bist du hier eingesperrt, weil dir auf keinen Fall etwas zustossen darf. Wenn dir etwas zustösst, würde die ganze Ordnung, die wir bei diesem Abenteuer sehr überstrapazieren, völlig zusammenbrechen und dann können nicht einmal wir Götter uns helfen."' Re lachte halbherzig auf. Es sollte die Worte weniger bedrohlich wirken lassen, was es nicht tat.

 "`Also, Vater?"' Maat richtet sich auf. "` Die Zeit in der wir uns befinden, ist die Zwischenzeit der Rauhnächte. Diese Zeit ist in Europa magisch und Berta ist eine der Hüterinnen dieses jährlichen Zeitabschnittes. Sie kann nicht alles, aber vieles, was in dieser Zeit passiert, lenken. Ausserdem hat unser König Geburtstag, ebenfalls eine magische Zeit,"' erklärte Re. "`Soll das heissen, wir müssen Berta in allem, was die Zeit betrifft freie Hand lassen?"' fragte Maat. Sie sah nachdenklich aus. "`Nicht ganz wir haben in den Tag- und Nachtstunden unsere eigene Zeit mitgebracht. Aber wir müssen uns genau an die Regeln halten. Wir müssen uns genau an das Amduat\footnote{Das Amduat ist die 'Schrift des verborgenen Raumes' und eines der ägyptischen Totenbücher. Aus Sicht der Götter ist es eine 'To-Do-Liste' für die 12 Stunden der Nacht, die der Sonnengott Re mit seiner Barke und seinem Hofstart in der Duat verbringt.} halten und dürfen nicht vom Protokoll der Nachtfahrt abweichen, dann sind wir,\dots dann bist du vor allem sicher."' "` Toller Plan! Glaubst du das wird klappen."' Jetzt war es an Maat, halbherzig zu lachen.
 
  "`Allein schon die Sache mit Isfet. Sie hat sich heimlich in die Barke geschlichen und sollte nicht hier sein!"' "`Ja,"' antwortete Re, "`du hast recht. Aber wenn Isfet es nicht getan hätte, dann wäre die Fahrt vielleicht heute Abend wieder zu ende!"' "` Und das Monster und der Zauberer?"' fragte Maat aufgeregt. "`Ja, auch denen werden wir selbst hier die Stirn bieten, wie jede Nacht, \dots

In diesem Moment klopfte es an der Tür und Horus kam schwungvoll hinein. "`Grossvater, wir sollten die Nachtfahrt durchgehen. Vor allem die erste Stunde, die wir mit Amélie zusammen fahren. Ausserdem ist der Rhein nicht ganz ohne\dots"' "`Gut, wie ich sehe, bist du voller Tatendrang, Enkel.  Maat, mein Schatz,\dots"' "`Bin schon weg, Vater. Ich glaube, ich nehme ein Bad\dots, schliesslich machen wir Urlaub. Bin gespannt, wie sich so ein Bad anfühlt."' Murmelte sie und liess die beiden verdutzt dreinschauenden Götter zurück. 

\section*{7}
\addcontentsline{toc}{section}{7}


Behutsam klopfte Thot an Osiris Tür -'Osiris? Bist du wach?' '-Komm rein.' Thot betrat den abgedunkelten Raum. Er war froh, Isis war nicht da, ihr wollte er erst wieder begegnen, wenn Amélie wieder aufgetaucht war.

-'Wenn die Nacht begonnen hat, wird es für dich leichter werden', tröstete Thot. -'Dann tauchen wir in die Zwischenzeit der Rauhnächte ein und die Heilige Nacht gibt zusätzlich Kraft und Schutz.'

In dem Moment bemerkte Thot die Tannenzweige, die ihren harzigen Duft verströmten. -'Hans hat dich besucht,' stellte er fest. -'Ja, er hat mir die Tannenzweige gebracht und etwas Mistel, daraus wird mir Isis einen Tee machen,' antwortete Osiris. -'Du wirst mit Isis sprechen müssen,' Thot seufzte, 'wenn unser Vorhaben glückt, wirst du nicht mehr der alte sein.' -'Das will ich schwer hoffen', Osiris sah Thot fest an, -'für sie wird es leichter werden\dots und für mich.' 

-'Ich habe eine gute Nachricht für dich, ich habe den Ort gefunden, an dem genug Lebenskraft fliesst, um dich aus der Vergangenheit in die heutige Zeit zu bringen.' Thots Augen glänzten, -'wenn der richtige Zeitpunkt gekommen ist, können wir im Bereich des Münsters die blockierte Zeit lösen und du kannst in die Gegenwart durchkommen.' -'Wenn Amélie ihre Aufgabe erfüllt.' -'Wenn Amélie ihre Aufgabe als Menschenvertreterin erfüllt.'

\section*{8}
\addcontentsline{toc}{section}{8}

Die Verkäuferin in dem grossen, mehrstöckigen Buchladen wurde unruhig. Amélie sah, wie sie mit einer Kollegin zu tuscheln anfing und in ihre Richtung zeigte. Immerhin habe ich hier zwei warme, unterhaltsame Stunden verbracht, dachte Amélie. Und war nicht allein, fügte sie hinzu und schluckte.

Sie wusste nicht wie lange sie durch das Gittertor auf das blaue Haus gestarrt hatte. Und wie oft sie all die Namen gerufen hatte, an die sie sich erinnern konnte. Sie wollte vergessen, wie laut sie nach Berta gerufen hatte und, dass sie heimlich geweint hatte. 

Ihren Aufenthalt hier in der fremden Stadt, hatte sie Berta zu verdanken. Das hiess, sie musste versuchen, auf Berta-Art an die Dinge heranzugehen: Sie hatte den Groll weg geschoben und die Verzweiflung und dann hatte sie das Tor und den Innenhof noch einmal betrachtet\dots Aber Amélie konnte nichts entdecken: Keinen Garten, und kein Leben. Das Haus schien unbewohnt. Amélie hatte sich durch die Gasse einen Weg um das Haus herum gesucht. Es stand in einer Häuserzeile, aber neben dem Nachbarhaus links, führte eine schmale Gasse zur Vorderfront des Hauses.

Amélie war den Hügel bergauf gestiegen, bis sie wieder vor dem blauen Haus stand, vor dem sich die Gasse zu einer Terrasse über den Rhein öffnete. Das Haus war offensichtlich ein Amtsgebäude und die Tür verschlossen. Auch bei dem Nebenhaus, das mit 'weisses Haus' angeschrieben war, waren die Türen versperrt. Auch dieses Haus war ein Amt. Amélie versuchte durch die Fenster zu schauen. Sie  hatte gerufen und an die Tür gehämmert, bis einige Passanten stehen gelieben waren und ein Mann pöbelte, er würde gleich der Polizei anrufen. Amélie hatte überlegt, ob sie weiter randalieren sollte, vielleicht konnte die Polizei ihr helfen? 

Ich muss einen Ort finden, an dem ich nachdenken kann. Und an dem ich nicht erfriere! Auf diese Weise war Amélie in den Buchladen geraten. Und nun fiel sie auf, nach zwei Stunden, welch Wunder! Sie hatte keine Jacke dabei und ihre Füsse steckten in Plüschpantoffeln. Ich seh' aus, als wäre ich wo ausgebrochen, dachte sie. Bin ich auch! Eben, meldete sich die Amélie-Vernunftstimme: Und deshalb hast du keinen Ausweis und kein Geld und keine warmen Kleider. Und dummer Weise bist du in einer fremden Stadt, in einem fremden Land und die einzigen Menschen, Personen,\dots Wesen, die du kennst, sind scheinbar aus einem ägyptischen Museum ausgebrochen, toll!

Amélie unterbrach ihre Gedanken und versuchte unauffällig zur Rolltreppe zu schlendern und zwar möglichst schnell, denn die Verkäuferin war mit ihrer Kollegin im Anmarsch. Ob die dritte, die zu ihr von der Kasse herüber sah und zum Telefon griff, die Polizei anrief, wollte Amélie nicht fragen. Amélie verschnaufte erst einige Läden weiter. Über dem Geschäft hing eine Uhr, es war kurz nach drei. Noch eine Stunde, dachte Amélie, dann werden die Läden geschlossen und dann ist für alle heilig Abend, ausser für mich. Ich? Ich bin allein, \dots trotzig wischte sie die Träne aus dem Augenwinkel. Sie spürte die Angst, sie wurde von Minute zu Minute stärker. Die Geräusche wurden lauter, je mehr sich die Strassen leerten. Die Menschen hasteten mit ihren letzten Weihnachtseinkäufen blicklos an ihr vorbei. 

\section*{9}
\addcontentsline{toc}{section}{9}


"`Isfet, das ist jetzt der zweite Laden, wo sie uns rausschmeissen! Was mir eigentlich egal ist, aber wir müssen Amélie finden."' "`Amsi, mach kein Stress, solange so viele Leute herumlaufen, finden wir sie eh nicht. Wir warten einfach, bis alle Läden schliessen und suchen dann."' Isfet hopste vergnügt vor Amset her, der missmutig hinterherstapfte. Anubis und Duamutef hatten sich alle Mühe gegeben Amélies Spur zu finden, aber es waren zu viele Menschen und Gerüche.

-'Amsi?' Ertönte die Stimme des Falken hinter Amsets Stirn. "`Isfet sei mal still, Kebi meldet sich!"' -'Amsi, ich muss zum Haus zurückfliegen. Es wird zu dunkel.' -'Hast du denn irgendwas gesehen?' fragte Amset verzweifelt, -'Nein, es sind zu viele Menschen. Bis später!' Ein Falkenruf ertönte hoch über ihnen. "`Hab' ich doch gesagt!"' Grinste Isfet.

-'Das Problem ist', mischte sich Anubis ein, -'Wir können Amélie nur finden, wenn sie bereit ist.' "`Was heisst denn das?"' Amset raufte sich die Haare. -'Amélie muss die Feuerprobe bestehen und das muss sie alleine tun. Sie muss den Schleier der sinnlichen Welt lüften, lüften wollen. Solange sie an ihrem Elend hängt und zagt und kämpft, ist ihr der Weg versperrt und uns auch. Und solange sie wie wild herumläuft und den Rückweg in dieser Zeit und in diesem Raum sucht, auch. Nur, wenn sie bereit ist, uns in ihr Schicksal einzuladen, können wir sie finden und sie zurückbringen.'

"`Und wir können nichts tun? Wir sind doch Götter?"' Amset kickte wütend eine leere Dose gegen die Hauswand. -'Nein, denn wenn wir es könnten, wie könnten die Menschen einen freien Willen haben?' "`Ich finde es gut, ich hab nämlich Hunger"' mischte sich Isfet ein. "`Ich will erst mal was essen! Bevor es weiter geht."' "`Isfet! Nein!"' "`Alles zu seiner Zeit!"' meinte sie und war in dem Burgerlokal verschwunden. Amset folgte ihr. Anubis rümpfte die Nase und wartete widerwillig am Eingang. Duamutef tat es ihm gleich. Er leckte sich über die Nase. Es roch himmlisch: Nach altem Fett, Fleisch und Dingen, die lange in Öl geschwommen waren und Essensresten, genau das richtige für einen Aasfresser.

\section*{10}
\addcontentsline{toc}{section}{10}

Amélie war erschöpft. Sie fand einen Brunnen, was in dieser Stadt nicht schwer war und trank. Das eiskalte Wasser belebte sie und füllte den leeren Magen. Ich muss mich konzentrieren, sonst bleibt mir nichts anderes übrig, als zur Polizei zu gehen, überlegte Amélie, aber das kann nicht die Lösung sein. Die Frage ist: Warum schickt Berta mich nach Basel zu einem wirklich durchgeknallten Haufen von ägyptischen Wesen? Bestimmt nicht, damit ich den Basler Polizeiposten kennenlerne. Amélie musste grinsen und irgendwie fühlte sie sich dadurch besser. 

Es ging um den Traum\dots aber für die Lösung des Traums brauchte sie scheinbar die ägyptischen Götter. Wie finde ich Götter? Was hatte Berta mir gesagt? 'Kind, wenn du etwas wirklich willst, dann gehe los und öffne alle Türen, die du finden kannst, aber warte, bevor du dich für eine entscheidest, ob dir von dort ein Wink entgegen kommt. Lass' den Nornen den Platz, den sie zum Spinnen  des Schicksals brauchen. Erzwinge nichts, aber sitz' auch nicht wehleidig herum, dann bekommst du alles, was du willst.'

Die Kirchenglocken begannen, eine nach der anderen zu läuten. Als Amélie sich umsah, bemerkte sie wie viele Läden schon geschlossen hatten. Sie schlang die Arme um sich. Warum hatte sie sich am Morgen nicht für den dicken Wollpullover entschieden, sondern für das schwarze, kurze Kordkleid, schwarze Strumpfhosen und einen weissen Rollkragen? Zum Glück habe ich, bevor ich in den Garten ging, die dünne Wolljacke angezogen, dachte sie.
 
Amélie stand still. Hinter dem Brunnen führte eine schmale Gasse den Hügel hinauf. Wie finde ich die Götter? Und dann wurde ihr klar: Sie konnte die Götter nicht finden! Jedenfalls nicht, indem sie durch die Stadt lief. "`Lasse den Nornen einen Platz\dots"' hatte Berta gesagt, aber was meinte sie damit? Wie kann ich den Göttern Platz machen? Berta hatte Amélie gelehrt still zu sein, Innen drinnen. "`Ist das Meditation?"' Hatte Amélie Berta damals gefragt und war sich sehr schlau vorgekommen. 

"`Nenne es wie du willst, Kind."' hatte Berta geantwortet "`Wenn du nicht weiter weisst, stell` dir vor, du bist ein Topf und warte. Die Götter und Nornen können nicht abwarten, etwas hinein zu tun. Sie wollen dich beschenken, du musst halt nur still halten. Du darfst Dich nicht wundern, oder denken, das gibt es nicht. Sei offen und stabil wie ein Topf."' Amélie musste kichern, nicht zum ersten mal. Sie sah die kleine, kugelrunde Berta vor sich, wie sie mit ihrem Stummelfinger vor ihrer Nase fuhrwerkte und mit beschwörender Stimme sagte:"`Sei ein Topf!"' Und dann fiel ihr die Ohrfeige ein, die Berta ihr gegeben hatte, als sie vor Lachen losgeprustete hatte. Ohne zu überlegen strich sie sich über die Wange, als ob diese schmerzte.

Und dann hielt Amélie still. Sie schloss ihre Augen. Ihre Zähne klapperten vor Kälte. Ihr Körper schlotterte. Sie spürte den eisigen Luftzug, der ihr durch Gesicht und Haare strich. Ihre Finger fühlten sich tot an. Schritte hörte sie. Sie hatte das Gefühl davon zu schweben\dots Dann mahnte eine Stimme, die verdächtig nach Bertas klang. "`Hab` ich was von träumen gesagt? Nein! Also! Und Zungenspitze an den Gaumen!"'

 Ach, ja, Amélie klappte die Zunge hoch und konzentrierte sich noch einmal. Diesmal spürte sie die Kälte nicht mehr und wusste dennoch, sie war da. Sie war sie selbst und eine zweite Amélie, die ihr dabei zuschaute, Amélie zu sein. Sie spürte ein Licht, eine Wärme in ihrer Brust. Ein kleines Fünkchen, das wuchs, -ich schaffe es! Freute sie sich und sofort wurde ihr kalt, aber es gelang ihr, alle Gedanken wieder zu verscheuchen\dots

"`Amset?"' Isfet blieb auf der Strasse stehen. Amset, der völlig in seinen Cheeseburger vertieft war, wer ist das nicht, wenn er zum ersten mal einen isst, lief prompt in sie hinein. Der Cheeseburger, der heruntergefallen war, hinterliess auf Isfets pinkem Pullover einen Ketchup-Käsefleck und einen zufrieden schmatzenden Duamutef. Anubis verzog angeekelt die Lefzen, 'Gott hin oder her, Schakale frassen wirklich alles!' Dachte er.

"`Was!"' fragte Amset gereizt. Er hatte die Nase voll von diesem ersten Ferientag. Er mochte Amélie und anstatt den Tag mit ihr zu verbringen, hatte er sie mit seiner chaotischen Grosstante Isfet in dieser riesigen, kalten Stadt gesucht. Der Cheeseburger war ein klitzekleiner Lichtblick gewesen, den jetzt die Schwerkraft und sein Bruder vernichtet hatten. "`Ich kann sie spüren!"' Isfet packte Amsets Arm und sah ihn mit leuchtenden, schwarzschillernden Augen an. "`Was? Wen? Amélie?"' fragte Amset aufgeregt. Anubis spitzte die Ohren und wedelte mit dem Schwanz. Er und Duamutef schnüffelten aufgeregt an der Göttin. "`Ich spüre sie! Sie hat die Prüfung geschafft!"' jauchzte Isfet. "`Ich hab es gewusst!"' 

 Die drei anderen sahen zu. Isfet begann sich mit ihrem Oberkörper im Kreis zu bewegen. Ihre Arme schwangen durch die Luft und hinterliessen leicht schillernde Schleier. Die Füsse fest auf dem Boden, den Oberkörper weiter und weiter im Kreis schwingend wurde der Schleier aus zartem Blau grösser, stärker und dehnte sich aus. Die drei Bänder, die in einem weiten Kreis um die Göttin schwangen, wirbelten in wachsenden Bahnen hoch in den Himmel. Über der Stadt schlossen sie sich hoch oben zu einem riesigen Ring, der sich majestätisch drehte und begann golden zu schillern. Die blauen Bänder lösten sich von den Händen der Göttin, die selbst bläulich-schwarz zu schimmern begonnen hatte und wurden von dem drehenden Rad aufgesogen. Das goldene Rad, mit den drei blauen Speichen wurde kleiner und kleiner und verschwand hinter den Dächern. 
 
 Es wirbelte, gemütlich wandernd über die Stadt, es senkte sich neben dem Brunnen direkt über Amélie nieder. Wie eine Hülle stülpte sich das Rad über sie und umgab sie mit einem blau-goldenem Schleier. Amélie öffnete die Augen. Sie wusste, wenn sie jetzt die Tür des blauen Hauses wieder fände, dann würden die Götter dort sein. Amélie Gehirn rebellierte dagegen, "`Woher willst du das wissen? Hä! Wo ist der Beweis?"' quietschte es ängstlich. -Gib Ruhe, du Nervensäge, schalt sich Amélie selbst und suchte den Rückweg. 

Amélie staunte wie schnell sie den Weg zurück zum blauen Haus fand. Nachdem sie die kleinen Gasse hinter dem Brunnen bergauf gestiegen war, kam sie auf einen grossen Platz mit einer riesigen Kirche, die einladend die Heilige Nacht einläutete. Und nicht weit davon entfernt fand sie die vordere Eingangstür des blauen Hauses wieder.

Sie setzte sich auf die Treppe und lehnte den Rücken gegen die helle Holztür. Die Welt in der die Götter sind, muss eine andere sein, dachte Amélie. Sie schloss die Augen und stellte sich alle Götter, Wesen vor, die sie in dem blauen Haus gesehen hatte. Den schönen Amset, die lustige Isfet und die kühle Maat, Duamutef, den Schakal, den Falken. Sie dachte an Hathor, die rund und klein in der grossen Küche herumgekugelt war und ihr das Frühstück serviert hatte. Frühstück! Essen!

Amélie lief das Wasser im Mund zusammen. Genau in diesem Moment würde sie gleich zwei Portionen vom Fuul mudammas essen. An das Gericht aus dicken Bohnen hatte sie sich heute Morgen nicht gewagt, sondern sich über ein Omelette mit Fleischstückchen und Schafkäse hergemacht. Dazu gab es ein dünnes Fladenbrot und Hummus mit frischer, knackiger Petersilie, Hummus\dots, Fladenbrot\dots, Amélie schmatzte laut. 

Die Tür in Amélies Rücken gab nach und sie kippte in die Eingangshalle. "`Ja, wen haben wir denn da?"' fragten sie zwei riesige Nasenlöcher die zwischen einem dichten, wilden Bart hervor auf sie herunter schauten. Amélies Kopf war auf Hans nackten, erdigen Füssen gelandet. Im gleichen Moment wurde sie von der weichen, feuchten und nach Cheeseburger stinkenden Zunge des Schakals abgeschleckt. Amset und Isfet standen in der Tür und grinsten wie Honigkuchenpferde. Selbst Anubis liess sich dazu hinreissen mit dem Schwanz zu wedeln und kurz zu bellen.

"`Amélie, du hast es geschafft!"' Amset strahlte wie sein Urgrossvater, der Sonnengott, und so stolz, als hätte er die Prüfung bestanden. Duamutef und Isfet blinzelten sich zu. Und Duamutef seufzte, wie nur ein Canidae über Menschen, bzw. Götter seufzen kann. "`Natürlich hat sie es geschafft und nun husch, alle an den Tisch! Bevor die Barke ablegt, müsst ihr euch alle stärken!"' zwitscherte Hathor, die sich ihre Hände in der Schürze mit Kuhmuster abwischte und wieder in die Küche eilte. Amélie stand auf, wobei sie Amsets Hand, die er ihr reichte, geflissentlich ignorierte: "`Barke?"' fragte sie stattdessen. "`Schätze wir sollten machen, was die alte Kuh sagt, sonst gibt`s heute nichts mehr zu essen."' Isfet legte den Arm um Amélie "`Weisst du, dass wir ein gutes Team sind, wir zwei?"' fragte sie und schob Amélie Richtung Küche. -'Das befürchte ich auch,' stöhnte Duamutef, dem ein Tropfen an der Nase hing schlotternd und trabte mit dem grinsenden Amset ebenfalls in Richtung Küche.

"`Und?"' fragte Hans den schwarzen Hund, der wie eine Statue in der Halle gesessen und alles beobachtet hatte. -'Wenn sie will, ist sie schon sehr stark, das ist gut! Isfet konnte sie erstaunlich schnell finden. Und sie konnte das Rad des Schicksals, das Isfet ihr zur Hilfe schickte sogar selbst benutzten.' Anubis erhob sich und schlenderte in die Küche. Hans bemerkte ein leichtes Zittern, aber schliesslich durfte ein Gott frieren, wenn er in dem dünnen Fell eines vornehmen, ägyptischen Hundes Stunden in der Basler Winterkälte verbracht hatte.

\chapter*{1. Nacht}
\addcontentsline{toc}{chapter}{1. Nacht}

\begin{quotation}

\emph{I Verum sine mendacio, certum et virissimum:\\ 1. Wahr ist es ohne Lüge, unzweifelhaft und wahrhaftig:\\Tabula Smaragdina}

\end{quotation}

\section*{1}

Das war ein feines Essen. Wobei, das musste Amélie zugeben, ihr in dem Moment jedes Essen geschmeckt hätte. Sie hatte mit Isfet und Amset in der Küche ein reichliches Mahl genossen. Hathor war aufgeregt, denn offensichtlich, war der Tag noch nicht vorbei, sondern noch eine Ausfahrt geplant. Amélie konnte sich unter einer Barkenfahrt nicht viel vorstellen. Allerdings zitterte sie bei dem Gedanken wieder hinaus in die Eiseskälte geschleppt zu werden. Hathor hatte ihr einen bitteren Tee eingeflösst, der gegen die Kälte helfen sollte.

"`So meine Lieben, jetzt ist es Zeit, der Himmel wird schon rot."' Sie erhoben sich. Und zu Amélies Überraschung legte ihr Hathor einen warmen, flauschigen Pelz um die Schultern. Im Gänsemarsch begaben sie sich zu einer schweren Tür im Treppenhaus, die in den Keller führte. 

Der Kellerraum war gross und hoch. Wie in einer Kirche, dachte Amélie, die staunend auf der Treppe stehengeblieben war. Zwei Reihen von Säulen durchschnitten den Raum und bildeten ein hohes Gewölbe. An den Säulen befanden sich Halter, in denen Fackeln den Raum erhellten. Es roch muffig. Aber das war Amélie recht, denn sie fühlte sich in eine Märchenwelt versetzt und der Geruch des alten, muffig-modrigen Gemäuers brachte Wirklichkeit. Sie folgte Duamutef quer durch den Raum unter den Fackeln hindurch bis zum hinteren Winkel. Dort war eine niedrige Tür, die in einen schmalen, engen Gang führte. Wenn Amélies Orientierungssinn sie nicht betrog, mussten sie sich unter der Strasse befinden, unter der Terrasse, von der aus sie auf den Rhein geschaut hatte. Tatsächlich kamen sie unterhalb der Terrasse wieder aus dem Gang heraus. 

Sie durchquerten einen verwilderten Garten, der ebenfalls auf mehreren Terrassenstufen angelegt war und erreichten schliesslich das Wasser des Flusses. Amélie stand und staunte. Es dämmerte, am Winterhimmel erschienen einige zarte, rosarote Streifen. Der Abendstern leuchtete und die blaue Stunde legte ihren Schleier über die Stadt und den Fluss. Ein leichter Abendwind strich Amélie durch das Haar, war aber nicht kalt. Amélie fühlte sich leicht und warm. Einige Möwen kreischten. Sie liessen sich ein Stück den Rhein bergab treiben und flogen dann wieder auf und liessen sich ein Stück weiter oben wieder auf dem Fluss nieder. 

Am Ufer lag ein Boot. Ein grosses Boot aus Schilf, -die Barke! dachte Amélie. Die Schilfgräser schimmerten gelblich. Bug und Heck der Barke schwangen in den Himmel. Die baumstarken Schilfrollen, aus denen das Schiff bestand waren an den Enden mit starken Seilen zusammen gebunden. Am Heck waren die Steuerruder zu sehen und die Umrisse von zwei Steuermännern, die die langen Stangen festhielten. Es knackte leise und quietschte, wenn das Schiff am Steg sachte auf und ab schaukelte und an das Holz gedrückt wurde. Das Wasser gluckste dazwischen. Der Fluss roch. Der Geruch erinnere Amélie an das Meer, Algen, Schlick und Wasser, das ein ganzes Land durchquert hatte. Der Geruch war alt.

Auf der Barke waren sie alle! Alle Götter, die aus Ägypten angereist waren, waren auf der Barke versammelt. Die Götter trugen nur einen weissen Leinenschurz, der ihnen bis zu den Knien reichte. Um ihren Hals waren kostbare Kragen aus Edelsteinen und Perlen gelegt. Ihre braune Haut glänzte wie poliert in dem blauen Licht. Die Göttinnen hatten lange, enge weisse Kleider an. Auch sie waren mit schweren Edelsteinkragen um den Hals geschmückt. Wie die Götter trugen sie unterschiedliche Zepter in den Händen. Die Göttinnen trugen ihre schwarz-glänzenden Haare offen. 

\section*{2}
\addcontentsline{toc}{section}{2}

Staunend kletterte Amélie in die Barke, die sich von der Strömung des Flusses ungeduldig hob und senkte. Amélie gelangte über einen kleinen Steg in den  vorderen Teil des Bootes. Der Boden der Barke bestand aus Holz. An den Seiten der Barke waren Bänke für die Mannschaft angebracht. Die Barke ist riesig, dachte Amélie, und ich hatte an ein kleines Ruderboot gedacht. Dabei ist dies Schiff gross genug, um ein kleines Haus zu transportieren. Wahnsinn! Cool!

In der Mitte der Barke war ein Baldachin. Darunter mit leuchtendem Glanz umgeben stand der grösste Gott. Amélie hatte ihn noch nie gesehen. Er trug einen mächtigen, gehörnten Widderkopf. Zwischen den Hörnern trug er eine Sonnenscheibe aus Gold. Einen Anch trug er in der Hand. Neben ihm stand Hathor und nichts erinnerte an die kugelrunde, gutmütige Grossmutter, die mit einer Kuhschürze durch die Küche fegte. Mit majestätischem Lächeln stand sie neben ihrem Gatten, um einiges grösser und schlanker. Aus ihrem Haupt ragten die Hörner einer Kuh. Sie waren lang und schön geschwungen. Das Horn glänzte schwarz und poliert. Eine Sonnenscheibe aus Gold thronte zwischen den aufragenden Hörnern. 

Amélie erkannte Isis, die vorne auf der linken Seite der Barke auf einer Bank sass. Sie hatte einen Stab in der Hand, sass kerzengerade und mit ernstem Gesicht. Sie trug einen Reif aus Gold auf dem Kopf, der die Form einer Kobra hatte. Die Schlange umringelte das Haupt der Göttin und richtete sich auf ihrer Stirn bedrohlich und blähend auf. Während Isis auf den Fluss sah, wendete die Schlange züngelnd den Kopf und witterte zu Amélie. 

Maat stand vorne am Bug. Sie hatte goldene Armreife an den Oberarmen, die im letzten Tageslicht blitzten. Ihre schneeweisse Feder hatte sie mit einem Band aus Goldfäden am Kopf befestigt. Amélie roch einen zarten Duft nach Badesalz und Rosenblüten. Sie war wie alle Göttinnen im besten Frauenalter und von dem 12 jährigen Mädchen, war keine Spur zu sehen. 

Hinter dem Sonnengott bemerkte Amélie mit Grausen eine kräftige, männliche Gestalt mit breiten Oberkörper und starken Armen, die anstelle eines Menschenkopfes, den eines Falken auf den Schultern hatte. Breitbeinig stand er hinter dem Widderköpfigen, wie ein Leibwächter. Er wendete unmerklich den Kopf und sah Amélie direkt mit seinen Falkenaugen an. Ihr Herz begann wild zu klopfen, so musste sich die Maus fühlen, kurz bevor sie von einem Raubvogel gepackt und in die Höhe gerissen wurde.  

Duamutef war an dem Baldachin vorbei geschlüpft und im Dunkel, das sich immer schneller herabsenkte, nicht mehr zu erkennen. Einer der Ruderer, der eine menschliche Gestalt hatte und auch einen Schurz um die Hüften gebunden hatte, löste das Seil. Die Strömung war stark und die Barke glitt schnell in die Flussmitte. Es schwankte.

Amélies Beine gaben nach und sie lies sich auf die Schiffsplanken sinken, bevor sie fiel. Amèlie blieb liegen und atmete den Holzduft der Planken ein. Sie spürte mit den Fingern das Holz, das glatt poliert war und sich warm anfühlte. Vorsichtig zog sie sich an der Bank hoch, die auf beiden Seiten für die Passagiere angebracht war und setzte sich. Sie dachte an Amset. Ob er auch auf der Barke war? Wie sah er aus? Mit Schaudern blickte sie zum Baldachin unter dem der Sonnengott mit dem mächtigen Widderschädel und der menschlichen Gestalt majestätisch hoch erhobenen Hauptes stand.  Und mit seinem Zepter, ein langer Stab mit einem Schlangenkopf am Ende den Ruderern Anweisungen gab.

Amélie blickte auf. Die Häuser und Uferbefestigungen waren mit Nebel verhüllt. Die kalte Luft strich über ihre Wangen und sie spürte den Pelz, der sie wärmte und ihren Hals kitzelte. Das Wasser glitt schnell an der Barke entlang, es gluggste, wenn eine kleine Welle sich im Schilf der Barke verfing. Amélie sah einen Stern am samtschwarzen Himmel leuchten. 

Sie reckte den Kopf weit in den Nacken und ein Stern nach dem anderen tauchte auf. Wie die Lichtpunkte auf dem Fluss, nur das die Sterne an ihrem Platz blieben. Schneller und schneller tauchten die Sternenlichter auf. Amélie staunte wie viele es waren. Wer hat gesagt, in der Nacht sei es dunkel, wunderte sie sich. Zart und wie ein durchgehender Schleier, enthüllte sich die Milchstrasse. Amélie schnappte nach Luft, sie hatte vergessen zu atmen. Sie klappte ihren Mund zu, er musste die Zeit über offengestanden sein\dots 

Sie sah sich um, die Häuser zu beiden Seiten waren verschwunden! An den Ufern waren Bäume zu erkennen. An manchen Stellen lichtete sich der Wald und reichte nicht ganz an das Ufer. Hinter dem Wald erhoben sich zu beiden Seiten Hügel. Die Hügel von Basel. In die eine Richtung ragte auf der linken Seite das Hörnli weit hinter der Flussbiegung auf. An der Kannte an der höchsten Stelle schimmerte ein Tempel hell am dunklen Waldrand. Und direkt vor der Barke auf der rechten Seite wuchs der Felsen, der nun Münsterhügel genannt wurde steil in die Höhe.

Die Barke drehte sich sanft und begann leicht mit der Strömung zu schwimmen. Amélie spürte die Wellen und die Strömung stärker. Sie hörte ein leises Murmeln. Je mehr sie auf das Murmeln hörte, umso kräftiger wurde die Vibration an ihrem Körper. -Das ist das Wasser! Das Geräusch vom Wasser, es ist so tief und gewaltig, meine Ohren hören es nicht. 

Die Barke glitt langsam weiter. An den Ufern erschienen einige Wildschweine. Zwischen den Schweinen bemerkte Amélie im fahlen Licht eine kräftige Frauengestalt. Nur ihren Schatten mit den langen Haar und dem weiten Rock und den winkenden Arme konnte Amélie erkennen. Ein Stück weiter tauchten Rehe und Hirsche am Ufer auf und zwischen ihnen eine grosse, kräftige, zottige Gestalt, die Amélie trotz der Dunkelheit bekannt vor kam. Die Bewegungen und der Geruch, der vom Ufer her wehte\dots 

Die Barke stoppte schliesslich an einem Felsen, der dicht am rechten Ufer steil aufragte. Er schimmerte hellgrau in der Dunkelheit. Am Fusse des Felsen, dicht am Ufer, war eine dunkler Höhle. Amélie reckte sich, etwas war in der Höhle.

Eine Gestalt kam heraus und dann noch eine und noch eine. Unzählige Menschen! Einer nach dem anderen traten sie aus der Höhle. Kein Geräusch war zu hören, lautlos, füllte sich der Kiesstrand am Ufer mit Menschen. Amélie sah Kinder, Erwachsene und Alte. Sie sah Menschen, die aus einem Mittelalterfilm zu sein schienen und welche in ganz normalen Kleidern. Sie sah Menschen, die als einziges Kleidungsstück einen Pelz trugen und andere in einem einfachen Kittel und mit Sandalen an den Füssen. Vornehme Herrschaften, die eine gepuderte Perücke trugen und Kirchenmänner in schwarzen und weissen Gewändern. Sie alle versammelten sich und blickten still und erwartungsvoll zu der Barke. 

Der Sonnengott trat an den Bug des Schiffes, neben sich Maat und Hathor. Isis und der Falkenköpfige traten hinter die drei. Der Sonnengott hob die Arme und sein Zepter und begrüsste die Menschenwesen am Ufer huldvoll. Die anderen Götter machten es ihm gleich. Zwischen den erhobenen Armen der Götter begann ein rotgoldenes Glühen, das stärker und stärker in den Himmel leuchtete. Langsam senkten sie ihre Arme, der Glanz bewegte sich wie Seifenblasen an ihren Armen entlang und in die Höhe. Als das lichte Schillern der Götter die Gruppe der Menschen erreichte, begannen diese zu jubeln und zu klatschen. Die Barke kehrte sich aus der Flussmitte dem Ufer zu und hielt darauf zu. 

Amélie rutschte langsam unter die Bank. Dieser Tag war ihr ein Rätsel, von vorne bis hinten. Je näher die Barke ans Ufer kam, umso genauer konnte Amélie, die vorsichtig unter der Bank hervorschaute, die seltsamen Menschen sehen. Die Menschen waren nicht nur aus den unterschiedlichsten Zeiten und in den unterschiedlichsten Altern, sie waren nicht richtig. Was stimmt nicht? Amélie blinzelte. Die Menschen waren durchsichtig. Wie ein Hologramm. Sie waren unterschiedlich hell und dadurch waren die helleren besser zu sehen. An einigen Stellen, an denen Amélie niemanden erkannt hatte, konnte sie dunkle Konturen von menschlichen Gestalten erkennen. Einige schienen dichter zu sein, andere wie ein Nebeldunst.

Amélie wurde es plötzlich eiskalt, denn jetzt hatte sie in der ersten Reihe, der wartenden, einen Mann entdeckt, der in der erhobenen Hand, mit der er winkte, seinen Kopf hielt. Der Kopf trug eine schwarze Kappe und einen langen feuerroten Bart, der in zwei Strängen geteilt war. Zu seinen Füssen sassen zwei schwarze Hunde, die unruhig zur Barke blickten.

Was war denn das für ein Gruselkabinett? Amélie sah zufällig auf ihre Hand, die sich am Rand der Barke festkrallten. Sie waren holografisch. Sie sah die Hand, sie sah darunter das Schilf der Barke und in der Hand schimmerten die blaulichen und rötlichen Adern und das Blut bewegte sich darin\dots Sch\dots  Bin ich tot? war das letzte, was Amélie dachte, bevor es dunkel um sie herum wurde.   

