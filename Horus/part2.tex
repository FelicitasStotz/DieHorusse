\part*{Zweite Stunde\\"`Kluge, die ihren Herrn beschützt"'}
\addcontentsline{toc}{part}{Zweite Stunde}

\chapter*{2. Tag, Christi Geburtstag}
\addcontentsline{toc}{chapter}{25. Dezember, Christi Geburtstag}

\section*{1}

Amélie öffnete die Augen, was für ein Traum\dots

Sie lag im Bett. In dem quadratischen Zimmer in dem sie gestern aufgewacht war. - Ich bin nicht tot! Dachte Amélie. Sie zog die Hand unter der Decke vor und betrachtete sie. Sie war wie immer. Undurchsichtig, fest, warm und mit weicher Haut. Wo war Berta? Ich will nach Hause! Zugegeben, zu Hause auf dem Gutshof der Eltern im Norden Deutschland fühlte sich Amélie wie das fünfte Rad am Wagen, seit ihr Bruder geboren worden war. Aber da war Berta und ein Leben, das -fast- so war, wie es mit 16 Jahren sein sollte.

Es klopfte sachte an der Tür. "`Nein, Herr Amset, Sie dürfen nicht mit in das Zimmer von Fräulein Amélie. Sicher liegt sie noch im Bett!"' Amélie hörte Amsets protestierende Stimme durch die Tür. Dann entfernten sich Schritte. Ich habe Amset gestern Abend nach dem Essen nicht mehr gesehen. Oder doch? Wenn seine Brüder Falken, Schakale und Affen waren, in was hatte er sich verwandelt? Amélie schob den Gedanken weit weg. Sie hatte keine Zeit an Amset zu denken. 

Es klopfte wieder. "`Jaa?"' rief Amélie zaghaft, wer steht heute hinter der Tür? Isfet, Maat, Hathor? Jemand drückte mit dem Ellenbogen die Tür auf und als erster kam Duamutef ins Zimmer gesprungen. Er machte eine Runde, die Nase dicht am Boden und schnupperte an dem Pelz, der über dem Holzstuhl hing. Dann kam er zum Bett und starrte Amélie an. Er hob die Lefzen, als ob er mich angrinst, dachte Amélie und wedelte mit dem Schwanz. 

Amélie streckte die Hand nach ihm aus. Doch er trat sofort zurück. Und schüttelte unwirsch den Kopf. Amélie zog schnell die Hand zurück. Ich habe vergessen, er ist doch ein Schakal, ein Wildtier, dachte sie, ich habe ihn mit einem Hund verwechselt, peinlich. -'He! Ich bin kein Wildtier, ich bin ein GOTT! Don't-touch-a-jackal-God!' dröhnte es plötzlich in Amélie Kopf. Sie wurde knallrot.

 "`Entschuldige. Ich hielt dich für einen Hund\dots, Ooups!"' Duamutef grunzte und lief aus dem Zimmer. "He! Oh, nein! Sorry, Tschuldigung!"' "`Nun mache dir mal keinen Kopf. Wenn die ägyptischen Herrschaften sich im Schakalskleid zeigen, müssen sie sich nicht wundern, wenn unsereins sie verwechselt."' Während Amélie bei Duamutef ins Fettnäpfchen getreten war, war eine ältere Frau in das Zimmer getreten. Sie trug ein Tablett, auf dem ein grosser dampfender Becher stand. Er war weiss und das Basler Münster war darauf abgebildet. Es roch verführerisch nach heissem Kakao. Neben dem Becher lag auf einem blauen Teller ein grosses Croissant mit Nussstücken darauf. Die Frau stellte das Tablett mit einem aufmunternden Lächeln auf Amélies Schoss.
 
 Dann wendete sie sich dem Fenster zu und öffnete den Vorhang. Sie trug ein dunkelgrünes Kleid, das bis zum Boden reichte und unter dem eine weisse Bluse hervorschaute. Der Kragen war hoch geschlossen. Das freundliche Gesicht hatte einige Falten, die von einem arbeitsamen und schweren Leben zeugten. Die Lachfältchen, um Augen und Mund und das entschlossene Kinn zeigten, wie die Frau mit diesem umgegangen war. Die Haare hatte sie unter einem weissen Kopftuch verborgen. Sie trug das Tablett nicht wie eine Angestellte, sondern wie eine sorgende Gastgeberin. Sie reichte Amélie ihr kräftige, vom Arbeiten raue und sonnengebräunte Hand.
 
"`Ich bin die Wibrandis und schaue mit dem Hans zusammen, das unsere feinen,  ausländischen Gäste alles haben, was sie brauchen und nicht zu viel Durcheinander anstellen. Du bist die Amélie?"' Amélie nickte, sie hatte den Mund voll mit dem Schoko-Croissant und reichte Wibrandis stumm die Hand. "`Wenn du gegessen hast, bring ich dich zu Thot."' sagte Wibrandis und wischte sich die Hand automatisch am Kleid ab. Es fielen einige Croissantkrümel zu Boden.

 Amélie schluckte schnell und sagte: "`Wibrandis, sag\dots, bist du ein echter Mensch?"' Amélie merkte wie ihre Stimme zitterte. Wibrandis drehte sich zu ihr um und setzte sich zu ihr auf das Bett. Sie sah Amélie mit ihren ruhigen Augen an. "`Ja, ich bin ein Mensch!"' Amélie wollte etwas sagen, aber Wibrandis sprach weiter "`Wie du aber bemerkt hast, bin ich kein Mensch aus deiner Zeit. So wie du mich hier siehst, habe ich vor 500 Jahren gelebt."' Amélie schluckte, ihr Herz klopfte rasch, unmöglich!
 
Aber die Frau sass auf ihrem Bett. Sie strömte einen strengen Geruch aus, Deodorant schien sie jedenfalls keines zu kennen. Sie roch nach Schweiss und leicht nach Rauch, aber auch nach Erde und Arbeit. Sie war nicht dreckig, ihre Haut war sauber und rosig. Ihre Kleider schienen jedoch lange nicht gewaschen, nur gelüftet. 

"`Aber, wie funktioniert das? Du kannst nicht 500 Jahre alt sein. Also bist du\dots tot? Wibrandis, ich verstehe nicht, was los ist. Ich bin in einem Horrorfilm und weiss nicht wieso!"' Amélie hatte heftig mit ihrem Croissant durch die Luft herumgefuchtelt und biss wieder hinein. Alles voll verrückt, dachte sie, aber ich gewöhne mich dran.  

Wibrandis war aufgestanden und hatte begonnen, Kleider für Amélie aus der Kommode zu nehmen, sie legte sie auf den Stuhl und nahm den Pelz an sich, während sie sagte: "`Schau, ich bin von der hohen Frau gefragt worden, ob ich bei einer göttlichen Sache mithelfen könnte und ich habe ja gesagt. Ich hatte im Jenseits nicht viel zu tun. Frag den Thot, wie alles genau funktioniert. Ich habe versprochen für die Rauhnächte hier in Basel die Gäste zu bewirten. Dafür kann ich nicht als Geist herum schweben und deshalb durfte ich wieder in diesen Körper schlüpfen. Dieser Körper ist sehr praktisch, denn er hat viel erlebt und ausgehalten. Von einem 'Orrofilm' weiss ich nichts."'

 Wibrandis blieb in der Tür stehen, die Klinke in der Hand: "`Frage Thot, oder besser noch den Herrn."' Sie schloss die Tür. Und Amélie hörte die erboste Stimme von Wibrandis: "`Amseeet! Du hast vor Amélie Tür nichts zu suchen. Verschwinde, oder ich mach dir Beine!"' Amélie grinste, die gute Wibrandis liess sich von Göttern nicht ins Bockshorn jagen.  "`Autsch, ich gehe ja schon! Lass mein Ohr los!"' jammerte Amset und mehrere Schritte entfernten sich im Gang. -'Hihi,' hörte Amélie in ihrem Kopf und dann tappten Hunde-\dots, Verzeihung, Schakalspfoten an der Tür vorbei nach unten. 
 
 -Ich kann ihre Gedanken hören, Amélie strahlte, endlich mal eine Sache, die nützlich war. Ich glaube, ich glaube, ich werde es ihnen nicht sofort erzählen, dachte Amélie. Während sie in die Kleider schlüpfte, spürte sie gute Laune im Bauch aufsteigen und wusste nicht, ob es am Frühstück, an Wibrandis oder daran gelegen hatte, die Gedanken von Duamutef zu lesen.
 
\section*{2}
\addcontentsline{toc}{section}{2}
 
 Mit klopfendem Herzen stand Amélie vor Thot Zimmertür. Wibrandis, die wie aus dem Nichts erschienen war, als Amélie ihr Zimmer verliess, hatte sie zu dem Zimmer des gelehrten Gottes geführt. Amélie war froh darüber. Das Haus war gross und verwinkelt. Im Gegensatz zu der äusseren Fassade, war es im Inneren nicht symmetrisch, sondern verschachtelt und wirkte, wie aus mehreren verschiedenen Häusern und Stilen zusammen gesetzt. Sie waren durch eine Tür mit sehr breitem Türsturz ins Nachbarhaus gegangen.
 
 "`Herein"' rief Thot und Amélie betrat sein Arbeitszimmer. Der Raum war achteckig! Das Fenster, das in den Hof des weissen Hauses zeigte, brachte wenig Licht in das Zimmer. Ausser an der Fensterseite und der Wand der Eingangstüre, waren alle Wände mit Holzregalen bis unter die Zimmerdecke versehen, die alles Licht aufsaugten. Thot sass, das Fenster im Rücken an einem grossen, massiven Holzschreibtisch. Das Licht, das aus dem Fenster seine Silhouette nachzeichnete, liess sein Gesicht im Dunkeln, ein feiner Lichtkranz umgab ihn.
 
 Als Amélie das Zimmer betreten hatte und die Tür geschlossen, kam er hinter seinem Schreibtisch vor und bot ihr einen der drei hohen Ohrensessel an. Er setzte sich ihr gegenüber und schlug die langen, schlanken Beine übereinander. Neben der Tür bemerkte Amélie nun den grossen Hundekorb, in dem der schwarze Hund lag, der Amset, Isfet und Duamutef in die Stadt begleitet hatte.
 
 "`Anubis und ich wollen dir einiges erklären."' sagte Thot und zeigte auf den Hund, der den Kopf hob und Amélie aufmerksam anschaute. -'Guten Morgen, Amélie,' tönte es in ihrem Kopf und ohne zu überlegen sagte sie "`Guten Morgen! Oh!"' Anubis schnaufte zufrieden und Amélie wurde rot. Soviel zu ihrem Geheimnis. "`Es ist sehr schwer seine Gedanken zu kontrollieren, damit sie nicht gehört werden können, Amélie!"' Na, bravo, dachte sie, und dann, -oh, nein?!
 
 "`Ich schlage vor, wir benutzten die Sprache. Anubis wird sich natürlich weiter mit seinen Gedanken mitteilen. Aber du musst nicht besorgt sein. Es gehört bei uns zum guten Ton, haha, sich aus den Gedanken, der anderen heraus zuhalten, wenn man nichts zu sagen hat."' meinte Thot, der über Amélie Mimik schmunzeln musste, als ihr klar wurde, was es bedeuten musste, wenn alle Gedanken gelesen würden. 
 
"`Ich denke, du hast einige Fragen, nach dem gestrigen Tag?"' Amélie schnaubte "`Ich denke schon, was mache ich hier? Wer seit ihr? Und wo ist Berta? Und vor allem, warum Ich? Und bin ich tot?"' 

Thot beugte sich etwas vor. "`Verstehe! Du bist hier, weil Berta deine Träume aufgefallen sind. Sie handeln von Ägypten, vor vielen tausend Jahren, hat sie erzählt. Sie sind voller Details und Wissen, das du nur haben kannst, wenn du damals in Ägypten gewesen wärst. Stimmt das?"' "`Ja, das stimmt!"' Amélie fielen Amsets Worte ein, die er im Garten zu ihr gesagt hatte. "`Ja. Und ich träumte, mir wurde das Herz gestohlen und eine Hund brächte es mir zurück. Aber eine Stimme sagt, das Herz sei nur geliehen, ich müsste es mir erst verdienen. Es war schrecklich, weil es sich anfühlte, als wollte mir jemand das Herz wieder herausreissen."

"`War es ein Hund?"' fragte Thot scharf und Anubis winselte leise. "`Oder, war es ein Schakal wie Duamutef? Du weisst ja jetzt, wie ein Schakal aussieht"' Thot war ernst und streng. Amélie befürchtet, sie hätte etwas Falsches gesagt. Sie überlegte "`Es könnte ein Schakal gewesen sein\dots"' "`Amélie, es ist sehr wichtig!"'

Amélie versuchte sich an den Traum zu erinnern: Es war ein schmaler, unterirdischer Gang gewesen, durch den das Tier mit einem Bündel im Maul zu ihr gekommen war. Es hatte im ersten Moment ausgesehen wie ein Schäferhund\dots, dann erinnerte sie sich genauer, für einen Hund hatte das Tier zu grosse Ohren und war dünner, auch die Beine waren länger. "`Ich glaube es war tatsächlich ein Schakal"' Anubis jaulte leise auf und auch Thot schien das für eine schlechte Nachricht zu halten.

"`Aber, was bedeutet das denn?"' fragte Amélie. "`Ich weiss es nicht genau, Amélie. Aber das ist nicht gut, nicht gut."' antwortete Thot. 
 
"` Zu deinen Fragen möchte ich vorerst soviel sagen"' sprach er, während er sich erhob und Anubis aus dem Zimmer liess. "`Berta, deine Amme, hat mich um Hilfe gebeten und da Osiris und ich hier in Europa etwas erledigen wollen, können wir dir direkt bei dem Problem helfen. Allerdings brauchen wir deine Hilfe ebenso."' "`Ihr Götter braucht meine Hilfe?"' Amélie wusste nicht, ob sie lachen sollte. "`Wie soll ich euch bitte helfen? Und die anderen Götter, brauchen die auch Hilfe?"' "`Oh, du meinst die restliche Reisegesellschaft? Nein! Sie sind wegen Osiris mitgekommen und weil sie mal wieder ein richtiges Abenteuer erleben möchten. Die Hauptarbeitszeit der ägyptischen Götterschaften ist doch einige Jährchen her, um nicht zu sagen, Jahrtausende."' Thot lächelt wieder "`Mehr kann ich dir nicht sagen."' "`Toll!"' schmollte Amélie "`und wieso ich?"' "`Später."' "`Aber\dots!"'

"`Kein aber, Amélie."' Thot legte die Fingerspitzen aneinander: "`Was ich dir sagen kann, ist: Nein, du bist nicht tot. Allerdings wirst du einige Prüfungen bestehen müssen. Diese Prüfungen finden in der Duat statt, hier in Europa sagt ihr Jenseits dazu. Gestern Abend hast du, wie du richtig bemerkt hast, einige der Toten gesehen, die die Götter begrüssten, als wir an dem Eingang der Unterwelt vorbei kamen. Die Götter und die Toten bewegen sich im gleichen Raum. Das ist völlig normal."' "`Völlig normal?"' begehrte Amélie auf. "`Amélie, wir haben nicht viel Zeit, wir haben viel zu wenig Zeit, um all die Rätsel zu lösen, die wir lösen müssen. Denn wenn wir sie nicht lösen, dann wirst du mit grosser Sicherheit tatsächlich dein Herz verlieren und sterben und Osiris wird weiter leiden."' Thot war wieder ernst und diesmal ungeduldig.

Amélie verschränkte schwungvoll die Arme und schob die Unterlippe vor. Sie hätte am liebsten gesagt, er solle doch sein Kram allein machen, wenn er ihr nichts sagen wollte, aber sie traut sich nicht.

in diesem Augenblick öffnete sich die Tür und die schlanke, kleine Isis trat in das Zimmer, gefolgt von Anubis. Mit staunen bemerkte Amélie, wie Thot beim Geräusch der Tür leicht zusammengezuckt war und als Isis eintrat plötzlich nervös wirkte. "`Störe ich?"' fragte Isis und schaute von Thot zu Amélie, die immer noch die Arme verschränkt und den Mund zusammen gepresst hatte.

\section*{3}
\addcontentsline{toc}{section}{3}
 
"`Nein, nein, Isis. Komm! Wir haben dich erwartet!"' -'Ach, ja?' dachte Amélie und sah Thot an, der flehentlich zurückschaute, -'sag nichts, bitte!' Also, stand Amélie auf und reichte Isis die Hand. "`Hallo, ich bin Amélie!"' Isis nahm mit kühlen, trockenen Händen die Amélies. "`Amélie! Wie geht es dir, nach deiner ersten Prüfung gestern?"' "`Prüfung?"' Amélie drehte sich zu Thot um, der zog den Kopf ein. "`Ah, du weisst nichts von einer Prüfung!"' stellte Isis fest. 

Sie hielt Amélie Hand noch immer fest. Nun sah sie ihr in die Augen. Isis Augen waren grün. Sie schillerten wie ein Opal. Ihre Haut war glatt und hell. Ihre Haare, die sie offen trug waren schwarz und leicht gewellt. Sie glänzten, wie die Federn einer Krähe, jedoch rötlich und grünlich, als ob Blumen darin versteckt wären. Sie wurden durch einen dünnen Reif aus Gold gehalten, der wie eine Schlange geformt war. Vor der Stirn der Göttin reckte sich die Goldene Schlange nach oben. Sie strömte eine frischen, süssen Duft aus, der mit dem Geruch von Weihrauch und verschiedenen Heilpflanzen gemischt war. Sie hatte ihr ägyptisches Leinenkleid in ein Blumenkleid und einen blauen Wollpullover getauscht. Sie war barfuss. Und hatte eine Schürze mit Monden drauf umgebunden.

Amélie wurde schwindelig. Ein tiefer und heftiger Schmerz in ihrer Brust flammte auf und zog sich enger und fester wie ein eiserner Ring um ihren Brustkorb. Sie schnappt nach Luft. Mit der freien Hand riss Amélie an ihrem Kleid. Luft! Ich habe keine Luft! Ihr Herz hämmerte lauter und schneller, lauter und schneller und der Eisenring zog sich enger und fester. Amélies Beine knickten unter ihr weg. "`Bitte!"' krächzte sie, "`Hilfe!"' 

Amélie lag vor Isis auf dem Boden, die Göttin hielt noch immer die rechte Hand fest in der ihren und stand still, entspannt da. Sie schien nach etwas zu lauschen. Dann klärte sich der ernste Gesichtsausdruck der Göttin und sie lächelte zart. Amélie hörte auf sich zu winden und zu krümmen. Der Schmerz lies nach und sie streckte sich vorsichtig auf dem Teppichboden aus. Isis zog ein Tuch aus ihrer Schürze, tränkte es mit einer bläulichen, nach Lavendel riechenden Flüssigkeit aus einem kleinen Glasfläschen und tupfte damit Amélies Stirn ab.

"`Arme Amélie"' seufzte sie. "`Was hast du herausgefunden?"' fragte Thot, der zunehmend blasser Amélies Schmerz verfolgt hatte. Isis half Amélie sich aufzusetzten und den Rücken gegen den Ohrensessel zu lehnen. Anubis legte sich neben Amélie und blickte aufmerksam die Göttin an. 

Thot konnte nicht mehr still sitzen, erhob sich und ging auf und ab. "`Amélie, ich habe versucht in dein Herz zu schauen! Es tut mir leid, du hattest starke Schmerzen. Dein Herz ist wie eine Rosine vertrocknet und zusammengezogen. Es liegt ein Schatten darauf und selbst ich kann nicht genau erkennen, was es ist. Etwas sehr dunkles hat einen Ring um dich und dein Herz gezogen."' Isis hatte leise und schnell gesprochen. Ihre Stirn glänzte. Sie hat sich genauso angestrengt wie ich, dachte Amélie. Isis beugte sich dicht an Amélies Ohr, während sie sanft Amélies Arme und Beine massierte, flüsterte sie: "`Ich habe ein kleines, wunderschönes Licht in deinem Herzen gesehen. Hüte es gut!"'

"`Amélie ist sich sicher, einen Schakal im Traum gesehen zu haben, der ihr das geborgte Herz bringt!"' sagte Thot bedrückt. Isis hielt in der Bewegung innen. "`Ist das wahr?!"' Amélie nickte ängstlich. "`Dann ist es gefährlicher als ich dachte. Thot, du weisst, Amélie muss die Prüfung heute Abend machen. Du musst sie vorbereiten. Ob sie es schafft, weiss ich nicht. Ihr Herz ist sehr schwach und für solche Prüfungen kaum zu gebrauchen. Dann ist die Reise heute zu ende."'

"`Adieu, kleine Amélie, wir sehen uns dann heute Abend. Versuche am Nachmittag zu schlafen, du kannst alle Kräfte brauchen."' Isis verliess leise das Zimmer. Thot beugte sich zu Amélie hinunter und nahm sie steif in den Arm. Junge Mädchen trösten war nicht die Hauptbeschäftigung des göttlichen Gerichtsschreibers, aber Amélie war froh. Sie konnte nicht anders und begann zu schluchzen.

"`Thot, was wird denn jetzt?"' schniefte sie. Thot ging zu einem Regal und holte eine Holzschatulle hervor. Er nahm eine Tafel Schokolade mit Mandeln heraus und setzte sich neben Amélie auf den feinen Teppichboden lehnte den Rücken gegen den grossen Ohrensessel. Er brach ein grosses Stück von der Schokolade ab und reichte es Amélie, nahm sich selbst ein Stück davon und erklärte Amélie, was sie in dieser Nacht erwartete. Anubis lag auf der anderen Seite des Mädchens und liess sich, was sonst strikt gegen seine göttliche Gewohnheit war, von ihr den Rücken kraulen.

Zur Mittagszeit verliess Amélie Thots Zimmer. Obwohl es im 1. Stock lag, drang der Duft des Mittagessens in ihre Nase. "`Amélie?"' rief Thot hinter ihr her, "`Amset ist sicherlich im Garten, bei seinen Brüdern!"' Amélie schaute verächtlich. "`Na und?"' sagte sie spitz und schlug die Tür lauter als nötig zu. Anubis schnaufte, es hörte sich an, als würde er lachen. Thot grinste und räusperte sich laut, bevor er mit seinem Freund Amélie langsam Richtung Küche folgte.

"`Amélie, meine Liebe, würdest du den Brüdern Bescheid sagen, dass sie zum Essen kommen sollen?"' trällerte Hathor, sobald Amélie die Küche betrat. -Was zum Teufel, ist denn mit denen los, dachte Amélie, natürlich wusste sie es selbst, aber sie wollte nicht.

\section*{4}
\addcontentsline{toc}{section}{4}
 
"`Amélie, was machst du hier?"' fragte Amset. "`Was mache ich wohl? Ich sitze hier!"' antwortete Amélie pampig und wünschte sich, sie wäre freundlicher gewesen. Amset war der einzige, der sie fragte, was sie machte und wie es ihr ging.

Sie war nach dem Essen in den Garten gegangen und hatte sich am Rand des Teiches auf der Steinbank niedergelassen. Die hohen Bäume schützten vor der Sicht von der Strasse aus. Der Teich, der dicht an der zweiflügeligen Treppe lag, die zur Eingangstüre auf der Hofseite führte, war zum Haus hin mit einem Schilfgürtel abgeschirmt. 

Amélie hatte der Pavianfrau zugesehen, die gemütlich mit ihren drei Kindern im Schatten hockte und an Früchten knabberte. Als sie Amélie gesehen hatte, war sie zu ihr gekommen und hatte ihr schüchtern eine Handvoll roter Beeren gegeben.

Amset war unsicher stehen geblieben. "`Tschuldigung,"' nuschelte Amélie und er bemerkte eine Träne, die ihr über die Wange lief. "`Hei, hei, nicht traurig sein!"' er setzte sich zu ihr auf die Bank und legte vorsichtig den Arm um sie. Amélie schluchzte. Plötzlich kam eine Hand aus dem Gebüsch und hielt ein grosses, geblümtes Taschentuch. Es raschelte und Hans trat aus den Büschen, gefolgt von dem Pavian Hapi, Amsets Bruder.

"`Joo, nei, Amélie, was isch los? Muesst nit brüelle!"' Der kräftige, wilde Mann beugte sich zu Amélie und gab ihr das Taschentuch. Sie schnäuzte sich, die Affenkind schauten erschrocken auf. "`Ich hab Angst! Ich hab echt Angst!"' sagte Amélie. "`Ich verstehe nicht, was passiert und heute Abend soll ich eine Prüfung machen und Isis sagt, ich wäre zu schwach dafür."' "`Ach, Meidli, isch nit so schlimm! Meinscht du Berta würde dich in Gefahr bringe?"' meinte Hans, aber er sah unglücklich aus dabei. "`Wo ist Berta denn?"' "`Sie kummt sobald s goot, sie muess no etwas erledigen."' Hans  klopfte Amélie mit seiner Pranke auf die Schulter. Amset konnte sie rechtzeitig festhalten, bevor sie von der Bank fiel.

Nachdem der wilde Mann wieder in den Büschen verschwunden war. Kamen die anderen beiden Brüder, Duamutef und Kebi dazu. Kebi setzte sich auf Amsets Schulter und Duamutef und Hapi hockten vor der Bank. "`Amélie, wir wollen dir helfen, aber dann müssen wir wissen, was Isis gesehen hat."' -'Ich habe zwar am Fenster gelauscht, konnte aber nicht alles verstehen,' meinte Kebi in Gedanken. "`Was! Du, Ihr belauscht und beschattet mich"' Amélie fuhr auf. "`He, bleib mal locker!"' Amset drückte sie zurück auf die Bank. "`Kebi hat erzählt, du hast einen Schakal im Traum gesehen"' flüsterte er. "` Wenn das wahr ist, dann könnte dieser Schakal Duamutef sein. Und wenn es Tef ist, der in deinen Träumen aufgetaucht ist, dann haben wir, dann hat er ein Problem."'

In den Augen von Amset funkelte es wütend und Duamutef war aufgestanden und lief unruhig hin und her. Kebi, der unruhig seine Flügel  ausgebreitet hatte, legte sie wieder an. Hapi knabberte an seinen  Fingern. "`Ich verstehe nicht, was du meinst, was für ein Problem?"' fragte Amélie. -'Es ist so, unsere Arbeit im Jenseits ist es die Eingeweide zu bewachen,' erklärte Duamutef. -'Wenn ein Schakal in deinen Träumen aus der ägyptischen Zeit dein Herz aus einem Grab trägt, dann ist da etwas sehr schlimmes passiert!' "`Aber was denn?"' Amélie verstand nicht. -'Wir wissen es nicht, noch nicht,' mischte sich Kebi ein. "`Wir Brüder sind die unbestechlichen Wächter der Kanopen und unseres Grossvaters Osiris. An uns dürfen keine Feinde vorbei kommen! Und so wie es aussieht, ist aber genau das passiert!"' Amset sah Amélie an, alle Brüder sahen sie an. 

Amélie knetete ihre Hände. Eingeweide? Amset und seine Brüder, waren die Hüter der Eingeweide? Kanopen? Sie erinnerte sich an ihrem Kindergeschichts-Atlas, den sie mit Berta früher oft angeschaut hatte. In dem Kapitel über Ägypten, gab es ein Abschnitt über Mumien und die Einbalsamierung. Amélie fiel ein Bild ein, auf dem vier Krüge abgebildet waren. Die Deckel der Krüge waren wie Köpfe geformt gewesen: Ein Menschen-, ein Falken-, ein Affen-, und ein Schakalkopf.

Für einen kurzen Augenblick spürte Amélie eine Gänsehaut und bemerkte ein saugendes schwarzes Loch aus Nichts, in einem Grab lauern. Als sie die Augen öffnete, starrten die Brüder sie immer noch an.

"`Also, passt auf!"' sagte sie. Amélie erzählte den Brüdern von dem Traum und dem, was Isis gesagt hatte. "`Aber ich weiss nicht, was sie damit meint\dots"' schloss sie. Die Brüder schwiegen betreten, dann tauschten sie ihre Blicke. -'Ich werde nachforschen', meinte Duamutef. -'Sollen wir mitkommen,' wollte Hapi wissen, -'Nein, ich gehe alleine. Bevor wir nicht wissen, wer oder was dahinter steckt, soll niemand wissen, was wir machen!' Duamutef verschwand in den Büschen

"`Wir müssen dich vor heute Abend warnen. Ich will dir keine Angst machen, aber\dots"`, sagte Amset. "`Toll, hat leider nicht geklappt!"' antwortete Amélie. "`Was?"' fragte Amset erstaunt. "`Keine Angst machen! Jetzt habe ich mehr Angst!"' Amélie schaute Amset an: "`Sag mir lieber, wie ich die Wasserprüfung überstehen soll, heute Nacht!"' 

"`Ich weiss es nicht, aber ich werde da sein."' Amset nahm zart Amélies Hand in seine. Er suchte ihren Blick. Amélie nahm einen grossen Schluck davon in ihr Herz.

\section*{5}
\addcontentsline{toc}{section}{5}

Duamutef trat hinter einer Palme, dicht am Nilufer vor. Er hoffte, er hatte die richtige Zeit erwischt. Für Götter war es kein Problem an verschiedene Orte zu verschiedenen Zeitpunkten zu reisen.

Er trabte zum Tempelbezirk der Einbalsamierung. Dieser war dicht am Nil, schliesslich brauchten die Priester und Balsamierer viel Wasser bei ihrer Arbeit.

Bevor er vom blauen Haus abgereist war, hatte er seine Tante Maat besucht. Er hatte ihr erzählt, er wolle sich in der Zeit umsehen, um herauszufinden, ob Amélie schon einmal in Ägypten gelebt hätte und wann genau das gewesen wäre. Als unbestechlicher Wächter durfte er niemals lügen, aber er musste nicht alles sagen, was er wusste! 

Maat hatte ein Buch gelesen. Es war eine Zeichnung darauf mit einem Jungen, der eine Brille trug und eine Zickzack-Linie auf der Stirn hatte. Sie hockte auf der breiten Fensterbank. Von der schönen Aussicht ihrer Fenster über den Rhein und weit ins Kleinbasel und den Schwarzwald, bekam Duamutef nicht viel mit, obwohl er neben sie auf die Fensterbank gesprungen war. 

Es war gut so, denn sonst hätte er den grossen Schatten gesehen, der an der Anlegestelle der Barke, die am Tag nicht zu sehen war, in das Rheinwasser platschte. Für einen kurzen Moment ragte ein braungrüner, schuppiger Leib aus dem Wasser, der gleich in der Tiefe verschwand. Nur wer aufmerksam den Fluss weiter beobachtet hätte, dem wäre aufgefallen, wie sich das Wasser flussabwärts in regelmässigen Abständen kräuselte, als ob ein riesiges Tier sich im Fluss vergnügte.

-'In der 18. Dynastie meinst du?' Duamutef legte sich mit der Zunge über die Nase, schliesslich hatte er keine Finger, um sich nachdenklich am Kinn zu kratzen. "`Da ist einiges durcheinander geraten, nachdem der unsägliche Echnaton wieder verschwunden war, war nichts mehr wie vorher. Die Priester waren zwar wieder mächtig, hatten aber einiges an wichtigem Wissen verloren. Du weisst es ja selbst, Duamutef, ab da hatte ich, als Ordnungsprinzip, nicht mehr leicht."'

-'Ich erinnere mich', meinte der Schakal. Auch er dachte mit Grausen an die Zeit des Pharao Echnaton zurück, der in kurzer Zeit alle Tempel im ganzen Land geschlossen hatte. Er verbot die Gottesdienste für die Götter. Nur noch seinem Sonnengott Aton durfte in seiner neu erbauten Stadt Amarna gehuldigt werden.

Er hatte sich weder bei den übrigen Göttern, noch bei deren Priestern beliebt gemacht. Nachdem er gestorben war, hatten sich alle bemüht, die alte Ordnung wieder herzustellen. Aber die Linie war unterbrochen worden, die Linie der Pharaonen, die von den Priestern geweiht worden waren, um zwischen Menschen und Göttern zu vermitteln. Die Pharaonen nach dieser Zeit, waren immer noch Eingeweihte und Vermittler, aber das Erbe der Götter konnte nicht mehr vollständig weitergereicht werden. 

"`Haremhab!"' meinte Maat. "`Die Zeit könnte passen."' Also war Duamutef ins alte Ägypten der 18. Dynastie zur Zeit des Pharao Haremhab gereist. 

Er schlüpfte durch die Tür der Halle des Totentempels. Die Priester in weissen Gewändern und Leopardenfell eilten an ihm vorbei. Sie konnten ihn nur sehen, wenn sie die Kanopenzeremonie durchführten und die Kanope seinem Schutz unterstellten. Er suchte die zuständigen Götter, die die Priester, ohne in sichtbare Erscheinung zu treten, unterstützten. In einer schattigen, ruhigeren Ecke des Säulenganges hörte er mehrere Stimmen und schaute vorsichtige um die Säule, hinter der er sich verborgen hatte. Er sah drei Göttinnen, die für das Wohlergehen der Lebenden, aber auch der Toten im Jenseits zuständig waren.

Von Säule zu Säule huschte der Schakal näher. Vielleicht konnte er die Göttinnen belauschen, ohne sich zu zeigen. Er wollte neugierige Fragen vermeiden. Schliesslich wussten sie nicht, wer Amélies Herz gestohlen hatte und warum.

\section*{6}
\addcontentsline{toc}{section}{6}

Duamutef schien einen guten Zeitpunkt erwischt zu haben, denn die Göttinnen waren durcheinander. Sie schnatterten aufgeregt wie Nilgänse. "`Es ist unheimlich, dieses Grab. Ich wollte nach dem rechten sehen und habe mich auf dem Weg zur Gruft verlaufen. Das ist passiert mir nie!"' Entrüstete sich die eine, die eine Krone auf dem Kopf trug. "`Ich wollte die Opferspeisen versorgen und bin plötzlich vor einer unsichtbaren Wand gestanden und konnte nicht weiter,"' meinte eine junge Göttin, die ein Anchzeichen trug und einen Was-Zepter. "`Ich habe mich richtig gefürchtet!"' So sprachen die beiden auf eine dritte Göttin ein, die ein Löwenkopf auf den Schultern trug und die beiden aus ihren gelben Löwenaugen wachsam anschaute. "`Sachmet, was sollen wir denn machen?"' 

Duamutef spitzte die Ohren und atmete so leise er konnte. Er hatte Glück! Sachmet, die Löwengöttin, war eine geschickte Kämpferin und mächtige Göttin, sie würde es sicher bemerken, wenn etwas merkwürdiges passiert war. Sie war nicht so weibisch, wie die anderen beiden, die nur Blumen und Opfergaben im Kopf hatten.

Sachmet sagte zu den beiden anderen: "`Wir treffen uns in einer Stunde wieder hier. Ich will selbst schauen, wie es in der Gruft aussieht."' Die Göttinnen verabschiedeten sich voneinander und Sachmet wendete sich eilig dem hinteren Bezirk des Tempels zu. Die Göttin mit dem gewaltigen Haupt einer Löwin verschwand in einem Bereich, der unter die Erde führte und in dem die Körper der Toten ruhten bis sie vollständig für ihre Reise in die Duat gerüstet waren.

 Die werdenden Mumien mussten eine Zeit in Salz liegen, damit ihnen die Flüssigkeiten aus dem Körper gesogen werden konnten. In dieser Zeit wurden sie in den unterirdischen, kühlen Gruften untergebracht und von den Göttinnen hin und wieder besucht und mit Opfergaben bei Laune gehalten.
 
Duamutef folgte Sachmet die Treppe in den unterirdischen Gang hinein, das war nicht der Gang, den er durch Amélies Traum besucht hatte. Dieser Gang war breit, schliesslich wurden die Toten auf Bahren täglich hindurchgetragen. Es gab etliche Abzweigungen in engere Gänge, oder niedrige Räume, in denen einer oder mehrere Tote lagen. Sie lagen in Särgen, die mit Salz randvoll waren.

Ein süsslicher und rauchiger Geruch bohrte sich in Duamutefs Nase. Die Toten rochen kaum nach Verwesung, dennoch durchzog die gesamte Tempelanlage der Totengeruch. In den Kellergewölben, in denen er sich jetzt befand, wurde stark geräuchert. Der Rauch diente den Toten nicht nur als Parfum, sondern auch als zusätzlicher Schutz vor Zersetzung.

Duamutef mochte den Geruch. Er war Aasfresser und wusste Räucherfleisch durchaus zu schätzen. Allerdings würde er sich niemals an einem seiner toten Schützlingen vergreifen. Aber er war froh über seine feine Nase, die sich an seinem Beruf als Totenwächter nicht störte. Amset und Hapi hatten sich ab und zu über den Geruch beklagt\dots

Duamutef bog um eine Ecke und blieb erschrocken stehen. Sachmet stand ein Stück weiter hinten im Gang vor einer Türöffnung. Duamutef zog sich hinter die Ecke zurück und schob vorsichtig nur den Kopf um die Ecke. Die Göttin war zu weit weg, er konnte nicht genau erkennen, was sie machte. Aber er konnte ihr nicht in den glatten Gang folgen, sie hätte ihn sofort entdeckt.

Sie streckte ihre Hände aus, als wollte sie etwas, das in der Türöffnung war abtasten. Sie drückte mit beiden Händen und Duamutef begriff, sie kam nicht weiter. So wie es ihm in dem Gang, durch den er in Amélies Traum geschlüpft war auch ergangen war.

Die Göttin nahm ihren Anch, das sie am Gürtel getragen hatte und versuchte damit die unsichtbare Wand zu durchdringen, vergeblich. Es blitzte und Funken sprühten. Die Göttin nahm ihr Was-Zepter, ein ebenso mächtiges und kraftvolles Instrument der Magie, wie der Anch. Duamutef zog die Nase aus dem Gang, denn es blitzte grell und ein heulender Ton übertönte die Zaubersprüche der Löwengöttin.

Duamutef machte sich auf den Rückweg. Wenn Sachmet nicht in den Raum hinein konnte, dann konnte er es auch nicht. Er verbarg sich zwischen den Säulen und wartete wie die beiden anderen Göttinnen auf Sachmet.

\section*{7}
\addcontentsline{toc}{section}{7}

-'Was machst du hier?' Während die Worte in Duamutefs Kopf widerhallten, packte ihn eine gewaltige Hand im Nacken und hob den ausgewachsenen Schakal hoch. Er blickte in die scharfen Falkenaugen seines Vaters. Horus, der seinen Falkenkopf auf den Schultern trug, dampfte. -'Wir haben eine heikle Mission zu erfüllen und mein Sohn treibt sich irgendwo zuhause in der 18. Dynastie herum. Wir brauchen jeden Mann in Basel!' bauste es hinter Duamutefs Stirn.

-Vater, so hör doch\dots -Kein Wort! Der Falkenköpfig schüttelte den Schakal, dem hören und sehen verging. Prompt kamen einige Wächter angelaufen. Sie trugen lange Messer mit sich. "`Was ist hier los!"' fragte der eine. 

Sie sind nervös, die Wächter! Bemerkte Duamutef, sie kommen sonst nicht gleich angerannt! Horus setzte sein strahlenstes Lächeln auf, soweit er das mit dem Falkenschnabel hinbekam und säuselte -'Nichts, nichts. Ihr wisst ja wie das ist, man kann die Jugend nicht genug im Auge behalten, hehe!' Um seine Worte zu bekräftigen, schüttelte er seinen Sohn noch einmal. -'Wenn ihr erlaubt, dann würden wir uns wieder auf unsere Arbeit konzentrieren.' Er machte kehrt und lies den Nacken von Duamutef los, der eilig ausser Reichweite seines Vaters sprang.

-'Aua, das tat weh!' -'Das sollte es mein Sohn, das sollte es,' der Falkenkopf starrte auf den Schakal und klappte mit dem Schnabel. Die Wächter beobachteten sie skeptisch, die Messer erhoben. Sie berieten sich und es war nicht klar, ob sie sich mit der Antwort zufrieden geben würden. 

Kurzentschlossen verwandelte sich Horus in seine Falkengestalt. Seine Klauen griffen Duamutef an Rücken und Nacken. Der riesige Falke schwang sich in die Luft, den winselnden und zappelnden Schakal schleppte er wie eine Puppe mit.

-'Vater, verdammt lass mich los!' -'Nein, wohl kaum in dieser Höhe!' -'Was ist denn?'

Obwohl die Götter keinen irdischen Körper hatten und dadurch keine irdischen Hindernisse kannten, hatte Thot vorsichtshalber sein Fenster weit geöffnet. Man wusste nie genau, wie weit sich die Materie daran erinnerte, Götter ohne weiteres passieren zu lassen.

Horus sauste durch das Fenster, liess Duamutef los, der auf den Schreibtisch des Gerichtsschreibers fiel und brauste in die Wand und das Büchergestell  gegenüber  des Fensters. Horus hatte es geschafft sich zu entmaterialisieren. Der Schwanz des Falken und seine Klauen ragten aus der Wand in das Zimmer, während sich der Rest in der Wand und auf dem Gang davor befand.

 Duamutef hatte weniger Glück gehabt. Er hatte die Beschleunigung des Fluges falsch eingeschätzt und war wie eine pelzige Kanonenkugel auf den Schreibtisch geprallt. Bücher, Schriftrollen und Papiere waren in einer explosionsartig Wolke vom Tisch in die Höhe geschleudert worden und segelten und fielen rund um den Schakal auf den Boden. Braune Gänsefedern, Thots liebste Schreibgeräte, schwebten im Zimmer umher und von der Nase des Schakals tropfte schwarze Tinte. Vor dem Schreibtisch breitete sich ein dunkler Tintenfleck aus, der aus dem Tintenfass auf den schönen Perserteppich floss. Ein letztes Buch knallte Duamutef auf den Kopf. Dann war es still.

Thot sass kerzengerade in seinem Ohrensessel. Die Beine übereinander geschlagen mit einer Lesebrille auf der langen Nase. "`Tse, tse, tse! Ging es nicht etwas langsamer?"' fragte er. Er erhob sich, öffnete die Tür und half seinem Freund Horus, der mit Falkengestalt in der Wand fest hing.

 Thot zog ihn vorsichtig in den Gang, wo Horus sich in seine menschliche Gestalt brachte. "`Thot, alter Junge, tut mir leid. Die Kinder, du weisst ja, wie das ist. Machen immer Dummheiten!"' Horus ging durch den Gang Richtung Küche davon, er schwankte leicht. Wenn er sich aufregte, dann half nur eine leckere Zwischenmahlzeit in Hathors Küche.
 
 Während Thot den Falken aus der Wand befreit hatte, hatte Duamutef die Gelegenheit genutzt, um vom Schreibtisch zu klettern, sich zu schütteln. Ihm war schwindelig\dots
 
 "`Was hast du gemacht!"' Duamutef sträubte sich das Nackenfell vor Schreck, so schnell, war das Gesicht des Gerichtsschreibers vor ihm erschienen. Ihre Nasen berührten sich fast. Er jaulte auf. -'Ich hab nichts\dots Maat hat gesagt\dots' Thots Augen funkelten wie Eiskristalle. 

Duamutef holte tief Luft. -'Ich war in der Zeit von Pharao Haremhab. Maat hat mir den Rat gegeben. Einer muss doch herausfinden, was mit dem Herz passiert ist.' "`Ja,"' knurrte Thot, "`aber nicht so auffällig, dass sämtliche Götter der ägyptischen Unterwelt aufgescheucht werden!"' -'Was? Wer?' "`Die Richter!"' Thot begann auf und ab zu wandern, "`Wie konntest du in einem Bereich herum schnüffeln, wo sich die Richter herumtreiben und sofort Lunte riechen?"' -'Die Richter, welcher Richter?'

 "`Du glaubst doch nicht ernsthaft, du könntest dich um die Herrin des Zitterns, dem Auge des Re, der Löwenköpfigen herum schleichen, ohne dass sie es bemerkt? -'Oh, Sch\dots?!' "`Ja, genau, oh, Schlamassel, wolltest du doch sagen? Mensch, Junge!"' Thot sah auf den Schakal runter, dessen Rücken von den Krallen des Falken blutig gerissen war und von dessen Nase Tinte tropfte. "`Solltest du die Idee gehabt haben, unauffällig zu sein, so kann ich dir sagen: Es hat nicht funktioniert!"'

"`Gehe zu Wibrandis und bitte sie, dir den Pelz zu säubern und deine Wunden zu verarzten. Sie soll dir einen ordentlichen Brocken Fleisch geben, denn ich brauch dich heute noch. Und dann, dann kommst du zu mir, wir müssen dringend reden!"' Duamutef seufzte und leckte kurz über Thots Hand, dann verschwand er dankbar aus dem Zimmer. "`Vergiss nicht, Wibrandis zu bitten, ob sie mir dann helfen kann, das Chaos zu richten!"' Nun war es an Thot zu seufzen\dots

\chapter*{2. Nacht}
\addcontentsline{toc}{chapter}{2. Nacht}


\begin{quotation}

\emph{II Quod est inferius, est sicut quod est superius, et quod est superius, est sicut quod est inferius, ad perpetranda miracula rei unius.\\2. Das, was unten ist, ist so wie das, was oben ist, und das, was oben ist, ist wie das, was unten ist, um die Wunder des Einen zu vollziehen. \\Tabula Smaragdina}

\end{quotation}


\section*{1}

Thot sass auf dem Bett von Osiris. Anubis sein treuer Freund und Begleiter, lag vor dem Bett des Unterweltgottes. -'Wie weit ist es mit den Vorbereitungen?' Osiris lag schwach und elend zwischen den Kissen. "`Es ist soweit alles vorbereitet. Amélie wird heute Nacht ihre Lebenskräfte erneuern, wenn alles gut läuft. Die Kräfte, die dabei freigesetzt werden, können wir dann zu dir in die Barke leiten."'

-'Es ist sehr wichtig, dass Amélie ihre Lebenskräfte kennenlernt und weiss, wie sie funktionieren,' bemerkte Osiris. -'Amélie wird sich ab morgen viel besser an alles erinnern können. Die Weckung der Lebenskräfte, weckt auch die Erinnerung', meinte Anubis. "`Ich hoffe, sie weiss dann, was sie da sieht und erinnert. Aber Berta wird sie vorbereitet haben\dots"' sagte Thot. -'Weisst du das sicher, hat Berta dir das gesagt?' fragte Anubis nach, "`Nein"' gab Thot zu. 

-'Wo ist Berta denn?' fragte Osiris ungeduldig. "`Sie ist im Norden, sie hat einen Verdacht, wer hinter Amélies Schwierigkeiten mit dem Herzen steckt, den will sie überprüfen."' -'Hat sie gesagt, wer?' fragte Anubis und spitzte die Ohren. "`Nein! Ihr Verdacht sei zu vage. Sie kommt sobald sie mehr weiss, spätestens in der Gerichtsnacht."'

Die drei Götter schwiegen. Es war eine neue Zeit angebrochen. In den guten alten Zeiten, da herrschten sie über die Menschen. Sie hatten die Priester, die die Pharaonen lehrten und unterstützten ihren göttlichen Willen in die menschliche Welt zu bringen. Eine lange Zeit hatte das wunderbar funktioniert. Dann war der Kontakt abgebrochen, das war, als der Pharao Echnaton regierte.

Mit dem königlichen Haremhab konnten die Götter und Priester einige Verbindungen wieder aufnehmen, aber es waren Lücken entstanden. Die Pharaonen vorher waren in das Wissen über die Götter eingeweiht, sie hatten gelernt, welche wichtige Aufgabe sie als Vermittler hatten. Echnaton war nicht vorgesehen, niemand hatte ihn richtig vorbereitet und als er die Macht ergriff, zerstörte er, die Verbindung zu den Göttern.

Er tat es nicht böswillig, nein, ihm fehlte die Schule der Priester, die Schule mit den Göttern direkt zu sprechen. So war er der erste Pharao, der glaubte, statt zu wissen. 

Sie alle wussten, es war ein unausweichlicher Schritt gewesen. Echnaton war nicht vorgesehen und gleichzeitig notwendig, um die Menschen weiter zu bringen. Deshalb hatte sie ihn auch am Leben gelassen. Aber es tat ihnen immer noch weh. Ebenso wie die Menschen mehr und mehr die Trennung zwischen den Göttern und ihrer Welt kennenlernten und darunter litten, so litten die ägyptischen Götter, weil sie den Kontakt zu ihrem Volk verloren.

Sogar die Ägypter selbst, hatten schon vieles vergessen, was in den Millionen von Jahren eingeschrieben worden war in die Chronik der Welt, aber dank dem Wissen und der Leitung ihres Gottes Thot, war ein grosser Teil des göttlichen Wissens lebendig. Dafür musste ein Preis bezahlt werden: Es konnten nicht alle an diesem Wissen teilhaben. Das Volk, die Mehrzahl der Menschen war in einer anderen Entwicklung.

In den tausenden Jahren ägyptische Geschichte lernten die Menschen, die nicht zu den wenigen Priestern gehörten, sich selbst bewusster zu werden. Und damit die Menschen sich selbst und im Gegenzug dazu auch ihre Welt, ihre Erde besser kennenlernen konnten, mussten die ägyptischen, wie auch alle anderen Götter ein Opfer bringen.

Sie mussten zulassen, vergessen zu werden, sie mussten ihre Menschenkinder, die sie auf dem Saturn wie Glucken auf dem Ei bebrühtet und umsorgt hatten, selbstständig werden lassen.

Die Priester trugen ihr Wissen jedoch weiter, bis in die jetzige Zeit. Die Menschen waren sich bewusster geworden, hatten vergessen und begonnen zu glauben. Aber glauben war nicht wissen!

Da sie Götter waren, konnten sie die Erinnerung wie ein Buch aufschlagen und die Seiten darin durch die Finger rinnen lassen und sich erinnern. Sie seufzten alle drei\dots 

"`Tja"' meinte Thot "`und nun sind die Götter abhängig von dem wankelmütigen Herz eines einzelnen Menschenmädchens."' -'Früher wäre es jedenfalls eine\dots Priesterin gewesen,' dachte Anubis. Sie seufzten und dachten wehmütig an die eine oder andere junge Frau, die zu ihren Ehren tanzte und jauchzte und das alles oben ohne\dots

Isis betrat das Zimmer ihres Mannes. Eine ganze Schar leicht bekleideter, durchsichtiger Priestermädchen löste sich leichtfüssig in Luft aus. Während Osiris grünliche Hautfarbe dunkler wurde, wurden Thots Wangen kräftig rot. Anubis gab vor nicht anwesend zu sein. "`Osiris, Liebster, ich störe euch ungern, aber es ist Zeit für den Stein der Weisen. Oder hat dich eurer gemeinsames Erinnungsstündchen an die Priesterinnen schon genug belebt?"' 

Osiris wechselte seine Farbe von kräftigem Tannengrün zu gesundem Apfelgrün -'Ja, Schatz! Ich meine natürlich, nein Schatz!' Sie sahen ein leichtes Lächeln auf Isis Gesicht und atmeten auf. Sie schien nicht wütend zu sein. "`Ich wünschte, es wäre schon morgen!"' meinte Isis. Sie holte den grossen Löffel aus der Schürzentasche mit den Monden und tropfte aus der Phiole die blutroten Flüssigkeit darauf. Sie drehte die Flasche auf den Kopf und der letzte Tropfen löste sich und fiel auf den Löffel. Sie alle starten kurz auf die Flasche, die auf den Kopf gestellt jedoch nichts mehr von sich gab.

Die anderen nickten und Anubis, der seine Tante und Ziehmutter besonders verehrte stupste ihr zart mit der Schnauze in die Hand. Gedankenvoll legte sie ihm die Hand auf den Kopf. Sie war eine der wenigen, die das tun konnten, ohne danach eine Hand weniger zu besitzen.

"`Anubis? Hast du alles für die Nacht, was du brauchst? Ist das Imiut bereit? Es hängt soviel davon ab, ob du die nötige Zeit gewinnen kannst?"' -'Keine Sorge, Mutter,' antwortete er, -'der Balg des Hundes, das Imiut ist bereit. Osiris kann darin Schutz finden, bis Amélie die Lebenskräfte in Bewegung gebracht hat. Ich werde Osiris darin mit auf die Barke nehmen, dann können wir ihn sofort beleben, sobald es geht.'"`Danke!"' Isis verliess leise das Zimmer.

\section*{2}
\addcontentsline{toc}{section}{2}

Amélie stand auf dem Brücke des Birsfeldener Wasserkraftwerks. Der See, der Staustufe lag glatt und dunkel auf der einen Seite, der Rhein floss träge durch die Turbinen des Kraftwerkes flussabwärts. Im Dämmerlicht schwammen die Silhouetten einiger Enten auf dem Wasser. Es gluckste leise. Es hatte lange nicht geregnet, deshalb waren alle Wehre geschlossen und der Fluss musste sich durch die Turbinen des Wasserkraftwerkes quetschen, um weiter zu fliessen.

Amélie war mit Duamutef, Amset und Thot bis vor das mittlere der fünf Wehre gegangen. Sie schaute über das Geländer. Das Wasser war auf dieser Seite einige Meter unter ihr. Sie zitterte, obwohl sie den flauschigen Fuchspelz wieder um die Schultern trug. Zugegeben, da drunter war sie im Badeanzug.

"`Ich habe Angst."' sagte sie, während ihr die Zähne heftig klapperten. "`Ich weiss!"' sagte Thot. "`Aber nun konzentriere dich auf das, was ich dir heute morgen erklärt habe: Wenn du in den Bereich der Lebenskräfte kommen willst, musst du ganz ruhig werden. Du kannst es schaffen deine Lebenskräfte zu beherrschen, wenn du ruhig und zentriert bleibst. Es ist unbedingt nötig, für die weiteren Tage und Nächte, deine Lebenskräfte zu kennen und zu führen, denn sie werden deine Hilfsmittel, dein Handwerkszeug sein."' Amélies Augen waren riesig und rund geworden. Sie war blass, ihre Haut leuchtete weiss. 

Duamutef setzte sich dicht neben sie und stupste sie freundlich am Bein. Thot gab Amset einen leichten Wink mit der Hand. Für eine kurzen Moment schien es, als wollte der junge Gott davonlaufen, aber statt dessen trat er dicht zu Amélie und nahm ihre kalten, zitternden Hände in die seinen. Ihre Hände zuckten, als hätte sie ein elektrischer Schlag getroffen. Es schien als wollte sie ihre Hände fortziehen, aber dann blieben sie in seiner Wärme.

"`Amélie, ich bin ganz sicher! Du schaffst es! Ich\dots"', es verschlug ihm die Sprache. Sie schauten sich an. Es war mucksmäuschenstill, kein Lüftchen regte sich, die Welt wurde langsamer und hörte für einen Augenblick auf sich zu drehen. Als sie wieder damit begann, die Abendluft sanft den kühlen Wind hauchte und eine einsame Ente auf dem See klagte, da war etwas passiert. Thot lächelte Anubis zu: Amélies Herz leuchtete!

Eine kleine aber stetige Flamme glühte ruhig in ihrer Brust. Sie liess Amset Hände los und lächelte ihn schüchtern und mit geröteten Wangen an. "`Ich muss es jetzt tun, sonst verliere ich den Mut."' flüsterte sie. "`Amélie\dots, hüte dich vor Sobek. Er ist ein guter Kerl, aber er ist ein strenger Lehrer und bestraft Unachtsamkeit mit dem Tod!"' Amélie nickte. Sie streifte den Pelzmantel ab und kletterte über das Geländer. 

Sie trat vorsichtig an den Rand des Betonpfeilers. Für einen Moment fror sie bis auf die Knochen, als die Kälte ihr Herz berührte, bewegte sich der Funke darin. Er begann sich auszubreiten und die Kälte aus ihrem Körper zu verscheuchen. Sie zögerte keine Sekunde, sobald die innere Wärme und Ruhe die Zehen und Fingerspitzen erreicht hatte, hob sie die Arme über den Kopf. 

Sie spürte sie zuerst wie einen Lufthauch und dann sah Amélie sie. Die Bewohner in der Welt der Lebenskräfte. Dort wo Wasser und Luft zusammenstiessen, im Dämmerlicht wie Nebel, tanzte das Wasservolk der Nymphen. Die Wasserleute waren an verschiedenen Stellen im Wasser versammelt. Eine grössere Schar hatte sie bemerkt und kam neugierig geschwommen.

Mitten im Fluss, ein Stück entfernt bemerkte sie eine grosse, breite Gestalt mit langem, lockigen Haaren und einem langen Bart, mit einem Dreizack in der Hand. Der Mann schien auf dem Fluss zu sitzen, dann bewegte sich das Wasser um ihn und Amélie konnte seinen mächtigen Fischschwanz gegen das Abendlicht schattenhaft sehen und hörte ihn laut auf den Fluss klatschte. Sie beugte sich etwas vor, bereit für den Sprung, da bemerkte sie die Barke der Götter. Sie musste die ganze Zeit über dort, am rechten Ufer unter den Bäumen gelegen haben. 

Amélie holte tief Luft und stiess sich ab. Während sie sich kopfüber ins Wasser stürzte, hob sich der riesige Kopf eines mächtigen Krokodils aus dem Wasser. Das Ungetüm machte einen Sprung und verschwand an der Stelle im Wasser, an der Amélie untergetaucht war. Es schlug gewaltig mit dem Schwanz, es klatschte laut in die Abendruhe. 

Nach kurzer Zeit waren nur die Ringe auf dem Wasser übrig. Von Amélie und dem Krokodil war keine Spur zu sehen. Die Wassernymphen, die neugierig auf Amélie gewartet hatten, schwammen wild schnattern und planschend flussabwärts. Dabei blieben sie dicht zusammen über einer bestimmten Stelle. Als die Wasserringe die Barke erreicht hatten, drehte diese bei und richtete sich flussabwärts. Maat und Isis stand vorne im Bug, sie hielten Ausschau.

\section*{3}
\addcontentsline{toc}{section}{3}

Amset stand bleich die Finger um das Geländer gekrallt. "`Amélie!"' brüllte er. Er wollte über das Geländer auf den Betonpfeiler springen. Duamutef packte sein Bein und zerrte ihn laut knurrend zurück. Er liess erst los, als Thot den Jungen packte. "`Du kannst ihr nicht helfen!"' zischte er.

"`Jedenfalls nicht so!"' Amset sackte zusammen und hielt sich die Hände vor das Gesicht: "`Sobek!\dots"' Thot war vor ihm in die Knie gegangen und griff seine Hände. Er zwang Amset ihn anzusehen: "`Schluss!"' flüsterte er. Amset wollte sich wegdrehen, aber Thot schüttelte ihn. Duamutef knurrte leise.

"`Konzentriere dich! Wenn du ihr helfen willst, dann konzentriere dich auf Amélies Herz."' Amset schluchzte. Dann wischte er sich mit dem Ärmel über das Gesicht. "`Thot? Es tut weh. Es ist schlimmer als der heftigste Kampf mit Apophis!"' "`Ich weiss,"' antwortete Thot leise. Er stand auf und zog den jungen Gott auf die Beine. "`Aber du kannst kämpfen, indem du ihr alle Wärme und Kraft schickst, die du hast! Kämpfe! Gewinne!"' er hatte die Hände auf Amsets Schulter gelegt und strahlte ihn mit den hellblauen, kristallklaren Augen an.

"`Es ist die einzige Prüfung bei der Amélie ihren Körper braucht. Bei den anderen wird sie mit ihrer Seele reisen. Das ist auch gefährlich, aber wir können den Körper schützen. Sie kann es schaffen!"'

Thot, Amset und Duamutef wendeten sich und gingen Richtung Basel davon. Der Fluss lag im Dunkeln. Nur die Lichter der Stadt spiegelten sich an einigen Stellen.

\section*{4}
\addcontentsline{toc}{section}{4}

Amélie tauchte ins Wasser. Die Eiseskälte presste alle Luft aus ihrer Brust und sie sank betäubt und unfähig sich zu bewegen in die Flut. Ein gewaltiger Schatten kam. Er tauchte unter sie und hob sie an die Oberfläche.

Amélie atmete tief ein. Sie sah die schwach erleuchtete Brücke und spürte die Strömung des Flusses, die sie ergriff. Sie bemerkte Wassernymphen, die um sie herumschwammen und sie neugierig beobachteten. Zarte Schleier, wie Nebelhauche. Durchsichtig und fein schillernd in wasserfarbenem Blau und grün. 

Sie spürte wie die Kälte ihre Beine lähmte und merkte wie ihr Herz anfing zu rasen, was, wenn sie einfach gleich hier und jetzt ertrinken würde? Sie begann wild mit den Armen und Beinen zu rudern. Sie wollte es wenigstens versuchen!

Plötzlich berührte etwas hartes, spitzes ihren Fuss. Sie schwamm schneller,  aber sie war nicht schnell genug. Etwas rammte sie an den Beinen, sie schaute sich um. In der Dunkelheit war nichts zu sehen ausser einem Buckel im Wasser.

Amélie kam unter die Brücke. Komme nicht zu dicht an die Pfeiler, hatte Thot ihr eingeschärft. Die Strömung war stark und Amélie schaffte es nicht gegen sie anzuschwimmen. Das Ding, das sie gerammt hatte, hatte sie abgelenkt. Die Strömung schleifte sie am Beton des Pfeilers entlang und dann als dieser endete wurde Amélie in den Strudel gesogen. Sie war nicht vorbereitet und vergeudete die letzte Luft mit einem Schrei, der ihren Mund mit dem braunen Wasser füllte. Es ging wieder aufwärts und sie schnappte kurz nach Luft, bevor die Wasserwalze sie erneut packte und abwärts zerrte.

Plötzlich war etwas über ihr. sie wurde tiefer gedrückt, als würde eine kräftige Hand an einem starken Arme sie zwingen tiefer zu tauchen. Verzweifelt begann Amélie erneut zu strampeln und um sich von der Hand zu befreien. Da hörte der Sog der Wasserwalze auf. Sie schwamm an die Oberfläche. Die Strömung hatte sie wieder übernommen und führte sie schnell von der Brücke weg.

Es blieb nur eine kurze Zeit, um zu verschnaufen. Denn sie bemerkte hinter sich eine Bewegung und als sie sich umdrehte, sah sie im Licht des hohen Büroturmes auf der rechten Seite ein baumsttammgrossen Umriss im Wasser. Ein riesiges mit fingerlangen Zähnen bewährtes Maul öffnete sich. 

Amélie hatte das Gefühl, als würde durch die Panik eine Kraft in ihrem Bauch explodieren und sich im Körper ausbreiten. Wild schlug sie um sich. Die Strömung hatte sie ein ganzes Stück weiter flussabwärts getrieben. Sie sah den Rücken des Krokodils auftauchen und zog die Füsse an den Körper. "`Hilfe!"' schrie sie. 

Das Krokodil sprang aus dem Wasser und presste Amélie dann mit seiner Schnauze tief unter Wasser bis auf den Grund. Amélie fühlte wie etwas ihre Beine packte und sie festhielt. Die letzte Luft in ihren Lungen stieg in Blasen aus ihrem Mund auf, während sie im Schlamm des Flusses verschwand.

Amélie öffnete die Augen. Sie lag in einer feuchten Erdhöhle auf dem matschigen Boden. Um sie herum wuselten kleine, runde Gestalten in allen erdenklichen Brauntönungen. Die Wesen waren leicht durchsichtig. Ihre Köpfe, oder das, was Amélie für ihre Köpfe hielt, weil es sich oben befand, war kugelig und für die kleinen Körper darunter zu gross.

Es war dämmrig in der Höhle und die kleinen Wesen zappelten und schienen zu lachen. Amélie spukte einen Brocken Lehm aus. Die kleinen Kerle fuchtelten ihr vor der Nase herum und knufften sie. Einer von ihnen, er war grösser und goldbraun hatte eine Münze in den verschwommen sichtbaren Händchen. 

'Fang!' rief der Gnom. 'Fang! Fang!' Und er begann Amélie die silberne, glänzende Münze auf die Nase zu hauen, die direkt vor ihm war. 'Willst-du-die-Münze?' rief er und bei jedem Wort hieb er Amélie ins Gesicht. Die anderen Gnome wurden auch mutiger und riefen ihrerseits 'Fang!' und 'Lahmarsch!' und schlugen und zerrten an allem, was sie von Amélie zu fassen bekommen konnten.

Amélie versuchte sich zu bewegen, aber die Erdhöhle war zu eng und zu matschig. Nur vor ihrem Gesicht schien sich eine Art Luftblase zu befinden. Ihr restlicher Körper steckte fest. Amélie spürte Hass in sich aufsteigen. Er wandt sich durch den Körper von den Ohren bis zu den Zehen. "`Ihr kleinen Mistkerle!"' zischte sie schliesslich. Darauf hatten die Gnome gewartet und stopften ihr laut gackernd eine Fuhre Lehm in den Mund.

Amélie schluckte und würgte. Was hatte Thot gesagt: Ruhig bleiben, aber nicht die Gefühle verdrängen. Benutze sie! -Sehr witzig, dachte Amélie, wie soll ich Hass benutzten? Und obwohl sie das Gefühl hatte platzen zu müssen, bemerkte sie, wie sich der Druck der eisigen Kälte verwandelte. Sie wurde wirklich ruhiger, aber ihre Muskeln blieben angespannt. Die Kälte wandelte sich mehr und mehr in Klarheit. Und je kälter Amélie wurde, umso klarer wurden ihre Gedanken und umso schärfer konnte sie sehen. 

"`He, Du Zwerg!"' rief sie dem goldbraunen mit der Münze zu, "`ist das alles?"' Die Gnome wurden unruhig und während Amélie sie weiter beschimpfte und anstachelte, tanzten und hopsten sie dichter und dichter um sie herum. Sie pufften und knufften. Aber Amélie behielt nur die Münze in den Augen.

Dann war es endlich soweit, der Gnom mit der Münze liess sich zu einem waghalsigen Manöver provozieren und Amélie schnappte die Münze mit der Zunge und versteckte sie im Mund. Die Gnome gerieten ausser Rand und Band sie schrien und zerrten wie wahnsinnig an Amélies Augenlidern, Nase und Ohren. Einige versuchten ihren Mund auf zu sperren. Amélie schluckte!

Ein letztes Kreischen klingelte in ihren Ohren, dann fand sie sich auf den Planken der Basler 'Wilde-Maa-Fähre' wieder. Hansens grosses, gutmütiges Gesicht tauchte vor ihr auf. Sie blubberte und spuckte. "`Supr, Meidli, hesch s bis zur erschtn Fääri gschafft!"' strahlte der Wilde Mann. "`Weiter so!"' Amélie konnte nichts erwidern.

Hans stellte sie vorsichtig auf die Füsse. "`Hab uf din Körper ufgpasst, als du dich mit däne Gnome amüsiert hesch!"' schmunzelte er zufrieden über Amélies zorniges Funkeln in den Augen. "`Amüsiert, ja?"' fragte sie zurück und ballte die Fäuste. "`Joh, mei, sie mache halt scho emool s klises Schärzli!"' "`Ich kann drauf verzichten!"' keifte Amélie und klappte den Mund zu und staunte. Auf Hans ausgestreckter Hand stand ein Zwerg. Genauso wie Amélie sie sich immer vorgestellt hatte, wenn Berta ihr Märchen erzählte. "`Do! lueg! Es hät sich extra für dich schön agleit und useputzt!"' bemerkte Hans.

Der Zwerg trug eine kleine Hose aus braunem Stoff, Kindergummistiefel in blau-gelb mit einem Schiffchen drauf. Oben hatte er ein blaues Hemd angezogen und trug eine Pelzweste. Sein rundes Gesicht mit geröteten Apfelwangen und verschmitzten Äuglein, schaute verlegen aus. Auf dem Kopf trug er eine hohe, spitze rote Kappe. Er streckte versöhnlich die Hand aus. Vorsichtig nahm Amélie sie, sie war so gross, wie die eines Säuglings, allerdings nicht so zart, sondern schwielig und rau.

"`Gratuliere! Du hast die wacker geschlagen. Das Element der Erde und seine Bewohner stehen von nun an zu deiner Verfügung."' Sagte der Gnomenkönig. Amélie bekam keinen Ton raus. Der kleine König, der sich sichtlich in ihrer Bewunderung sonnte und sich wie ein Minimannequin hin und her drehte, grinste verschmitzt. Er stupste Amélie mit dem kräftigen Finger an ihre Nase, die ihm immer näher gekommen war und meinte frech: "`Jo, also Meitli, aus Silber s Kacki mache, hesch jetzt gelernt! Jetzt muesch s numme noch umkehre und aus Kacki Gold mache!"' Mit gackerndem Lachen löste er sich auf. "` Haha, aus Kak\dots"' Hans schlug sich lachend auf den Schenkel, dann bemerkte er Amélies finstere Miene und wurde ernst.

Aber dann musste Amélie selbst grinsen. Sie begann zu strahlen und ein wohliges, warmes Gefühl breitete sich aus. Wie ein kleiner Fahrstuhl erzeugte es kleine Schauer, die Amélie in ihrem Bauch kitzelten. Sie lachte und Hans lachte. Amélie spürte das Glück. Und überrascht bemerkte sie, dass es fest in ihr verankert war, wie eine gelbe Glückskugel in ihrem Bauch, die leuchtete, wenn sie sich freute. Sie blickte zu Hans. Auch Hans hatte einen leuchtenden Lichtball unterhalb seiner nackten Brust, der kräftig leuchtete und blasser wurde, als Hans wieder ruhiger wurde. Hans bemerkte Amélies Blick."`Vergiss es nicht, Meitli! Glück isch asteckend!"'

"`Guet gmacht Meitli! D erschte Prüfig hesch du supr gemeistert!"' Hans sah Amélie an. "`Machs bei der zweiten genauso guet, ich drück dir d Dume!"' Amélie balancierte mit Hanses Hilfe auf dem Rand der Fähre und sprang wieder ins Wasser, gefolgt von vergnügtem Gelächter und dem klatschenden Geräusch von Schenkelklopfern.

\section*{5}
\addcontentsline{toc}{section}{5}

Sobald Amélie den Fluss berührte, rauschte der breite geschuppte Rücken von Sobek heran. Das gewaltige Krokodil stiess sie vor sich her, wie ein Seehund im Circus seinen Ball balancierte. Was Amélie auch versuchte, das Krokodil war stärker, schneller und schleuderte sie immer erneut in die Luft.

Zwischendurch liess es Amélie eine kleine Verschnaufpause, genug Zeit, damit sie wieder und wieder ansehen musste, wie die dicke Schnauze auf sie zu schwamm und sie wieder auf die Schippe nahm und fortschleuderte. Es tat nicht weh. Aber Amélie fühlte sich, nachdem sie den ersten Schreck überwunden hatte und sie nicht sofort im Maul des Reptils auf nimmer wiedersehen verschwunden war, hilflos. Und die Hilflosigkeit machte sie von mal zu mal wütender. Als das Krokodil erneut angriff, kochte die Wut über und Amélie brüllte.

Seltsamerweise gab es ein viel lauteres und kräftigeres Echo. Bis Amélie einen Löwen bemerkte, der in der 'Leuen-Fähre' an der Pfalz des Münsters stand und sich ihr entgegen reckte. Er trug einen Stecken im Maul, der lichterloh brannte.

Amélie nahm alle Kräfte zusammen und schwamm auf die Fähre zu, die mitten im Fluss stand. Die Wut gab ihr ungeahnte Macht über ihren Körper. Als sie bei der Fähre anlangte, griff sie nach dem brennenden Stab und riss ihn dem Löwen, der sich ihr entgegen gereckt hatte aus dem Maul.

Als das Krokodil auftauchte und auf sie zu schwamm, schlug sie so gut sie konnte die brennende Fackel nach der Schnauze. "`Nimm das, du Ungeheuer!"' brüllte sie. Jedoch das Krokodil tauchte ab und die Fackel erlosch, sobald sie das Wasser berührte. Sie entzündete sich wieder, wenn sie wieder an die Luft kam. Amélie schrie erneut. Die Wut wurde übermächtig und sie glaubte daran zu ersticken. Das Wasser um sie, brodelte und kochte. Dann  verschwand es.

Amélie war plötzlich von Feuer umgeben. Dann wurde sie Gestalten darin gewahr, die wie kleinere und grössere Flammen schnell und kräftig hin und her huschten. Sie sprangen an den Stab und entzündeten ihn sie sprangen über Amélie. Ihre Haare standen in alle Richtungen, als ob sie in die Steckdose gefasst hätte. Die Wesen waren sehr schön und fein, obwohl sie feurig waren. Amélie bemerkte noch etwas anderes.

Die Wesen der Feuersalamander schienen sich vor einem Vorhang zu bewegen. Vor einem Durchgang von dem Amélie sich magisch angezogen fühlte. Sie kämpfte sich durch die Feuersalamander dem Tor entgegen, doch die Schar der Feuerwesen drängte dichter zusammen, die Hitze war kaum mehr zu ertragen. Amélie schrie vor Wut und Enttäuschung und der Stab loderte hell auf und verbrannte ihr die Hand. Sie liess ihn fallen.

Durch einen dichten Nebel vernahm sie Thots Stimme: Ruhig bleiben, aber nicht die Gefühle verdrängen. Benutze sie! Amélie hustete, ich muss hier weg, ich halte die Hitze nicht aus\dots Sie nahm den Stab in die andere Hand, dieses mal vorsichtig. Sie hielt ihn ausgestreckt vor sich und konzentrierte sich. Die Wut kreiste in ihrem Bauch und von dort durchspülte sie den ganzen Körper. Amélie versuchte dem Fluss zu folgen. Sie sah den roten Strom und als er ihren Arm entlang glitt, sprang er auch auf den Stab über. Amélie zitterte, blieb aber ruhig und gelassen: Die Spitze des Stabes begann zu glühen! Gleichmässig und heiss!

In diesem Moment stürzte sie wieder zurück in den Fluss, wo sie von Sobek, dem Krokodilsgott bereits erwartet wurde. Doch mit der Ruhe, die sie gewonnen hatte, war der Stab endlich zur einer brauchbaren Waffe geworden. Die Glut erlosch nicht mehr, wenn sie das Krokodil schlug. Ihre Gedanken waren konzentriert, ihr Körper bündelte die Kraft, die die heisse Wut in die Muskeln schickte und so schaffte Amélie es sich auf den Rücken des Krokodils zu schwingen.

Wie eine Amazone sass sie aufrecht auf dem Reptil und schlug ihm den Stab auf den Kopf, sobald es Anstalten machte, sie abzuschütteln. "`Nichts da!"' zischte sie. In dem Augenblick kamen sie unter der Mittleren Brücke durch. Ein Schiff kam auf der linken Seite des Rheins hinauf gefahren. Die Wellen brachen sich an den Brückenpfeilern und schwappten über das Reptil und seine Reiterin. Der Strudel hinter der Brücke verstärkte sich und das Krokodil nutzte die Gelegenheit Amélie abzuschütteln. 

Sie kämpfte verzweifelt gegen den Sog der Walze an, doch es war zu spät sie ging unter. Aber sie war vorbereitet, statt nach oben, tauchte sie so tief sie konnte und die Strömung der Walze aufhörte. Sie spürte den sandig, schlammigen Boden und stiess sich ab. An der Oberfläche atmete sie tief durch. Verdammtes Schiff! Sie hatte den Stab verloren! 


\section*{6}
\addcontentsline{toc}{section}{6}


Die Strömung hatte sie schon weiter getrieben und als sie sich umschaute, wo das Krokodil geblieben war, stiess sie mit dem Rücken gegen die Gryffenfähre. Sie hatte Glück, denn sie bekam ein Seil zu fassen, das von der Fähre herab hing und konnte es halten bevor die Strömung sie unter das Boot drückte.

Ihr Kopf prallte an die Planken der Fähre, aber sie schaffte es das Seil festzuhalten. Am hinteren Ende der Fähre schlug sie gegen das Ruderblatt. Ein Schatten war im hinteren Fenster der der Kajüte aufgetaucht. Er hatte einen gewaltigen Schnabel. Amélie wollte vor Schreck los lassen, als sie den breiten Rücken des Reptils aus der Flussmitte herankommen sah. Sie packte das Seil fester und zog sich so gut sie konnte daran hoch.

Der grosse Vogel beugte sich durch das Fenster weit zu ihr runter, er hatte ein längliches Bündel im Schnabel. Amélie packte es. Es war ein Schwert. Ein kleines Schwert in einer Scheide. Sie liess das Seil los. Der Fluss packte sie sofort. Das Krokodil kam schnell auf sie zu geschossen. Amélie versuchte das Schwert aus der Scheide zu ziehen, aber sie sank dabei. 

Noch ehe sie wusste, was geschah, hatte der Krokodilgott sie hoch in die Luft geschleudert. Zu ihrer Verblüffung fiel sie nicht wieder ins Wasser, sonder stieg weiter und weiter auf, bis sie schliesslich hoch über der Stadt schwebte. Sie kreiste in der Luft, die zahlreichen Lichter unter ihr, bildeten ein grossartiges Muster. Der Fluss, war mal als dunkles, mal als an den Rändern beleuchtete Ader bis weit an den Horizont zu erkennen. 

Am Ende des Flusses leuchtete am Himmel der Nordstern. Von dort aus dem Norden war sie hier her gekommen. Eine Welle von Heimweh überkam sie. Sie dachte an ihre Eltern und an ihren Bruder. Der kleine 'Prinz' wie in alle nannten, war schwächlich und empfindsam geboren. Der kleinste Luftzug konnte ihn für Wochen auf das Krankenlager strecken. Und dennoch schien es zu Hause nur ihn zu geben. 

Amélie zog das Schwert aus der Scheide. Die Scheide war ein dunkler Schatten vor den Lichtern. Amélie fühlte sich elend. Wieso konnten alle ihren Bruder besser leiden als sie? Wieso durfte er in seinem warmen Bettchen im Gutshof selig schlafen und sie musste weit fort von ihrer Familie diese Abenteuer bestehen? Missmutig schlug sie mit dem Schwert in die Luft. Es hinterliess einen schwarzen Schweif, als hätte sie mit dem Schlag alle Lichter in dieser Richtung ausgelöscht. 

Warum? Warum? Warum? Dreimal schlug sie mit der Klinge zu. Amélie stand im Dunklen. Allein. Die Eifersucht wirbelte in ihrem Körper und hinterliess giftig-grüne Schlieren. Amélie war absolut still. Sie hörte ihr Herz schlagen, warum habe ich so ein verfluchtes Herz? Begehrte sie auf. Doch die Stille verschluckte die Gedanken.

Amélie erinnerte sich an Thots Worte: Ruhig bleiben, aber nicht die Gefühle verdrängen. Benutze sie! Und gleichzeitig war ihr als würde sie eine sanfte Stimme vernehmen, 'du hast schon soviel geschafft!'

Unsicher schlug Amélie mit dem Schwert. Die Klinge hatte einen Teil des Schattens weggewischt. Sie bemerkte den Wind, der sie streichelte. Wie viele zarte, sanfte Hände, die sie streichelten. Amélie spürte wie sich die Verkrampfung der Eifersucht löste. Sie sah ihren kleinen Bruder, der schwach und blass in seinem Bett lag und stöhnte, weil er wieder einen Alptraum hatte. "`Vielleicht möchte ich nur auch so tapfer und geduldig sein wie du!"' dachte sie. Die Eifersucht in ihrem Bauch wandelte sich und wurde zu einem schönen grün. Einem Grün der Bewunderung.

Amélie spürte die Sylphen, die Luftgeister, die sie sachte, sachte wieder dem Fluss entgegen gleiten liessen. 'Bald hast du es geschafft, Menschenwesen' wisperten sie. 'Rufe uns, wenn du klare Gedanken brauchst!'

Amélie glitt langsam ins Wasser. Aber da kam Sobek schon wieder! So gut es ging ruderte sie mit einem Arm, das Schwert in der Scheide in der Hand. Das Krokodil war direkt bei ihr. Sie zappelte und fühlte Grund unter den Füssen! 

Es blieb keine Zeit das Schwert zu ziehen, also rammte Amélie es mit der Scheide in den weit geöffneten Rachen hinein. Das Krokodil bäumte sich auf und klatschte rückwärts ins Wasser, dann war es verschwunden. Es liess eine Spur von Blut und aufgewühltem Schlamm hinter sich. 

 Amélie stand in der Nähe des Ufers im Wasser und bemerkte die Wassernymphen, die ihr aus einiger Entfernung zugesehen hatten. Sie hatten die Hände vor den Mund geschlagen und ihre grossen Augen schillerten ängstlich in der Dunkelheit. 
 
Neben ihnen tauchte ein breiter, kräftiger Männerkörper aus dem Wasser, mit dem Dreizack in der Hand und winkte Amélie. Sie seufzte und glitt wieder ins Wasser. Der Mann lachte und schwang seinen Dreizack. 

Die Strömung schaukelte Amélie unter der Johanniterbrücke durch. Von dem Krokodil war weit und breit keine Spur zu sehen. Dafür wurde sie schwächer und schwächer. Ihr Kopf verschwand unter Wasser und sie konnte nichts dagegen tun, weil sie erschöpft war. Sie fühlte die Bewegungen des Wassers, des Flusses und der Strömung und war plötzlich müde.

\section*{7}
\addcontentsline{toc}{section}{7}

Auf der Barke, die während der Prüfung nebenher den Fluss hinuntergeglitten war, war es mucksmäuschen still. Von dem Kampf den Amélie mit dem Herrscher der Lebenskräfte geführt hatte, hatten die Götter nicht mehr gesehen, als das schäumende Flusswasser und die aufgescheuchten Nymphen.

Der Flussgott Rhenus hatte zwar versprochen sie auf dem Laufenden zu halten, war aber über den spannenden Kampf in seinem Fluss in Aufregung geraten. Es fiel ihm offensichtlich schwer Amélie nicht mit dem Dreizack zur Hilfe zu eilen. Dann tauchte er an verschiedenen Stellen aus dem Wasser und scheuchte aufgeregt seine Nymphen durcheinander, oder schüttelte seine Fäuste. Während des Kampfes waren ihm nostalgische Erinnerungen an seine frühere Macht und sein Ansehen gekommen und hatten ihn für kurze Zeit sein heutiges, langweiliges und von Staustufen unterbrochenes Flussgottleben vergessen lassen.  Aber er wusste, wie alle anderen Wesen und Götter, Amélie musste diese Kampf selbst ausfechten. 

Nur mit der Hilfe der vier Elemente und der damit verbundenen Insignien der verschiedenen, menschlichen Kräfte, durfte und konnte sie den Sieg davontragen. Und nur die Vertreter der Elemente, in Basel waren es die vier Wappentiere der Fähren über den Lebensstrom der Stadt, durften die Insignien: Münze, Stab, Schwert und Kelch übergeben.

Um die weiteren Abenteuer zu bestehen, brauchte es einen Menschen, der seine Lebenskraft und die damit verbundenen Fähigkeiten nicht nur wahrnehmen, sondern auch steuern konnte. Es gab nur einen ägyptischen Gott, der für diese Prüfung als Lehrer in Frage kam, den strengen und gefürchteten und zum Leidwesen seiner Prüflinge sehr gefrässigen Sobek, der sein Leben als Krokodil in vollen Zügen lebte und auskostete. Im wahrsten Sinne des Wortes auskostete, wie Thot zu predigen pflegte und es auch Amélie mehrmals eingebläut hatte.

Wenn er keine Prüfungen abnahm, war Sobek bei allen Göttern beliebt, denn er hatte einen deftigen Humor, scherzte gerne und war sehr praktisch veranlagt, was seine Problemlösungsstrategie betraf: Entweder man konnte Dinge essen, dann war es gut. Oder man konnte sie nicht essen, dann verschluckte man sie im Stück, oder vergrub sie im Schlamm bis sie essbar wurden. Dann war es auch gut.

Die Götter hielten den Atem an, denn die Prüfung war sowohl für den Lehrer als auch für den Prüfling gefährlich. An der Flussseite der Barke, die zwischen Johanniterbrücke und der Uelifähre an der Grossbaslerseite vertäut lag kräuselte sich das Wasser. "`Hrrngh, hraaanng!"' Der weit geöffneter Rachen des Krokodilgottes tauchte auf und erschreckte einen der Ruderer, der ins Wasser fiel. Die Götter reckten die Köpfe, die Barke schwankte gefährlich auf die Flussseite. 

Der mutige Horus getraute sich am dichtesten an das offene Maul. "`Das Schwert steckt in seinem Schlund fest"' rief er. 'Ich kann es rausholen' bemerkte Duamutef. Es blitzte kurz im Auge des Krokodils, was Horus dazu brachte das riesige Reptil an seinem Hals zu packen. Er versuchte das Krokodil zu schütteln, gab es aber auf und begnügte sich es anzubrüllen:"`Wenn du meinem Sohn ein Härchen krümmst, dann bekommen alle Ladys auf der Barke morgen eine Krokotasche kapiert?"' Sobek versuchte zu nicken, ohne den wütenden Gott, der sich an seinen Hals klammerte, ins Wasser zu werfen und Gefahr zu laufen, tatsächlich als Handtasche zu Enden. Es war halt Reflex, grummelte er für sich, vermutlich konnte er die ganze nächste Woche nichts zu sich nehmen\dots, zumindest nichts, was nicht mehrere Tage im Wasser vergammelt war\dots Natürlich hatte er vorgesorgt\dots

Während Horus das Krokodil an der Gurgel gepackt hielt, fasste Thot Duamutefs Schwanz, denn der Schakal kletterte kopfüber in den Rachen und versuchte das Schwert zu schnappen und heraus zu ziehen. Sobald sich das Schwert lockerte und der Schakal es fest zwischen den Zähnen hatte, begann er zu strampeln. Thot riss an dem Schwanz und schleuderte den Schakal im hohen Bogen über seinen Kopf in die Barke. Laut krachte das Gebiss des Krokodils zusammen. Die Barkenbesatzung schauderte synchron. Sobek richtete sich auf und warf sich rückwärts in den Fluss und verschwand. Im Licht das sich auf dem Wasser spiegelte schimmerte ein blutroter Streifen.

Duamutef war winselnd mit dem Schwert unter den Sitz der Barke geflüchtet und leckte sich die Rute. Dies war nicht sein Tag gewesen und scheinbar war es auch nicht seine Nacht.

\section*{8}
\addcontentsline{toc}{section}{8}

Anubis hatte einen Pelzsack auf dem Rücken, der aus dem Fell eines schwarzen Hundes bestand. Es sah aus, als würde er einen zweiten Hund tragen. Er trug den Imiut. In dem Imiut war Osiris. Er war der sicherste Ort an dem Osiris reisen konnte. Damals, vor unzähligen Zeiten, als sein Bruder Seth ihn umgebracht und zerstückelt hatte, hatten Isis und Anubis jeden seiner einzeln verteilten Knochen am Nil zusammengesucht und im Imiut zurück gebracht. 

Anubis war Osiris Sohn, allerdings der Sohn, den Osiris mit Isis Schwester Nephtys hatte. Nephtys war die Schwester von Isis, Osiris und Seth und mit Osiris grösstem Feind und Bruder Seth vermählt\footnote{Wie den meisten Lesern bekannt sein dürfte, herrschte, um es im westlichen, religiösen Kontext zu sagen, in der ägyptischen Götterfamilie Sodom und Gomorah. Aber andere Länder, andere Sitten, andere Zeiten\dots ,Zeiten als Blut tatsächlich dicker war, als Wasser\dots}. Nephtys hatte Anubis heimlich geboren und ihn dann Isis anvertraut, die ihn vor Seth versteckte und mit Horus zusammen wie einen Sohn aufzog. Osiris war nach dem Mord, den sein Bruder an ihm verübt hatte, vom Herrscher des Diesseits zum Herrscher des Jenseits geworden. Sein Bruder konnte ihn im menschlichen Sinn nicht töten, aber aus der Welt der Physis verdrängen, indem er ihm den Körper raubte und unbrauchbar machte. Umso mehr mussten Nephtys und Isis ihre Kinder  vor Seth schützen, bis sie gross genug waren, sich selbst zu verteidigen.

Horus zog aus und kämpfte mit Seth. Er wollte das verlorene Erbe seines Vaters wieder zurück holen. Der Kampf war gewaltig und am Ende musste der Sieger mit Hilfe der göttlichen Richter bestimmt werden, da beide Kämpfer gleichstark waren. 

Seth war jedoch nicht für seine Fairness bekannt und versuchte Horus mit einem üblen Trick kampfunfähig zu machen. Doch Isis, die ihren Bruder nur zu gut kannte, war auf der Hut, bekam Wind von den Machenschaften und heckte ihrerseits mit Horus eine List aus. Die göttlichen Richter entschieden sich schliesslich für Horus als Sieger und somit wurde er Herrscher über das Diesseits. Allerdings fürchteten sie die Kräfte und den unbeherrschten Charakter Seths und er bekam als Abfindung einen grosszügigen Palast am Himmel. Dort konnte er ungestört donnern und grollen, ohne Schaden anzurichten

Anubis schlug einen anderen Weg ein als sein Halbbruder Horus. Er half seiner Stiefmutter Isis, der er sehr verbunden war, die Teile von Osiris zu finden und zusammen zu fügen. Dabei entwickelte er die Technik der Einbalsamierung. Seine Berufung wurde es die Toten, die das Jenseits betraten, zu empfangen und ihnen den Weg zu weisen und ihre Körper und Seelen für die zahlreichen Abenteuer vorzubereiten, die die Toten im Totenreich erlebten bevor sie sich nach Sechet-iaru, dem Teil der Duat, den ein Christ als Paradies bezeichnen würde, zurückziehen und den Tot geniessen konnten\footnote{Ja, auch das Paradies entsprach nicht ganz den christlichen Wertevorstellungen. Es enthielt weder flauschige Wolken, Nachthemden noch Harfen. Dafür durfte man nach Herzenslust saufen (Bier) und Essen und sich mit seiner Frau vergnügen, nachdem man auf Herz und Nieren geprüft worden war. Man musste im Paradies auch arbeiten, da man ein Anrecht auf eine Parzelle fruchtbares Sumpfufer hatte, aber der kluge Lebende hatte vorgesorgt und darauf bestanden Uschebtis mit ins Grab zu nehmen. Diese kleinen, menschlichen Figuren wurden im Jenseits lebendig und verrichteten die mühsamen, unbeliebten Arbeiten: Planst du dein Leben nach dem Tot, so hast du später keine Not, war das Motto des damaligen Ägypters. Alles in allem, lebte es sich im alten Ägypten nicht schlecht, sowohl vor als auch nach dem Tot. Vorausgesetzt jedoch, man hatte tadellos gelebt, sich für das Jenseits eingerichtet und verfügte im Diesseits über nette und attraktive Geschwister.}.

Er wurde schliesslich gemeinsam mit Thot zum Obersten Totenrichter. 

Jetzt war er jedoch ein besorgter Sohn und Neffe, denn er war für seinen Onkel Osiris verantwortlich. Sie wollten Osiris so dicht wie möglich dabei haben, wenn die Lebenskräfte von Amélie geweckt würden. Noch war nicht sicher, ob diese ein Opfer, ein Todesopfer forderten. Sollte das geschehen, wäre auch Osiris in Gefahr. Seine schwachen Kräfte im diesseitigen Raum würden im selben Augenblick verschwinden.

Anubis wartete angespannt mit Isis. Sie starte in den Imiut, indem sich der Herr der Duat befand. Wobei das falsch ausgedrückt war: Es befanden sich seine Knochen darin. Es war ihnen nicht möglich gewesen mehr von dem grünen Gott mit zu nehmen\dots

"`Es ist gleich soweit!"' flüsterte Isis. Sie zitterte vor Aufregung. "`Wenn es schief geht\dots"' -'Dann springe ich sofort ins Jenseits, Mutter!' Anubis stand an Deck und starrte neben Isis zur Uelifähre hinüber, die sich der Mitte des Flusses näherte. "`Lass' es gelingen!"' Isis packte ihren Stab fester. Sie war die grösste Magierin, die die Götter hatten. Obwohl, es gab einen, einen der so mächtige Magie kannte wie sie: Ihr Bruder Seth!

Doch in diesem Fall war Magie nutzlos. Das wussten sie alle. Die Lebenskraft konnte verlängert, gedehnt, geliehen und aufgeblasen werden, aber dafür musste sie jemand opfern, zur Verfügung stellen und es war nur kurze Zeit möglich. Und es brauchte die Kraft der Lebenden. Der lebenden Menschen, eines lebenden Menschen, die Amélies.

Es brauchte einen klitzekleinen Funken lebendige Herzens- und Lebenskraft, Liebeskraft eines Menschen\dots

\section*{9}
\addcontentsline{toc}{section}{9}

Sie zwang sich die Augen zu öffnen und bemerkte dicht um sich herum die Nymphen, die eng um sie schwammen, sich lebendig und weich, warm und kühl gleichzeitig, quirlig und neugierig um sie scharrten. Amélie drehte sich auf den Rücken und liess sich treiben. Sie blickte hoch zu den Lichtern am Himmel und an den Ufern. Ein Spritzer, ein lustiger Strudel traf sie an Hand und Gesicht.

Sie war bei der letzten Fähre angekommen. Der Fährmann zog sie ins Boot und legte ihr eine Decke um. Er selbst war in einen Mantel mit Kapuze gehüllt. Er zitterte. Auch Amélie zitterte, vor Erschöpfung und Kälte.

Der Fährmann brachte ihr einen Schale aus klarem, durchsichtigen Kristall. Darin befand sich eine rote Flüssigkeit. Amélie nahm dankbar die Schale. Sie war schwer und Amélie wäre sie beinah aus der Hand gerutscht, wenn der Fährmann nicht schnell seine Hände um ihre und die Schale gelegt hätte. Vorsichtig hob er die Schale an ihren Mund und sie trank sie in einem Zug leer.

Es schmeckte metallisch, nach Blut. Während die Flüssigkeit durch ihre Kehle ran, wurde es immer kälter und kälter. Sie bekam keine Luft. Und als sie nach Atmen rang, merkte Amélie wie die Flüssigkeit einem eisernen Ring gleich ihren Brustkorb einschnürte. Amélie knallte auf die Planken, der Kelch zersprang, funkelnde Stücke, wie Diamanten sprangen über den Boden der Fähre. Jetzt ist es aus, dachte sie, auf der letzten Fähre\dots 

Amélie schwebte über dem Wasser. Über der Fähre. Sie sah den Fährmann, der sich über ihren regungslosen Körper beugte. Sie sah einen dunklen Schatten, der ihren Körper umschlossen hielt und huschte in ihrer durchsichtigen, lichten Gestalt dichter heran. Es gab eine Verbindung zwischen ihr und ihrem Körper. Ein lichtes, blaugrünes Band in sich gedreht, wie eine Nabelschnur. Es wurde blasser und blasser, den das Herz, ihr Herz, war im Zentrum des schwarzen Schattens.

Verzweifelt beugte sich Amélie über ihren Körper, sie wollte genau sehen, was das für ein Schatten war. Sie schrie, stimmlos, das Herz war in einer uralten, verstaubten Binde eingewickelt und mit schwarzem Pech und Teer verklebt. Wie ein Stein. Sie griff trotzdem danach, obwohl sie vor Grausen wie gelähmt war. 

Da spürte sie noch etwas. Ein zaghafte Bewegung. Ein kleines Vögelchen schien  im Inneren der festen, schwarzen Kruste zu flattern. Das Herz, es schlug! Noch!

Amélie bemerkte die Schnur, die aus dem Herzensstein kam. Auch sie pulsierte. Es dauerte ein Augenblick und dann begriff Amélie: Die Pulsschläge, die das Herz lebendig hielten, kamen nicht aus dem Herzen, sondern von Ihr, ihrer durchsichtigen, leuchtenden Gestalt. Die Lebenskraft.  Amélie erinnerte sich an Thots Worte. Die Prüfung bestand darin ihre Lebenskraft in dem Lebensstrom einer ganzen Stadt, die durch das verfluchte Herz abgeschnürt und zerdrückt wurde, zu erneuern.

Ich muss in meinen echten Körper zurück. Ich bin ausgesperrt, der Fluch lässt mich nicht zurückkehren! Amélie versuchte mit ihren durchsichtigen Fingern die Öffnung in der schwarzen, harten Panzerung, durch die die Schnur zu ihrem Herzen führte, zu weiten. Doch es war unmöglich. Sie schaute sich wild um, da bemerkte sie eine Scherbe des Kelchs. Es klebte von dem Blut daran. Die Finger griffen, wie sie es erwartet hatte durch die Scherbe hindurch. Aber sie hielten dennoch eine Klinge! Eine blutrote Klinge. 

Da setzte der Pulsschlag aus und Amélie spürte die Verbindung mit dem Bündel Mensch am Boden zerreissen. Der Lebensleib gehorchte plötzlich anderen Gesetzten und begann sich aufzulösen und zu schweben\dots Amélie nahm alle Kraft zusammen in diese eine Hand in diese blutrote Klinge und fuhr damit in den Spalt des Fluches und mitten ins Herz.

Amélie öffnete die Augen. Sie sah die Planken direkt vor sich. Ich bin wieder da! Ich habe es geschafft! Ich lebe. Sie rappelte sich auf. Der Fährmann war auf die Schiffsbank gesackt und schluchzte. Er trug immer noch die Kapuze. Scheu legte Amélie ihm die Hand auf die Schulter. Er zuckte zusammen und schaute auf! 

Es war Amset. Er erhob sich und legte den Arm um ihre Schultern. Über ihnen rief ein Falke. Und als Amélie ein helles Licht in Amsets Augen spiegeln sah und sich umdrehte, kam die Barke der Götter auf sie zu. Maat stand mit erhobenen Armen am Bug und hinter ihr Isis und das ganze Schiff schimmerte in zart rotem Morgenlicht. Unter dem Baldachin war nicht der Widderköpfige zu sehen, sondern ein in allen Farben des Morgenhimmels glänzender Käfer.

Amélie staunte mit offenem Mund, den der Käfer wurde grösser und grösser und durchsichtiger und schob mit den Hinterbeinen einen riesigen goldenen Ball in die Höhe. Gleichzeitig sah sie die Stadt im Dämmerlicht des Morgens.

Nun bemerkte sie Thot, er streckte den Daumen nach oben. Er grinste und wischte sich über die Augen. Anubis sass still und majestätisch daneben. Auf seinem Rücken leuchtete der Imiut in blaugrünem Licht. "`Ich habe es geschafft!"' flüsterte Amélie. "`Ja, du hast es geschafft!"' Amset packte Amélie und wirbelte sie im Kreis: "`Du hast es geschafft!"'. "`Ich habe es geschafft!"' Er schaute ihr tief in die Augen\dots

"`Amsi?"' raunte sie. "`Ja, Amélie?"' hauchte er. "`Ich,\dots Wir, wir sind nicht allein!"' Flüsterte sie und heftete den Blick auf etwas in Armsets Rücken. Amset drehte sich um, wobei er Amélie immer noch an sich gedrückt hielt. Ihre Füsse baumelten in der Luft. Die Barke hatte neben der Fähre angehalten und die Götterschar schaute verzückt auf das Paar in der Fähre. Sie sahen sich an und wussten beide, der Moment für Romantik war vorüber. Amset stellte Amélie sacht auf die Planken. Die Scherben des Kelchs funkelten rosa im Morgenlicht rund um ihre Füsse. Sie hörten ein vereinzeltes kichern aus der Barke. Dann lachten sie beide, bis sie sich die Bäuche halten mussten. 
