\part*{Sechste Stunde\\"`Ankunft, die den rechten Weg gibt"'}
\addcontentsline{toc}{part}{Sechste Stunde}

\chapter*{6. Tag, Jonathanstag}
\addcontentsline{toc}{chapter}{29. Dezember, Jonathanstag}

\section*{1}
\addcontentsline{toc}{section}{1}

Sie starrten sich an. Schliesslich hielt Luise den Blick ihres Mannes nicht mehr aus, den gekränkter-Dackel-Blick. ,,Was ist?'' fragte sie spitz. ,,Warum trägst du Handschuhe?'' fragte Wilfried zurück. Luise sah wie der Dackel im Inneren ihres Mannes  plötzlich die Zähne fletschte und sie anknurrte. ,,Weil ich es halt so will!'' Sie drehte sich um und beugte sich über das grosse Bett in dem der Junge schlief. 

,,Aufwachen, Goldkind!'' sagte sie und strich dem Buben sanft über das blonde Haar. Er hatte viel vom Aussehen seiner Mutter. Die blonden, feinen Haare, den feingliedrigen Knochenbau. Die langen dünnen Gliedmassen und Finger. Amélie hat mehr von mir, dachte Wilfried. Die schwarzen, kräftigen Haare und den sehnigen, muskulösen Körper\dots Wo war sie, seine Tochter? 

Sobald Wilfried an Amélie dachte, hatte er das zuckende, blutige Herz vor Augen, das er im Traum in den Tonkrug geworfen hatte. Er zog langsam die linke Hand ein Stück aus der Hosentasche, weit genug, bis er die rostrote Hautfärbung sehen konnte.

,,Luise, wir\dots '' begann er. ,,Wir haben nicht viel Zeit, wir müssen Amélie suchen!'' antwortete sie schnell. Sie drehte Wilfried den Rücken zu und tat als würde sie dem Jungen beim Anziehen helfen. 

Wilfried ging zur Tür: ,,Gut! Ich warte unten im Frühstücksraum auf euch!'' Er verliess das Zimmer. Sobald er fort war, eilte Luise in das Badezimmer und zog sich die Handschuhe aus. Ihr Spiegelbild reckte ihr vorwurfsvoll zwei blutrot befleckte Hände entgegen. Sie versuchte sich die Hände sauber zu waschen, aber es gelang ihr auch diesmal nicht. 

,,Mama?'' der Junge klopfte an die Badezimmertür. ,,Ich muss mal!'' Luise öffnete ihm die Tür. ,,Warum hast du die Handschuhe schon an?'' fragte der Kleine, als er sich auf das WC setzte. ,,Damit wir Amélie schneller suchen können!'' antwortete Luise. Als sie das Zimmer verliessen und zum Frühstücksraum gingen, zog sich der Junge seine dicken Fäustlinge über und stapfte grimmig vor seiner Mutter her. Eine kleine Träne rollte über Luises Wange und streifte ein winziges Lächeln.

\section*{2}
\addcontentsline{toc}{section}{2}






 Es war stockdunkel. Durch den dichten Wald blitzten die Scheinwerfer eines Autos auf. Die Räder des alten Leichenwagens quitschten, als er die engen Kurven ins Tal hinunter brauste. Aus den Lautsprechern rumpelten rockmusische Bässe und schoben ihrerseits den Wagen vorwärts. 

Tef jaulte und Hapi kreischte. Amset brüllte den Text mit und riss im Takt das Lenkrad hin und her in den scharfen Kurven der Serpentinen. Dann rasten sie auf dem Kamm der Berge ins Tal.

Sobi sagte nichts. Er rumpelte im hinteren Teil des Wagens hin und her, nachdem er es aufgegeben hatte, sich mit dem kräftigen Schwanz und seinen vier Tatzen an den Wänden abzustützen. Ihm war nicht direkt schlecht, denn Krokodile sind zu gierig, um ihre Beute auszukotzen. Vielleicht schimmerte seine grünliche Haut um die Schnauze grüner als sonst. Sobi sagte nichts, was sollte er sagen? Er hoffte, sie würden trotz Amsets Gefühlschaos heil unten ankommen. Amélie war noch in dem Merowingergrab und keiner von ihnen wusste zu sagen, in welchem Zustand sie sich befand. Lebte sie? War sie bei Verstand?

Als sie an der Kreuzung auf die Route de Champs de Feu trafen, bogen sie ab Richtung Klingenthal. Sobald sie am Waldrand eine geschützte Stelle ausgemacht hatten, bremste Amset den Wagen. 

Sobis Schnauze lag nun auf dem Amaturenbrett, Hapi, der noch vor wenigen Sekunden dort gesessen hatte, hing an dem Lederbezug des Wagendaches und schrie.

Tef hatte sich im Fussraum des Beifahrersitzes zu einer Kugel zusammengerollt und schielte über seinen Schweif. Sobis Vordertatze ruderte hilflos in der Luft und verfehlt immer wieder knapp Tefs Nase. 

Amset war schon aus dem Wagen gesprungen und öffnete die Heckklappe. Er zog energisch am Schwanz des Reptils, was nicht viel nützte.

Vorsichtig schob sich Sobi rückwärts aus dem Wagen. Er seufzte glücklich, als er die Ehn sah. Schnell wie der Blitz war der Krokogott in den kleinen Fluss geklettert und lies sich von der Strömung Richtung Strassburg weisen.

Der kleine Fluss war nicht überall tief genug, aber Sobi machte es nicht viel aus ein Stück im flachen Wasser zu waten. Er grunzte und stapfte fröhlich los.

Ein kleines Hüngerchen machte sich bemerkbar und Sobi hielt freudig Ausschau nach den kulinarischen Leckerbissen, die ihm auf seiner Flusswanderung begegnen würden. 

\sterne

An diesem Morgen gingen bei der Polizei Bas rhin und Haute rhin mehrere Meldungen ein. Entlang des kleinen Flusses Ehn und des grossen Flusses Ill, der durch Strassburg floss, wurden zahlreiche Hunde vermisst, die am Ufer dieser Flüsse mit ihren Besitzern spazieren gegangen waren. An vielen Orten vermissten die Spaziergänger die zahlreichen Schwäne. Ein, zwei Rehe verschwanden, aber das fiel keinem Menschen auf\dots 

Ein kleines Mädchen, das mit seinen Eltern einen Spaziergang gemacht hatte und am Ufer des Flüsschens Ehn Steine in das Wasser geworfen hatte, erzählte seinen eigenen Kindern später folgendes Geschichte. 

Sie war felsenfest davon überzeugt ein riesiges Krokodil im Fluss gesehen zu haben. Das Krokodil war auf sie zugeschwommen und hatte ihr in die Augen geschaut. Sie hatte sich nicht bewegen können, ihr Herz schlug schmerzhaft gegen den Brustkorb, als wollte es raus springen. Das Krokodil flüsterte: ,,Lecker!'' Das Mädchen schloss die Augen und wartete darauf von den grossen Zähnen gepackt und unter Wasser gezogen zu werden. Sie hörte ihre Knochen brechen.

Als der Schmerz aus blieb, hatte die Augen geöffnet. Das Maul des Krokodils war direkt neben ihr, aber es hatte nicht ihr Bein gepackt, sondern einen dicken Stock- der war es, der unter dem Gebiss des Krokodils krachte.
Das Krokodil schielte mit dem einen Auge auf das Mädchen, eine Träne lief über seine Schnauze. Es krachte ein letztes mal, als es den Ast endgültig zermalmte. Holzsplitter flogen in alle Richtungen.

,,Susanne?'' hatte die Stimme der Mutter vom Weg ausgerufen. ,,Susanne, was machst Du?'' Eilige Schritte hatten sich auf dem Kies genähert, es hatte zwischen den Büschen am Uferrand geraschelt. Die Mutter hatte sich auf Susanne gestürzt und sie am Arm zu sich herangezogen. ,,Ich habe dir doch gesagt, du sollst nicht so dicht an das Wasser gehen!''

,,Maman! Schau, das Krokodil!''
Doch da war kein Krokodil! ,,Susanne!'' der Stiefvater war auf das sandige Ufer getreten. ,,Erzählst du wieder Lügengeschichten?'' Er hatte hämisch gelächelt und sich die Lippen geleckt. Susanne hatte eine Gänsehaut bekommen. Sie hatte nicht genau gewusst, warum, aber  sie mochte den neuen Mann ihrer Mutter nicht. Seit er im Haus war, hatte sie sich vor Raubtieren gefürchtet. Sie hatte sogar Angst vor ihrer Rotkäppchen-CD bekommen. 

Susanne hatte verlegen auf den Sand vor ihren Gummistiefeln geschaut. Und da in dem Sand hatte ein weisser, glänzender Gegenstand gelegen. Sie hatte sich gebückte und ihn schnell hoch gehoben. Als sie die Hand geöffnet hatte, befand sich darin ein Zahn. Ein grosser Zahn!

Susanne hatte ihrer Mutter fest in die Augen gesehen: ,,Siehst du! Ich lüge nicht! Auch wenn er das immer sagt!'' Sie hatte auf den Stiefvater gezeigt. Der war unter dem Blick von Mutter und Tochter blass geworden. Er hatte sich nervös über die Lippen geleckt. Susanne hatte gespürt, wie ihre Mutter zusammengezuckt war und war zufrieden gewesen. Sie hatte gespürt, wie sich der Blick der Mutter auf den Stiefvater plötzlich gelichtet hatte.
Im Aquarium les Naiades, das ganz in der Nähe lag und das Susanne und ihre Eltern oft am Wochenende besucht hatten, hatten Susanne und Ihre Mutter den Tierarzt gefragt von welchem Tier der Zahn stamme, den Susanne aufgehoben hatte. Der Tierarzt bestätigte Susanne, dass es ein Krokodilszahn war, allerdings wollte er die Geschichte wie Susanne ihn bekommen hatte nicht glauben. 

\sterne

Wie ging Susannes Leben weiter? Kurz nach der Begegnung mit Sobi, gerieten die Mutter und Stiefvater immer öfter in Streit und trennten sich schliesslich. Susanne war froh, als er weg war. Obwohl sie nach dem Erlebnis am Fluss keine Angst mehr hatte, vor wilden, gefährlichen Tieren nicht und auch nicht mehr vor dem Stiefvater.

Susanne wurde später eine berühmte Tierforscherin. Sie untersuchte viele unglaubliche Phänomene, ob sie wahr waren, oder nur ein Gerücht. Den Zahn liess sie sich von einem Goldschmied zu einem Kettenanhänger machen und trug ihn den Rest ihres Lebens.

Susanne traf einen anderen Forscher, der ihr sehr gefiel. Sie heiratet ihn und zusammen bekamen sie Zwillinge, einen Jungen und ein Mädchen.

Wenn Susanne vor schwierigen Entscheidungen stand, dann konnte es passieren, dass sie von einer merkwürdigen Gestalt träumte: die einen männlichen, menschlichen Körper und den Kopf eine Krokodils hatte\dots



\section*{3}
\addcontentsline{toc}{section}{3}

Amélie schlug die Augen auf. Dann schlug sie die Augen wieder auf. Es blieb stockdunkel. Amélie hob den Kopf und rammte ihn gegen Stein. Benommen fiel sie zurück. 

Als die Sterne vor ihren Augen aufhörten zu tanzen, streckte sie vorsichtig die Hände vor sich aus. Sie fühlte die eiskalte Steinplatte. Der Stein, der um sie herum war, strömte den Geruch von modrigem Laub und Erde aus.

Mit einem Schlag wurde ihr die Stille bewusst. Und doch\dots Es klopfte. Es rauschte. 

Für einen Moment traten die Bilder der Nacht vor sie. Amélie konnte sich nicht rühren, während die Bilder ihrer Ermordung sich vor ihr ausbreiteten. Sie wollte schreien, aber ihre Kehle gab keinen Laut von sich. Das Rauschen und Klopfen, wurden nun lauter und lauter. Es rauschte, es klopfte, es schabte: Amélie japste nach Luft, -ich ersticke!

Amélie schrie!

,,Psssst! Nun sei halt nicht so laut! Du weckst ja alle auf!'' Amélie hob geblendet den Arm vor ihr Gesicht. Obwohl es noch dunkel war, waren ihre Augen an die absolute Dunkelheit im Grab gewöhnt. Das Licht, das die Wolkendecke von den umliegenden Städten zurückwarf, langte aus, um sie zu blenden.

,,Berta!'' Amélie reckte die Arme und Berta zog sie aus dem Grab. Sie taumelte und Isfet fing sie auf. ,,Ja, hallo, meine schöne Chaosbraut!'' Begrüsste sie Amélie. Doch die fiel ihr um den Hals und schluchzte.

Berta ächzte und hob die Grabplatte ab und legte sie an die alte Stelle zurück. Sie klopfte Amélie sachte auf den Rücken. ,,Dein Schatz kommt!'' flüsterte Isfet. Amélie versteifte sich sofort und löste sich von Isfet. Sie wischte sich mit dem Jackenärmel trotzig über da Gesicht.

Sie kam nicht dazu zickig zu sein, den Amset war im vollen Lauf über das Geländer gesprungen und riss Amélie von den Füssen. Er schluchzte. Und zerquetschte sie fast.

Tef und die Brüder schnauften hinterher. Hapi kletterte an Isfet hoch und setzte sich auf ihre Schulter. Seine Zähne, die recht beeindruckend waren, klapperten in der Kälte. Berta hatte den Arm ausgestreckt damit Kebi einen guten Landeplatz hatte. der Falke rief ein letztes mal und sortierte seine Federn, bevor er sich aufplusterte.

Tef schleckte mit der Zunge Amélies schlaffe Hand.

,,Amélie, ich\dots '' Amélie löste sich vorsichtig von Amset und schaute ihm ins Gesicht. Ein unendlich Schmerz breitete sich in ihrer Brust aus. Die Tränen liefen, es war ihr gleich. Aber Amsets Schmerz zu sehen, war um vieles schlimmer, als die Schrecken, die sie in dieser Nacht erlebt hatte.

Amset legte seine Stirn an ihre. ,,Nicht weinen, Amsi! Alles ist gut, ich bin ja da! Es geht mir gut!'' Sie sah seinen Zweifel und strich ihm über die Wange. Von seinen Augen, die im Dunklen glühten, angezogen, hauchte Amélie Amset einen Kuss auf den Mund\dots

Dann befreite sie sich aus Amsets Umarmung und sagte: ,,Ich habe Hunger!'' Amélie kletterte über die Abgrenzung auf den Platz und machte sich auf den Weg zum Speisesaal. Tef lief ihr hinterher. Als er sie eingeholt hatte, legte sie ihm den Hand auf den Rücken.

\sterne

Amset blieb vom Donner gerührt stehen. 

\sterne

Isfet und Berta folgten Amélie ins Kloster.

\sterne

Kebi schüttelte den Kopf und das Gefieder. Er hatte genauso grossen Hunger wie Amélie und beschloss, dass es Zeit für eine winterfette Morgenmaus wäre.

Er hüpfte flatternd auf die Umfriedungsmauer. Er genoss den weiten Blick, über die Rheinebene. Er breitete seine Flügel aus und liess sich in den Morgendunst fallen. Bevor seine Füsse die Wipfel der Fichten streiften, schlug er mit den Flügeln und steuerte elegant in den Wald um die Odilienquelle.

Kebi schlüpfte durch das Gitter zu Becken der Quelle. Er nahm einige Schlucke von dem frischen, eiskalten Wasser, das in einem kleinen Rinnsal beständig aus dem Berg lief. Vorsichtig rieb er den Kopf und die geschlossenen Augen an dem quellenfeuchten, bemoosten Stein. 

Er kletterte zurück und schwang sich in die Luft. Als er auf die offene Matte der Ruine des Klosters Niedermünster kam, entdeckte er zu seinem Vergnügen, sofort eine dicke Maus.

Während Kebi auf der Wiese sass, beschloss er, dem Flüsschen Ehn zu folgen.

\sterne

Nachdem sich die anderen auf dem Weg zum Speisesaal gemacht hatten blieben Hapi bei Amset. Er kletterte auf dessen Schultern, ohne das sein Bruder reagiert hätte. Hapi winkte mit seiner Pfote vor Amsets Gesicht auf und ab\dots nichts.
Da zuckte Hapi mit den Schultern und eilte den anderen hinterher. Schliesslich hatte er auch eine lange, kalte Nacht hinter sich.

\section*{4}
\addcontentsline{toc}{section}{4}

Auf dem Odilienberg wurde gefrühstückt. Die Mannschaft hatte tüchtig Hunger. Die Götter hatten auf den Kakao verzichtet (Meidli-Chichi) und hatten sich Kaffee, Tee und warmes Honigbier bestellt.\footnote{Der Spruch, der die ,Schnäbelfressigkeit' der Bauern aufs Korn nimmt, gilt auch für Götter. Dieser lautet dann: ,Was der Gott nicht kennt, das trinkt er nicht!' Denn, was dem Bauer seine Mahlzeit, ist dem Gott sein Trank. -Ausser, es hat Alkohol darin. Schliesslich ist aus der Duat ein berühmter Trinkspruch bis in die heutige Zeit mündlich überliefert: ,,Bier her, Bier her, oder ich fall' um!'' Was damals wörtlich zu verstehen war, wenn er von einer feiernden Mumie zum Besten gegeben wurde. } 

Sie alle hatten sich in die Nähe des Kamins gesetzt und assen genüsslich frische Croissants mit Erdbeermarmeladenfüllung. Es knusperte und schmatze zufrieden. Selbst Isis, die von Odilia eine Ordenstracht bekommen hatte, wollene, graue Strumpfhosen und ein paar Stiefel wie sie die Ordensschwestern trugen, wenn sie um diese Zeit den Schnee schippten, sass entspannt in einem Sessel und schaute in die Flammen. Hin und wieder fielen ihr die Augen zu. Die kleine goldene Schlange, die um ihr Haupt gewunden war, reckte ihr Köpfchen empor und wachte.

Berta, Hathor und Odilia tauschten sich über die verschiedenen Kulte aus, die sie erlebt und gefeiert hatten. Hathor war sehr interessiert, was Odilia über ihren Gottesdienst berichtete. Sie hatte ihre Rituale von irischen Mönchen gelernt. Berta bemerkte Parallelen zu den Kulten, die sie kannte. Sie war am weitesten von allen herumgekommen und hatte an verschiedenen Kulten teilgenommen\footnote{Es ist erstaunlich wie viele Kulte und Rituale in den vergangenen Zeiten von kleinen, kugelförmigen Frauen handelten.}. 

Als Amélie den Speisesaal betrat wurde sie von einem Gott zum nächsten weitergereicht. Jeder wollte sie umarmen. Wibrandis hatte dafür ein wunderbares Frühstück gezaubert. Rührei und Speck, genau das richtige, wenn man die Nacht in einem Merowingergrab verbringen musste.

Nachdem sie die Hälfte ihres Frühstücks verschlungen hatte, blickte Amélie auf. Thot sah ihr von seinem Platz am Kaminfeuer aus zu. ,,Gut gemacht! Wie geht es dir?'' Amélie überlegte. Sie war die ganze nacht bei Minusgraden in einem steinernen Grab in einem Berg in Frankreich gelegen\dots Dafür, fand sie, fühlte sie sich verblüffend gut.

Dann spürte sie wie sich die Erinnerung näherte. Die Erinnerung an den Mord, bei dem ihr das schlagende Herz aus dem Leib gerissen worden und mit seinen letzten Schlägen in die Kanope von Tef geworfen worden war.

Amélie schüttelte sich, sie wollte die Bilder los werden, wegschieben. Sie schloss die Augen, aber da fühlte sie eine knochige, kräftige Hand auf der Schulter. Es war Thots Hand. ,,Du musst dich erinnern, Amélie! Sonst war die Nacht im Grab umsonst!''

\section*{5}
\addcontentsline{toc}{section}{5}

Währenddessen stand Luise vor dem Spiegel. Sie hatte den Jungen bei Wilfried gelassen und behauptet sie hätte keinen Hunger. Sie starrte auf die silbrige Fläche.

Eine Zeit lang, als sie jung war, hatte Berta ihr einige Tricks gezeigt, wie sie es nannte. Dann als das Schreckliche passiert war, hatte Berta damit aufgehört. Die törichte alte Frau hatte ihr die Hilfe verwehrt, als sie sie am dringendsten gebraucht hätte. Aber Berta hatte behauptet, Luise würde die Tricks für ihre Rache nutzen und hat sich geweigert Luise weiter auszubilden.

Luise hatte getobt. Sie hatte gebittelt, gebettelt und hatte sie Berta verflucht. Es half ihr alles nichts. Denn natürlich hatte Berta recht, dachte Luise bitter, als sie sich im Spiegle betrachtete.

Sie hätte ihre Seele verkauft um an Bertas richtige Tricks zu kommen. Stattdessen hatte sich Berta in den letzten Jahren, seit der Bub geboren worden war, ihrer Tochter zugewandt. 

-vermutlich hat sie all ihr Wissen an Amélie weitergegeben, dachte Luise. Zu ihrem erstaunen, spürte sie einen Stich in der Brust. Sie hatte geglaubt, es würde ihr nichts mehr ausmachen. Luise biss sich auf die Lippe. Sie wollte nicht heulen, sie wollte Berta nicht vermissen und Amélie erst recht nicht.

Solange Amélie und Berta da waren, würde sie sich nie sicher fühlen. Luise fühlte sich nur sicher, wenn sie alles und jeden um sich kontrollieren konnte. Das war bei Wilfried keine Frage, er war ihr ergeben, er würde alles für sie tun! Und der Bub? Der war viel zu klein und zu schwach, um sich gegen seine Mutter durchzusetzten. Es gab keinen Grund für ihn. Lästig war nur, dass der Kleine seine Schwester tatsächlich mochte.

-Was bin ich für ein Biest! dachte Luise und blickte sich mit halb geschlossenen Augen und schiefgelegtem Kopf an. Es war so einfach Männer im Griff zu haben. Sie verzog unwillkürlich angeekelt ihren Mund.

Luise sah auf die weissen Wollhandschuhe, sie bemerkte rote Flecken, die von ihren blutig geschrubbten Fingern stammten. Sie ballte die Faust, wo war dieses missratene Kind und diese blöde Amme?

Berta kannte Luise zu gut. Viel zu gut. Sie war die einzige, die begriffen hatte, wie eisig kalt es in Luise wirklich war. die einzige, die wusste, zu was Luise fähig war\dots

Und Amélie? Amélie war ein naives Huhn. Aber sie war fast erwachsen, also jung und schön. Es würde nicht mehr lange dauern, bis sie begreifen würde, was das für eine junge Frau bedeuten konnte. Luise wollte keine Konkurrentin, auch nicht, wenn diese ihr eigene Tochter war.

Luise wollte nie wieder ausgeliefert sein. Und sie wusste, als Frau brauchte sie dazu einen Beschützer. Einen erträglichen Beschützer, der dumm genug war, für ein geheucheltes Lächeln, alles zu geben. Und der stark genug war, sie zu beschützen.

\section*{6}
\addcontentsline{toc}{section}{6}



Amélie wanderte mit Odilia an der Heidenmauer entlang. Die Luft war frostig. Ihr Atem bildete Wolken vor ihren kalten Gesichtern, die, angeregt von ihrem Gespräch und der Bewegung, erhitzt leuchteten.

,,Du bist sehr mutig, Amélie!'' sagte Odilia, die auf dem schmalen Trampelpfad, der zwischen der Kannte des Felsen und der Mauer aus gewaltigen Sandsteinen entlangführte vorausging. Sie blieb stehen und wendete sich zu Amélie um. Sie schauten durch die Fichten auf das ausgebreitete Tal.

,,Ich habe es mir nicht ausgesucht!'' antwortete Amélie. ,,Verstehe!'' sagte Odilia. Und sie verstand es wirklich.
Odilia nahm Amélies Hand und plötzlich löste sich etwas in Amélies Brust. Die Traurigkeit von der sie sich bis jetzt hatte ablenken können, überschwemmt sie. Amélie schluchzte und dann schrie sie. Die Vögel, die leise im Unterholz gesessen hatten, flogen auf. 

Odilia legte sanft ihren Arm um Amélies Schulter und hielt sie fest. bis sie fertig geschrien hatte. Dann langte sie in die falten ihres wollenen Umhanges und zog ein Stofftuch heraus. Sie reichte es Amélie. Sie schnäuzte sich ausgiebig.

,,Sie sind meine Eltern!'' Amélie sprach leise und ruhig. ,,Ich habe schreckliche Dinge gesehen, die Menschen mit mir gemacht haben und das waren meine Eltern! Auch wenn wir alle anders ausgesehen haben. ich weiss, sie waren es!'' ,,Als ich geboren wurde, war ich blind. Meine Vater Etticho, der aus der heidnischen Tradition stammt, sah darin ein Beweis für seine Schande. er verlangte von meiner Mutter mich zu töten. Meine Mutter Bereswinda liess mich heimlich zu ihrer Amme bringen und diese zog mich auf, bis ich alt genug wurde, um im Kloster Baume-les-Dames meine Ausbildung zu beginnen. Ich musste einige Male fliehen und um mein Leben fürchten, bevor ich meine Aufgabe, die Kloster hier auf diesem heiligen Grund zu errichten, ausführen konnte.''

,,Wie hast du das geschafft?'' fragte Amélie. ,,Wie hast du das gemacht, ganz allein?'' Amélie spürte wie ihre Kehle sich wieder zu schnürte. Sie konnte den grossen Kloss der Beklommenheit und Angst nicht schlucken.

,,Indem ich begriff, ich bin nie allein!'' antwortete Odilia. ,,Aber ich bin nicht so religiös wie du! Und ich hab auch nicht vor es zu werden!'' rief Amélie. ,,Wie meinst du das?'' fragte odilia verwirrt. ,,Du glaubst halt an Jesus und so und betest und bist sogar eine Heilige!'' antwortete Amélie.

Odilia drehte Amélie zu sich um und sah sie mit ihren hellen, fast weissen Augen an. ,,Sag', du denkst, weil ich glaube und bete, denke ich, ich wäre nicht einsam?'' ,,Genau!'' Amélie hielt Odilias Blick nicht stand, sie blickte mit verschrenkten Armen zu Boden.

Odilia starrte Amélie an. Sie lächelte. ,,Ich find es nicht lustig!'' rief Amélie. ,,Ich schon!'' Odilia machte eine weite Geste über die Rheinebene, die in der Sonne schimmerte. ,,Schau dich um, Kind! Sieh hin, mach die Augen auf! Du kannst nicht alleine sein! Das ist unmöglich!''

Widerwillig gehorchte Amélie und schaute ebenfalls in die Weite. Odilia legte ihr die hand leicht auf die Schulter. Ihr Atem glitt zart in die kalte Luft. Ein kleiner Vogel zwitscherte sein Lied, emsig. Der Duft der Erde war scharf gewürzt durch den Frost. Es Knackte. Die dunkle rotbraune Erde und die grossen, rostroten Steine mit dem grünen Moos lagen still da. Aus der grünen, mattgrünen Ebene schallte ein leises, beständiges Summen herauf. Autos, winzig wie Ameisen, folgten den dunklen Strassenbändern.

Amélie spürte wie sich die Angst auflöste, je mehr sie ihren Körper und ihre Umgebung zu spüren bereit war. Doch sobald sie sich wieder spürte, tauchte die Traurigkeit auf und sie schlang die Arme um sich. ,,Aber es macht weh!'' flüsterte sie. Odilia drückte ihre Schulter stärker. ,,Ich weiss! Amélie! Du wirst den Schmerz aushalten lernen! Weisst du auch warum?'' Amélie schaute Odilia fragend an. ,,Weil du menschlich bist! Wenn du aufhörst den Schmerz zu fühlen, dann, dann erst, wirst du wirklich allein sein, in deiner eigenen Hölle!''

Amélie zog unwirsch die Schulter hoch. ,,Benenne die Dinge als das, was sie sind. Du bist Schmerz, denn du hast ein liebgewonnenes Wissen als Enttäuschung entlarvt. Du hast auch nicht deine Eltern verloren, sondern den Glauben, sie müssten gut zu dir sein. Aber deshalb bist du nicht allein. Sondern du musst den Blick in eine andere Richtung lenken und deinen Weg neu ausrichten!'' Odilia schwieg und auch die Vögel und die Bäume rings um sie schienen erwartungsvoll zu schweigen.

Es dauerte eine Weile, doch Amélie merkte wie Odilias Worte in ihr wirkten. ,,Weisst du, was ich wirklich seltsam finde?'' fragte Odilia, nachdem Amélie sie schüchtern angelächelt hatte. ,,Was?'' ,,Wie es dir nach all dem, was dir passiert ist, möglich ist, nicht an Götter zu glauben!'' Amélie schaute Odilia überrascht an.

Ihr fröhliches Gelächter, das sie schüttelte bis sie das Kloster erreicht hatten, drang hinauf bis zu Kebi, der auf dem Rückflug von seiner Frühstückssuche war. Er antwortete ihnen mit zufriedenem Falkenruf.

\section*{7}
\addcontentsline{toc}{section}{7}

Luise und Wilfried schlenderten missmutig hinter dem Kleinen her durch die Strassen. Sie hatten das Hotel klug gewählt, denn es war Teil der Altstadt. Sie waren durch die engen Gässchen runter gewandert. Schliesslich blieben sie auf dem Barfüsserplatz vor der Museumskirche stehen. 

,,Und nun?'' fragte Wilfried. ,,Wie sollen wir in dieser riesigen Stadt unsere Tochter und Berta finden?'' Luise schenkte ihm einen abfälligen Blick und liess ihren Blick dann weiter über den Platz und das Gedränge um die Tramstation schweifen.Sie war lange genug bei Berta in die Schule gegangen, um ein untrügliches Gespür zu entwickeln, wann sie wo zur rechten zeit an einen rechten Ort geraten war. Und im Moment hatte sie das Gefühl, sie müsste einzig und allein warten. 

Sie hielt weiter Ausschau, ohne zu wissen wonach und steckte Wilfried an. Gleichzeitig entdeckten sie einen Mann, der sich aus der Menge löste und in ihre Richtung schlenderte. Er war sehr blass, was durch die schwarzen Kleider (Jeans und T-Shirt) und die ascheschwarzen Haare verstärkte. Sein Gesicht war schmal, die Augen schienen blassgelb. Die Nase, schlank und lang krümmte sich leicht. Zwischen den Lippen, die wie ein Strich in das kräftige, spitze Kinn geschnitten schienen, ragte eine Zigarette.

Luise blinzelte. Plötzlich fröstelte sie, obwohl der Platz von der Wintersonne sanft gewärmt worden war. Ein leichter, kalter Schatten schien sich vor den Schritten des Fremden auf sie zu zu bewegen. Luise schaute hoch an den Himmel, der jedoch kein Wölkchen zeigte. Auch Wilfried schien es zu spüren, denn er schaute sich ratlos um. Nur der Kleine, der an der Kirchenmauer sein Spielzeugauto entlang fahren liess, schien nichts zu spüren. Luise schauderte. Die Tramstation wirkte wie durch eine Seifenblase gesehen, verschmiert, undeutlich\dots

In schlichten, schwarze Boots stieg der Mann die Treppe hinauf und hielt direkt auf Luise und Wilfried zu. ,,Hallo!'' sagte er. ,,Ich bin Seth!'' ,,Hallo, Seth!'' säuselte Luise und reichte ihm ihre behandschuhte Hand. Er nahm sie galant und hauchte einen Luftkuss darauf. Luise lief ein kalter Schauer über den Rücken. Sie konnte sich nicht erinnern, wann sie sich das letzte mal so gefürchtet hatte. Die Kälte und Starre, die Seth ausstrahlte, presste ihr die Luft zum Atmen aus der Brust.

Wilfried stand einige Sekunden wie vom Blitz getroffen. Er wurde leichenblass und alles, was an ihm lebendig geblieben war, floss aus ihm heraus. Den letzten Rest an Wärme verlor er, als auch er Seth die Hand reichte. ,,So ist es recht, Wilfried!'' sagte Seth. Der schmale Mund grinste, lächeln konnte er nicht.

Der kleine hatte aus einiger Entfernung den Fremden, der seine Eltern begrüsste, beobachtet. Auch er stand erstarrt. ,,Der Junior?'' fragte Seth und wendete sich dem Kind zu. Doch Luise war schneller, eilte zu dem Jungen und nahm ihn auf den Arm. Sie blitzte Seth aus den Augen an, was dieser mit einem amüsierten Blick erwiderte.

Wilfried erwachte aus der Erstarrung und legte schnell seinen Arm um Luises Schulter, er spürte seine Verwunderung, als sie seine Hand nicht sogleich abschüttelte.

Seth lachte. ,,Wir haben das gleiche ziel!'' sagte er dann. ,,Welches Ziel haben wir denn?'' fragte Luise. ,,Ihr wollt eure Göre wieder zurück, die sich gerade bei meinem lieben Brüderchen und meiner Sippschaft einschleimt zusammen mit der alten Vettel und ich will, dass sie wieder verschwinden!'' Luises Augen blitzten auf und Wilfried bekam einen Hauch Farbe. Seth nickte zufrieden.

,,Und wie genau willst du das machen?'' fragte Luise. ,,Die alte Vettel, wie du sie nennst, sollte man nicht unterschätzen!'' Ein Schatten huschte über Seths Gesicht und er langte instinktiv an seine Augenbraue, die von einer feinen Narbe unterbrochen wurde. ,,Ich weiss'', sagte der Kriegsgott. ,,deshalb werden wir uns zusammen tun!'' ,,Ach?! Sicher?!'' Luise hob arrogant die Augenbraue. 

Wilfried, dessen Gespür viel feiner war, als das seiner Frau, zuckte. Er wollte sich zwischen den Fremden und Luise stellen, doch es war zu spät. Seth packte Luises Kinn und flüsterte:,, Ganz sicher. Im Gegensatz zu deiner Tochter, ist dir dieser kleine Hosenscheisser ja nicht egal!'' Luise wurde bleich. Der Kleine, eingeklemmt zwischen seiner Mutter und Seth begann zu schluchzen.

Wilfried fand aus seiner Starre und schob sich sachte, aber bestimmt zwischen den den Gott und Luise. ,, Was willst du?'' fragte er. Seth, der ein Kopf grösser war als Wilfried, beugte sich zu ihm bis ihre Nasen sich berührten. ,,So gefallt ihr mir schon besser! Wir treffen uns heute um Mitternacht am Gerberberglein. Wir werden einen alten Freund von mir besuchen. Euch menschlichen würde ich raten Gummistiefel und eine spiegelndes Sonnenbrille mitzunehmen.''

,,Und der Sohn?'' fragte Luise. Seth blickte sie kalt an. ,,Mein  Freund hat sicher nichts gegen einen kleinen Appetithappen!'' Seth machte kehrt und war verschwunden ehe Luise und Wilfried durchatmen konnten. Sie standen aneinander gedrängt auf dem Platz und versuchten das zitternde Kind zu trösten.

Als würde sich eine Wolke von der Sonne wegbewegen, wurde der Platz wieder heller und wärmer. ,,Wer zum Teufel ist er?'' fragte Luise. Sie schaute in Wilfrieds Gesicht. Es war totenbleich und seine Augen starr. Ein Ruck lief durch den Mann und er antwortete leise: ,,Es ist der Teufel! Es ist Seth!'' Luise spürte seit langer Zeit zum ersten mal wieder Zuneigung zu ihrem Mann. Sie wollte Wilfried zu trösten und griff zart nach seiner Hand. Doch er riss erschrocken die Hand zurück. Betroffen rückten sie von einander ab. Der Junge zappelte unruhig und Luise stellte ihn ab.

Schweigend besorgten sie sich Gummistiefel, Trekkingkleidung und spiegelnde Sonnenbrillen. Schliesslich fragten sie im Hotel nach und es fand sich eine Auszubildende, die mit Erlaubnis des Chefs als Babysitterin einspringen sollte.


\section*{8}
\addcontentsline{toc}{section}{8}


Auf dem Parkplatz vor dem Kloster herrschte Hochbetrieb, die Göttinnen scheuchten die Ordensschwestern, die Odilia geheissen hatte das Gepäck der Gäste zu bringen, hin und her.

Obwohl Sobi den fuss- und Schwimmweg eingeschlagen hatte, schien der Platz im Kofferraum geschrumpft zu sein. Die lag an den riesigen Proviantkörben, die die Schwestern mit  vereinten Kräften in jeden Zwischenraum zwängten. 

Amélie stand fror neben dem schwarzen Wagen. Sie wollte sich von Odilia verabschieden. Doch die Äbtissin war nicht zu sehen. Amélie war enttäuscht, war es Odilia egal, was aus ihnen wurde?

Da sah sie die Klosterfrau in ihren dicken, grauen Wollumhang gehüllt durch das Tor eilen, sie trug eine kleine lederne Tasche. ,,Odilia!'' Horus öffnete ihr die Beifahrertür zum Leichenwagen. ,,Du kannst auf meinem Platz sitzen, ich will mir die Flügel lüften und mit Kebi den Luftraum ausspähen!'' Odilia nickte und liess sich von dem kräftigen Gott die Tür halten, als sie zu Thot in den Wagen stieg.

,,Odilia Kommt mit?'' rief Amélie überrascht. ,,Freust du dich?'' fragte Berta. Amélie blickte Berta durch den Rückspiegel ins Gesicht, war sie traurig deswegen? Doch Berta grinste zufrieden, trotz ihres Stumpens im Mund. ,,Ja!'' antwortete Amélie da. ,,Odilia hat mir von ihrem Vater erzählt!'' ,,Oh, ja!''seufzte Berta. ,,Verstehe! Geteiltes Leid ist halbes Leid!''

,,Odilia wird uns heute Nacht begleiten. Sie wird die Göttin, die ihr Bild verbirgt, vertreten.'' sagte Isis. ,,Die Göttin die alles verbirgt?'' fragte Amélie. ,,Genau! Die hilft Re sich mit seinen Augen vollständig zu verbinden.'' ,, Und das macht die Göttin, die  verbirgt?'' ,,Genau!'' antwortete Isis und schaute aus dem Fenster in das enge Tal, durch das sich der Wagen hinunter schlängelte.

\sterne

Jeder hing seinen Gedanken nach\dots

Den Ereignissen der letzten Nacht und denen, die kommen würden\dots Die ruhige Fahrt wurde gelegentlich unterbrochen, von Wibrandis, die kurz frische Luft auf dem Grünstreifen schnappte\dots

So erreichten am Nachmittag Strasbourg. Sie hielten vor dem weiss-braunen Eckhaus direkt am Quai Saint Nicolas. Sie parkten ihre drei Wagen direkt vor dem Haus, vor dem grauen Tor\footnote{Wir wissen, was den Göttern natürlich nicht bewusst war: Es war ein Wunder, dass sie auf Anhieb drei freie Parkplätze hintereinander gefunden hatten!}.

Horus und Kebi warteten schon auf sie. Horus stand an Geländer der Ill gelehnt und Kebi hatte sich auf dem gegenüberliegenden Dach des  alten Zollgebäudes niedergelassen. Aus den Fenstern des langen grauen Gebäudes drang warmes, gelbes Licht aus dem Restaurant in die blaue Stunde. Kebi rief seinen Brüdern einen Gruss zu. Tef, der zwischen den Stangen des Geländers witterte, winselte\footnote{Es roch verführerisch nach einer Hundedame, Huskymix mit blauen Augen!}. Amset hatte Hapi auf den Arm genommen. Der Pavian, dem Isfet einen rosa Hoody geliehen hatte, hatte die Kapuze tief über die Schnauze gezogen. Die Brüder winkten Kebi fröhlich zu.

Anubis sortierte seine langen Beine aus dem Wagen und streckte sich. Dann gähnte er ausgiebig. Er lief zum Geländer und schob Tef auf die Seite. Er nahm nur eine kurze Nase von der Huskydame und starrte dann hoch konzentriert auf das Wasser und nahm die Witterung des gegenüberliegenden Ufers auf. Er blickte zu Thot und wies mit der Schnauze auf die Brücke aus Stein, die links von ihnen auf die Altstadtinsel führte.

Am Brückenkopf auf der anderen Seite war schwach eine Steintreppe zu erkennen die auf den alten Chemin de Halage (Treidelpfad) führte, der direkt unten am Fluss an dem alten Zollhaus vorbei rund um die Altstadtinsel führte. Anubis ging steifbeinig auf die Brücke zu und Thot folgte ihm.

Genau vor dem grossen Eckhaus führten zwei Treppen runter zum Fluss. Horus stieg eine zum Fluss runter. Während Re und die Frauen sich zwischen den beiden Treppen, oben auf dem Trottoir versammelten und in dem Dämmerlicht auf das träge, dunkle Wasser schauten.

Plötzlich bemerkte Amélie, die zwischen den Zwillingen Maat und Isfet stand, wie sich die Oberfläche des Flusses kräuselte. Und schliesslich, tauchten kaum sichtbar aus den kleinen Wellen die gewaltigen Schuppen Sobeks auf. Nur die gelben Augen des Krokodilgottes funkelten, wie kleine Laternen. Amélie schauderte, als der Gott sie anstarrte und dann den Oberkiefer öffnete und seine Zähne zeigte.

,,Sobi!'' rief Horus und beugte sich zu seinem Freund, der zur Treppe geschwommen war und tätschelte die breite Schnauze: ,,Gut gereist?'' Sobi hob die Schnauze kurz aus dem Wasser, es spritzte und er klappte mit seinem Gebiss. ,,Gut gegessen!'' sagte Horus zufrieden. ,,Anubis und Thot erwarten Dich!'' sagte Horus und wies auf den Treidelpfad auf dem Thot und Anubis inzwischen angekommen waren. ,,See you in a while\dots '' Horus stieg zurück auf das Trottoir.

Sobek durchquerte die Ill zu Thot und Anubis, die sich direkt neben dem Fluss niedergelassen hatten. Es gab einiges zu besprechen\dots

\chapter*{6. Nacht}
\addcontentsline{toc}{chapter}{6. Nacht}

\begin{quotation}

\emph{VI Vis eius integra est, si versa fuerit in terram.\\6. Seine Kraft ist vollständig, wenn sie in der Erde umgekehrt worden ist.  \\Tabula Smaragdina}

\end{quotation}

\section*{1}
\addcontentsline{toc}{section}{1}

Die Götter und Amélie wendeten sich dem grossen Haus zu, das auf der anderen Strassenseite zu. Es war ein hübsches Eckhaus. Zwei Männer traten auf den Gehweg und winkten ihnen zu.

Die Reisegruppe überquerte die Strasse und begrüsste die beiden. ,,Johannes! Was für eine Freude!'' Odilia, die ihre Ordenstracht trug und einen grauen Wollumhang begrüsste den Dominikanermönche in der weissen Wollkutte und dem schwarzen Umhang fröhlich und stellte ihn den anderen vor. ,,Johannes Tauler, ein beeindruckender Prediger!'' ,, Nicht doch, nicht doch, liebe Odilia\dots '' beschwichtigte der Mönch und reichte allen fröhlich die Hand. nur bei Bertas Anblick zuckte er unmerklich, hatte sich aber sofort wieder im Griff.

Gleichzeitig begrüssten sie auch den anderen Mann. Der trug einen grossen, dreieckigen Hut auf dem Kopf, der seine rotbraunen Locken bedeckte und sein ge- und belebtes Gesicht rahmte. Bis auf den Hut hatte Sebastian Brandt beschlossen, Komplikationen zu meiden und sich in neuzeitliche Kleidung zu hüllen. Er trug einen gewöhnlichen, grauen Anzug und einen dicken, weiten, schwarzen Wollmantel. 

Er schmunzelte über den bunten Haufen. Im Gegensatz zu seinem Freund Tauler, hatte er keine Berührungsängste und umarmte Berta herzlich. ,,Ich bin froh Euch zu sehen! Es wird nicht einfach werden heute Nacht!'' sagte Berta düster. ,,Aber, aber, ma grande, c'est la vie! N'aie pas peur, tout marchera bien!'' ,,J'espère, mon ami!''

,,Bevor wir in unser nächtliches Abenteuer starten, habe ich eine kleine Überraschung für euch! Tretet ein!'' mit diesen Worten öffnete er eine grosse Holztüre und ein Schwall warmer, rauchiger Essensduft strömte auf die Strasse. Alle wendeten ihre Köpf und beeilten sich dann in die Gaststube zu kommen.


Als Amélie durch die Tür schlüpfen wollte, hielt Berta sie zurück. ,,Amélie! Ich muss weg!'' ,,Was!'' Amélie zog es den Boden unter den Füssen weg! Berta konnte sie nicht wieder allein lassen! ,,Aber! Berta, ich habe Angst! Ich bin fertig! Lass mich nicht allein!'' Amélie klammerte sich an Bertas Schürze und schluchzte. ,,Du bist die einzige, die ich habe!''

Berta packte Amélie an der Schulter und schüttelte sie ungeduldig. ,,Kind, was redest du da! Du bist von den höchsten Göttern umgeben! Und ich muss zurück! Ich muss mit Luise reden!'' ,,Du willst zu Mama? Ich will mit! Berta, ich will zu meiner Mutter. Zu meinen Eltern!'' Amélie zerrte an Bertas Schürze. Die kleine Göttin schlug ihr auf die Hände. ,,Nein!'' Sie packte Amélie wieder und flüsterte dicht an ihrem Ohr: ,,Reiss dich zusammen! Und vor allem, denke daran: Sie sind nicht nur deine Eltern, sondern vermutlich auch deine Mörder!'' ,,Neeeein!'' schrie Amélie. 

Amset stand plötzlich zwischen ihnen und legte vorsichtig seinen Arm um Amélie, die schluchzte und zitterte. ,,Was ist los?'' ,,Amset! Nehme Amélie mit hinein. Ich muss fort! Geht!'' Berta schob Amset und Amélie zur Tür. 

Ohne sich weiter um die beiden zu kümmern, öffnete sie eilig den Kofferraum des Leichenwagens. Sie verabschiedete sich kurz von Isis, die den Sarkophag ihres Mannes nicht verlassen hatte, seit die Autos Quai Saint Nicolas geparkt wurden. Berta klopfte auf Osiris Sarg und zog einen Besen raus.

Sie lief über die Strasse und sprang im Lauf auf den Besen, der sich in die Luft hob. Berta zog eine kurze Schleife um den Münsterturm und wendete den Besen nach Südosten. Sobald sie den Rhein erreicht hatte, der von oben als dunkles Band in der punktiert leuchtenden Landschaft lag, richtete sie sich nach Süden aus auf den hellen Schein, den Basel in den Nachthimmel schickte. -So ein Nachtlicht ist nicht schön, aber praktisch! Dachte Berta und genehmigte sich ein warmen Schluck aus ihrem Flachmann, den sie unter ihrem Strumpfband hervorgeholt hatte.


\section*{2}
\addcontentsline{toc}{section}{2}

,,Ahhh! Vivre comme un coq en pate!'' rief Re entzückt. Sie standen gedrängt in einer grossen Gaststube. ,,Mes amis!'' rief Sebastian, ,,ich habe mir erlaubt Euch un petit diner auftischen zu lassen in der Herberge meines Vaters!''

Ein grosses Oh und Ah begann! ,,In welchem Jahrhundert sind wir?'' fragte Thot Sebastian, nachdem er in die Rund geschaut hatte. ,,Oh, qui! Wir sind in dem 15-Jahrhundert in der ,l'auberge du lion d'Or'. Meinem Vater hat sie gehört. Das Haus ist neu, aber der Standort ist ungefähr der richtige!'' 

,,Magnifique!'' antwortete Thot. Der Gott war hingerissen, denn er liebte Reisen\footnote{Zeit und Raum sind relativ. Und in der Göttersphäre gänzlich unbekannt. Andererseits, was eben das Paradoxe des Glaubens in diesem Fall als Gegensatz zu den intellektuellen Glaubenstrukturen werden lässt, ist es dort substanziell dennoch möglich, dass ein verkörperter Gott durch Raum und Zeit reisen kann.}.

Es gab sogar einige mittelalterliche Gäste, die für die richtige Stimmung sorgten, indem sie würfelten, Karten spielten, oder an einem Tisch speisten und Bier tranken.

Zu aller Verwunderung ging plötzlich die Tür auf und ein weiterer Gast trat ein. Horus sprang ihm sofort in den Weg und auch Re war angespannt. Sebastian war einfach die Kinnlade nach unten geklappt, während Johannes Tauler die Nase rümpfte und ein Gesicht machte, als hätte er in eine Zitrone gebissen. Odilia grinste und Wibrandis sah aus, als wäre ihr ein Engel erschienen.

,,Ich habe mir erlaubt einen Überraschungsgast einzuladen!'' sagte Thot. Er und Anubis begrüssten dem Neuankömmling freudig. Er war recht klein und pummelig. Sein rundes Gesicht strahlte unter einer weissen Perücke mit Pferdeschwanz hervor und sein rundlicher Körper steckte in bordeauxfarbenen Kniebundhosen, einem weiten, weissen Leinenhemd mit  und einer schönen samtenen, bordeauxfarbenen Justaucorps. Seine strammen Waden steckten in weissen Strümpfen und seine Füsse in eckigen Schnallenschuhen.

,,Darf ich ihn Euch bekannt machen? Den grossen Meister der Hermetik, der uns mit seinen Künsten unser Feriendomizil in Basel eingerichtet hat und uns auch heute mit Rat und Tat zur Seite stehen wird: Der grosse Cagliostro!'' Die restliche Reisegesellschaft blieb vorerst skeptisch. Auf den ersten Blick schien der Fremde hauptsächlich eine Hilfe im leeren von Töpfen und Pfannen zu sein. Sie schwiegen.

Mit einem Ruck erinnerte sich Sebastian an seine Aufgabe als Gastgeber. ,,Also, nehmt Platz! Ich lasse Sabine gleich ein weiteres Gedeck holen''. ,,Ich mach es gerade!'' rief da Wibrandis und rannte mit hochrotem Kopf in Richtung Küche davon. Amélie erwiderte das Grinsen von Odilia, die sich die Hand vor den Mund gelegt hatte. 

An einer langen Tafel, weiter hinten im Gasthaus, war für die Götter eingedeckt. Und wie! Selbst Hathor, die Königin der Festtafeln, pfiff durch die Zähne und nickte zufrieden. 

Denn sie hatte bemerkt, dass für jeden von ihnen ein köstlicher Happen bereit stand. Was, das wusste sie nur zu gut aus eigener Erfahrung, nicht so leicht war bei den vielen verschiedenen Geschmäckern.

Es dauerte eine Zeit, aber dann hatten alle ihre Plätze gefunden und sich ihre Teller gefüllt. Es gab ein grosses hin und her. Die Flasche Riesling kreiste in die eine Richtung und der Krug mit frisch gebrautem Bier in die andere.

Tef und Anubis kümmerten sich nicht um das Gerangel am Tisch. Für sie gab es vor dem Kamin Decken und Kissen, auf denen sie sich ausstreckten, nachdem sie mit grossem Appetit ihre ,Wädele' geknurpst und zerkaut hatten. Thot hatte ihnen eine Tonschüssel mit Bier hingestellt. Sie würden, wie alle, viel Kraft für die Nacht brauchen.

Horus schmatze laut seinen ,Civet de Lièvre', Hasenpfeffer in Pinot Noir gekocht, in sich hinein. Dazu trank er Bier. Während sich Re mit Genuss an eine ,Tourte au Riesling' wagte. Zusätzlich hatte er sich ,Frommage du munster', ,Pate de foie gras' und einige Stück ,Carpe frite' auf seinem Holzteller angerichtet und brach glücklich das weisse Baguette.

Hathor begann ihre Schlemmerei mit einer ,Pate à la reine', wie es sich für die Gottesmutter gehörte. Nach einer Portion Spätzli mit ,Coq au Riesling, beendete sie ihr Diner mit einer besonderen Spezialität, die Sebastian der südlichen Herkunft gewidmet hatte\dots

\section*{3}
\addcontentsline{toc}{section}{3}

Berta landete auf dem Dach des Hotels. Sie balancierte über den Dachfirst und liess sich auf der Giebelgaube von Luises und Wilfrieds Zimmer nieder. Den Besen klemmte sie sich unter den Arm und dann suchte sie unter ihren Röcken den silbernen Flachmann. Sie zog den Stöpsel aus der kleinen Flasche und schnupperte daran. ,,L'eau de vie des baies de sureau... Mein liebe Sebastian, du hast nicht zu viel versprochen!'' 

Berta setzte die Suche zwischen den Rockfalten fort und zog eine weisse Stoffservierte heraus, die sie sich über den Schoss breitete und schliesslich eine kleine mit Enten bemalte Terrine. ,,Ouh, Rillettes de canard!'' seufzte sie. Sie band das Baguette vom Besenstiel los. Sie riss ein Stück vom Brot und tauchte es in die Terrine und biss hinein. Grimmig zermalmte Berta das Brot\dots 

Berta fragte sich, ob sie einen Fehler machte. Das kam nicht sehr oft vor. Sie band die Flasche Riesling vom Besen. Sie trank gerne, aber sie konnte sich nicht mehr erinnern, wann sie sich das letzte mal Mut angetrunken hatte. Sie senkte die Flasche und liess den Blick über die Dächer der alten Stadt schweifen. Sie wurde das Gefühl nicht los, einen schrecklichen Fehler zu begehen\dots aber sie konnte nicht anders, sie wollte einen letzten Versuch wagen, bevor sie Luise ganz verloren geben konnte\dots 

Berta schwankte, als sie sich auf den Weg machte. Sie blieb auf dem Dachfirst stehen und versank. Und tauchte mit den Füssen voran in Wilfrieds und Luises Zimmer auf. Luise bewies ihre Nerven, indem sie nicht schrie. Wilfried setzte sich kraftlos in einen der Sessel. ,,Guten Abend!'' sagte Berta. ,,Was willst du!'' fragte Luise. Ich will mit Dir und Wilfried reden!'' ,,Es gibt nichts zu reden, du hast meine Tochter verschleppt!'' rief Luise. Wilfried schaute überrascht auf. Hatte er Luise falsch eingeschätzt? Er sah den kalten Glanz in Luises Augen. 

,,Luise, hör mir zu! Du hast schlimmes durchgemacht und wirst die Wunden mit dir tragen, aber Amélie kannst du retten. Du kannst sie und deine zukünftigen Enkel vor Unglück bewahren, weil du es kennst!'' ,,Ich weiss nicht wovon du redest, alte Frau! Du hast mich fallen lassen, als ich dich am dringendsten gebraucht hätte und jetzt willst du mir sagen, was ich zu tun habe?'' Luise stampfte mit dem Fuss auf und ihr Rocksaum wischte einen Stadtplan von dem niedrigen Couchtisch. 

Für einen Moment schimmerten Tränen in Luises Augen, aus Trotz, aus Wut. Berta schöpfte einen Augenblick Hoffnung. ,,Geh! Alte Frau! ich will dich nie mehr sehen. Und wenn du mir nicht morgen dieses schreckliche Gör wiederbringst, dann mache ich dir die Höhle heiss!'' Berta blickte zu Boden, suchte Worte, ihr blick glitt über den Plan am Boden. In der Mitte, war eine rot markierte Stelle. 

,,Wilfried?'' Berta wendete sich dem Mann zu, der zusammengesackt in dem Sessel hockte. Er blickte sie müde an. ,,Geh! Berta! Bitte! Es wird nur alles schlimmer!'' sagte er.

Luise wies Berta die Tür und Berta ging auf den Gang hinaus. Sie schwebte nach oben auf das Dach und setzte sich. Von Luise hatte sie sich nicht viel mehr erwartet. Doch Wilfrieds Reaktion alarmierte sie zutiefst.

Wilfried liebte seine Tochter. Er und sie hatten Amélie seit gedenken vor ihrer Mutter in den Schutz genommen. Sie hatten Amélie nicht vor dem Hass und der Eifersucht bewahren können mit der Luise ihre Tochter attackierte, aber sie hatten Amélie geholfen damit fertig zu werden und sich Orte der Liebe zu schaffen, wo ihre Mutter sie nicht erreichte.

Wenn Wilfried sie wegschickte, dann war etwas schreckliches passiert\dots Berta bemerkte am Rande ihres Bewusstseins eine Unordnung. Dinge verlangten ihre Aufmerksamkeit, Dinge, die sie im Zimmer gesehen hatte\dots Luise und Wilfried hatten Handschuhe getragen! Dann fiel ihr der Plan ein\dots der rote Kreis in der Innenstadt\dots als Luise sie rausgeworfen hatte, war sie an zwei Paar neuen Gummistiefeln vorbei gegangen\dots

Berta sass auf dem Dachfirst. Sie hielt sich ihren Flachmann an die Nase und sog den zart-herben Geist der Holunderbeeren ein, der sich im l'eau de vie- Wasser des Lebens spiegelte\dots

Sie sass eine Weile, mäuschenstill\dots bis die Gummistiefel und der Stadtplan ihr ihre Geschichten zugeflüstert hatten. Nur, weil sie sich sehr im Griff hatte, fiel sie nicht vom Dach, als ihr des Rätsels Lösung bewusst wurde. ,,Merci, mes amies! Les aimes de sureau!''

Berta erhob sich und schwang sich auf den Besen. Sie stieg in die Luft und landete nach einem kurzen Flug im Innenhof des blauen Hauses. ,,Hans? Haaans!\dots ''

\section*{4}
\addcontentsline{toc}{section}{4}

Amélie fühlte sich schläfrig. Sie sass zwischen Hathor und Wibrandis auf der Sitzbank. Die schweren Gerüche von essen und Pfeifentabak, Bier und Feuer hüllten Amélie gemütlich ein. Wibrandis unterhielt sich angeregt mit Odilia und Hathor raunte mit Isis. Amélie, an Wibrandis und die Wand gelehnt schloss die Augen.

Die übrige Gesellschaft war topfit. Thot wies den Benediktiner und Odilia in den Ablauf der Nacht ein. Diese machten ernste und andächtige Gesichter. Anubis hatte sich von seinem gemütliche Platz vor dem Kamin erhoben und sich zu den beiden gesetzt. Auch er hatte seine Hundestirn in feine Falten gelegt. Ab und zu stupste er mit der Nase Thots oder Johannes Arme, womit er sie dazu brachte ihm geistesabwesend die letzten Fleischhappen von der geplünderten Festtafel zu reichen. Schliesslich schleckte er sich leise über die Nase und lobte im Stillen ein weiteres mal seine Reisegestalt.

Thot bemerkte zu seiner Erleichterung, wie gut Odilia und Cagliostro sich verstanden. Er hatte befürchtet, es könnte für sie ungewohnt sein miteinander zu arbeiten, da sie Zeit ihres Lebens im Zölibat waren. Doch Odilia waren unbefangen. ,,Ich habe zu meiner Zeit in Strassbourg einiges über den Monte Sante Odilie gehört,'' sagt der Alchemist: ,,War der heilige Columban tatsächlich Euer Lehrer?'' ,,In der Tat!'' antwortete die Äbtissin. Da er Odilia nicht nachstehen wollte, wurde der benediktiner Mönch auch gesprächiger. Thot atmet durch, denn heute Nacht mussten alle hand in Hand arbeiten, da konnten sie sich keine Ressentiments leisten. Letztlich vertraute er auf Cagliostros Charisma und Charme, von dem der kleine Mann erstaunlich viel hatte.

Tef schlief vor dem Kamin und schnarchte selig. Hapi hatte sich zu ihm geflätzt und knapperte an einer Schale Nüsse, die er sich geklaut hatte. 

Am anderen Tischende besprachen Sebastian, Re, Horus und Amset den Plan. nach einiger Zeit rutschten Odilia und Wibrandis zu Thot und Johannes auf und Hathor und Isis zu der anderen Gruppe. Amélie glitt auf die Bank unter den Tisch.

,,Scht!'' rief Johannes und Sebastian hob den Zeigefinger. Es wurde still in der Herberge. Anubis und Tef, der erwacht war und sich reckte, hörten es als nächste. Dann die anderen. ,,Die Zehnerglock vom Liebfrauen! unser Startzeichen!'' 

Sie huschten aus der Bank und sprangen von den Stühlen, schnell, aber ruhig hatte sich die Gruppe in Windeseile vor der Tür der Herberge eingefunden. Amset und Horus hatten den Kofferraum des Leichenwagens geöffnet und den Sarg mit Osiris ein Stück rausgezogen. Sie hielten den geöffneten Deckel hoch, damit Isis ihren Gatten versorgen konnte.

Isis legte ihre Hand auf Osiris bandagierten Leib. ,,Bist du bereit, Liebster?'' ,,Ich war selten so bereit!'' sagte Osiris und lächelte schief. Er war, wie jede Nacht von Kopf bis zu den Füssen in schneeweisse Mumienbinden gewickelt. Isis zupfte zärtlich den prächtigen Perlenkragen zurecht, den Osiris um den Hals trug.

Osiris grünen Hände schauten aus den Stoffbandagen. Isis suchte in den Seitentaschen des Kofferraumes, die mit feinem Rüschen den Innenraum des Leichenwagens ausschmückten und dem Bestatter die Möglichkeit bot, von der Thermokanne, einem Schminkset oder wie in diesem Fall die Insignien des Fahrgastes sicher für die Fahrt zu verstauen.

Sie wischte die Stäubchen vom Krummstab. Dieser war aus purem Gold und trug Streifen aus tiefblauem Lapislazuli. Osiris fasste ihn mit der linken. Isis schüttelte die Geissel, das zeichen der Fruchtbarkeit bis alle Perlen und goldenen Glieder geordnet waren. ,,Dein Nechacha!'' Isis reichte es ihm und kletterte aus dem Heck des Wagens.

,,es ist gut!'' sagte sie knapp. Sie stand am Quai und beobachtete streng wie Horus und Amset den Sarg behutsam aus dem Wagen wuchteten und ihn die steilen Treppen zum Wasser herab trugen. Sobald sie das Wasser erreicht hatten, kräuselte sich die schwarz schimmernde, lichtbesprenkelte Wasseroberfläche.

Sobek tauchte knapp auf. Der gewaltige Rücken des Krokodilgottes sah aus wie ein riesiger Baumstamm. Horus und Amset stellten vorsichtig den Sarg auf den breiten Rücken. ,,Gib auf euch acht!'' rief Thot von oben und Horus gab dem Krokodil einen freundlichen Klapps. Sobek klapperte mit den Zähnen und drehte langsam ab in die Mitte des Flusses. Sobald er sie erreicht hatte, tauchte er ab, der schwere Sarg versank mit ihm.

,,Husch, husch, meine Lieben!'' rief Re. Seine Augen blitzten hinter den Gläsern seiner Sonnenbrille auf. Isis löste sich von dem Geländer und warf einen letzten Blick über den Fluss. Re nahm seine Enkelin in den Arm. ,,Wir werden die beiden gleich wieder treffen, Isis.'' ,,Ich weiss!''seufzte sie. Hathor war neben die beiden getreten: ,,Kommt, wir wollen Osiris und Sobek an jenem Ort nicht zu lange allein lassen!'' Die Muttergöttin, die von hinten aussah wie Berta in einem Pelz, begann die Gruppe zu inspizieren.

Johannes Tauler und Sebastian Brant hatten sich schon an den Kopf der Gruppe begeben, um ihre Gäste auf dem schnellsten und unsichtbarsten Weg auf der Altstadtinsel zum Münster zu führen. Anubis und Tef schnüffelten. Aha! Die Huskydame wohnte wohl in der Nähe, den sie hatte eine weitere Duftmarke hinterlassen.

,,Wo ist denn Amélie?'' fragte Wibrandis plötzlich. ,,Amélie?'' ,,Amélie!'' Amset rutschte das Herz in die Hose, doch bevor sich jemand richtig Sorgen machen konnte, war Hathor in die Herberge geeilte und zog nun eine schlaftrunkene Amélie über die Strasse. 

Amélie sah zu Amset hinüber, aber dann hackte Wibrandis sich bei ihr ein und Odilia auf der anderen Seite. ,,Auf gehts!'' sagte Wibrandis und ihre Augen funkelten\dots 

\section*{5}
\addcontentsline{toc}{section}{5}

Berta eilte in die Küche des blaue Hauses. ,,Haans!'' Aus dem Augenwinkel bemerkte sie einen dunklen Schatten auf sie zu schiessen und verzweifeltes Jaulen, bevor sie der Länge\footnote{In ihrem Fall Kürze} auf den Boden prallte und mit dem Kopf an den Herd stiess, der in der Mitte der Küche thronte. Benommen rappelte sie sich hoch  blieb auf dem Boden sitzen. Der Dackel, den Berta als sie gestolpert war mit dem Fuss heftig gegen die Küchenschränke geschubst hatte, berappelte sich ebenfalls. Er blieb vorsichtshalber in einiger Entfernung und kläffte und jaulte hysterisch.

,,Waldi! Alter Freund! Komm her, komm zur alten Berta!'' Berta lockte den ängstlichen, verwirrten Hund. Und dieser kläffte und knurrte, während er sich näherte. Er drehte sich immer wieder und hüpfte. Berta nahm ihn sanft auf den Schoss. ,,Nun ist aber gut!'' Sie hielt den Hund, dessen flattriges Herz langsam ruhiger klopfte. Berta und der Hund sassen für einen Moment still da.

,,Oh, weh!'' sagte Berta schliesslich. Sie hob den Dackel vor ihr Gesicht und sagte freundlich: ,,Kein Wunder, dass du dich so !aufregen musstest!'' 

Sie setzte Waldi auf den Boden. Sofort sprang der Dackel aus dem Küche und bog links ab zum Frauenbad. Berta rannte hinterher. ,,Tefnut!'' rief sie erschrocken. Berta eilte zu dem grossen Schwimmbecken, in dem der Körper der Göttin leblos trieb. Das Bad war warm und feucht. Über dem Wasser webte ein Dunstschleier. einzelne Seifenblasen, die sich aus dem Badeschaum gelöst hatten, flogen sachte umher. 

Tefnut lag mit dem Gesicht nach unten im Wasser. Ihr feuerrotes Haar wogte sachte auf der Wasseroberfläche wie ein riesiger Fächer um ihren Kopf. Ihr Körper lag still mit ausgebreiteten Armen auf dem Wasser, das sich langsam rot färbte. Tefnut trug eine Wunde am Hinterkopf aus dem das Blut in die Haare und von dort in das Wasser ran. Waldi rannte wild um das Becken und bellte. Seine raue Dackelstimme halte unheimlich von den gekachelten Wänden durch den Raum. 

Berta stieg mit all ihren Kleidern in Wasser und zog Tefnut an den Rand des Beckens. Vorsichtig schob sie die reglose Göttin über den Rand. ,,Tefnut!'' flüsterte Berta und strich das klatschnasse Haar aus dem blassen Gesicht. Das Blut lief aus der Wunde an Tefnuts Hals in den Ablauf am Beckenrand. ,,Tefnut! Wach auf!'' Berta hatte einige Mühe ihr Strumpfband unter Wasser und unter all den vollgesogenen Röcken zu finden, aber schliesslich hielt sie triumphierend ihren silbernen Flachmann in die Höhe. 

Waldi hatte mit der Wiederbelebung begonnen und schleckte der Göttin eifrig das Gesicht ab. ,,Aargh! Jetzt!'' Berta hielt Tefnut die flasche unter die Nase. 

Die Augenlider der Katzengöttin zuckten und die feine Nase kräuselte sich! Der schön geschwungene Mund verzog sich angeekelt und dann nieste Tefnut. Waldi tappte erwartungsvoll von einem Stummelbein auf das andere. Und Berta seufzte und machte sich auf den Weg ihre vollgesogenen Kleider aus dem Schwimmbecken zu wuchten.

Es flatschte und patschte und klatschte, während Tefnut sich stöhnend erhob und sich den Kopf abtastete. ,,Miiaaauuuuh!'' Tefnut schaute auf ihre blutigen Finger und schlang sich schnell ein Handtuch um den Kopf. dann eilte sie zu Berta und umarmte sie fest. ,,Gut, dass du da bist!'' ,,Was zum Teufel ist hier los?'' fragte Berta und zappelte vergeblich herum, um ihre nassen Kleider abzustreifen. ,,Eben, der ist Los!'' antwortete Tefnut, ,,Wuah,wuah!'' meldete sich Waldi. 

Tefnut half Berta aus den Kleidern. ,,Der Teufel?!'' fragte Berta und wickelte sich in ein rosa Badetuch. Sie sah aus wie ein riesiger, flauschiger Ball. Tefnut zog ihren kurzen Satinmantel über. ,,Seth!'' knurrte sie knapp. Dann stürzte sie aus dem Bad. ,,Maat! Schu!'' rief sie. Berta eilte hinterher und holte auf. ,,Hans?'' rief sie. Tefnut und sie rannten durch die Eingangshalle zum Treppenhaus.

,,Ich weiss nicht!'' rief Tefnut über die Schulter und sprang jede zweite Stufe nehmend hinauf. Berta trippelte gleichauf hinterher. Sie beschleunigten als sie den ersten Stock erreicht hatten, denn aus dem zweiten Stockwerk war ein leises Stöhnen zu hören. ,,Schu!'' ,,Hans!''

Die beiden Göttinnen hatten die oberste Stufe erreicht und stürzten in Res Zimmer, dessen Tür offen stand. Im Treppenhaus blieb das emsige, unermüdliche kratzen von Waldis Pfoten auf den Holzstufen zurück.

\section*{6}
\addcontentsline{toc}{section}{6}

Sobald Luise vor dem Hotel auf der verlassenen Strasse stand, wich alle Farbe aus ihrem Gesicht. Bleich und zu Tode geängstigt rührte sie sich nicht von der Stelle. Für einen Augenblick keimte eine Spur Mitleid in Wilfried auf. ,,Was?'' zischte Luise. Wilfrieds Lächeln erstarb sofort. ,,Nichts!''sagte er kalt. Er spürte zum ersten mal, seit er Luise begegnet war Hass auf seine  Frau.

,,Los!'' sagte Wilfried und zog Luise grob am Arm mit sich. Die erwartete Gegenwehr blieb aus. Sollte Luise tatsächlich vor etwas Angst haben? Sie hasteten zu dem vereinbarten Treffpunkt. Als sie dort ankamen, war von dem Fremden, Seth, nichts zu sehen. Ein schwacher Lichtschein, der einen schmalem Eingang an der Seite des kleinen Platzes erhellte, der in den Hügel führte, erregte ihre Aufmerksamkeit. Sie begaben sich in den Gang und schlüpften durch die dicke Steintür. Sobald sie im Hügel waren schabte die Tür leise über den Boden und schloss sich mit leisem Klick!

Wilfried zog sein Smartphone raus, doch das Gerät machte keinen Mucks, obwohl er es frisch aufgeladen hatte. ,,Nichts!'' raunte er. Wohl oder Übel, sie tasteten sich Stück für Stück durch den stockdunklen Gang.

\section*{7}
\addcontentsline{toc}{section}{7}

,,Schu!'' schrie Tefnut. Hätte Berta die Göttin nicht gepackt, wäre sie ebenso aus dem Fenster gestürzt wie ihr Mann. Berta zerrte Tefnut in das Zimmer zurück. Der Sessel des Sonnengottes streckte alle Viere von sich. Das Buch, dessen Vorderseite ein Junge mit einer Brille zierte, lag zerrissen auf dem Boden. 

Die beiden Göttinnen machten kehrt und sausten die Treppenstufen zurück ins Erdgeschoss. Waldi, der gerade die letzte Stufe erklommen hatte, seufzte und machte sich ergeben an den Abstieg. Ein kurzer Augenblick des Gepolters, dann blieb kratzten seine Pfoten durch das stille Treppenhaus\dots

Waldi fand die Göttinnen vor der Tür auf der Strasse. Berta flösste dem Luftgott einen Schluck l'eau de vie ein, das ihn mit krächzen und husten auf die Beine brachte. Mit vereinten Kräften schleppten die Göttinnen den schwankenden Gott ins Haus. Wobei dieser, durch die unterschiedliche Grösse seiner Helferinnen mehrmals auf die kleine, flauschige kugel zu kippen drohte. Nur Bertas energisches Schubsen verhinderte, dass Schu auf sie fiel.

Tefnut gab der Tür, nachdem sie sich zu dritt durchgezwängt hatten, einen Tritt und sie fiel ins Schloss. Waldis Nase fuhr empor. Er kläffte kurz und empört. Er hatte doch nur kurz den Laternenpfahl vor dem Haus\dots

Der Hund seufzte, dann machte er sich auf den Weg zur Rückseite des blauen Hauses. Seine Pfoten tapsten auf dem Pflaster. Er schlüpfte durch das Schmiedeeiserne Tor und rannte durch den Seiteneingang.

Als Waldi in die erleuchtet Küche kam, hechelte er heftig. Er keuchte und hechelte und schlappte mit der Zunge. Schu und Tefnut sassen am Küchentisch, je ein Tuch mit Eiswürfeln an den Kopf gepresst. Berta. die sich Eiswürfel in ihren Drink gefüllt hatte, erhob sich und gab dem japsenden Hund Wasser in seiner Schüssel. In der Küche seufzte, stöhnte, ächzte und schlabberte es.

,,Es ist unglaublich! Aber Seth hat es geschafft uns zu überrumpeln!'' sagte Schu. ,,Das Schlimmste ist, dass Maat verschwunden ist!'' sagte Tefnut. ,,Und Hans!'' meinte Berta. ,,Es ist eine Katastrophe! Wenn der heiligen Ordnung, unserer Maat, etwas zugestossen ist, dann, dann bricht das absolute, tödliche Chaos aus!'' rief Tefnut. ,,Und wenn wir Hans nicht wiederfinden, dann wird der Frühling nicht kommen, die Bäume werden keine Blüten tragen.'' Berta schluckte schwer. sie schwiegen. Schu sah sich in der Küche um. ,,Maat muss sich versteckt haben! Wenn sie verschwunden wäre, dann wäre kein Stein mehr auf dem anderen! Aber, wo ist sie?''

Waldi hatte kaum ausgetrunken, da trippelte er zu Berta und stellte sich auf die kurzen Hinterbeine. Er jaulte und bellte, sein Schwanz wedelte wild und verfing sich in Bertas langen Reiseröcken. ,,Jaja, mein Gutster! Bist ein braver. Hättest ja nix könne mache.'' sagte Berta. Waldi hielt verdutzt inne. Was war nur in Berta gefahren. Wieso verstand die Erdgöttin, Berta, die Sprache der Tiere nicht mehr?

Waldi liess von Berta ab und rollte verzweifelt mit den Augen. er stellte sich in die Mitte der Küche und begann im Kreis zu rennen und zu kläffen. Ab und zu blieb erstehen und bellte herausfordernd. Als ihn alle drei Götter anschauten, als wenn sie ihm gleich den Hals umdrehen wollten, hielt Waldi es an der Zeit Phase zwei einzuleiten. Er lief zur Küchentür blieb dort erwartungsvoll stehen und jaulte kurz.\footnote{Er wollte sein Glück nicht strapazieren und in einer Stichflamme enden!}


\section*{8}
\addcontentsline{toc}{section}{8}

Die Gruppe, die, wenn sie sichtbar gewesen wäre, an einen kleinen Karnevalsumzug erinnerte, lief über die Pont Staint Nicolas und auf dem anderen Ufer der Ill die Treppen zum Treidelweg hinunter. Sie bewegten sich zügig. Thot beobachtete aufmerksam die Passanten, die ihnen auf der anderen Uferseite begegneten und die von der Terrasse des Ancienne Duoane rauchten und auf den Fluss schauten. Keiner von ihnen beachtete sie. Er entspannte sich. ,,Isis Schutzschild hat gehalten, denke ich. Wir können uns scheinbar unsichtbar durch die Stadt bewegen.'' sagte er zu Re. Der Sonnengott blieb einen Moment stehen und musterte seinerseits die Menschen, die sich in ihrer Umgebung befanden. ,,Du hast recht! Und das ist gut!'' antwortete Re und wandte sich sogleich a Horus und seine Söhne.

Sie stiegen die Treppe vom Pier zum Fischmark hoch und durch die enge Rue de Rohan, am Palast vorbei, direkt auf die Kathedrale zu. Der Platz war leer, es hatte zu Regnen begonnen\dots Ein Windhauch wehte ihnen ins Gesicht, der unermüdlich um die Säulen der gewaltigen Kirche strich.

Sebastian zeigte auf ein Fachwerkhaus, das an der Ecke zur Rue Merciére stand. ,,Dort war früher eine Apotheke, eine sehr gute und alte eingesessene!'' ,,Stimmt!'' rief Wibrandis erstaunt aus. Sie hatte mit zwei ihrer Ehemänner auch zwanzig Jahre in Strassburg gelebt, ebenso wie Sebastian, der zwar in Strassburg geboren war, aber lange zeit in Basel gelebt hatte, bis er schliesslich in seine Heimatstadt zurückkehrte und dort Ratsherr wurde. ,,Allerdings konnte ich mir die Medizin nur selten leisten!'' sagte Wibrandis. ,,Ich kenne die Apotheke auch!'' fügte der Dominikaner an. ,,Wir Mönche haben uns jedoch eher aus unserer eigenen Apotheke vom Konvent versorgt.'' ,,Selbst ich kenne die Apotheke noch!'' rief Cagliostro. Er besann sich: ,,Ich kann mich  an den alten Heraclides IV. erinnern! So sein wissenschaftlicher Name.'' Des Grafen Gesicht verdunkelte sich: ,,Ja, damals haben wir viel diskutiert. Schliesslich war ich dann zu Johanns Beerdigung.''

,,Kommt! Kinderchen!'' sagte Horus. Seine tiefe Stimme liess die Erinnerungsseifenblasen platzen und die Gesichter wendeten sich ihm zu. ,,Es ist Zeit! Sobi und Osiris sind sicher schon auf dem See!'' 

,,Alessandro! nun ist dein Job gefragt. Du bist und bleibst unser Zeitspezialist,'' sagte Thot und der kleine dicke Graf stellte sich vor der Gruppe auf. ,,Also! Meine Damen und Herren!'' begann er, Wibrandis kicherte verstohlen. ,,Unter dem Nachbarhaus der Apotheke befindet sich der Einstieg zu dem sagenumwobenen See, der sich unter der Kathedrale und dem Platz befindet. Es wird ein sehr starker Wind aus dem Eingang herausblasen und Nebelschwaden werden aufsteigen. Deshalb müssen wir dicht beieinander bleiben. Für die Menschen unter uns, lasst euch nicht beeindrucken. Die Angsthasen der letzten Jahrhunderte haben das Geheimnis des Sees nicht mehr gekannt und vor lauter Furcht haben sie endlich den Eingang zugemauert. Ich werde die Häuser in einen Zeitraum versetzten, indem der Eingang offen ist. Damit sich niemand anderes dorthin verirrt, werde ich Das Zeitfenster einige Zentimeter in die Hauswand verlegen.''

Der Graf war so begeistert von diesem Auftrag und seiner Idee, dass er sich auf die Zehenspitzen gestellt hatte. Nun zappelte er aufgeregt und zog ein langes Seil aus seinem Justaucorps. Er gab den Seilanfang Anubis ins Maul, dann reihten sich alle ein und fassten das Seil. Das Seilende nahm Tef zwischen die Zähen. Cagliostro trat ein Schritt zurück und betrachtete sein Werk: Anubis, Isis, Re, Hathor, Odilia, Wibrandis, Amélie, Thot, Johannes Tauler, Sebastian Brant, Cagliostro, Horus mit Kebi, Amset auf seinen Schultern Hapi und Tef. 

Horus hob plötzlich den Kopf und schaute alle genau an: ,,Es fehlt doch jemand!'' sagte er unsicher. Alles sahen sich um. ,,Isfet!'' rief Amélie. ,,Isfet fehlt!'' Thot schaute auf seine Uhr. ,,Wir müssen! Es hat keinen Zweck und keine Zeit Isfet zu suchen: Wir gehen! Jetzt!'' Doch ihm war zu tiefst unbehaglich! Und ein Blick auf die anderen Götter, Re, Isis und Horus zeigte ihm, dass es ihnen ebenso erging.

Cagliostro öffnete den Raum und sie sammelten sich in einem schmalen Gang des Nachbarhauses. Sie blieben vor einer schweren mit grossen Eisenscharnieren versehenen Tür stehen: ,,Hereinspaziert in la pègre de la Cathédrale Notre Dame de Strassbourg!'' sagte Sebastian mit leicht zittriger Stimme.

Sie zwängten sich nacheinander durch die schmale Tür. Aus der, sobald sie geöffnete worden war, ein eisiger, heftiger, heulender Wind gebraust kam.

Amélie packte das Seil fester. Obwohl sie schon so viele Abenteuer erlebt hatte in den letzten nächten, ging ihr die Kälte des Windes durch Mark und Bein. Das Heulen schien ihr durch die Ohren direkt jeden einzelnen Nerv und Gedanken wegzufegen. Sie wollte sich die Hände auf die Ohren pressen\dots da hörte sie leise Thots Stimme durch das lärmende Getöse:

\textit{Verweilen der Majestät dieses grossen Gottes\\
in der Wassertiefe ,,Herrin der Unterweltlichen''.\\
Er er teilt Weisungen den Gottheiten, die in ihr sind,\\
er befiehlt, dass sie sich ihrer Gottesopfer bemächtigen bei dieser Stätte.}

Und Odilia antwortete ihm:

\textit{lala}

Johannes Tauler fiel ein:

\textit{Es kommt ein Schiff, geladen\\
bis an sein höchsten Bord;\\
es trägt den Sohn des Vaters,\\
das ewige wahre Wort.\\
\\
Das Schiff, das kommt so stille,\\
es trägt ein' teure Last;\\
das Segel ist die Liebe,\\
der Heilige Geist der Mast.}



\section*{9}
\addcontentsline{toc}{section}{9}

,,Du! ich glaub der Dackel will uns etwas sagen!'' sagte Schu, der sich sichtlich erholt hatte. ,,Jetzt, wo du es sagst! Waldi! Du weisst etwas, stimmts!'' fragte Berta. Waldi trat ungeduldig von einer Pfote auf die andere und wedelte mit dem Schwanz. ,,Meint ihr?'' fragte Tefnut. Sie, als Katzengöttin hatte ein natürliches Misstrauen zu Hunden. Waldi jaulte und sprang auf!

,,Ich gehe mal schauen, was der Hund will!'' beschloss Schu. Er und Berta erhoben sich und Waldi seufzte und sauste los Richtung Dienstbotenaufgang, der vom Gang zur Küche abzweigte und der schnellste Weg in die oberen Stockwerke war.

Tefnut verwandelte sich betont langsam in eine Katze. Ihr Stolz war heute Abend genug verletzt worden, als dass sie einem Hund so ohne weiteres das Feld überlassen konnte. Sie sprang leichtfüssig und elegant aus der Küche und kam punktgenau zwei Stufen vor den drei andern im ersten Stock an. Waldi ignorierte die rote Katze, die leise miaute und eine grosse Beule mit eine Platzwunde zwischen den Ohren trug.

Wie es sich für einen Jagdhund gehörte, blieb Waldi in hab-Acht-Stellung vor der kleinen, unscheinbaren Holztür im Treppenhaus stehen. Berta schaute zweifelnd auf die Zwergentür. Waldi begann zu bellen, und Schu öffnete vorsichtig die Tür, die ihm nur bis an den auch reichte. Im Inneren wurde ein schmaler Gang sichtbar, der rechts und links offensichtlich zwischen den Wänden des Treppenhauses und den Zimmern verlief. ,,Na, endlich!'' rief eine helle Stimme aus dem Dunkel.

,,Maat!'' rief Schu. Er eilte in die Finsternis und schrie vor Freude: ,,Maat! Maat! Oh, Mann!'' Berta, die ein Stück in den schmalen Raum getreten war, nahm die junge Göttin der Ordnung in Empfang und zog sie dann in ihre Arme. Waldi umkreiste die beiden mit Gebell. Die Katze sass gelangweilt auf der Fensterbank und schleckte sich die Pfote. Waldi wusste, der Tag würde kommen, der Tag, an dem die Katze büssen würde, Katzengöttin hin oder her\dots


\section*{10}
\addcontentsline{toc}{section}{10}

Es war stockfinster. Der Wind, der ihnen aus dem Gang unter die Kathedrale entgegen peitschte war kalt und scharf. Eigenartige Geräusche brachen sich and den Wänden zu einem vielstimmigen Jammern und Klagen. Amélie war es, als würde sie wieder und wieder von eisigen, feuchten Händen gestreift. Modrige Spinnenweben legten sich auf ihr Gesicht und klebten an der Haut und den Augen. 

Das schlimmste waren jedoch die Gefühle, die der Wind im Inneren aufweckte. Es war das hoffnungsloseste Sehnen, die quälendsten Gedanken,die schärfsten Selbstvorwürfe und der ärgste Selbsthass, die Amélie, die sich durchaus für eine Expertin auf diesem Gebiet hielt, je gefühlt hatte. Wie hatte Wibrandis geraunt, als der erste Windstoss sie gestreift hatte: ,,Pesthauch!'' 

Wibrandis, die vor Amélie ging, quitschte und kreischte von Zeit zu Zeit und Amélie wurde vor dem nächsten Pesthauch gewarnt. Es tröstete sie, dass selbst ein erfahrene und tote Frau wie Wibrandis hier an ihre Grenze kam. Odilia, soweit Amélie es im Brausen hören konnte, betete unablässig vor ihr, genauso wie Johannes Tauler und Ratsherr Brant hinter ihr.

Amélie schlich mit gesenktem Kopf weiter. Plötzlich bemerkte sie einen Lichtfaden. Sie schaute überrascht auf. Aus Odilias Mund, der weiterhin Gebete murmeldte, wandt sich ein feiner, goldener nebelstreifen. Als würde sie eine zigarre rauchen und der Qualm käme golden aus dem Mund. es waren zarte Gebilde, die sich hartnäckig in der rauen, fauligen und bewegten Luft hielten und diejenigen sanft berührten, die dirch sie hindurch gingen.

Amélie dreht sich um und bemerkte dieselben goldenen Rauchwolken bei Johannes und bei Sebastian. Sie stolperte, konnte sich rechtzeitig fangen bevor sie auf Wibrandis stürzte. Der Gang aus glitschigen, von Algen grüngefärbten Steinen führte jetzt steiler hinunter.

Thot hingegen schien den Spaziergang zu geniessen, er murmelte nur hin und wieder etwas über die sagenhafte Akustik in diesem sagenhaften Höhlensysthem. 

-Höhle? dachte Amélie. -Ja, meine Liebe! Wir sind gleich an unserem Ziel angelangt. Amélie bemerkte eine Unruhe in der Gruppe. Sie hörte die anderen murmeln und verwundert rufen.

Sie hörte Res kraftvolle Basstimme ein klangvolles ,,Ahoi, Kameraden!\dots '' schmettern, das sich ins unendliche ausdehnte und in vielzahligen Echos fröhlich zurückkehrte und in den Gang hallte.

,,Kommt, kommt, meine Lieben, das müsst ihr euch ansehen!'' Hathor zog die sie in eine riesige Höhle, der Ende nicht zu sehen war. Sie standen am Ufer eines unendlichen Sees. das Wasser schien zu leuchten und strahlte ein grüne-gelbes Licht aus, dass sich an einigen Stellen rot reflektierte\dots oder waren es die weit entfernt sichtbaren Wände? Die Decke von der tropfenförmige Kalkgebilde herab hingen?

,,Ohh!'' raunte Amélie. Sie spürte einen Arm um ihre Schultern. Amset war neben sie getreten und staunte mit offenem Mund. 

,,Mein Sohn, mach den Mund zu sonst kommen Flüche rein!'' lachte Horus und gab Amset einen kleinen Kinnharken. Johannes und Sebastian rissen ihre Augen auf. Johannes bekreuzigte sich. ,,Wer hätte gedacht, dass es diesen See wirklich gibt?'' fragte er. Sebastian grunzte zustimmend. Auch er bekam den Mund nicht zu. er fasste die feuchte, warme, schimmernde Wand an und fuhr an ihr entlang, als fürchtete er, sie könnte sich wieder auflösen.

Kebi, der auf Amsets Schulter gesessen hatte, stiess sich ab und glitt mit einem Falkenruf auf den See. Er umflog die Kalksäulen die an einigen Stellen von der Decke in den See gewachsen waren und das Licht mit Schatten schmückten. ,,Er wird den See im Augen behalten!'' sagte Thot. 
,,Meine Freunde, wir müssen beginnen, wir haben nur noch wenig Zeit bis Mitternacht und das Ritual muss pünktlich beginnen!'' Amélie hob überrascht die Augenbraue, die Stimme des Alchemisten Cagliostro war warm, satt und wohltönend. Sie passte nicht zu der kleinen, dicklichen Gestalt in den geckenhaft bunten, berüschten Kleidern. 

Wibrandis drehte sich erstaunt um und Odilia lächelte spitzbübisch. ,,Cagliostro, mein lieber Freund, du hast ja so recht. Komm ich will dich auf unsere Barke einladen.''

Amélie schmunzelte, als sie Wibrandis pikierten Blick bemerkte, als die kleine Isis sich zwanglos und vertraulich bei Cagliostro einhängte und ihn zum Ufer führte.

Wie von Zauberhand war der Sarg von Osiris wie ein Boot auf sie zu geschwommen und stiess an das Ufer. Amélie schauderte, denn der Zauber entpuppte sich als die riesige Krokodilsschnauze von Sobek!

,,Sobek, mein Guter, ich danke dir, dass du meinen Enkelsohn und den Gemahl meiner Enkelin wohlbehalten an dieses Gestade gebracht hast.'' Re beugte sich zu dem Krokodil und klopfte ihm freundlich auf das Maul. Dann schüttelte es den Sonnengott und er stand in da in seinem zeremoniellen Schurz. Sein nackter Oberkörper glänzte golden-grün und schillerte in der Reflektion des Wassers.

Die Männer und Horus wateten in den See und hoben mit vereinten Kräften den Deckel des Sarges ab. Osiris lag darin als hielte er ein Nickerchen. In der Tat gähnte er und murmelte leise bewundernde Worte über die Schönheit der Höhle.

Die Männer drehten den Sargdeckel um und schoben das Behelfsboot an das Ufer. Amélie staunte, denn beide Sarghälften wurden grösser und grösser, so dass sie zu zwei stattlichen Barken wurden. 

Isis stieg in die Barke des Osiris, der in der Mitte erhöht ruhte. Sie winkte Cagliostro und der kleine Magier kletterte ächzend über die Reling der Barke an die Seite der Mondgöttin. Horus schob die Barke in das Wasser und stieg dann vorsichtig dazu. Er holte ein Ruder hervor und einen Augenblick später befand sich das Boot schon mitten auf dem See. Amélie hörte wie Isis und Cagliostro begannen ihren rituellen Formeln zu rezitieren.

Re hob die Arme über den Kopf und begann ebenfalls eine mächtige Formel zu sprechen. Die Augen, die nicht mehr von der Sonnenbrille bedeckt waren, gleissten in unglaublichem Licht auf und erloschen dann. Amélie erschrack und schaute ängstlich zu Thot, doch der lächelte ihr beruhigend zu. -Schau!

Der Sonnengott beugte sich zu seiner Frau hinunter die ihm sanft die Augen aus den Augenhöhlen nahm. Sie winkte Odilia und legte der Heiligen in jede Hand einen Augapfel. ,,Hüte sie gut! Du, als ,die ihr Bild verbirgt'!'' sprach Hathor feierlich und lächelte Odilia zu. Thot und Sebastian halfen der Göttin und der Heiligen an Bord der Barke des Re. Dann kam Anubis, der die ganze Zeit still im Hintergrund aufmerksam gelauscht und beobachtet hatte, ans Ufer und schob seinen hohen Rücken unter die Hand des Sonnengottes. Dieser liess sich von dem schwarzen Hund an Bord der Barke geleiten. Sie stiegen beide hinein und Re legte sich in der Mitte der Barke auf ein Podest nieder. Hathor und Odilia wachten an seiner Seite. Anubis stellte sich an das Heck und lenkte die Barke mit seiner Aufmerksamkeit auf die Mitte des Sees und auf die andere Barke zu.

,,Jetzt! Schnell!'' rief Thot den Verbliebenen am Ufer zu. ,,Johannes, Sebastian und Wibrandis behaltet die Barken in eurer Konzentration, betet, dass es uns gelingen möge an diesem besonderen Ort die Wiedergeburt der Sonne zu bewirken!'' Die drei Menschen schauten ehrfurchtsvoll auf den See. In weiter Entfernung hatten sich die Barken getroffen und begannen in wundersamen Licht und Nebelbändern zu leuchten. Das Murmeln von Sprüchen und beschwörungen halte von den Höhlenwänden und über das Wasser\dots und die drei Menschen fielen auf natürliche und andächtige Weise in die Stimmen der Götter ein.

Amélie spürte, wie Thot durchatmete. ,,Amset! Du gehst unverzüglich wieder hinauf und sicherst den Eingang der Höhle vom Münsterplatz aus. Verhalte dich so unauffällig wie möglich und lasse den Münsterturm nicht aus den Augen!'' Amset drückte kurz Amélies Schulter und war -verschwunden. Amélie zuckte zusammen.

,,Hapi und Duamtef ihr zwei sichert hier das Ufer! Wir anderen werden mit der Verwandlungen beschäftigt sein. Wenn ihr etwas entdeckt, dann kommt unverzüglich zu mir!'' Hapi schwang sich an den Felsen und Vorsprüngen der Höhle in die eine Richtung und Duamutef wendete sich in die andere.

Unversehens befanden sich Thot und Amélie alleine am Ufer. Amélie schauderte. Thot legte sich ein weisses Seidentuch über die Hand und machte mit der anderen darüber Zeichen in der Luft. Mit einem zarten ,,Blopp!'' befand sich plötzlich etwas in der verdeckten Hand.

-Nimm das Tuch weg! Amélie zog vorsichtig an dem Tuch und der Kelch kam zum Vorschein! Amélie holte tief Luft\dots Thot reichte ihr den Kelch, dessen diamantene schale sanft und strahlend die Funken des göttliehcne Lichtes, das vom See herüber blitzte zurück warf.

Thot griff in den Kelch und holte den klaren Rubin an der goldenen Kette heraus. Er legte sie Amélie um den Hals. -Er wird dich wärmen und unterstützen, wenn Du Kraft verlierst, dann richte einen Augenblick deine Aufmerksamkeit auf den Stein. Dennoch wirst du eine schwierige Aufgabe vor dir haben! -Ich weiss, antwortete Amélie, -du hast es mir ja erklärt! 

-Sobek wird dir helfen! Amélie dreht sich zum Ufer um, dort wartete das Krokodil schon auf sie! Sie ging zu ihm hin und bemerkte ein Lächeln in den Augen des Reptils. -Hallo, junge Frau, kann ich Sie irgendwohin mitnehmen? Da musste Amélie lachen! -ja, gerne! Einmal in die Mitte des Sees bitte!

Sie schwang sich mit einem kleinen Seufzer auf den Rücken des riesigen Reptiliengottes und er schwamm vorsichtig mit ihr zur Mitte des Sees. -Sobi? Mmhh? -Kannst Du mir sagen, warum es in den Geschichten immer um Jungs geht, die Abenteuer bestehen? fragte Amélie und liess ihre Blick über den Unterirdischen See gleiten. -Vielleicht, weil die Jungen früher in die Schule gehen durften und vor den Mädels lesen und schreiben lernten? - \dots ? \dots !

Krokodil und junge Frau seufzten. Und dann genossen sie einen Moment, den Moment auf der spiegelglatten Wasserfläche des unterirdischen Sees. Durch die hauchzarte Wellen liefen, die von den Barken ausstrahlten, die sachte, sachte schaukelten. Das Ufer lag im Halbdunkel. Die Gestalten der Männer und die von Thot waren Umrisse, in mattes goldenes Licht gehaucht.

Die Wände der Höhle strahlten dieses warme Licht zurück, das keinen Ursprung zu haben schien\dots Die Farben des lapisblauen Halsschmuckes leuchtete und die Gewänder der Götter strahlte weiss. Odilias weisser Schleier strahlte am hellsten. 

Amélie fragte sich einen Augenblick, ob die Äbtissin einen echten Heiligenschein hätte, doch das Licht kam aus einer anderen Richtung. Odilia musste die Lichtquellen in den Händen in ihrem Schoss halten!
Amélie erinnerte sich: Die Augen von Re!

\section*{11}
\addcontentsline{toc}{section}{11}

während die anderen auf den See hinausfuhren, hatte Amset sich in dem schmalen Gang, der zum Eingangstor der Höhle führte, im schatten vorgewagt. Er schaute vorsichtig auf den Münsterlatz hinaus, ohne den Schatten zu verlassen. In absoluter Stille verharrte er und beobachtete den Platz und die stille Kirche\dots

Amset meinte, das Münster würde gleich einmal tief Luft holen und alle die kleinen Häuser, die so eng um es herum standen, ausversehen freundlich umschubsen. er legte den Kopf in den Nacken. Die Spitze des einen Turmes war hinter dem dunkelgrau verhangenen Nachthimmel kaum zu sehen. 

-Ich werde beobachtet! Amset hatte es sofort bemerkt, sobald er die dicke Tür in den schmalen Durchgang, der zum Tor der Höhle führte, durchdrungen hatte. Wo? Amset spürte sein Herz klopfen. Wo war er, wo war \dots es?

Eine Millisekunde und ein Atemzug\dots Nur für die Sinne eines göttlichen Wesens wahrnehmbar, bewegte sich ein ebenso göttliches Wesen auf der Spitze des Turmes! Amset bündelte seine Aufmerksamkeit und liess sie wie ein Pfeil zu der dunklen, verborgenen Spitze fliegen.

\begin{Large}
-Amset!
\end{Large}
 brüllte es in seinem Kopf!
 \begin{Large}
 -Hast Du mich bemerkt! Spatzenhirn! Richte den anderen meinen Gruss aus! Mir ist es zu langweilig! \dots gehe lieber schauen, was mein Schwesterchen macht! Au revoir, mon Chouchou!
 \end{Large}
 
 
\begin{Large}
,,Isfet!''
\end{Large}

Der Schrei von Amset hallte über den Platz! ,,Ruhe!'' brüllte es aus einem offenen Fenster zurück. Doch die Göttin des Chaos liess sich kreischend fallen und spannte riesige Fledermausfügel auf. Sie raste an Amsets Versteckt vorbei und war in der Dunkelheit verschwunden.

Amset wurde es schlecht! Isfet! irgendwie war sie in den letzten Stunden vergessen gegangen! Das war nicht gut! Das war garnicht gut! Horus würde toben, wenn er es erfuhr\dots

Aber Amset konnte nichts tun. Er durfte seinen Wachposten nicht verlassen und schlimmer,er durfte Horus und die anderen nicht mehr stören\dots die Münsteruhr schlug 12 mal! Mitternacht! Die neuen Sonne musste nun geboren werden! Dies war der heikelste Moment jede Nacht! Nichts, nichts durfte diesen Vorgang gefährden, nichts, nichts durfte sich feindlich nähern, alles, alles musste daher unter grösster Geheimhaltung und mit grösster Sicherheit stattfinden! Deshalb waren sie nach Frankreich gewechselt\dots Basel war kein sicherer Ort mehr, nachdem Seth seinen Anschlag verübt hatte und die Dunkelheit, der Fluch aus dem Norden die Stadt erreicht hatte\dots

Seth! Oh, nein, Seth! Er war in Basel! Er und Isfet! Zusammen konnten sie grosses Unheil wirken. Amset raufte sich verzweifelt die Haare, Isis Zauber, den sie an der Grenze gewirkt hatte, würde seine Gedanken stoppen, er konnte Schu und Tefnut nicht warnen! Und er durfte die anderen nicht stören!

Ich muss den Brüdern Bescheid sagen! Sie sind in der Höhle und können sofort Kunde geben, sobald die Zeremonie beendet ist!

-Horussöhne, Horussöhne! Hört mich meine Brüder! Ein Schatten ist auf dem Weg zur Maat! Isfet ist auf dem Weg zu Maat, sie will die Ordnung  zerstören!

So schwer es Amset fiel, so blieb er, wo er war! er konnte nicht helfen\dots Maat nicht, Osiris nicht und Amélie, \dots er seufzte, nicht!

\section*{12}
\addcontentsline{toc}{section}{12}

,,Aahh! Da bist DU ja!'' raunte eine leise, zischen Stimme an Luises Ohr. Sie schrie auf! ,,das ist nicht nötig!'' sagte die Stimme. Ein Licht ging an. Vor Luise stand der Fremde, dem sie am Morgen auf dem Barfüsserplatz begegnet waren. Er trug eine Taschenlampe. Er reichte Wilfried und Luise zwei weitere Lampen. ,,Seid leise und folgt mir! ihr solltet den Bewohner der Höhle nicht mit eurem jämmerlichen Gekreische nerven!'' Er drehte sich um und machte einige Schritte in den Gang. Luise und Wilfried folgten. Der Boden war glitschig. Die Gummistiefel rutschten über Schleim und Algen. Luise glitt aus und wollte sich an der Höhlenwand abstützen. Seth war schneller und packte ihren Arm und riss sie von der Wand weg! ,,Ach ja'' raunte er ,,ihr solltet hier nichts berühren, wenn ihr am Leben bleiben wollt!'' Er liess Luise los, die sich den Arm rieb. Seth hatte die gleiche Stelle erwischt, an der Wilfried sie zuvor gepackt hatte\dots

Seth verschwand im Gang und Luise stolperte sie schnell sie konnte mit Wilfried hinter dem dunklen Gott her. Fr einen Moment besann sie sich und fragte sich, warum sie hinter diesem unangenehmen Fremden durch diese Höhle tiefer und tiefer unter die Erde kroch. Was machte es denn, dass Amélie mit Berta abgehauen war? Sollte sie nicht froh sein, dass sie die beiden los war? War nicht alles so, wie sie es wollte. Heile Welt mit dem Jungen und, nun ja, Wilfried?

Dann fiel ihr ein, warum sie sich auf den Weg gemacht hatte. Warum sie Wilfried gedrängt hatte, Berta und Amélie zu folgen. 

Es war eine kalte Novembernacht gewesen. Luise erinnerte sich, wie besorgt Berta gewesen war, weil Amélie wieder einen schlechten Traum gehabt hatte. Luise fand es lästig. Sie hatte kein Interesse an den wirren Träumen einer sechzehnjährigen\dots

Sie war froh, als die Alte endlich aufgab und aufhörte ihr ein Gespräch über Träume auf zu zwängen und mit einem schweren seufzen aus der Wohnstube in ihr Bett schlich.

Luise hatte sich ein Glas Wein eingeschenkt und sich in das breite Polster des Sofas und ihre Lieblingsflauschdecke eingekuschelt. Und plötzlich-

Plötzlich sass eine seltsame junge Frauen ihr gegenüber. Die eine mit einem perfekt geschnittenen Pagenkopf. Sie trug ein schlichtes, schwarz Kleid mit Rollkragen. Es reichte bis zu ihren Knien und darunter trug sie eine schwarze Wollstrumpfhose und schwarze Ballerinas. Das einzige farbige war ihr schönes, ebenmässiges Gesicht mit den roten Lippen, und die von der Kälte geröteten Wangen. Ihre dunklen Augen waren mit kräftigem Kajal umrahmt und in ihrem Haar schimmerte eine königsblaue Straussenfeder. Sie hatte eine dünne Goldkette umgelegt and der ein Ankhzeichen aus Gold hing.

,,Luise!'' die Stimme war leise, aber gleichzeitig mächtig. Luise fühlte sich, als wäre sie unter einer Glocke, die angeschlagen würde. ,,Wer bist Du?'' Auch Luise hatte mit Berta schon einiges erlebt. Sie hatte gelernt die Dinge bei ihrem Namen zu nennen oder den Namen zu erfragen. Das war ein magisches Gesetz: Fragst Du nach dem Namen, so antworten die Dinge und Wesen. Fragst Du nicht, so halten sie es, wie es ihnen richtig erscheint\dots

,,Ich bin die Ma'at!'' wieder die Glockenstimme. Luise spürte wie die Töne sich durch ihre Haut bis in ihren innersten Kern vibrierten. Ihr Körper sass erstarrt und gleichzeitig im Inneren wie im wildesten Ozean bewegt auf der flauschigen Decke. ,,Du hast die Ordnung gestört!'' tönte es. ,,Du wirst sterben!'' Luise erbleichte. ,,Was für ein Unsinn!'' zischte sie. ,,Du! Oder deine Tochter!'' die Frau stand auf. Sie wurde grösser und älter. Ihr Gesicht war völlig ruhig. Und ihre Worte unerschütterlich! Sie sprach was wahr war, was wahr werden würde. Es war keinerlei Frage!

Luise zitterte und holte tief Luft. Sie blickte kurz auf den Boden, weil sie die Präsenz der Fremden nicht mehr ertrug. Als sie aufschaute war Maat verschwunden\dots Luise stürzte zur Terassentüre. Diese war geschlossen. Sie presste ihr Gesicht an die Scheibe, nichts! Nur Dunkel!

Luise kehrte zum Sofa zurück. Ihr Herz raste. Ein feiner Geruch nach Sandelholz und Honig hing in der Luft, aber er löste sich auf und Luise fragte sich nach einer viertel Stunde, ob sie kurz eingenickt war und nur geträumt hatte. 

Am nächsten Tag jedoch fand sie ein kleines feines Fläumchen. Ein kleines, hauchfeines königsblaues Federfläumchen. Und da Luise nicht viel von Berta gelernt hatte, weil diese ihr das wirkliche magische Wissen verweigert hatte, wusste sie genug, \dots Sie legte das Fläumchen unter ihr Kopfkissen und konzentrierte sich vor dem Einschlafen darauf. Am nächsten Morgen nahm sie das blaue etwas und verbranntes in der Küche in dem alten Ofen. Sie hatte nichts geträumt. Nicht nichts, sondern ein abgrundtiefes, leeres, heulendes Nichts. Es war das Schrecklichste, was sie je geträumt hatte, es war das schlimmste, was sie je gefühlt hatte, obwohl sie schon einiges erlebt hatte. Dieses Gefühl war schlimmer, als alles, was sie je auf dieser Erde erleben konnte. Und Luise beschloss eines: Sie würde nicht sterben! Denn was anderes als der Tod konnte dieses Gefühl gewesen sein? 

Luise hatte einiges überlebt und hatte ihr Herz vor allem verschlossen, was ihr schmerzhaft erschien. Sie würde sich dieser Aufgabe stellen, wenn nötig! 

Und dann eine Nacht später waren Amélie und Berta verschwunden!

Vielleicht, vielleicht würde alles das, was jetzt passiert, nicht passieren, wenn diese merkwürdige Frau nicht zu mir gekommen wäre, dachte Luise. Sie stolperte hinter Seth her durch die dunkle Höhle, tiefer und tiefer\dots gut das Wilfried nicht wusste, dass sie ihre Tochter nicht suchte, um sie zurück nach hause zu bringen\dots

-ja! Das ist gut! denn sonst würde dir dein Mann wohl deinen dünnen Hals umdrehen! Seth drehte sich kurz zu Luise an und zwinkerte ihr zu. Er fuhr leicht mit dem Zeigefinger wie mit einer Klinge über seinen Hals\dots und lächelte. ,,Wir sind da!'' sagte er mit kalter Stimme.

Sie waren in einem Raum angekommen. Luise und Wilfried schauderten. In der Mitte des Raumes lag ein Mann, ein menschliches Wesen am Boden. Es war nackt und die Haut schimmerte im weissen Licht der Taschenlampen weisslich-grün. Der Mann war echt gross und kräftig, es war kein Geräusch zu hören.

-Er atmet nicht! Dachte Luise. Sie warf Wilfried einen Blick zu. Er war kreideweiss im Gesicht. ,,Er blutet!'' flüsterte er und richtete den Strahl seiner Taschenlampe auf den Boden unter dem Fremden. Um ihn und unter ihm, war der Boden rot! Das Blut schimmerte im Licht wie ein Rubin.

,,Mörder!'' zischte Luise zu Seth herber. Der stand lässig da und genoss ihre entsetzten Gesichter. -Nein! Das verstehst du völlig falsch. Erstens ist er nicht tot, zweitens bin ich völlig unschuldig! Seth grinste.

Wilfried war inzwischen zu dem Liegenden gegangen und beugte sich über den Körper. Vorsichtig berhrte er dei Schulter. ,,Es ist kein Mensch! Es ist ein STein!'' rief er. Luise fixierte Seth: ,,Was hat das alles zu bedeuten? Was soll das ganze?'' ZU Wilfried meinte sie: ,,Wilfried wir gehen! Ich weiss nicht, was du von uns willst, aber wir gehen jetzt!''

Wilfried erhob sich und sie strebten auf den Ausgang der Höhle zu! -Stop! Das würde ich nicht tun! Hallte es in ihren Köpfen wieder. Ihr zwei Menschlein würdet nicht mit dem Leben davon kommen! Ihr würdet nicht zu Stein erstarren, wie unser wilder Mann hier! Nien! Ihr zwei werdet krepieren, sobald der Basilisk erscheint!

,,Du widerliche Kreatur!'' kreischte Luise und ging mit den Fäusten auf Seth los. ,,was willst DU denn von uns, Mann?'' rief Wilfried dazwischen. Seth packte Luise am Arm und drehte ihn bis sie sich krümmte.

,,Sehr schlau, dein Mann!'' zischte er. ,,Ihr beide, ihr dürft Euch glücklich schätzen! Denn ihr werdet von dem Blut des Wilden Mann ein kleines Schlückchen trinken und dann werde ich dafür sorgen, dass ihr heil wieder aus der Höhle heraus kommt!''

,,Nein!'' keifte Luise und riss sich los!
In der kurzen Stille, schabte etwas den Gang herauf. ,,Das ist der Basilisk! Er wird sich freuen, Euch zu fressen?'' ,,Wilfried, du Memme, mach' was!'' furh Luise auf ihren Mann los. Wilfried stolperte auf Seth zu. Der Gott packte ihn am Hals und hob ihn in die Luft!

,,Es reicht jetzt! Ihr werdet beide vom Blut des Wilden Mannes trinken, jetzt! Es wird euch zusätzliche Kraft geben und bis auf unseren Freund den Basilisken unverwundbar machen!'' ,,Warum?'' fragte Luise. Sie bewegte sich auf den anderen Ausgang der Höhle zu, das Basiliskengeräusch war verdächtig  nah! ,,Weil mir viel daran gelegen ist, dass du dein liebes Töchterchen wieder in deine Hände bekommst! Und du auch!'' Er gab Wilfried einen Stoss und der Mann fiel kraftlos in das Blut, das sich weiter auf dem Boden ausgebreitet hatte. ,,Mach schon!'' Seth fegte Luise um und sie stolperte neben Wilfried in das Blut.

Seth packte ihre Köpfe und tauchte sie in die Blutlache.Dann riss er sie hoch und zerrte die beiden Menschen aus der Höhle in dem Augenblick, als der Basilisk erschien.

\sterne

Luise und Wilfried erwachten in dem kleinen Schacht vor dem Gerberbrunnen. Sie rappelten sich mühsam auf. Und dann starrten sie beide auf ihre Hände, die nicht nur ihre Hautfarbe wieder hatten, sondern von starken, grünlich schimmernden Adern durchzogen waren\dots

Als sie sich zum Hotel zurückschlichen, sahen sie weder die die Göttin des Chaos, Isfet, die im Schatten des Gerbergässleins stand und sie sahen nicht die runde Alte, die Gegenüber dem Brunnen auf dem Giebel eines Hauses hockte mit einem leise knurrenden Dackel unter dem Arm\dots

\section*{13}
\addcontentsline{toc}{section}{13}

,,Waldi, das sieht nicht gut aus und fühlt sich umso schlimmer an!'' murmelte Berta. ,,Oh, nein!'' Berta schlug sich die Hand vor den Mund, als sie Seth sah, der als letzter durch den Gerberbrunnen aus der Höhle des Basilisken schlüpfte. 

,,Seth! Neffe!'' rief eine helle Stimme und Berta sah Isfet aus dem Schatten treten. Im Gegensatz zu Seth blickte Isfet direkt zu Berta hoch. Diese liess sich schnell den Giebel herabgleiten und schwang sich auf den Besen bevor sie auf den Boden der Falknerstrasse aufschlug. Wie der Blitz raste Berta zum blauen Haus.

Sie landeten zu Waldis Entsetzen in Sobis Ferienteich. Berta schien es nicht zu bemerken und rannte über den Hof zum Eingang der Küche. Waldi schwamm an das Ufer des seichten Wassers und schüttelte sich ausgiebig. Dann folgte er Berta.

Sobald er den Geruch Seths in die Nase bekam, raste Waldi knurrend und Zähnefletschend in die Küche. Er wusste, dass er keine Chance gegen den dunklen Wüstengott hatte, aber er stürzte sich auf dessen dünne Wade und schlug die Zähne hinein als wolle er niemals wieder los lassen.

,,Du verdammtes Vieh!'' Brüllte Seth, der abgelenkt durch durch Berta, Schu und Tefnut, tatsächlich überrascht worden war. Er packte mit seinen gewaltigen Händen den Hund. Doch der Dackel hatte sich festgebissen und Tefnut liess es sich nicht entgehen und fuhr dem Neffen als  Katze mit all ihren Krallen in das blasse, hagere Gesicht.

Erst das wilde und bedrohliche Lachen von Isfet, die nun die Küche betrat, brachte alles sofort zum Stillstand. Seth verharrt beim Anblick seiner Tante, die in ihrer jetzigen Mädchengestalt um einiges jünger Aussah als er.

Waldi liess das Bein los. Blut tropfte aus seinem Maul und sein rechter Hinterlauf hing schlaf herunter. Seth hatte den Hund gegen den Küchenschrank geschlagen. Berta hob ihn sanft auf. Er biss ihr in die Hand, rasend vor Schmerz und Angst um seinen Herrn. Die Göttin verzieh ihm und hielt ihn fest und flüsterte ruhige Worte.

Tefnut hatte die Stille genutzt, um von Seths Kopf auf den Boden zu springen und hatte sich in ihre menschliche Gestalt verwandelt. ,,Isfet!'' knurrte Schu. ,,Warum so unfreundlich?'' antwortete diese. ,,Weil du immer dort bist, wo das Unheil stattfindet!'' zischte Tefnut und zeigte auf Seth. ,,Es ist meine liebste Aufgabe, liebe Tefnut, da zu sein, wo Unheil ist, das gehört zu meinem Job!''

Seth lachte. ,,Tantchen, Tantchen\dots !'' rief er. Doch Isfet blieb ernst. ,;Was hast du mit dem Wilden Mann gemacht?'' ,,Ich? Ich habe nichts mit ihm gemacht!'' antwortete Seth hämisch. ,,Wo ist Hans?'' fragte Berta. ,,Dein Hans zog es vor seine hässliche Nase in anderer Leute Angelegenheiten zu stecken!'' antwortete Seth. ,,Und er ist dabei dummerweise dem Basilisken über den Weg gelaufen! Wenn die Höhle nicht verschlossen wäre, könntest du ihn dir als Schmuck in deinen Garten stellen, alte Frau!'' zischte Seth und beugte sich dicht zu Berta. Ihre Nasen berührten sich fast. 

Berta blieb stumm. etwas an dem, was Seth gesagt hatte, schrie laut in ihrem Gedächtnis, winkte mit der roten Flagge\dots Was hast du mit der Höhle des Basilisken gemacht?'' zischte Berta bleich. ,,Ich? Frag dich lieber, was dein Kumpel gemacht hat, als er sich vom Basilisken versteinern liess!''

Alle waren stumm. 

Bis Seth sagte: ,,Adios Amigos! Schön, wenn ihr euch selbst so im Weg steht, dass ich garnichts mehr tun muss!'' Er lachte und verliess die Küche.

Zurück blieben betretene Gesichter, ein Dackel, der schrie und eine Chaosgöttin, die sich ein Sandwich aus allem machte, was ihr in der grossen Küche Hathors zwischen die Finger kam\dots

\section*{14}
\addcontentsline{toc}{section}{14}

\section*{15}
\addcontentsline{toc}{section}{15}

\section*{16}
\addcontentsline{toc}{section}{16}



