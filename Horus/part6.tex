 \part*{Sechste Stunde\\"`Ankunft, die den rechten Weg gibt"'}
\addcontentsline{toc}{part}{Sechste Stunde}

\chapter*{6. Tag, Jonathanstag}
\addcontentsline{toc}{chapter}{29. Dezember, Jonathanstag}

\section*{1}
\addcontentsline{toc}{section}{1}

Sie starrten sich an. Schliesslich hielt Luise den Blick ihres Mannes nicht mehr aus, den gekränkter-Dackel-Blick. ,,Was ist?'' fragte sie spitz. ,,Warum trägst du Handschuhe?'' fragte Wilfried zurück. Luise sah wie der Dackel im Inneren ihres Mannes  plötzlich die Zähne fletschte und sie anknurrte. ,,Weil ich es halt so will!'' Sie drehte sich um und beugte sich über das grosse Bett in dem der Junge schlief. 

,,Aufwachen, Goldkind!'' sagte sie und strich dem Buben sanft über das blonde Haar. Er hatte viel vom Aussehen seiner Mutter. Die blonden, feinen Haare, den feingliedrigen Knochenbau. Die langen dünnen Gliedmassen und Finger. Amélie hat mehr von mir, dachte Wilfried. Die schwarzen, kräftigen Haare und den sehnigen, muskulösen Körper\dots Wo war sie, seine Tochter? 

Sobald Wilfried an Amélie dachte, hatte er das zuckende, blutige Herz vor Augen, das er im Traum in den Tonkrug geworfen hatte. Er zog langsam die linke Hand ein Stück aus der Hosentasche, weit genug, bis er die rostrote Hautfärbung sehen konnte.

,,Luise, wir\dots '' begann er. ,,Wir haben nicht viel Zeit, wir müssen Amélie suchen!'' antwortete sie schnell. Sie drehte Wilfried den Rücken zu und tat als würde sie dem Jungen beim Anziehen helfen. 

Wilfried ging zur Tür: ,,Gut! Ich warte unten im Frühstücksraum auf euch!'' Er verliess das Zimmer. Sobald er fort war, eilte Luise in das Badezimmer und zog sich die Handschuhe aus. Ihr Spiegelbild reckte ihr vorwurfsvoll zwei blutrot befleckte Hände entgegen. Sie versuchte sich die Hände sauber zu waschen, aber es gelang ihr auch diesmal nicht. 

,,Mama?'' der Junge klopfte an die Badezimmertür. ,,Ich muss mal!'' Luise öffnete ihm die Tür. ,,Warum hast du die Handschuhe schon an?'' fragte der Kleine, als er sich auf das WC setzte. ,,Damit wir Amélie schneller suchen können!'' antwortete Luise. Als sie das Zimmer verliessen und zum Frühstücksraum gingen, zog sich der Junge seine dicken Fäustlinge über und stapfte grimmig vor seiner Mutter her. Eine kleine Träne rollte über Luises Wange und streifte ein winziges Lächeln.

\section*{2}
\addcontentsline{toc}{section}{2}






 Es war stockdunkel. Durch den dichten Wald blitzten die Scheinwerfer eines Autos auf. Die Räder des alten Leichenwagens quitschten, als er die engen Kurven ins Tal hinunter brauste. Aus den Lautsprechern rumpelten rockmusische Bässe und schoben ihrerseits den Wagen vorwärts. 

Tef jaulte und Hapi kreischte. Amset brüllte den Text mit und riss im Takt das Lenkrad hin und her in den scharfen Kurven der Serpentinen. Dann rasten sie auf dem Kamm der Berge ins Tal.

Sobi sagte nichts. Er rumpelte im hinteren Teil des Wagens hin und her, nachdem er es aufgegeben hatte, sich mit dem kräftigen Schwanz und seinen vier Tatzen an den Wänden abzustützen. Ihm war nicht direkt schlecht, denn Krokodile sind zu gierig, um ihre Beute auszukotzen. Vielleicht schimmerte seine grünliche Haut um die Schnauze grüner als sonst. Sobi sagte nichts, was sollte er sagen? Er hoffte, sie würden trotz Amsets Gefühlschaos heil unten ankommen. Amélie war noch in dem Merowingergrab und keiner von ihnen wusste zu sagen, in welchem Zustand sie sich befand. Lebte sie? War sie bei Verstand?

Als sie an der Kreuzung auf die Route de Champs de Feu trafen, bogen sie ab Richtung Klingenthal. Sobald sie am Waldrand eine geschützte Stelle ausgemacht hatten, bremste Amset den Wagen. 

Sobis Schnauze lag nun auf dem Amaturenbrett, Hapi, der noch vor wenigen Sekunden dort gesessen hatte, hing an dem Lederbezug des Wagendaches und schrie.

Tef hatte sich im Fussraum des Beifahrersitzes zu einer Kugel zusammengerollt und schielte über seinen Schweif. Sobis Vordertatze ruderte hilflos in der Luft und verfehlt immer wieder knapp Tefs Nase. 

Amset war schon aus dem Wagen gesprungen und öffnete die Heckklappe. Er zog energisch am Schwanz des Reptils, was nicht viel nützte.

Vorsichtig schob sich Sobi rückwärts aus dem Wagen. Er seufzte glücklich, als er die Ehn sah. Schnell wie der Blitz war der Krokogott in den kleinen Fluss geklettert und lies sich von der Strömung Richtung Strassburg weisen.

Der kleine Fluss war nicht überall tief genug, aber Sobi machte es nicht viel aus ein Stück im flachen Wasser zu waten. Er grunzte und stapfte fröhlich los.

Ein kleines Hüngerchen machte sich bemerkbar und Sobi hielt freudig Ausschau nach den kulinarischen Leckerbissen, die ihm auf seiner Flusswanderung begegnen würden. 

\sterne

An diesem Morgen gingen bei der Polizei Bas rhin und Haute rhin mehrere Meldungen ein. Entlang des kleinen Flusses Ehn und des grossen Flusses Ill, der durch Strassburg floss, wurden zahlreiche Hunde vermisst, die am Ufer dieser Flüsse mit ihren Besitzern spazieren gegangen waren. An vielen Orten vermissten die Spaziergänger die zahlreichen Schwäne. Ein, zwei Rehe verschwanden, aber das fiel keinem Menschen auf\dots 

Ein kleines Mädchen, das mit seinen Eltern einen Spaziergang gemacht hatte und am Ufer des Flüsschens Ehn Steine in das Wasser geworfen hatte, erzählte seinen eigenen Kindern später folgendes Geschichte. 

Sie war felsenfest davon überzeugt ein riesiges Krokodil im Fluss gesehen zu haben. Das Krokodil war auf sie zugeschwommen und hatte ihr in die Augen geschaut. Sie hatte sich nicht bewegen können, ihr Herz schlug schmerzhaft gegen den Brustkorb, als wollte es raus springen. Das Krokodil flüsterte: ,,Lecker!'' Das Mädchen schloss die Augen und wartete darauf von den grossen Zähnen gepackt und unter Wasser gezogen zu werden. Sie hörte ihre Knochen brechen.

Als der Schmerz aus blieb, hatte die Augen geöffnet. Das Maul des Krokodils war direkt neben ihr, aber es hatte nicht ihr Bein gepackt, sondern einen dicken Stock- der war es, der unter dem Gebiss des Krokodils krachte.
Das Krokodil schielte mit dem einen Auge auf das Mädchen, eine Träne lief über seine Schnauze. Es krachte ein letztes mal, als es den Ast endgültig zermalmte. Holzsplitter flogen in alle Richtungen.

,,Susanne?'' hatte die Stimme der Mutter vom Weg ausgerufen. ,,Susanne, was machst Du?'' Eilige Schritte hatten sich auf dem Kies genähert, es hatte zwischen den Büschen am Uferrand geraschelt. Die Mutter hatte sich auf Susanne gestürzt und sie am Arm zu sich herangezogen. ,,Ich habe dir doch gesagt, du sollst nicht so dicht an das Wasser gehen!''

,,Maman! Schau, das Krokodil!''
Doch da war kein Krokodil! ,,Susanne!'' der Stiefvater war auf das sandige Ufer getreten. ,,Erzählst du wieder Lügengeschichten?'' Er hatte hämisch gelächelt und sich die Lippen geleckt. Susanne hatte eine Gänsehaut bekommen. Sie hatte nicht genau gewusst, warum, aber  sie mochte den neuen Mann ihrer Mutter nicht. Seit er im Haus war, hatte sie sich vor Raubtieren gefürchtet. Sie hatte sogar Angst vor ihrer Rotkäppchen-CD bekommen. 

Susanne hatte verlegen auf den Sand vor ihren Gummistiefeln geschaut. Und da in dem Sand hatte ein weisser, glänzender Gegenstand gelegen. Sie hatte sich gebückte und ihn schnell hoch gehoben. Als sie die Hand geöffnet hatte, befand sich darin ein Zahn. Ein grosser Zahn!

Susanne hatte ihrer Mutter fest in die Augen gesehen: ,,Siehst du! Ich lüge nicht! Auch wenn er das immer sagt!'' Sie hatte auf den Stiefvater gezeigt. Der war unter dem Blick von Mutter und Tochter blass geworden. Er hatte sich nervös über die Lippen geleckt. Susanne hatte gespürt, wie ihre Mutter zusammengezuckt war und war zufrieden gewesen. Sie hatte gespürt, wie sich der Blick der Mutter auf den Stiefvater plötzlich gelichtet hatte.
Im Aquarium les Naiades, das ganz in der Nähe lag und das Susanne und ihre Eltern oft am Wochenende besucht hatten, hatten Susanne und Ihre Mutter den Tierarzt gefragt von welchem Tier der Zahn stamme, den Susanne aufgehoben hatte. Der Tierarzt bestätigte Susanne, dass es ein Krokodilszahn war, allerdings wollte er die Geschichte wie Susanne ihn bekommen hatte nicht glauben. 

\sterne

Wie ging Susannes Leben weiter? Kurz nach der Begegnung mit Sobi, gerieten die Mutter und Stiefvater immer öfter in Streit und trennten sich schliesslich. Susanne war froh, als er weg war. Obwohl sie nach dem Erlebnis am Fluss keine Angst mehr hatte, vor wilden, gefährlichen Tieren nicht und auch nicht mehr vor dem Stiefvater.

Susanne wurde später eine berühmte Tierforscherin. Sie untersuchte viele unglaubliche Phänomene, ob sie wahr waren, oder nur ein Gerücht. Den Zahn liess sie sich von einem Goldschmied zu einem Kettenanhänger machen und trug ihn den Rest ihres Lebens.

Susanne traf einen anderen Forscher, der ihr sehr gefiel. Sie heiratet ihn und zusammen bekamen sie Zwillinge, einen Jungen und ein Mädchen.

Wenn Susanne vor schwierigen Entscheidungen stand, dann konnte es passieren, dass sie von einer merkwürdigen Gestalt träumte: die einen männlichen, menschlichen Körper und den Kopf eine Krokodils hatte\dots



\section*{3}
\addcontentsline{toc}{section}{3}

Amélie schlug die Augen auf. Dann schlug sie die Augen wieder auf. Es blieb stockdunkel. Amélie hob den Kopf und rammte ihn gegen Stein. Benommen fiel sie zurück. 

Als die Sterne vor ihren Augen aufhörten zu tanzen, streckte sie vorsichtig die Hände vor sich aus. Sie fühlte die eiskalte Steinplatte. Der Stein, der um sie herum war, strömte den Geruch von modrigem Laub und Erde aus.

Mit einem Schlag wurde ihr die Stille bewusst. Und doch\dots Es klopfte. Es rauschte. 

Für einen Moment traten die Bilder der Nacht vor sie. Amélie konnte sich nicht rühren, während die Bilder ihrer Ermordung sich vor ihr ausbreiteten. Sie wollte schreien, aber ihre Kehle gab keinen Laut von sich. Das Rauschen und Klopfen, wurden nun lauter und lauter. Es rauschte, es klopfte, es schabte: Amélie japste nach Luft, -ich ersticke!

Amélie schrie!

,,Psssst! Nun sei halt nicht so laut! Du weckst ja alle auf!'' Amélie hob geblendet den Arm vor ihr Gesicht. Obwohl es noch dunkel war, waren ihre Augen an die absolute Dunkelheit im Grab gewöhnt. Das Licht, das die Wolkendecke von den umliegenden Städten zurückwarf, langte aus, um sie zu blenden.

,,Berta!'' Amélie reckte die Arme und Berta zog sie aus dem Grab. Sie taumelte und Isfet fing sie auf. ,,Ja, hallo, meine schöne Chaosbraut!'' Begrüsste sie Amélie. Doch die fiel ihr um den Hals und schluchzte.

Berta ächzte und hob die Grabplatte ab und legte sie an die alte Stelle zurück. Sie klopfte Amélie sachte auf den Rücken. ,,Dein Schatz kommt!'' flüsterte Isfet. Amélie versteifte sich sofort und löste sich von Isfet. Sie wischte sich mit dem Jackenärmel trotzig über da Gesicht.

Sie kam nicht dazu zickig zu sein, den Amset war im vollen Lauf über das Geländer gesprungen und riss Amélie von den Füssen. Er schluchzte. Und zerquetschte sie fast.

Tef und die Brüder schnauften hinterher. Hapi kletterte an Isfet hoch und setzte sich auf ihre Schulter. Seine Zähne, die recht beeindruckend waren, klapperten in der Kälte. Berta hatte den Arm ausgestreckt damit Kebi einen guten Landeplatz hatte. der Falke rief ein letztes mal und sortierte seine Federn, bevor er sich aufplusterte.

Tef schleckte mit der Zunge Amélies schlaffe Hand.

,,Amélie, ich\dots '' Amélie löste sich vorsichtig von Amset und schaute ihm ins Gesicht. Ein unendlich Schmerz breitete sich in ihrer Brust aus. Die Tränen liefen, es war ihr gleich. Aber Amsets Schmerz zu sehen, war um vieles schlimmer, als die Schrecken, die sie in dieser Nacht erlebt hatte.

Amset legte seine Stirn an ihre. ,,Nicht weinen, Amsi! Alles ist gut, ich bin ja da! Es geht mir gut!'' Sie sah seinen Zweifel und strich ihm über die Wange. Von seinen Augen, die im Dunklen glühten, angezogen, hauchte Amélie Amset einen Kuss auf den Mund\dots

Dann befreite sie sich aus Amsets Umarmung und sagte: ,,Ich habe Hunger!'' Amélie kletterte über die Abgrenzung auf den Platz und machte sich auf den Weg zum Speisesaal. Tef lief ihr hinterher. Als er sie eingeholt hatte, legte sie ihm den Hand auf den Rücken.

\sterne

Amset blieb vom Donner gerührt stehen. 

\sterne

Isfet und Berta folgten Amélie ins Kloster.

\sterne

Kebi schüttelte den Kopf und das Gefieder. Er hatte genauso grossen Hunger wie Amélie und beschloss, dass es Zeit für eine winterfette Morgenmaus wäre.

Er hüpfte flatternd auf die Umfriedungsmauer. Er genoss den weiten Blick, über die Rheinebene. Er breitete seine Flügel aus und liess sich in den Morgendunst fallen. Bevor seine Füsse die Wipfel der Fichten streiften, schlug er mit den Flügeln und steuerte elegant in den Wald um die Odilienquelle.

Kebi schlüpfte durch das Gitter zu Becken der Quelle. Er nahm einige Schlucke von dem frischen, eiskalten Wasser, das in einem kleinen Rinnsal beständig aus dem Berg lief. Vorsichtig rieb er den Kopf und die geschlossenen Augen an dem quellenfeuchten, bemoosten Stein. 

Er kletterte zurück und schwang sich in die Luft. Als er auf die offene Matte der Ruine des Klosters Niedermünster kam, entdeckte er zu seinem Vergnügen, sofort eine dicke Maus.

Während Kebi auf der Wiese sass, beschloss er, dem Flüsschen Ehn zu folgen.

\sterne

Nachdem sich die anderen auf dem Weg zum Speisesaal gemacht hatten blieben Hapi bei Amset. Er kletterte auf dessen Schultern, ohne das sein Bruder reagiert hätte. Hapi winkte mit seiner Pfote vor Amsets Gesicht auf und ab\dots nichts.
Da zuckte Hapi mit den Schultern und eilte den anderen hinterher. Schliesslich hatte er auch eine lange, kalte Nacht hinter sich.

\section*{4}
\addcontentsline{toc}{section}{4}

Auf dem Odilienberg wurde gefrühstückt. Die Mannschaft hatte tüchtig Hunger. Die Götter hatten auf den Kakao verzichtet (Meidli-Chichi) und hatten sich Kaffee, Tee und warmes Honigbier bestellt.\footnote{Der Spruch, der die ,Schnäbelfressigkeit' der Bauern aufs Korn nimmt, gilt auch für Götter. Dieser lautet dann: ,Was der Gott nicht kennt, das trinkt er nicht!' Denn, was dem Bauer seine Mahlzeit, ist dem Gott sein Trank. -Ausser, es hat Alkohol darin. Schliesslich ist aus der Duat ein berühmter Trinkspruch bis in die heutige Zeit mündlich überliefert: ,,Bier her, Bier her, oder ich fall' um!'' Was damals wörtlich zu verstehen war, wenn er von einer feiernden Mumie zum Besten gegeben wurde. } 

Sie alle hatten sich in die Nähe des Kamins gesetzt und assen genüsslich frische Croissants mit Erdbeermarmeladenfüllung. Es knusperte und schmatze zufrieden. Selbst Isis, die von Odilia eine Ordenstracht bekommen hatte, wollene, graue Strumpfhosen und ein paar Stiefel wie sie die Ordensschwestern trugen, wenn sie um diese Zeit den Schnee schippten, sass entspannt in einem Sessel und schaute in die Flammen. Hin und wieder fielen ihr die Augen zu. Die kleine goldene Schlange, die um ihr Haupt gewunden war, reckte ihr Köpfchen empor und wachte.

Berta, Hathor und Odilia tauschten sich über die verschiedenen Kulte aus, die sie erlebt und gefeiert hatten. Hathor war sehr interessiert, was Odilia über ihren Gottesdienst berichtete. Sie hatte ihre Rituale von irischen Mönchen gelernt. Berta bemerkte Parallelen zu den Kulten, die sie kannte. Sie war am weitesten von allen herumgekommen und hatte an verschiedenen Kulten teilgenommen\footnote{Es ist erstaunlich wie viele Kulte und Rituale in den vergangenen Zeiten von kleinen, kugelförmigen Frauen handelten.}. 

Als Amélie den Speisesaal betrat wurde sie von einem Gott zum nächsten weitergereicht. Jeder wollte sie umarmen. Wibrandis hatte dafür ein wunderbares Frühstück gezaubert. Rührei und Speck, genau das richtige, wenn man die Nacht in einem Merowingergrab verbringen musste.

Nachdem sie die Hälfte ihres Frühstücks verschlungen hatte, blickte Amélie auf. Thot sah ihr von seinem Platz am Kaminfeuer aus zu. ,,Gut gemacht! Wie geht es dir?'' Amélie überlegte. Sie war die ganze nacht bei Minusgraden in einem steinernen Grab in einem Berg in Frankreich gelegen\dots Dafür, fand sie, fühlte sie sich verblüffend gut.

Dann spürte sie wie sich die Erinnerung näherte. Die Erinnerung an den Mord, bei dem ihr das schlagende Herz aus dem Leib gerissen worden und mit seinen letzten Schlägen in die Kanope von Tef geworfen worden war.

Amélie schüttelte sich, sie wollte die Bilder los werden, wegschieben. Sie schloss die Augen, aber da fühlte sie eine knochige, kräftige Hand auf der Schulter. Es war Thots Hand. ,,Du musst dich erinnern, Amélie! Sonst war die Nacht im Grab umsonst!''

\section*{5}
\addcontentsline{toc}{section}{5}

Währenddessen stand Luise vor dem Spiegel. Sie hatte den Jungen bei Wilfried gelassen und behauptet sie hätte keinen Hunger. Sie starrte auf die silbrige Fläche.

Eine Zeit lang, als sie jung war, hatte Berta ihr einige Tricks gezeigt, wie sie es nannte. Dann als das Schreckliche passiert war, hatte Berta damit aufgehört. Die törichte alte Frau hatte ihr die Hilfe verwehrt, als sie sie am dringendsten gebraucht hätte. Aber Berta hatte behauptet, Luise würde die Tricks für ihre Rache nutzen und hat sich geweigert Luise weiter auszubilden.

Luise hatte getobt. Sie hatte gebittelt, gebettelt und hatte sie Berta verflucht. Es half ihr alles nichts. Denn natürlich hatte Berta recht, dachte Luise bitter, als sie sich im Spiegle betrachtete.

Sie hätte ihre Seele verkauft um an Bertas richtige Tricks zu kommen. Stattdessen hatte sich Berta in den letzten Jahren, seit der Bub geboren worden war, ihrer Tochter zugewandt. 

-vermutlich hat sie all ihr Wissen an Amélie weitergegeben, dachte Luise. Zu ihrem erstaunen, spürte sie einen Stich in der Brust. Sie hatte geglaubt, es würde ihr nichts mehr ausmachen. Luise biss sich auf die Lippe. Sie wollte nicht heulen, sie wollte Berta nicht vermissen und Amélie erst recht nicht.

Solange Amélie und Berta da waren, würde sie sich nie sicher fühlen. Luise fühlte sich nur sicher, wenn sie alles und jeden um sich kontrollieren konnte. Das war bei Wilfried keine Frage, er war ihr ergeben, er würde alles für sie tun! Und der Bub? Der war viel zu klein und zu schwach, um sich gegen seine Mutter durchzusetzten. Es gab keinen Grund für ihn. Lästig war nur, dass der Kleine seine Schwester tatsächlich mochte.

-Was bin ich für ein Biest! dachte Luise und blickte sich mit halb geschlossenen Augen und schiefgelegtem Kopf an. Es war so einfach Männer im Griff zu haben. Sie verzog unwillkürlich angeekelt ihren Mund.

Luise sah auf die weissen Wollhandschuhe, sie bemerkte rote Flecken, die von ihren blutig geschrubbten Fingern stammten. Sie ballte die Faust, wo war dieses missratene Kind und diese blöde Amme?

Berta kannte Luise zu gut. Viel zu gut. Sie war die einzige, die begriffen hatte, wie eisig kalt es in Luise wirklich war. die einzige, die wusste, zu was Luise fähig war\dots

Und Amélie? Amélie war ein naives Huhn. Aber sie war fast erwachsen, also jung und schön. Es würde nicht mehr lange dauern, bis sie begreifen würde, was das für eine junge Frau bedeuten konnte. Luise wollte keine Konkurrentin, auch nicht, wenn diese ihr eigene Tochter war.

Luise wollte nie wieder ausgeliefert sein. Und sie wusste, als Frau brauchte sie dazu einen Beschützer. Einen erträglichen Beschützer, der dumm genug war, für ein geheucheltes Lächeln, alles zu geben. Und der stark genug war, sie zu beschützen.

\section*{6}
\addcontentsline{toc}{section}{6}



Amélie wanderte mit Odilia an der Heidenmauer entlang. Die Luft war frostig. Ihr Atem bildete Wolken vor ihren kalten Gesichtern, die, angeregt von ihrem Gespräch und der Bewegung, erhitzt leuchteten.

,,Du bist sehr mutig, Amélie!'' sagte Odilia, die auf dem schmalen Trampelpfad, der zwischen der Kannte des Felsen und der Mauer aus gewaltigen Sandsteinen entlangführte vorausging. Sie blieb stehen und wendete sich zu Amélie um. Sie schauten durch die Fichten auf das ausgebreitete Tal.

,,Ich habe es mir nicht ausgesucht!'' antwortete Amélie. ,,Verstehe!'' sagte Odilia. Und sie verstand es wirklich.
Odilia nahm Amélies Hand und plötzlich löste sich etwas in Amélies Brust. Die Traurigkeit von der sie sich bis jetzt hatte ablenken können, überschwemmt sie. Amélie schluchzte und dann schrie sie. Die Vögel, die leise im Unterholz gesessen hatten, flogen auf. 

Odilia legte sanft ihren Arm um Amélies Schulter und hielt sie fest. bis sie fertig geschrien hatte. Dann langte sie in die falten ihres wollenen Umhanges und zog ein Stofftuch heraus. Sie reichte es Amélie. Sie schnäuzte sich ausgiebig.

,,Sie sind meine Eltern!'' Amélie sprach leise und ruhig. ,,Ich habe schreckliche Dinge gesehen, die Menschen mit mir gemacht haben und das waren meine Eltern! Auch wenn wir alle anders ausgesehen haben. ich weiss, sie waren es!'' ,,Als ich geboren wurde, war ich blind. Meine Vater Etticho, der aus der heidnischen Tradition stammt, sah darin ein Beweis für seine Schande. er verlangte von meiner Mutter mich zu töten. Meine Mutter Bereswinda liess mich heimlich zu ihrer Amme bringen und diese zog mich auf, bis ich alt genug wurde, um im Kloster Baume-les-Dames meine Ausbildung zu beginnen. Ich musste einige Male fliehen und um mein Leben fürchten, bevor ich meine Aufgabe, die Kloster hier auf diesem heiligen Grund zu errichten, ausführen konnte.''

,,Wie hast du das geschafft?'' fragte Amélie. ,,Wie hast du das gemacht, ganz allein?'' Amélie spürte wie ihre Kehle sich wieder zu schnürte. Sie konnte den grossen Kloss der Beklommenheit und Angst nicht schlucken.

,,Indem ich begriff, ich bin nie allein!'' antwortete Odilia. ,,Aber ich bin nicht so religiös wie du! Und ich hab auch nicht vor es zu werden!'' rief Amélie. ,,Wie meinst du das?'' fragte odilia verwirrt. ,,Du glaubst halt an Jesus und so und betest und bist sogar eine Heilige!'' antwortete Amélie.

Odilia drehte Amélie zu sich um und sah sie mit ihren hellen, fast weissen Augen an. ,,Sag', du denkst, weil ich glaube und bete, denke ich, ich wäre nicht einsam?'' ,,Genau!'' Amélie hielt Odilias Blick nicht stand, sie blickte mit verschrenkten Armen zu Boden.

Odilia starrte Amélie an. Sie lächelte. ,,Ich find es nicht lustig!'' rief Amélie. ,,Ich schon!'' Odilia machte eine weite Geste über die Rheinebene, die in der Sonne schimmerte. ,,Schau dich um, Kind! Sieh hin, mach die Augen auf! Du kannst nicht alleine sein! Das ist unmöglich!''

Widerwillig gehorchte Amélie und schaute ebenfalls in die Weite. Odilia legte ihr die hand leicht auf die Schulter. Ihr Atem glitt zart in die kalte Luft. Ein kleiner Vogel zwitscherte sein Lied, emsig. Der Duft der Erde war scharf gewürzt durch den Frost. Es Knackte. Die dunkle rotbraune Erde und die grossen, rostroten Steine mit dem grünen Moos lagen still da. Aus der grünen, mattgrünen Ebene schallte ein leises, beständiges Summen herauf. Autos, winzig wie Ameisen, folgten den dunklen Strassenbändern.

Amélie spürte wie sich die Angst auflöste, je mehr sie ihren Körper und ihre Umgebung zu spüren bereit war. Doch sobald sie sich wieder spürte, tauchte die Traurigkeit auf und sie schlang die Arme um sich. ,,Aber es macht weh!'' flüsterte sie. Odilia drückte ihre Schulter stärker. ,,Ich weiss! Amélie! Du wirst den Schmerz aushalten lernen! Weisst du auch warum?'' Amélie schaute Odilia fragend an. ,,Weil du menschlich bist! Wenn du aufhörst den Schmerz zu fühlen, dann, dann erst, wirst du wirklich allein sein, in deiner eigenen Hölle!''

Amélie zog unwirsch die Schulter hoch. ,,Benenne die Dinge als das, was sie sind. Du bist Schmerz, denn du hast ein liebgewonnenes Wissen als Enttäuschung entlarvt. Du hast auch nicht deine Eltern verloren, sondern den Glauben, sie müssten gut zu dir sein. Aber deshalb bist du nicht allein. Sondern du musst den Blick in eine andere Richtung lenken und deinen Weg neu ausrichten!'' Odilia schwieg und auch die Vögel und die Bäume rings um sie schienen erwartungsvoll zu schweigen.

Es dauerte eine Weile, doch Amélie merkte wie Odilias Worte in ihr wirkten. ,,Weisst du, was ich wirklich seltsam finde?'' fragte Odilia, nachdem Amélie sie schüchtern angelächelt hatte. ,,Was?'' ,,Wie es dir nach all dem, was dir passiert ist, möglich ist, nicht an Götter zu glauben!'' Amélie schaute Odilia überrascht an.

Ihr fröhliches Gelächter, das sie schüttelte bis sie das Kloster erreicht hatten, drang hinauf bis zu Kebi, der auf dem Rückflug von seiner Frühstückssuche war. Er antwortete ihnen mit zufriedenem Falkenruf.

\section*{7}
\addcontentsline{toc}{section}{7}

Luise und Wilfried schlenderten missmutig hinter dem Kleinen her durch die Strassen. Sie hatten das Hotel klug gewählt, denn es war Teil der Altstadt. Sie waren durch die engen Gässchen runter gewandert. Schliesslich blieben sie auf dem Barfüsserplatz vor der Museumskirche stehen. 

,,Und nun?'' fragte Wilfried. ,,Wie sollen wir in dieser riesigen Stadt unsere Tochter und Berta finden?'' Luise schenkte ihm einen abfälligen Blick und liess ihren Blick dann weiter über den Platz und das Gedränge um die Tramstation schweifen.Sie war lange genug bei Berta in die Schule gegangen, um ein untrügliches Gespür zu entwickeln, wann sie wo zur rechten zeit an einen rechten Ort geraten war. Und im Moment hatte sie das Gefühl, sie müsste einzig und allein warten. 

Sie hielt weiter Ausschau, ohne zu wissen wonach und steckte Wilfried an. Gleichzeitig entdeckten sie einen Mann, der sich aus der Menge löste und in ihre Richtung schlenderte. Er war sehr blass, was durch die schwarzen Kleider (Jeans und T-Shirt) und die ascheschwarzen Haare verstärkte. Sein Gesicht war schmal, die Augen schienen blassgelb. Die Nase, schlank und lang krümmte sich leicht. Zwischen den Lippen, die wie ein Strich in das kräftige, spitze Kinn geschnitten schienen, ragte eine Zigarette.

Luise blinzelte. Plötzlich fröstelte sie, obwohl der Platz von der Wintersonne sanft gewärmt worden war. Ein leichter, kalter Schatten schien sich vor den Schritten des Fremden auf sie zu zu bewegen. Luise schaute hoch an den Himmel, der jedoch kein Wölkchen zeigte. Auch Wilfried schien es zu spüren, denn er schaute sich ratlos um. Nur der Kleine, der an der Kirchenmauer sein Spielzeugauto entlang fahren liess, schien nichts zu spüren. Luise schauderte. Die Tramstation wirkte wie durch eine Seifenblase gesehen, verschmiert, undeutlich\dots

In schlichten, schwarze Boots stieg der Mann die Treppe hinauf und hielt direkt auf Luise und Wilfried zu. ,,Hallo!'' sagte er. ,,Ich bin Seth!'' ,,Hallo, Seth!'' säuselte Luise und reichte ihm ihre behandschuhte Hand. Er nahm sie galant und hauchte einen Luftkuss darauf. Luise lief ein kalter Schauer über den Rücken. Sie konnte sich nicht erinnern, wann sie sich das letzte mal so gefürchtet hatte. Die Kälte und Starre, die Seth ausstrahlte, presste ihr die Luft zum Atmen aus der Brust.

Wilfried stand einige Sekunden wie vom Blitz getroffen. Er wurde leichenblass und alles, was an ihm lebendig geblieben war, floss aus ihm heraus. Den letzten Rest an Wärme verlor er, als auch er Seth die Hand reichte. ,,So ist es recht, Wilfried!'' sagte Seth. Der schmale Mund grinste, lächeln konnte er nicht.

Der kleine hatte aus einiger Entfernung den Fremden, der seine Eltern begrüsste, beobachtet. Auch er stand erstarrt. ,,Der Junior?'' fragte Seth und wendete sich dem Kind zu. Doch Luise war schneller, eilte zu dem Jungen und nahm ihn auf den Arm. Sie blitzte Seth aus den Augen an, was dieser mit einem amüsierten Blick erwiderte.

Wilfried erwachte aus der Erstarrung und legte schnell seinen Arm um Luises Schulter, er spürte seine Verwunderung, als sie seine Hand nicht sogleich abschüttelte.

Seth lachte. ,,Wir haben das gleiche ziel!'' sagte er dann. ,,Welches Ziel haben wir denn?'' fragte Luise. ,,Ihr wollt eure Göre wieder zurück, die sich gerade bei meinem lieben Brüderchen und meiner Sippschaft einschleimt zusammen mit der alten Vettel und ich will, dass sie wieder verschwinden!'' Luises Augen blitzten auf und Wilfried bekam einen Hauch Farbe. Seth nickte zufrieden.

,,Und wie genau willst du das machen?'' fragte Luise. ,,Die alte Vettel, wie du sie nennst, sollte man nicht unterschätzen!'' Ein Schatten huschte über Seths Gesicht und er langte instinktiv an seine Augenbraue, die von einer feinen Narbe unterbrochen wurde. ,,Ich weiss'', sagte der Kriegsgott. ,,deshalb werden wir uns zusammen tun!'' ,,Ach?! Sicher?!'' Luise hob arrogant die Augenbraue. 

Wilfried, dessen Gespür viel feiner war, als das seiner Frau, zuckte. Er wollte sich zwischen den Fremden und Luise stellen, doch es war zu spät. Seth packte Luises Kinn und flüsterte:,, Ganz sicher. Im Gegensatz zu deiner Tochter, ist dir dieser kleine Hosenscheisser ja nicht egal!'' Luise wurde bleich. Der Kleine, eingeklemmt zwischen seiner Mutter und Seth begann zu schluchzen.

Wilfried fand aus seiner Starre und schob sich sachte, aber bestimmt zwischen den den Gott und Luise. ,, Was willst du?'' fragte er. Seth, der ein Kopf grösser war als Wilfried, beugte sich zu ihm bis ihre Nasen sich berührten. ,,So gefallt ihr mir schon besser! Wir treffen uns heute um Mitternacht am Gerberberglein. Wir werden einen alten Freund von mir besuchen. Euch menschlichen würde ich raten Gummistiefel und eine spiegelndes Sonnenbrille mitzunehmen.''

,,Und der Sohn?'' fragte Luise. Seth blickte sie kalt an. ,,Mein  Freund hat sicher nichts gegen einen kleinen Appetithappen!'' Seth machte kehrt und war verschwunden ehe Luise und Wilfried durchatmen konnten. Sie standen aneinander gedrängt auf dem Platz und versuchten das zitternde Kind zu trösten.

Als würde sich eine Wolke von der Sonne wegbewegen, wurde der Platz wieder heller und wärmer. ,,Wer zum Teufel ist er?'' fragte Luise. Sie schaute in Wilfrieds Gesicht. Es war totenbleich und seine Augen starr. Ein Ruck lief durch den Mann und er antwortete leise: ,,Es ist der Teufel! Es ist Seth!'' Luise spürte seit langer Zeit zum ersten mal wieder Zuneigung zu ihrem Mann. Sie wollte Wilfried zu trösten und griff zart nach seiner Hand. Doch er riss erschrocken die Hand zurück. Betroffen rückten sie von einander ab. Der Junge zappelte unruhig und Luise stellte ihn ab.

Schweigend besorgten sie sich Gummistiefel, Trekkingkleidung und spiegelnde Sonnenbrillen. Schliesslich fragten sie im Hotel nach und es fand sich eine Auszubildende, die mit Erlaubnis des Chefs als Babysitterin einspringen sollte.


\section*{8}
\addcontentsline{toc}{section}{8}


Auf dem Parkplatz vor dem Kloster herrschte Hochbetrieb, die Göttinnen scheuchten die Ordensschwestern, die Odilia geheissen hatte das Gepäck der Gäste zu bringen, hin und her.

Obwohl Sobi den fuss- und Schwimmweg eingeschlagen hatte, schien der Platz im Kofferraum geschrumpft zu sein. Die lag an den riesigen Proviantkörben, die die Schwestern mit  vereinten Kräften in jeden Zwischenraum zwängten. 

Amélie stand fror neben dem schwarzen Wagen. Sie wollte sich von Odilia verabschieden. Doch die Äbtissin war nicht zu sehen. Amélie war enttäuscht, war es Odilia egal, was aus ihnen wurde?

Da sah sie die Klosterfrau in ihren dicken, grauen Wollumhang gehüllt durch das Tor eilen, sie trug eine kleine lederne Tasche. ,,Odilia!'' Horus öffnete ihr die Beifahrertür zum Leichenwagen. ,,Du kannst auf meinem Platz sitzen, ich will mir die Flügel lüften und mit Kebi den Luftraum ausspähen!'' Odilia nickte und liess sich von dem kräftigen Gott die Tür halten, als sie zu Thot in den Wagen stieg.

,,Odilia Kommt mit?'' rief Amélie überrascht. ,,Freust du dich?'' fragte Berta. Amélie blickte Berta durch den Rückspiegel ins Gesicht, war sie traurig deswegen? Doch Berta grinste zufrieden, trotz ihres Stumpens im Mund. ,,Ja!'' antwortete Amélie da. ,,Odilia hat mir von ihrem Vater erzählt!'' ,,Oh, ja!''seufzte Berta. ,,Verstehe! Geteiltes Leid ist halbes Leid!''

,,Odilia wird uns heute Nacht begleiten. Sie wird die Göttin, die ihr Bild verbirgt, vertreten.'' sagte Isis. ,,Die Göttin die alles verbirgt?'' fragte Amélie. ,,Genau! Die hilft Re sich mit seinen Augen vollständig zu verbinden.'' ,, Und das macht die Göttin, die  verbirgt?'' ,,Genau!'' antwortete Isis und schaute aus dem Fenster in das enge Tal, durch das sich der Wagen hinunter schlängelte.

\sterne

Jeder hing seinen Gedanken nach\dots

Den Ereignissen der letzten Nacht und denen, die kommen würden\dots Die ruhige Fahrt wurde gelegentlich unterbrochen, von Wibrandis, die kurz frische Luft auf dem Grünstreifen schnappte\dots

So erreichten am Nachmittag Strasbourg. Sie hielten vor dem weiss-braunen Eckhaus direkt am Quai Saint Nicolas. Sie parkten ihre drei Wagen direkt vor dem Haus, vor dem grauen Tor\footnote{Wir wissen, was den Göttern natürlich nicht bewusst war: Es war ein Wunder, dass sie auf Anhieb drei freie Parkplätze hintereinander gefunden hatten!}.

Horus und Kebi warteten schon auf sie. Horus stand an Geländer der Ill gelehnt und Kebi hatte sich auf dem gegenüberliegenden Dach des  alten Zollgebäudes niedergelassen. Aus den Fenstern des langen grauen Gebäudes drang warmes, gelbes Licht aus dem Restaurant in die blaue Stunde. Kebi rief seinen Brüdern einen Gruss zu. Tef, der zwischen den Stangen des Geländers witterte, winselte\footnote{Es roch verführerisch nach einer Hundedame, Huskymix mit blauen Augen!}. Amset hatte Hapi auf den Arm genommen. Der Pavian, dem Isfet einen rosa Hoody geliehen hatte, hatte die Kapuze tief über die Schnauze gezogen. Die Brüder winkten Kebi fröhlich zu.

Anubis sortierte seine langen Beine aus dem Wagen und streckte sich. Dann gähnte er ausgiebig. Er lief zum Geländer und schob Tef auf die Seite. Er nahm nur eine kurze Nase von der Huskydame und starrte dann hoch konzentriert auf das Wasser und nahm die Witterung des gegenüberliegenden Ufers auf. Er blickte zu Thot und wies mit der Schnauze auf die Brücke aus Stein, die links von ihnen auf die Altstadtinsel führte.

Am Brückenkopf auf der anderen Seite war schwach eine Steintreppe zu erkennen die auf den alten Chemin de Halage (Treidelpfad) führte, der direkt unten am Fluss an dem alten Zollhaus vorbei rund um die Altstadtinsel führte. Anubis ging steifbeinig auf die Brücke zu und Thot folgte ihm.

Genau vor dem grossen Eckhaus führten zwei Treppen runter zum Fluss. Horus stieg eine zum Fluss runter. Während Re und die Frauen sich zwischen den beiden Treppen, oben auf dem Trottoir versammelten und in dem Dämmerlicht auf das träge, dunkle Wasser schauten.

Plötzlich bemerkte Amélie, die zwischen den Zwillingen Maat und Isfet stand, wie sich die Oberfläche des Flusses kräuselte. Und schliesslich, tauchten kaum sichtbar aus den kleinen Wellen die gewaltigen Schuppen Sobeks auf. Nur die gelben Augen des Krokodilgottes funkelten, wie kleine Laternen. Amélie schauderte, als der Gott sie anstarrte und dann den Oberkiefer öffnete und seine Zähne zeigte.

,,Sobi!'' rief Horus und beugte sich zu seinem Freund, der zur Treppe geschwommen war und tätschelte die breite Schnauze: ,,Gut gereist?'' Sobi hob die Schnauze kurz aus dem Wasser, es spritzte und er klappte mit seinem Gebiss. ,,Gut gegessen!'' sagte Horus zufrieden. ,,Anubis und Thot erwarten Dich!'' sagte Horus und wies auf den Treidelpfad auf dem Thot und Anubis inzwischen angekommen waren. ,,See you in a while\dots '' Horus stieg zurück auf das Trottoir.

Sobek durchquerte die Ill zu Thot und Anubis, die sich direkt neben dem Fluss niedergelassen hatten. Es gab einiges zu besprechen\dots

\chapter*{6. Nacht}
\addcontentsline{toc}{chapter}{6. Nacht}

\begin{quotation}

\emph{VI Vis eius integra est, si versa fuerit in terram.\\6. Seine Kraft ist vollständig, wenn sie in der Erde umgekehrt worden ist.  \\Tabula Smaragdina}

\end{quotation}

\section*{1}
\addcontentsline{toc}{section}{1}

Die Götter und Amélie wendeten sich dem grossen Haus zu, das auf der anderen Strassenseite zu. Es war ein hübsches Eckhaus. Zwei Männer traten auf den Gehweg und winkten ihnen zu.

Die Reisegruppe überquerte die Strasse und begrüsste die beiden. ,,Johannes! Was für eine Freude!'' Odilia, die ihre Ordenstracht trug und einen grauen Wollumhang begrüsste den Dominikanermönche in der weissen Wollkutte und dem schwarzen Umhang fröhlich und stellte ihn den anderen vor. ,,Johannes Tauler, ein beeindruckender Prediger!'' ,, Nicht doch, nicht doch, liebe Odilia\dots '' beschwichtigte der Mönch und reichte allen fröhlich die Hand. nur bei Bertas Anblick zuckte er unmerklich, hatte sich aber sofort wieder im Griff.

Gleichzeitig begrüssten sie auch den anderen Mann. Der trug einen grossen, dreieckigen Hut auf dem Kopf, der seine rotbraunen Locken bedeckte und sein ge- und belebtes Gesicht rahmte. Bis auf den Hut hatte Sebastian Brandt beschlossen, Komplikationen zu meiden und sich in neuzeitliche Kleidung zu hüllen. Er trug einen gewöhnlichen, grauen Anzug und einen dicken, weiten, schwarzen Wollmantel. 

Er schmunzelte über den bunten Haufen. Im Gegensatz zu seinem Freund Tauler, hatte er keine Berührungsängste und umarmte Berta herzlich. ,,Ich bin froh Euch zu sehen! Es wird nicht einfach werden heute Nacht!'' sagte Berta düster. ,,Aber, aber, ma grande, c'est la vie! N'aie pas peur, tout marchera bien!'' ,,J'espère, mon ami!''

,,Bevor wir in unser nächtliches Abenteuer starten, habe ich eine kleine Überraschung für euch! Tretet ein!'' mit diesen Worten öffnete er eine grosse Holztüre und ein Schwall warmer, rauchiger Essensduft strömte auf die Strasse. Alle wendeten ihre Köpf und beeilten sich dann in die Gaststube zu kommen.

,,Ich habe mir erlaubt einen Überraschungsgast einzuladen'' sagte Thot. Er hielt die Türe auf und ein weiterer Mann betrat den Gastraum. Er war recht klein und pummelig. Sein rundes Gesicht strahlte unter einer weissen Perücke mit Pferdeschwanz hervor und sein rundlicher Körper steckte in bordeauxfarbenen Kniebundhosen, einem weiten, weissen Leinenhemd mit  und einer schönen samtenen, bordeauxfarbenen Justaucorps. Seine strammen Waden steckten in weissen Strümpfen und seine Füsse in eckigen Schnallenschuhen.

 

Als Amélie durch die Tür schlüpfen wollte, hielt Berta sie zurück. ,,Amélie! Ich muss weg!'' ,,Was!'' Amélie zog es den Boden unter den Füssen weg! Berta konnte sie nicht wieder allein lassen! ,,Aber! Berta, ich habe Angst! Ich bin fertig! Lass mich nicht allein!'' Amélie klammerte sich an Bertas Schürze und schluchzte. ,,Du bist die einzige, die ich habe!''

Berta packte Amélie an der Schulter und schüttelte sie ungeduldig. ,,Kind, was redest du da! Du bist von den höchsten Göttern umgeben! Und ich muss zurück! Ich muss mit Luise reden!'' ,,Du willst zu Mama? Ich will mit! Berta, ich will zu meiner Mutter. Zu meinen Eltern!'' Amélie zerrte an Bertas Schürze. Die kleine Göttin schlug ihr auf die Hände. ,,Nein!'' Sie packte Amélie wieder und flüsterte dicht an ihrem Ohr: ,,Reiss dich zusammen! Und vor allem, denke daran: Sie sind nicht nur deine Eltern, sondern vermutlich auch deine Mörder!'' ,,Neeeein!'' schrie Amélie. 

Amset stand plötzlich zwischen ihnen und legte vorsichtig seinen Arm um Amélie, die schluchzte und zitterte. ,,Was ist los?'' ,,Amset! Nehme Amélie mit hinein. Ich muss fort! Geht!'' Berta schob Amset und Amélie zur Tür. 

Ohne sich weiter um die beiden zu kümmern, öffnete sie eilig den Kofferraum des Leichenwagens. Sie verabschiedete sich kurz von Isis, die den Sarkophag ihres Mannes nicht verlassen hatte, seit die Autos Quai Saint Nicolas geparkt wurden. Berta klopfte auf Osiris Sarg und zog einen Besen raus.

Sie lief über die Strasse und sprang im Lauf auf den Besen, der sich in die Luft hob. Berta zog eine kurze Schleife um den Münsterturm und wendete den Besen nach Südosten. Sobald sie den Rhein erreicht hatte, der von oben als dunkles Band in der punktiert leuchtenden Landschaft lag, richtete sie sich nach Süden aus auf den hellen Schein, den Basel in den Nachthimmel schickte. -So ein Nachtlicht ist nicht schön, aber praktisch! Dachte Berta und genehmigte sich ein warmen Schluck aus ihrem Flachmann, den sie unter ihrem Strumpfband hervorgeholt hatte.


\section*{2}
\addcontentsline{toc}{section}{2}

,,Ahhh! Vivre comme un coq en pate!'' rief Re entzückt. Sie standen gedrängt in einer grossen Gaststube. ,,Mes amis!'' rief Sebastian, ,,ich habe mir erlaubt Euch un petit diner auftischen zu lassen in der Herberge meines Vaters!''

Ein grosses Oh und Ah begann! ,,In welchem Jahrhundert sind wir?'' fragte Thot Sebastian, nachdem er in die Rund geschaut hatte. ,,Oh, qui! Wir sind in dem 15-Jahrhundert in der ,l'auberge du lion d'Or'. Meinem Vater hat sie gehört. Das Haus ist neu, aber der Standort ist ungefähr der richtige!'' 

,,Magnifique!'' antwortete Thot. Der Gott war hingerissen, denn er liebte Reisen\footnote{Zeit und Raum sind relativ. Und in der Göttersphäre gänzlich unbekannt. Andererseits, was eben das Paradoxe des Glaubens in diesem Fall als Gegensatz zu den intellektuellen Glaubenstrukturen werden lässt, ist es dort substanziell dennoch möglich, dass ein verkörperter Gott durch Raum und Zeit reisen kann.}.

Es gab sogar einige mittelalterliche Gäste, die für die richtige Stimmung sorgten, indem sie würfelten, Karten spielten, oder an einem Tisch speisten und Bier tranken.

An einer langen tafel, weiter hinten im Gasthaus, war für die Götter eingedeckt. Und wie! Selbst Hathor, die Königin der Festtafeln, pfiff durch die Zähne und nickte zufrieden. 

Denn sie hatte bemerkt, dass für jeden von ihnen ein köstlicher Happen bereit stand. Was, das wusste sie nur zu gut aus eigener Erfahrung, nicht so leicht war bei den vielen verschiedenen Geschmäckern.

Es dauerte eine Zeit, aber dann hatten alle ihre Plätze gefunden und sich ihre Teller gefüllt. es gab ein grosses hin und her. Die Rieslingflasche kreiste in die eine Richtung und der Krug mit frisch gebrautem Bier.

Tef und Anubis kümmerten sich nicht um das Gerangel am Tisch. Für sie gab es vor dem Kamin decken und Kissen, auf denen sie sich ausstreckten, nachdem sie mit grossem Appetit ihre ,Wädele' geknurpst und zerkaut hatten. Thot hatte ihnen eine Tonschüssel mit Bier hingestellt. Sie würden, wie alle, viel Kraft für die nacht brauchen.

Horus schmatze laut seinen ,Civet de Lièvre', Hasenpfeffer in Pinot Noir gekocht, in sich hinein. Dazu trank er Bier. Während sich Re mit Genuss an eine ,Tourte au Riesling' wagte. Er bevorzugte in diesem Fall natürlich Riesling. Zusätzlich hatte er sich ,Frommage du munster', ,Pate de foie gras' und einige Stück ,Carpe frite' auf seinem Holzteller angerichtet.

Hathor begann ihre Schlemmerei mit einer ,Pate à la reine', wie es sich für die Gottesmutter gehörte. Nach einer Portion Spätzli mit ,Coq au Riesling, beendete sie ihr Diner mit einer besonderen Spezialität, die Sebastian der südlichen Herkunft gewidmet hatte\dots

\section*{3}
\addcontentsline{toc}{section}{3}

Berta landete auf dem Dach des Hotels. Sie balancierte über den Dachfirst und liess sich auf der Giebelgaube von Luises und Wilfrieds Zimmer nieder. Den Besen klemmte sie sich unter den Arm und dann suchte sie unter ihren Röcken den silbernen Flachmann. Sie zog den Stöpsel aus der kleinen Flasche und schnupperte daran. ,,L'eau de vie des baies de sureau... Mein liebe Sebastian, du hast nicht zu viel versprochen!'' 

Berta setzte die Suche zwischen den Rockfalten fort und zog eine weisse Stoffservierte heraus, die sie sich über den Schoss breitete und schliesslich eine kleine mit Enten bemalte Terrine. ,,Ouh, Rillettes de canard!'' seufzte sie. Sie band das Baguette vom Besenstiel los. Sie riss ein Stück vom Brot und tauchte es in die Terrine und biss hinein. Grimmig zermalmte Berta das Brot\dots 

Berta fragte sich, ob sie einen Fehler machte. Das kam nicht sehr oft vor. Sie band die Flasche Riesling vom Besen. Sie trank gerne, aber sie konnte sich nicht mehr erinnern, wann sie sich das letzte mal Mut angetrunken hatte. Sie senkte die Flasche und liess den Blick über die Dächer der alten Stadt schweifen. Sie wurde das Gefühl nicht los, einen schrecklichen Fehler zu begehen\dots aber sie konnte nicht anders, sie wollte einen letzten Versuch wagen, bevor sie Luise ganz verloren geben konnte\dots 

Berta schwankte, als sie sich auf den Weg machte. Sie blieb auf dem Dachfirst stehen und versank. Und tauchte mit den Füssen voran in Wilfrieds und Luises Zimmer auf. Luise bewies ihre Nerven, indem sie nicht schrie. Wilfried setzte sich kraftlos in einen der Sessel. ,,Guten Abend!'' sagte Berta. ,,Was willst du!'' fragte Luise. Ich will mit Dir und Wilfried reden!'' ,,Es gibt nichts zu reden, du hast meine Tochter verschleppt!'' rief Luise. Wilfried schaute überrascht auf. Hatte er Luise falsch eingeschätzt? Er sah den kalten Glanz in Luises Augen. 

,,Luise, hör mir zu! Du hast schlimmes durchgemacht und wirst die Wunden mit dir tragen, aber Amélie kannst du retten. Du kannst sie und deine zukünftigen Enkel vor Unglück bewahren, weil du es kennst!'' ,,Ich weiss nicht wovon du redest, alte Frau! Du hast mich fallen lassen, als ich dich am dringendsten gebraucht hätte und jetzt willst du mir sagen, was ich zu tun habe?'' Luise stampfte mit dem Fuss auf und ihr Rocksaum wischte einen Stadtplan von dem niedrigen Couchtisch. 

Für einen Moment schimmerten Tränen in Luises Augen, aus Trotz, aus Wut. Berta schöpfte einen Augenblick Hoffnung. ,,Geh! Alte Frau! ich will dich nie mehr sehen. Und wenn du mir nicht morgen dieses schreckliche Gör wiederbringst, dann mache ich dir die Höhle heiss!'' Berta blickte zu Boden, suchte Worte, ihr blick glitt über den Plan am Boden. In der Mitte, war eine rot markierte Stelle. 

,,Wilfried?'' Berta wendete sich dem Mann zu, der zusammengesackt in dem Sessel hockte. Er blickte sie müde an. ,,Geh! Berta! Bitte! Es wird nur alles schlimmer!'' sagte er.

Luise wies Berta die Tür und Berta ging auf den Gang hinaus. Sie schwebte nach oben auf das Dach und setzte sich. Von Luise hatte sie sich nicht viel mehr erwartet. Doch Wilfrieds Reaktion alarmierte sie zutiefst.

Wilfried liebte seine Tochter. Er und sie hatten Amélie seit gedenken vor ihrer Mutter in den Schutz genommen. Sie hatten Amélie nicht vor dem Hass und der Eifersucht bewahren können mit der Luise ihre Tochter attackierte, aber sie hatten Amélie geholfen damit fertig zu werden und sich Orte der Liebe zu schaffen, wo ihre Mutter sie nicht erreichte.

Wenn Wilfried sie wegschickte, dann war etwas schreckliches passiert\dots Berta bemerkte am Rande ihres Bewusstseins eine Unordnung. Dinge verlangten ihre Aufmerksamkeit, Dinge, die sie im Zimmer gesehen hatte\dots Luise und Wilfried hatten Handschuhe getragen! Dann fiel ihr der Plan ein\dots der rote Kreis in der Innenstadt\dots als Luise sie rausgeworfen hatte, war sie an zwei Paar neuen Gummistiefeln vorbei gegangen\dots

Berta sass auf dem Dachfirst. Sie hielt sich ihren Flachmann an die Nase und sog den zart-herben Geist der Holunderbeeren ein, der sich im l'eau de vie- Wasser des Lebens spiegelte\dots

Sie sass eine Weile, mäuschenstill\dots bis die Gummistiefel und der Stadtplan ihr ihre Geschichten zugeflüstert hatten. Nur, weil sie sich sehr im Griff hatte, fiel sie nicht vom Dach, als ihr des Rätsels Lösung bewusst wurde. ,,Merci, mes amies! Les aimes de sureau!''

Berta erhob sich und schwang sich auf den Besen. Sie stieg in die Luft und landete nach einem kurzen Flug im Innenhof des blauen Hauses. ,,Hans? Haaans!\dots ''

\section*{4}
\addcontentsline{toc}{section}{4}

Amélie fühlte sich schläfrig. Sie sass zwischen Hathor und Wibrandis auf der Sitzbank. Die schweren Gerüche von essen und Pfeifentabak, Bier und Feuer hüllten sie gemütlich ein. Wibrandis unterhielt sich angeregt mit Odilia und Hathor raunte mit Isis. Amélie, an Wibrandis und die Wand gelehnt schloss die Augen.

Die übrige Gesellschaft war topfit. Thot wies den Benediktiner und Odilia in den Ablauf der Nacht ein. Diese machten ernste und andächtige Gesichter. Anubis hatte sich von seinem gemütliche Platz vor dem Kamin erhoben und sich zu den beiden gesetzt. Auch er hatte seine Hundestirn in feine Falten gelegt. Ab und zu stupste er mit der Nase Thots oder Johannes Arme, womit er sie dazu brachte ihm geistesabwesend die letzten Fleischhappen von der geplünderten Festtafel zu reichen. Schliesslich schleckte er sich leise über die Nase und lobte im Stillen ein weiteres mal seine Reisegestalt.

Thot bemerkte zu seiner Erleichterung, wie gut Odilia und der Dominikaner sich verstanden. Er hatte befürchtet, sie könnte für sie ungewohnt sein miteinander zu arbeiten, da sie Zeit ihres Lebens im Zölibat waren. Doch sie waren unbefangen. ,,Ich habe zu meiner Zeit in Starssbourg einiges über den Monte Sante Odilie gehört, sagt, war der heilige Columban tatsächlich Euer Lehrer?''

Tef schlief vor dem Kamin und schnarchte selig. Hapi hatte sich zu ihm geflätzt und knapperte an einer Schale Nüsse, die er sich geklaut hatte. 

Am anderen Tischende besprachen Sebastian, Re, Horus und Amset den Plan. nach einiger Zeit rutschten Odilia und Wibrandis zu Thot und Johannes auf und Hathor und Isis zu der anderen Gruppe. Amélie glitt auf die Bank unter den Tisch.

,,Scht!'' rief Johannes und Sebastian hob den Zeigefinger. Es wurde still in der herberge. Anubis und Tef, der erwacht war und sich reckte, hörten es als nächste. Dann die anderen. ,,Die Zehnerglock vom Liebfrauen! unser Startzeichen!'' 

Sie huschten aus der Bank und sprangen von den Stühlen, schnell, aber ruhig hatte sich die Gruppe in Windeseile vor der Tür der Herberge eingefunden. Amset und Horus hatten den Kofferraum des Leichenwagens geöffnet und den Sarg mit Osiris ein Stück rausgezogen. Sie hielten den geöffneten Deckel hoch, damit Isis ihren Gatten versorgen konnte.

Isis legte ihre Hand auf Osiris bandagierten Leib. ,,Bist du bereit, Liebster?'' ,,Ich war selten so bereit!'' sagte Osiris und lächelte schief. Er war, wie jede Nacht von Kopf bis zu den Füssen in schneeweisse Mumienbinden gewickelt. Isis zupfte zärtlich den prächtigen Perlenkragen zurecht, den Osiris um den Hals trug.

Osiris grünen Hände schauten aus den Stoffbandagen. Isis suchte in den Seitentaschen des Kofferraumes, die mit feinem Rüschen den Innenraum des Leichenwagens ausschmückten und dem Bestatter die Möglichkeit bot, von der Thermokanne, einem Schminkset oder wie in diesem Fall die Insignien des Fahrgastes sicher für die Fahrt zu verstauen.

Sie wischte die Stäubchen vom Krummstab. Dieser war aus purem Gold und trug Streifen aus tiefblauem Lapislazuli. Osiris fasste ihn mit der linken. Isis schüttelte die Geissel, das zeichen der Fruchtbarkeit bis alle Perlen und goldenen Glieder geordnet waren. ,,Dein Nechacha!'' Isis reichte es ihm und kletterte aus dem Heck des Wagens.

,,es ist gut!'' sagte sie knapp. Sie stand am Quai und beobachtete streng wie Horus und Amset den Sarg behutsam aus dem Wagen wuchteten und ihn die steilen Treppen zum Wasser herab trugen. Sobald sie das Wasser erreicht hatten, kräuselte sich die schwarz schimmernde, lichtbesprenkelte Wasseroberfläche.

Sobek tauchte knapp auf. Der gewaltige Rücken des Krokodilgottes sah aus wie ein riesiger Baumstamm. Horus und Amset stellten vorsichtig den Sarg auf den breiten Rücken. ,,Gib auf euch acht!'' rief Thot von oben und Horus gab dem Krokodil einen freundlichen Klapps. Sobek klapperte mit den Zähnen und drehte langsam ab in die Mitte des Flusses. Sobald er sie erreicht hatte, tauchte er ab, der schwere Sarg versank mit ihm.

,,Husch, husch, meine Lieben!'' rief Re. Seine Augen blitzten hinter den Gläsern seiner Sonnenbrille auf. Isis löste sich von dem Geländer und warf einen letzten Blick über den Fluss. Re nahm seine Enkelin in den Arm. ,,Wir werden die beiden gleich wieder treffen, Isis.'' ,,Ich weiss!''seufzte sie. Hathor war neben die beiden getreten: ,,Kommt, wir wollen Osiris und Sobek an jenem Ort nicht zu lange allein lassen!'' Die Muttergöttin, die von hinten aussah wie Berta in einem Pelz, begann die Gruppe zu inspizieren.

Johannes Tauler und Sebastian Brant hatten sich schon an den Kopf der Gruppe begeben, um ihre Gäste auf dem schnellsten und unsichtbarsten Weg auf der Altstadtinsel zum Münster zu führen. Anubis und Tef schnüffelten. Aha! Die Huskydame wohnte wohl in der Nähe, den sie hatte eine weitere Duftmarke hinterlassen.

,,Wo ist denn Amélie?'' fragte Wibrandis plötzlich. ,,Amélie?'' ,,Amélie!'' Amset rutschte das Herz in die Hose, doch bevor sich jemand richtig Sorgen machen konnte, war Hathor in die Herberge geeilte und zog nun eine schlaftrunkene Amélie über die Strasse. 

Amélie sah zu Amset hinüber, aber dann hackte Wibrandis sich bei ihr ein und Odilia auf der anderen Seite. ,,Auf gehts!'' sagte Wibrandis und ihre Augen funkelten\dots 

\section*{5}
\addcontentsline{toc}{section}{5}

Berta eilte in die Küche des blaue Hauses. ,,Haans!'' Aus dem Augenwinkel bemerkte sie einen dunklen Schatten auf sie zu schiessen und verzweifeltes Jaulen, bevor sie der Länge\footnote{In ihrem Fall Kürze} auf den Boden prallte und mit dem Kopf an den Herd stiess, der in der Mitte der Küche thronte. Benommen rappelte sie sich hoch  blieb auf dem Boden sitzen. Der Dackel, den Berta als sie gestolpert war mit dem Fuss heftig gegen die Küchenschränke geschubst hatte, berappelte sich ebenfalls. Er blieb vorsichtshalber in einiger Entfernung und kläffte und jaulte hysterisch.

,,Waldi! Alter Freund! Komm her, komm zur alten Berta!'' Berta lockte den ängstlichen, verwirrten Hund. Und dieser kläffte und knurrte, während er sich näherte. Er drehte sich immer wieder und hüpfte. Berta nahm ihn sanft auf den Schoss. ,,Nun ist aber gut!'' Sie hielt den Hund, dessen flattriges Herz langsam ruhiger klopfte. Berta und der Hund sassen für einen Moment still da.

,,Oh, weh!'' sagte Berta schliesslich. Sie hob den Dackel vor ihr Gesicht und sagte freundlich: ,,Kein Wunder, dass du dich so !aufregen musstest!'' 

Sie setzte Waldi auf den Boden. Sofort sprang der Dackel aus dem Küche und bog links ab zum Frauenbad. Berta rannte hinterher. ,,Tefnut!'' rief sie erschrocken. Berta eilte zu dem grossen Schwimmbecken, in dem der Körper der Göttin leblos trieb. Das Bad war warm und feucht. Über dem Wasser webte ein Dunstschleier. einzelne Seifenblasen, die sich aus dem Badeschaum gelöst hatten, flogen sachte umher. 

Tefnut lag mit dem Gesicht nach unten im Wasser. Ihr feuerrotes Haar wogte sachte auf der Wasseroberfläche wie ein riesiger Fächer um ihren Kopf. Ihr Körper lag still mit ausgebreiteten Armen auf dem Wasser, das sich langsam rot färbte. Tefnut trug eine Wunde am Hinterkopf aus dem das Blut in die Haare und von dort in das Wasser ran. Waldi rannte wild um das Becken und bellte. Seine raue Dackelstimme halte unheimlich von den gekachelten Wänden durch den Raum. 

Berta stieg mit all ihren Kleidern in Wasser und zog Tefnut an den Rand des Beckens. Vorsichtig schob sie die reglose Göttin über den Rand. ,,Tefnut!'' flüsterte Berta und strich das klatschnasse Haar aus dem blassen Gesicht. Das Blut lief aus der Wunde an Tefnuts Hals in den Ablauf am Beckenrand. ,,Tefnut! Wach auf!'' Berta hatte einige Mühe ihr Strumpfband unter Wasser und unter all den vollgesogenen Röcken zu finden, aber schliesslich hielt sie triumphierend ihren silbernen Flachmann in die Höhe. 

Waldi hatte mit der Wiederbelebung begonnen und schleckte der Göttin eifrig das Gesicht ab. ,,Aargh! Jetzt!'' Berta hielt Tefnut die flasche unter die Nase. 

Die Augenlider der Katzengöttin zuckten und die feine Nase kräuselte sich! Der schön geschwungene Mund verzog sich angeekelt und dann nieste Tefnut. Waldi tappte erwartungsvoll von einem Stummelbein auf das andere. Und Berta seufzte und machte sich auf den Weg ihre vollgesogenen Kleider aus dem Schwimmbecken zu wuchten.

Es flatschte und patschte und klatschte, während Tefnut sich stöhnend erhob und sich den Kopf abtastete. ,,Miiaaauuuuh!'' Tefnut schaute auf ihre blutigen Finger und schlang sich schnell ein Handtuch um den Kopf. dann eilte sie zu Berta und umarmte sie fest. ,,Gut, dass du da bist!'' ,,Was zum Teufel ist hier los?'' fragte Berta und zappelte vergeblich herum, um ihre nassen Kleider abzustreifen. ,,Eben, der ist Los!'' antwortete Tefnut, ,,Wuah,wuah!'' meldete sich Waldi. 

Tefnut half Berta aus den Kleidern. ,,Der Teufel?!'' fragte Berta und wickelte sich in ein rosa Badetuch. Sie sah aus wie ein riesiger, flauschiger Ball. Tefnut zog ihren kurzen Satinmantel über. ,,Seth!'' knurrte sie knapp. Dann stürzte sie aus dem Bad. ,,Maat! Schu!'' rief sie. Berta eilte hinterher und holte auf. ,,Hans?'' rief sie. Tefnut und sie rannten durch die Eingangshalle zum Treppenhaus.

,,Ich weiss nicht!'' rief Tefnut über die Schulter und sprang jede zweite Stufe nehmend hinauf. Berta trippelte gleichauf hinterher. Sie beschleunigten als sie den ersten Stock erreicht hatten, denn aus dem zweiten Stockwerk war ein leises Stöhnen zu hören. ,,Schu!'' ,,Hans!''

Die beiden Göttinnen hatten die oberste Stufe erreicht und stürzten in Res Zimmer, dessen Tür offen stand. Im Treppenhaus blieb das emsige, unermüdliche kratzen von Waldis Pfoten auf den Holzstufen zurück.

\section*{6}
\addcontentsline{toc}{section}{6}

Sobald Luise vor dem Hotel auf der verlassenen Strasse stand, wich alle Farbe aus ihrem Gesicht. Bleich und zu Tode geängstigt rührte sie sich nicht von der Stelle. Für einen Augenblick keimte eine Spur Mitleid in Wilfried auf. ,,Was?'' zischte Luise. Wilfrieds Lächeln erstarb sofort. ,,Nichts!''sagte er kalt. Er spürte zum ersten mal, seit er Luise begegnet war Hass auf seine  Frau.

all die Jahre, die sie zusammen verbracht hatten, war es ihm gelungen, das schöne, aber zutiefst verletzte Mädchen, das Luise in sich verbarg und das sie unberechenbar und bösartig machte, nicht aus den Augen zu verlieren. Doch die Ereignisse der letzten Nacht zeigten Wilfried auch den von unbewusster Schuld getriebenen jungen Mann, der vergeblich um einen Hauch von Liebe kämpfte und zu einem resignierten, seelischen Wrack mutiert war. -Vielleicht war ich schon ein seelischer Krüppel, bevor meine Mutter mich in dies Leben gespien hat, dachte er bitter.

,,Los!'' sagte Wilfried und zog Luise grob am Arm mit sich. Die erwartete Gegenwehr blieb aus. Sollte Luise tatsächlich vor etwas Angst haben? Sie hasteten zu dem vereinbarten Treffpunkt. Als sie dort ankamen, war von dem Fremden, Seth, nichts zu sehen. Ein schwacher Lichtschein, der einen schmalem Eingang an der Seite des kleinen Platzes erhellte, der in den Hügel führte, erregte ihre Aufmerksamkeit. Sie begaben sich in den Gang und schlüpften durch die dicke Steintür. Sobald sie im Hügel waren schabte die Tür leise über den Boden und schloss sich mit leisem Klick!

Wilfried zog sein Smartphone raus, doch das Gerät machte keinen Mucks, obwohl er es frisch aufgeladen hatte. ,,Nichts!'' raunte er. Wohl oder Übel, sie tasteten sich Stück für Stück durch den stockdunklen Gang.

\section*{7}
\addcontentsline{toc}{section}{7}

,,Schu!'' schrie Tefnut. Hätte Berta die Göttin nicht gepackt, wäre sie ebenso aus dem Fenster gestürzt wie ihr Mann. Berta zerrte Tefnut in das Zimmer zurück. Der Sessel des Sonnengottes streckte alle Viere von sich. Das Buch, dessen Vorderseite ein Junge mit einer Brille zierte, lag zerrissen auf dem Boden. 

Die beiden Göttinnen machten kehrt und sausten die Treppenstufen zurück ins Erdgeschoss. Waldi, der gerade die letzte Stufe erklommen hatte, seufzte und machte sich ergeben an den Abstieg. Ein kurzer Augenblick des Gepolters, dann blieb kratzten seine Pfoten durch das stille Treppenhaus\dots

Waldi fand die Göttinnen vor der Tür auf der Strasse. Berta flösste dem Luftgott einen Schluck l'eau de vie ein, das ihn mit krächzen und husten auf die Beine brachte. Mit vereinten Kräften schleppten die Göttinnen den schwankenden Gott ins Haus. Wobei dieser, durch die unterschiedliche Grösse seiner Helferinnen mehrmals auf die kleine, flauschige kugel zu kippen drohte. Nur Bertas energisches Schubsen verhinderte, dass Schu auf sie fiel.

Tefnut gab der Tür, nachdem sie sich zu dritt durchgezwängt hatten, einen Tritt und sie fiel ins Schloss. Waldis Nase fuhr empor. Er kläffte kurz und empört. Er hatte doch nur kurz den Laternenpfahl vor dem Haus\dots

Der Hund seufzte, dann machte er sich auf den Weg zur Rückseite des blauen Hauses. Seine Pfoten tapsten auf dem Pflaster. Er schlüpfte durch das Schmiedeeiserne Tor und rannte durch den Seiteneingang.

Als Waldi in die erleuchtet Küche kam, hechelte er heftig. Er keuchte und hechelte und schlappte mit der Zunge. Schu und Tefnut sassen am Küchentisch, je ein Tuch mit Eiswürfeln an den Kopf gepresst. Berta. die sich Eiswürfel in ihren Drink gefüllt hatte, erhob sich und gab dem japsenden Hund Wasser in seiner Schüssel. In der Küche seufzte, stöhnte, ächzte und schlabberte es.

,,Es ist unglaublich! Aber Seth hat es geschafft uns zu überrumpeln!'' sagte Schu. ,,Das Schlimmste ist, dass Maat verschwunden ist!'' sagte Tefnut. ,,Und Hans!'' meinte Berta. ,,Es ist eine Katastrophe! Wenn der heiligen Ordnung, unserer Maat, etwas zugestossen ist, dann, dann bricht das absolute, tödliche Chaos aus!'' rief Tefnut. ,,Und wenn wir Hans nicht wiederfinden, dann wird der Frühling nicht kommen, die Bäume werden keine Blüten tragen.'' Berta schluckte schwer. sie schwiegen. Schu sah sich in der Küche um. ,,Maat muss sich versteckt haben! Wenn sie verschwunden wäre, dann wäre kein Stein mehr auf dem anderen! Aber, wo ist sie?''

Waldi hatte kaum ausgetrunken, da trippelte er zu Berta und stellte sich auf die kurzen Hinterbeine. Er jaulte und bellte, sein Schwanz wedelte wild und verfing sich in Bertas langen Reiseröcken. ,,Jaja, mein Gutster! Bist ein braver. Hättest ja nix könne mache.'' sagte Berta. Waldi hielt verdutzt inne. Was war nur in Berta gefahren. Wieso verstand die Erdgöttin, Berta, die Sprache der Tiere nicht mehr?

Waldi liess von Berta ab und rollte verzweifelt mit den Augen. er stellte sich in die Mitte der Küche und begann im Kreis zu rennen und zu kläffen. Ab und zu blieb erstehen und bellte herausfordernd. Als ihn alle drei Götter anschauten, als wenn sie ihm gleich den Hals umdrehen wollten, hielt Waldi es an der Zeit Phase zwei einzuleiten. Er lief zur Küchentür blieb dort erwartungsvoll stehen und jaulte kurz.\footnote{Er wollte sein Glück nicht strapazieren und in einer Stichflamme enden!}


\section*{8}
\addcontentsline{toc}{section}{8}



\section*{9}
\addcontentsline{toc}{section}{9}

,,Du! ich glaub der Dackel will uns etwas sagen!'' sagte Schu, der sich sichtlich erholt hatte. ,,Jetzt, wo du es sagst! Waldi! Du weisst etwas, stimmts!'' fragte Berta. Waldi trat ungeduldig von einer Pfote auf die andere und wedelte mit dem Schwanz. ,,Meint ihr?'' fragte Tefnut. Sie, als Katzengöttin hatte ein natürliches Misstrauen zu Hunden. Waldi jaulte und sprang auf!

,,Ich gehe mal schauen, was der Hund will!'' beschloss Schu. Er und Berta erhoben sich und Waldi seufzte und sauste los Richtung Dienstbotenaufgang, der vom Gang zur Küche abzweigte und der schnellste Weg in die oberen Stockwerke war.

Tefnut verwandelte sich betont langsam in eine Katze. Ihr Stolz war heute Abend genug verletzt worden, als dass sie einem Hund so ohne weiteres das Feld überlassen konnte. Sie sprang leichtfüssig und elegant aus der Küche und kam punktgenau zwei Stufen vor den drei andern im ersten Stock an. Waldi ignorierte die rote Katze, die leise miaute und eine grosse Beule mit eine Platzwunde zwischen den Ohren trug.

Wie es sich für einen Jagdhund gehörte, blieb Waldi in hab-Acht-Stellung vor der kleinen, unscheinbaren Holztür im Treppenhaus stehen. Berta schaute zweifelnd auf die Zwergentür. Waldi begann zu bellen, und Schu öffnete vorsichtig die Tür, die ihm nur bis an den auch reichte. Im Inneren wurde ein schmaler Gang sichtbar, der rechts und links offensichtlich zwischen den Wänden des Treppenhauses und den Zimmern verlief. ,,Na, endlich!'' rief eine helle Stimme aus dem Dunkel.

,,Maat!'' rief Schu. Er eilte in die Finsternis und schrie vor Freude: ,,Maat! Maat! Oh, Mann!'' Berta, die ein Stück in den schmalen Raum getreten war, nahm die junge Göttin der Ordnung in Empfang und zog sie dann in ihre Arme. Waldi umkreiste die beiden mit Gebell. Die Katze sass gelangweilt auf der Fensterbank und schleckte sich die Pfote. Waldi wusste, der Tag würde kommen, der Tag, an dem die Katze büssen würde, Katzengöttin hin oder her\dots


\section*{10}
\addcontentsline{toc}{section}{10}

\section*{11}
\addcontentsline{toc}{section}{11}

\section*{12}
\addcontentsline{toc}{section}{12}

\section*{13}
\addcontentsline{toc}{section}{13}

\section*{14}
\addcontentsline{toc}{section}{14}




