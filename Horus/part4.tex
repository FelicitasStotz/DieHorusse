\part*{Vierte Stunde\\"`Die gross ist in ihrer Macht"'}
\addcontentsline{toc}{part}{Vierte Stunde}

\chapter*{4. Tag, Johannestag}
\addcontentsline{toc}{chapter}{27. Dezember, Johannestag}

\section*{1}
\addcontentsline{toc}{section}{1}

Als Amélie erwachte stand die Sonne hoch und fahl am Winterhimmel. ,Ich habe Bärenhunger!' dachte sie und schwang die Beine aus dem Bett. Ihre Füsse landeten in einem Pelz. Schnell liess sie sie wieder unter der Bettdecke verschwinden. ,,Oh, Anubis! Sorry! Ich wollte dich nicht treten!''  sagte Amélie und beugte sich zu dem schwarzen Hund herunter, der vor ihrem Bett lag. Der Hundegott schaute mit seinen Augen den -Schau-was-für-ein-armer-geduldiger-braver-Hund-ich-bin-Blick zu ihr auf. ,,Du hast sicher auch Hunger, gell?'' Die Augen des Gottes begannen zu leuchten und er erlaubte sich einen eleganten Schleck mit der rosa Zunge über seine schwarze Nase.

Es klopfte an der Tür und bevor Amélie einen Mucks machen konnte, stürmte Wibrandis mit einem Kleiderbündel ins Zimmer. ,,Guhuten Morgen!'' rief sie und warf schwungvoll die Kleider auf Amélies Bettende. ,,Schnell, schnell, Liebes, es gibt einen Mittagsbrunch im Musikzimmer!'' Amélie bemerkte die Schürze, die Wibrandis trug, sie war weiss und mit vielen blauen Schneeflocken verziert.

,,Brunch?'' fragte sie und sah die Schwanzspitze des Anubis auf dem Gang verschwinden. ,Na, warte!' dachte sie. ,Du musst dir ja nichts anziehen.' 

,,Komm! Es gibt Neuigkeiten! Berta ist eingetroffen!'' Wibrandis hatte ihre Hand mütterlich auf Amélies Schulter gelegt. ,,Berta! Berta ist da!'' Amélie sprang aus dem Bett und rannte los. ,,Warte!'' Wibrandis lief hinter ihr her und schnappte sich im vorbeirennen Amélies Morgenmantel. Als sie Thots Stimme im Treppenhaus hörten, die Anubis einen guten Morgen wünschte, waren sie beide froh, dass Wibrandis Amélie rechtzeitig mit dem Morgenmantel eingeholt hatte.

Von überall im Haus ertönten Stimmen und Gemurmel, das sich dem Musikzimmer im Erdgeschoss rechts neben der Haustür näherte.

,,Berta!'' Amélie hatte sie entdeckt. Die kleine Frauenkugel kam vergnügt aus dem Trackt des Frauenbades. Als Amélie sich in ihre Arme warf, sog sie tief den Geruch nach Winter, Erde, Whisky und Red-Superduper-Magnolie-Rosen-Nasenexplodierer-Duschschaum ein. Dann konnte Amélie die Tränen, die sich in den letzten Tagen vergeblich zu Wort gemeldet hatten, nicht mehr zurückhalten. ,,Jaja, meine Kleine, wird schon wieder alles gut! Berta is' ja nun da!'' Die kleine Göttin tätschelte Amélie den Rücken. 

,,Komm, essen!'' befahl sie schliesslich. Zu ihrem Entsetzten bemerkte Amélie Amset und Duamutef, die zu Salzsäulen erstarrt im Entre standen und auf sie starrten. Automatisch wischte sie sich die Schnoddernase am Ärmel ab. Als sie bemerkte, was sie getan hatte, wurde sie puterrot. ,,Hi! Ihr zwei!'' quietschte Amélie. Sie gingen in das Musikzimmer und suchten sich einen Platz an der grossen Tafel aus, die die Göttinnen am Morgen gezaubert hatten.

Re sass am hinteren Ende der Tafel. Er hatte seinen Becher in der Hand und nippte von dem heissen Tee. Er hatte eine kleine Beule an der Stirn. Thot sass zu seiner linken und zu seiner rechten Isfet und Maat. Es folgten Amset, Duamutef und Anubis. Damit die beiden hundegestaltigen Götter das Tischgespräch im Auge haben konnten, ruhten sie auf einer langen, breiten Bank, die fast so hoch war, wie der Tisch. In der Mitte der Bank waren zwei Porzellanschüsseln eingelassen, in denen sich der Brunch für die Caniden befand. Sie konnten alles am Tisch verfolgen und ihr Mahl gemütlich im Liegen aus den Schüsseln schlecken. Nach der Hundebank blieben die Stühle für Hathor und Wibrandis frei.

Berta nahm neben Thot Platz und Amélie setzte sich neben sie. Neben Amélie setzte sich Tefnut, in ihrer Frauengestalt, allerdings trug sie heute morgen eine stattliche, gelbe Löwenmähne auf den Schultern. Amélie war beunruhigt, denn Tefnut schien sich jeden Moment in eine Furie verwandeln zu wollen. Dann fiel Amélie auf, dass nicht Schu neben Tefnut sass, sondern ein finster dreinblickender Gott, den sie noch nie gesehen hatte. Schu sass auf der anderen Seite des Fremden und daneben konnte Amélie Hans entdecken, der auf seiner Pfeife kaute. Der Hahn reckte sich auf seiner Schulter und verdrehte nach gockelart den Kopf zu dem Fremden.

Amélie musterte den Neuankömmling verstohlen. Er war im Gegensatz zu den anderen Göttern blass. Er hatte dumpfes, schwarzes Haar und seine Augen lagen dicht zusammen. Die Nase war sehr lang, schmal und gerade mit grossen Nasenlöchern, sein Kinn spitz. Seth hatte hohe Wangenknochen. Sein Körper war gross und stark. Er konnte es ohne weiteres mit der trainierten Kämpferstatur seines Neffen Horus aufnehmen. er trug eine schwarze, enge Jeans, die aussah, als wäre sie staubig und schwarze, schlichte Western Boots. Als er Amélies Blick bemerkte und zurückstarrte, musste sie schnell den Blick wenden. ,Mein Herz rast!' dachte sie und griff sich an die Brust. Seth beobachtete sie und grinste mit seinen schmalen Lippen. ,Er ist nicht hässlich,'dachte Amélie. ,Er ist absolut unheimlich! Er macht mir Angst!' 

 Sie liess ihren Blick durch den Raum gehen und bemerkte, trotz des leisen und höflichen Gemurmels, Anspannung. Re schien Isfet im Auge zu behalten, obwohl er mit Thot plauderte. Maat hatte die Arme vor der Brust verschränkt, während Amset vergeblich bemüht war, sie in ein Gespräch zu verwickeln. Die beiden Caniden schielten verstohlen über die Ränder ihrer Schüsseln und hatten gespitzte Ohren. Aber am schlimmsten schien es Schu und Tefnut zu gehen, sie sassen auf glühenden Kohlen. Schu wurde immer wieder von dem Flügel des aufgeregten Hahnes getroffen, bis dieser die Nerven verlor und auf den hohen Ofen flatterte. ,ich bin nicht die einzige, die sich fürchtet!' bemerkte Amélie verblüfft.

Hathor und Wibrandis erschienen. Wibrandis setzte sich an das unterste Ende der Tafel und Hathor stand neben ihr. Die Muttergöttin war, ähnlich wie Berta, klein und rund. Obwohl sie den Tisch kaum überragte, wurde es still als sie zufrieden lächelnd ihren Blick über die Versammelten und den Speisen gleiten liess.

 Ihr blick blieb eine Nanosekunde an Seth hängen und ihre grünen Katzenaugen blitzten auf. ,,Oh, ich vergass! Braucht jemand Lattich?'' Auf diese harmlose Frage hin, kicherte es zu Amélies Verwunderung vereinzelt am Tisch und Seths Gesicht bekam einen Hauch Farbe. ,,Lass es gut sein, meine Liebe! Es sieht fantastisch aus! Greift zu!'' sagte Re. Kam es Amélie nur so vor, oder blitzte es hinter der Sonnenbrille des Sonnengottes auch verdächtig? \footnote{Was Amélie nicht weiss: ,,Lattich'' ist in der Familie der altägyptischen Götter eine Art Safewort. Wobei es als Antisafewort bezeichnet werden müsste, wenn es um die Wirkung desselben auf Seth geht. Also, lieber Leser, solltest Du Seth jemals begegnen, weisst Du, welches Wort Du, als kleiner, zerbrechlicher Mensch, nicht sagen solltest! Selbst, wenn Du Dich zufällig mit dem Herren des Donners und Krieges in einem Lattichspezialrestaurant befindest, oder auf dem Markt Verkäufer von super-lecka-Lattich  sein solltest, oder so!} 

,Der einzige, der nicht angespannt wirkt ist Re\dots' dachte Amélie, als sie begann an ihrem Honigwabenfladen zu knappern. Sie wollte Berta so vieles fragen, aber sie brachte keinen Ton raus. ,Als ob zwei Filme gleichzeitig laufen! Die Komödie: das grosse Familienessen und der Psychothriller: das grosse Familienessen, wer frisst wen zuerst?'

Der finstere Gott zwischen Schu und Tefnut nahm geschmeidig das Messer, das für die Käseplatte gedacht war. Er fuhr mit den Fingern vorsichtig und lasziv über die Klinge. ,,Wo ist mein Brüderchen? Osiris!'' Er bellte den Namen, sprang auf und rammte das Messer in einen unschuldigen Camembert.

Dann ging alles schnell, so schnell, dass Amélies Gehirn die Slowmotion-Taste drückte, um nichts zu verpassen, während Amélies Stammhirn sie dazu brachte augenblicklich vom Stuhl unter den Tisch zu rutschen und über den Rand zu schielen: Schu sprang auf die Füsse und warf dabei seinen Stuhl um. Tefnut wechselte in Sekunden ihre Gestalt und setzte zu einem mächtige Löwenbrüllen an\dots Thot hob abwehrend den Arm und schien unter den Tisch rutschen zu wollen\dots Anubis und Duamutef sprangen auf die Tafel und fletschten die Zähne\dots Amset schnappte sich das Messer von der Wurstplatte und brachte es in einer fliessenden Bewegung in Wurfposition\dots Hathor erhob sich und wollte den mütterlichen Fluch des erhobenen Zeigefingers einsetzten\dots Hansens Hahn machte einen grossen Klecks auf den schicken Keramikofen und Hans sprang zur Tür, stolperte und viel auf den Po\dots Die Tür ging auf und Isis stand darin mit erhobenem Schlangenstab\dots Neben Amélie bewegte sich Berta und warf\dots

\dots das Nudelholz. Es traf Seth zwischen den Augen und er kippte nach hinten, wobei er mit dem Messer und dem aufgespiessten Camembert einen eleganten Bogen machte. Der Camembert klatschte schwungvoll an die Wand. Schu blickte auf seinen Enkelsohn runter von dem die anderen am Tisch nur die Füsse sahen. Tefnut verwandelte sich zurück und brachte ihre Haare in Ordnung. Weder Anubis noch Duamutef konnten ihren Sprung abbremsen. Anubis landete neben Seth und schleckte den Camembert von der Wand, schliesslich wollte er keinen Frommage francaise umkommen lassen. Duamutef landete aus Seths Gesicht und da er ein heiden Respekt vor seinem Onkel hatte und mit fliegenden Pfoten versuchte das Gesicht schnellstens wieder zu verlassen, hinterliess er tiefe Kratzspuren auf den Wangen.

Hathor stand mit erhobenem Zeigefinger und blickte streng. Wie Wibrandis in der Geschwindigkeit eine Waschschüssel, Tücher, Salbe und einen Verband besorgt hatte, war allen ein unbegreifliches Phänomen. Thot sollte es später mit ,Pastorenfrau-Magie' bezeichnen.

Stille! Isis stieg über Hans, der noch am Boden hockte und packte ihren Bruder am T-Shirtkragen. Sinnigerweise stand auf dem schwarzen Shirt: ,Bösewicht'. Sie zog den Gott, der drei Köpfe grösser war als sie wie eine Feder hoch, bis sich ihre Nasenspitzen trafen. Es gab einen gelb-roten Funkenregen. ,,Vergiss es!'' flüsterte sie, fast freundlich. Seth versuchte seine Schwester zu fokussieren, was ihm nicht gelang. Über dem linken Augen thronte eine bläulich-rote Beule mit einem tiefen Riss in der Mitte.

Isis liess Seth los und er fiel zu Boden. Wibrandis stand bereit, um den Verband anzulegen. Amélie grinste, Wibrandis war bei ihrer Verarztung nicht zimperlich. ,Die ist echt mutig, ich würde mich nicht trauen so unfreundlich zu sein\dots'

Noch bevor Amélie wieder auf ihrem Stuhl sass, befanden sich alle Götter wieder an der Tafel, freundlich, fröhlich plaudernd. Nur Berta sass auf ihrem Platz und streichelte verliebt über ihr Nudelholz, als wollte sie das Holz glätten und trösten. Zum ersten Mal an diesem Morgen begegnete Amélie Amsets intensiven Blick. Er zeigte mit dem Finger auf die Tür und dann auf Amélie. Amélie reckte den Daumen nach oben. 

Isis behäufte einen grossen Teller mit verschiedenen Leckereien und verliess das Musikzimmer. Erst jetzt fiel Amélie auf wie viele Götter an der Tafel fehlten und demnach im Haus wachen mussten! ,Meine Berta! Ha, die hat es geschafft!' Amélie sah stolz und liebevoll zu ihrer alten Amme hinüber und Berta zwinkerte zurück. ,,Husch, mein Mädchen, geh' Du mit dem Amset plaudern.'' ,,Wirklich?'' fragte Amélie. ,,Ja,ja! Die alte Berta muss vorher auch noch mit Thot sprechen und dann haben wir alle Zeit der Welt!''  Amélie fiel plötzlich eine Beule an Bertas Stirn auf und roch eine Hauch Whisky\dots ,,Geh' Amélie!'' sagte Berta und lächelte müde.

\section*{2}
\addcontentsline{toc}{section}{2}

Amélie trat aus der kleinen Seitentür gleich neben der grossen Eingangstreppe, die zu dem noblen Entré im ersten Stock führte. Wieder war sie verblüfft über die warme Luft, die wie eine Schutzglocke über dem Garten, den die Götter gezaubert hatten, lag. Amélie umkreiste den Weiher und die Bäume. Schliesslich sah sie Amset stand mit Duamutef am geöffneten Küchenfenster stehen. Sie hatten das schwere Eisengitter, das die Fenster im Parterre sonst vor Einbüchen schützen sollte, heraus genommen. Sie liessen sich drei riesige, dampfende Töpfe nach draussen reichen, die sie auf einen kleinen Leiterwagen stellten.

,,Amélie!'' rief Amset erfreut. ,,Wir bringen Sobek sein Mittagessen! Komm!'' ,,Sobeks Mittagessen? Zeig!'' Amélie hob vorsichtig den Deckel des ersten Topfes. ,,Mmmh! Das riecht aber gut!'' sagte sie überrascht. Duamutef legte sich die Schnauze und seufzte. ,,Heute gibts für Sobi Gänsebraten, das hat er am liebsten!''

Sie zockelten mit dem Wagen über den Hof auf die Bäume zu. Die kleinen Räder des Leiterwagens klapperten über die Steine. ,,Pssst!'' hörten sie es von oben, sobald sie den ersten Baum erreicht hatten. Hapi schwang sich vom Baum. ,Psst! Seth ist bei Sobek!' Die drei Ankömmlinge blieben mucksleise stehen. 

,Kommt!' winkte Hapi und verschwand ohne ein Geräusch zwischen den Büschen. Die anderen folgten ihm. ,Uhh! Ich mach Krach wie ein Elefant!' dachte Amélie. ,Tatsächlich! Aber am meisten mit deinen Gedanken, schalt die ab!' antwortete Amset.

,Warum bist du denn so rot im Gesicht?' fragte Amset, als Amélie  auf dem Bauch unter einem grossen Busch neben ihn rutschte. ,Ich versuche nicht zu denken!' dachte sie. Amset schaute sie von der Seite an und färbte sich rosa. ,Was ist?' dachte Amélie ärgerlich. ,Ich versuche nicht zu lachen!' antwortete er. ,Still!' Duamutef war ernst.

Sobald Amélie zu Seth und Sobek schaute lief es ihr kalt den Rücken runter. Die beiden schienen sich so sicher zu fühlen, dass sie miteinander sprachen. ,, Na, Sobi! Wie gefällt dir der Ausflug?'' fragte Seth, betont gleichgültig. ,,Kann nicht klagen!'' murmelte Sobek und schaffte es mit einem zufriedenen Grinsen auf seine Vorderkrallen, seinem Äquivalent zu Fingernägeln, zu schauen. ,,Und dir, liieber Papaa?'' fragte das Krokodil, das nur halb aus dem Teich geklettert war. Es stützte seine lange Schnauze auf seine Tatzen und lächelte zu Seth auf\footnote{Krokodile sind die Lächler, sie werden mit einem Lächeln auf der Schnauze geboren. Im Gegensatz dazu, nimmt man ihnen Tränen nicht ab. Vermutlich ist eine Schnauze voller spitzer, kräftiger Zähne wünschenswert, wenn die Leute meinen, man hätte nicht genug Herz, um zu weinen. Oder es liegt genau daran?!}. ,***!', dachte Amélie. ,Schhh!' hörte sie die drei Brüder.

Seth wischte sich unwirsch durch das zerkratzte Gesicht. ,,Ist schon recht!'' antwortete er. Dann beugte er sich runter zum versteckten Ohr des grossen Reptils. Sein Kopf war direkt neben dem riesigen Kiefer und den vielen, spitzten Zähnen, die auch aus dem geschlossenen Maul des Krokodils ragten. Sobek klappte sein Maul unruhig auf und zu, schliesslich war es Mittagszeit\footnote{Und wie sagt man bei Krokodils: Essenszeit macht dicker als Blut. Was heisst, dass sie kannibalischen Genüssen nicht abgeneigt sind, wenn dafür die Mahlzeiten eingehalten werden können.}

,,Wie geht es Osiris?''fragte Seth. ,,Ich habe ihn noch nicht gesehen, keine Ahnung!'' Sobek machte ein unbehagliches Gesicht. ,,Vatter, hör' zu!'' sagte er. ,,Ich kann Dir da nicht helfen!'' ,,Kannst du nicht, oder willst du nicht?'' knurrte Seth und packte das Krokodil fest am Unterkiefer und bog die Schnauze nach oben und starrte ihm in die kleine Augen, die wie wild rotierten und sowohl auf Seths Nase schielten, als auch panisch hinten in den Schädel rollten.

,,Immer auf die Kleinen, Seth?'' Amélie fuhr zusammen. Neben dem Busch in dem sie und die Horussöhne versteckt lagen, trat eine Göttin auf den Teich zu. ,,Neith!'' sagte Seth. Er liess seinen Sohn sanft in den Teich gleiten, wo dieser sich in die Mitte zurückzog. ,,Hei, Mom!'' sagte Sobek und klang wie ein Teenager, der froh ist, wenn seine Mutter im richtigen Moment auftaucht, um ihn von einer grossen Dummheit abzuhalten und gleichzeitig weiss, sie wird ihm den A***h aufreissen\dots aber mit etwas Glück wird es niemand erfahren\dots

,,Hi, mein Schatz!'' antwortete Neith. Sie schlenderte näher und aus dem Paar schlanker, kleiner Füsse, die gut manikürt waren und in grazilen Sandalen steckten, wuchs eine schlanke, kleine göttin vor Amélie empor. Sie sieht aus, als wäre sie von einer ägyptischen Tempelwand gestiegen, dachte Amélie.

Die schwarz-bleau schillernden Haare waren zu einem akkuraten Bob geschnitten. In den Haarspitzen trug sie goldene Perlen. Um die Haare zu bändigen, hatte sie ein roten Leinenstreifen um den Kopf gelegt. Ihre Augen waren mit Kohle dezent geschminkt. Sie war in ein Trägerkleid aus dem gleichen roten Leinen gehüllt, das einen feinen, weissen Spitzensaum oben und unten hatte. Ihre Arme, Handgelenke und Fesseln waren mit je zwei schweren Ringen aus Gold und Lapislazuli geschmückt, die leise und dumpf aneinander schlugen wenn sie sich bewegte. Es raschelte und Neith nahm einen grossen, hölzernen Bogen von der Schulter.

,,Willst du mir an den Kragen?'' fragte Seth seine Ex. ,,Nein, mein Lieber, ausser du versuchst wieder unseren Sohn zu Dingen anzustiften, die seiner zarten Konstitution schaden.'' antwortete sie und schaute zu Seth auf, der sie um drei Köpfe überragte. Amélie, die einen freien Blick auf die Beine der Göttin hatte, bemerkte das die hinter den langen Stoffbahnen des Kleides zitterten. Auch Amélies Arm zitterte, doch es war nicht ihr Arm der zitterte, sondern Duamutef, der dicht neben ihr flach an den Boden gepresst lag. ,Was ist, Tef?' dachte Amélie so leise sie konnte. ,ich kann Seth nicht ausstehen', antwortete der Schakal und konnte ein leises Knurren nicht unterdrücken.

,,Ach, herrjehh'' sagte Seth schliesslich betont gelangweilt: ,,Macht euch mal locker! Wir sehen uns!'' Er tippte sich leicht an die Stirn, als hätte er eine Kappe auf und verliess den kleinen Wald.

Als sie die Tür zum Haus hörten, kamen die drei Brüder und Amélie unter dem Busch hervor geklettert. ,,Ja, guck mal einer an!''schmunzelte Neith. Duamutef begrüsste sie besonders stürmisch. Er sprang an ihr hoch und schleckte ihr ins Gesicht, wie es unerzogene Caniden machen. ,,Pfff! Tef! Lass das!'' sagte die Göttin. ,,Das ist Amélie!'' stellte Amset vor. ,,Das Mädchen, das so laut denken kann, wie ein Elefant gross ist!'' sagte Neith und reichte Amélie die Hand, die sich erstaunlich kräftig und fest anfühlte. Aber dann fiel Amélie der Bogen wieder ein.

,,Sobi, dein Essen!'' rief Amset derweilen und der Krokogott kam eilends an das Ufer. ,,Endlich, Kinder! Yami, was riecht mein empfindliches Schnäuzchen?!'' Während Sobi sein Mahl genoss -Amset Amélie und Hapi füllten abwechselnd grosse Kellen mit dem Gänse-Fett-Saucen-und-vieles-mehr und schaufelten sie in das geöffnete Maul hinein, hatten Neith und Duamutef sich im Schatten der kleinen Bank nieder gelassen, die am Teich stand. Inzwischen wusste Amélie, dass die beiden miteinander Gedanken austauschten und auch, wenn sie den einen oder andere Gedanken selbst aufschnappen konnte, so reichte es nicht der Unterhaltung zu folgen. 

,,Neith ist Duamutefs Schutzgöttin. Jeder von uns Horussöhnen hat eine Göttin, die ihn berät und im Notfall hilft und beschützt.'' erklärte Amset Amélie, er hatte ihre Gedanken erraten. Amélie füllte ihre Kelle. ,,Was müsst ihr den machen, was so gefährlich werden kann?'' fragte sie stemmte die gefüllte kelle mit beiden Armen hoch und drehte si über dem weit aufgeklappten Krokomaul. Es platschte und schmatzte vergnügt.
 
,,Jeder von uns hat am Tag und in der Nacht eine Stunde in der er bei Osiris wache hält. Und dann müssen wir jede Nacht gegen das Monster kämpfen, bei diesen Dingen halt!'' ,,Ah!'' mehr sagte Amélie dazu nicht, denn sie konnte sich nicht vorstellen, wovon Amset sprach. Monster, Stundenwache\dots jede Nacht!? 

Aber da kam Duamutef zu ihnen. ,Neith wird heute Nacht dabei sein und will mir helfen, Seth zu befragen! Und Nehebkau!' ,Nehebkau? ja, der sollte etwas wissen, schliesslich ist er der Torwächter der Duat und einer der 42 Richter. Cool!' Amset strahlte Neith an, die zu ihnen getreten war. ,,Geht euren verletzten bruder besuchen!''sagte sie. ,,ich füttere den Sobi schon zuende!'' Sie nahm Amélie die Kelle aus der Hand und lächelte traurig: ,,Bis heute Nacht Elefantenmädchen!'' ,,Willst du nicht mit herein kommen, Neith? Kebi und Selket freuen sich bestimmt auf dich!'' ,,Das ist lieb gemeint, Amsi, aber ich werde hier draussen bleiben.'' 

,,Warum kommt Neith nicht ins Haus?'' fragte Amélie, als sie das Treppenhaus in den ersten Stock hinauf stiegen. Selbst Hapi machte seinem kleinen Bruder zuliebe eine Ausnahme und hangelte sich am Treppengeländer hinauf. Ab und zu keckerte er ängstlich, wohl war ihm nicht im Haus. ,,Wegen Sobek! Oder wegen Seth!'' antwortete Amset kryptisch. ,,Neith hat mit Seth Sobek zum Sohn bekommen, deshalb. Jeder der sich mit Seth einlässt ist verdächtig. Wobei es völliger Quatsch ist! Neith ist voll okay!'' sagte Amset. ,,Sie sieht aus wie eine ägyptische Diana!'' meinte Amélie. ,,Wer?'' ,,Vergiss es!''

\section*{3}
\addcontentsline{toc}{section}{3}

,,Was ist passiert?'' fragte Thot. Er hatte sich mit seinen langen, dünnen Beinen in einen der Sessel in seinem Bureau sortiert und die Finger zu einer Raute zusammengelegt. Re hatte in dem anderen Sessel einen Platz gefunden und hatte sich eine Pfeife angezündet. Auf seiner Sessellehne, leicht an seine Schulter geschmiegt, sass Maat. Sie spielte mit ihrer Feder, die sie über ihren Arm und den Kopf ihres Vaters wandern liess. 

Horus ging hinter den Polstermöbel an den Regalen entlang hin und her und blieb ab und zu stehen, um einen der vielen Gegenstände, die Thot um sich sammelte, zu betrachten. Auf dem Sofa sassen Hans und Berta.

Für Hans war es zu klein, seine Knie und sein Kinn kamen sich näher, als es der Gemütlichkeit zuträglich war. Für Berta war das Sofa zu hoch. Ihre Beine waren so kurz, das ihre Füsse geradeaus in die Luft ragten, wie bei einem Kind. Ihre Füsse steckten in  Hühnerpantoffeln und sie trug eine grüne Schürze, die mit kleinen Nudelhölzern und Fässern bestickt war. Auf den Fässern stand, wie Thot beunruhigt feststellte ,Whisky'.

 Hansens Hahn tappte um Bertas Pantoffeln, die auf und ab wippten, und versuchte die beiden stummen Schönheiten in ein Gegackel zu verwickeln. Der Dackel, der vor dem Schreibtisch ein ruhiges Plätzchen gefunden hatte, lag auf der Seite und schlief. Er schnarchte leise, wobei seine Lefzen hingebungsvoll vibrierten, chrrr\dots flappflappflapp\dots chrrr\dots
 
,,Wie ihr wisst, bin ich im Norden gewesen,'' begann Berta. ,,Wie erwartet traf ich Amélies Eltern in grosser Sorge vor. Nicht nur, weil Amélie über Nacht verschwunden war, sondern auch, weil es ihrem kleinen Bruder, der von schwächlicher Natur ist, schlechter ging als sonst. Ich konnte sie beruhigen, was Amélie betrifft, indem ich ihnen sagte, ich hätte Amélie in die Ferien geschickt, weil sie hier ungeahnte Kontakte knüpfen könnte. Aber sie wollen, dass sie zurückkommt, wegen dem kleinen Bruder.  Die Mutter sagt, sie brauche im Moment beide Kinder um sich.'' 

,,Wissen die Eltern von Amèlies Schwierigkeiten?'' fragte Thot. ,,Nein!'' antwortete Berta schlicht. ,,Aber warum?'' fragte Re. ,,Weil ich nicht weiss, was es ist, das Amélie bedrängt hat in ihren Träumen!'' sagte Berta. Sie fuhr fort: ,,Es waren keine richtigen Träume, dafür war die Reaktion von Amélies Körper, wenn sie aufgewacht war, zu stark. Hatte sie hier im Haus eigentlich einen solchen Traum?'' Thot blickte in die Runde, aber alle schüttelten den Kopf.

,,Berta, du hast doch eine Vermutung, oder?'' sagte Re. ,,Ich spüre doch, dass es einen Grund gibt, wieso du Amélie aus ihrem Haus fortgebracht hast. Wir alle kennen dich doch und wissen an dir kommt niemand so schnell vorbei! Nicht mal unser Seth! Also, raus damit! Wenn wir Amélie schützten sollen, dann müssen wir wissen wovor!'' Berta schaute jeden einzelnen an und rutschte unruhig auf dem Sofa. ,,Ich bin mir nicht sicher!'' sie schaute auf ihre abgebrochenen Fingernägel, die aufgeregt ,Wurzelbürste, Wurzelbürste!' riefen \dots Alle starrten sie an, sogar der Gockel liess von den Plüschhühnern ab und schielte zu Berta hoch: ,,Goooock?'' ,,Ich glaube,'' sagte Berta und alle sogen hörbar Luft ein, ,,ich glaube, es ist ein Fluch! Ein Fluch aus der Vergangenheit der zwischen Amélie und ihrer Familie liegt!'' 

Ein überraschtes Murmeln setzte ein. ,,,Ruhe!'' rief Thot und hob die Hände. ,,Willst du damit andeuten, Amélie könnte von ihrer eigenen Familie verflucht sein?'' ,,Nein!\dots Ja!'' antwortete Berta: ,,Ich will sagen, Amélie ist das Opfer eines Fluches! Und jeder Fluch ist ursprünglich von einem Menschen gesprochen worden!'' Die Götter verstummten und jeder hing seinen Gedanken nach. \footnote{Schliesslich kennen sich auch Götter mit Familie aus! In vielen Pantheonen gehört Familienflüche zum festen Bestandteil der Mythologie, sprich Familienchronik.} 

Ein vielstimmiges Seufzen erfüllte dann das Bureau, es gab niemanden im Raum, der nicht an die eigenen Familie-ist-die-Hölle-Situation dachte. ,,Berta, wie meinst du das?'' fragte Maat schliesslich. ,,Ich bin die Ordnung. Und die Familie ist der Ort, an denen ich einerseits immer vorhanden bin, egal wie es in der Familie aussieht und gleichzeitig ist es der Ort an dem auch meine Schwester Isfet, das Chaos und die Zerstörung ihren Ursprung haben.'' Re legte Maat eine Hand auf die Schulter. Sie strich sich die Straussenfeder über den Mund.

,,Es gibt unendlich viele Möglichkeiten, das ist das Problem!'' sagte Horus schliesslich. Er war hinter Thots Sessel stehen geblieben, stützte sich darauf und schaute seinem Freund über die Schulter. Dann stiess er sich ab und ging weiter hin und her. ,,Das Problem mit den Flüchen ist, dass praktische jeder seinen eigenen Krieg führt!'' sagte er. ,,Wir brauchen mehr Informationen!'' Horus war stehen geblieben und klatschte die Faust in die Hand. ,,Hast du eine Ahnung, wer den Fluch gesprochen hat?''

 Maat schaute mit grossen Augen zu ihrem Neffen auf. Thot schmunzelte, er liebte Horus, wenn er in den Kampfmodus umschwenkte und die Welt zu seinem Schlachtfeld machte. Egal, -welche Welt, egal, -welche Schlacht\dots Horus, der Falke, glitt durch die Lüfte und wenn er seine Ziel erspäht hatte, rüttelte er und stiess dann hinab\dots
 
,,Äähhm!'' Berta war, wie Thot amüsiert bemerkte, verwirrt. Aber dann schien ihr der Falkengott zu gefallen. ,,Nein! das Problem ist wohl, dass es ein alter Fluch ist. Ein Fluch der vor tausenden von Jahren gesprochen wurde. Sonst würdet ihr ja nicht damit in Verbindung stehen. Aber die Spuren sind verwischt, die Menschen sind gestorben und wieder geboren\dots '' Wieder mehrstimmiges Gemurmel. Horus schlug mit der Faust auf den Schreibtisch und dann in die Luft. Der Dackel, der einen tüchtigen Schreck bekommen hatte,  war im Nullkommanix auf den Stummelbeinen und kläffte empört. ,,Magie!'' murmelte Horus. ,,Priester!'' sagte Berta trocken. Re nahm die Pfeife aus dem Mund und grinste zu Thot. ,,Interessant!'' sagte Thot, der die Beine neu übereinander schlug, die Finger zur Raute legte und dann lächelte. 

,,Und? Wie weiter?'' fragte Re. ,,In der vierten Nacht stehen viele Magier zur Verfügung, die Auskunft geben können. Allen voran Seth!'' meinte Thot. ,,Ich werde die Söhne auf Trab bringen! ' sagte Horus und stürmte aus der Tür. ,,Ich werde es den Mädels und Mutter weitersagen.'' Maat kletterte von der Sessellehne und schlenderte aus dem Zimmer. So gingen sie ihrer Wege. 

Nur Berta blieb einen Moment auf dem Sofa sitzen. ,,Ich mache mir wirklich Sorgen, Thot!'' sagte sie und knetete das Glas mit der goldenen Flüssigkeit, das Thot ihr gereicht hatte, in den kleinen, aber starken Händen. ,,Was glaubst du, Berta? Wieso jetzt?'' ,,Ich weiss es nicht, vermutlich ist der Fluch einfach alt geworden\dots und Amélie ist sehr stark!'' ,,Mit Flüchen und Magie kennen wir uns doch aus,'' meinte Thot. ,,Ich glaube, Thot, ich glaube, es geht auf Leben und Tod! Und damit meine ich wirklich tot! Nicht über Reinkarnation gehen und 2000 Goldtaler oder Sünden einsammeln, sondern \dots Tod sein!'' Thot schwieg. ,,Ich bin nur froh, dass die Höhlen von heute Nacht sicher sind!'' sagte Berta. ,,Ja, das sind sie! Jedenfalls kann nichts von Aussen in sie hinein!'' antwortete Thot. Aber da sollte er sich täuschen. 

\section*{4}
\addcontentsline{toc}{section}{4}

,,Papa! Ich bin müde! Wann sind wir denn bei Amélie?'' fragte der blasse, schwarzhaarige Junge von der Rückbank des schwarzen, modernen Geländewagens. ,,Wilfried! Geht es nicht etwas schneller?'' fauchte die blonde Frau mit der eleganten Hochsteckfrisur, die in einem Chanelkostüm auf dem Beifahrersitz kerzengerade sass. ,,Luise, Schatz, ich fahre so schnell es geht!'' sagte der Mann am Steuer. Er war sportlich gekleidet in ein weisses Poloshirt der Marke mit dem Krokodil und Bluejeans. Er hatte die rotbraune Hautfarbe eines Südländers, die durch den Winter einen gräulichen Schimmer bekommen hatte und schwarze mit grauen Strähnen durchzogene, kurze Locken, die sich wie eine Kappe um seinen markanten Schädel legten. Er hatte eine kräftige Adlernase und die majestätische Haltung eines Mannes, der es gewohnt war, Dinge und Menschen zu führen.

,,Schätzchen! Kannst du denn noch? Wir machen bald eine Pause!'' sagte Luise und drehte sich zu ihrem Sohn um, der tapfer zurück lächelte. Der Junge hatte einige Schweissperlen auf der Stirn und hatte eine grosse, lange Klapperschlange aus Plüsch im Arm, die über seiner Schulter und der Rückenlehne lag und die nachfolgenden Autofahrer irritierte. ,,Es ist schon gut, Maman!'' antwortete der Junge leise. ,,Nichts ist gut!'' fauchte Luise. ,,Wie konnte Amélie uns das antun? Das Biest denkt nur an sich! und Berta! Die unterstützt noch ihre egoistischen Flausen!'' ,,es ist schon gut, Maman!'' murmelte der kleine Junge und lehnte seine heisse Stirn an die kühle Fensterscheibe.

,,Wirklich Wilfried, wenn das hier vorbei ist, sollten wir ernsthaft überlegen, ob wir Berta nicht rauswerfen!'' ,,Du willst die Frau, die dich grossgezogen hat und immer beschützt hat aus dem Haus werfen?'' fragte Wilfried. Er verriss das Steuer, es hupte und der Wagen, der sie überholte und den Wilfried knapp streifte, schoss an ihnen vorbei. Der Fahrer ballte die Faust. ,,Spinnst du! Wilfried! Glotzt auf die Strasse verdammt!''

Wilfried packte das Lenkrad fester, die Fingerknöchel wurden weiss. Er mahlte mit dem Unterkiefer. ,,Luise! Was ist los?'' Luise zuckte hilflos mit den Schultern, dann legte sie die schlanken, manikürten Hände vor das Gesicht und schluchzte leise. ,,Maman!'' Der Junge streckte seinen Arm so weit wie er konnte nach vorne und Luise ergriff die zarten, feingliedrigen Finger. ,,Ich wünschte wir wären zuhause,'' murmelte sie. 

Dann schlief sie ein. Sie hatte die gleiche zarte, sehnige Gestalt wie ihre beiden Kinder. Sie war eine auffallende Frau mit einer klassischen, geraden Nase und vollen Lippen. Jedoch waren die Wangenknochen sehr hoch und das Kinn sehr spitz. Ihr Gesicht bekam dadurch etwas faunisches, elfenhaftes. Wenn Luise lächelte wirkte sie schön, sobald sie ernst wurde, bekam sie einen harten Zug um den Mund und durch die schrägen, hellblauen Mandelaugen einen dämonischen Zug.

Wilfried dachte an seine Tochter. Er war auch traurig gewesen, als sie am Weihnachtsmorgen entdeckt hatten, dass Amélie nicht da war. Aber warum wollte Luise mit dem kranken Jungen Amélie unbedingt holen und dafür bis in die Schweiz fahren? Er warf einen Blick in den Rückspiegel, auch der Junge war eingeschlafen. 

\section*{5}
\addcontentsline{toc}{section}{5}

Amélie ahnte von all dem nichts. Sie sass mit Amset, Hapi und Duamutef auf einem grossen Baumstamm in Kebis Zimmer. Der Falke lag mit geschlossenen Augen in seinem Nest. Das Nest war in die linke Zimmerwand eingelassen, wie eine Felsnische. Es war mit Federn, goldgelben Stroh und Heu, das wunderbar nach Blumen roch, gepolstert.

Der Boden des Zimmers war mit einer Moos-Gras-Schicht bewachsen und die Wände grün gemalt mit einem Blättermuster. Innen vor dem Fenster war ein Ast quer angebracht, der Start- und Landeplatz, wenn Kebi von seinen Streifflügen zurück kam. Es war zugig im Zimmer, denn die Fensterscheiben waren ausgehängt.\footnote{Kein Falkengott hat es gerne, wenn er spät am Abend heim kommt und draussen übernachten muss, weil irgendein Idiot das Fenster geschlossen hat, oder es vom Wind zugeschlagen ist.}  Unter dem Ast war Sand ausgestreut, der offensichtlich als Toilette diente.

Für Gäste war an der rechten Zimmerwand ein dicker, breiter Baumstamm gelegt worden, der die ganze Wand ausfüllte. Es gab einige Schaffelle für die Caniden, Menschen und Götter. Amélie und Amset hockten zusammen auf einem Fell den Rücken an die Wand gelehnt. In der Mitte des Zimmers befand sich eine Baumscheibe, die, gross wie ein Lastwagenreifen, auf dem Boden lag und als Tisch diente.

 Neben der Nestnische war ein Regal angebracht und hinten links, neben dem Fenster auf dem Sandstreifen stand ein niedriges Sandsteinbecken, das zwei Vertiefungen hatte. Die eine war mit weissem, feinen Quarzsand gefüllt, die andere mit Wasser. Ein Falkenboudoir, dachte Amélie. Sie konnte sich nicht satt sehen. Der Querast hatte einige Astgabeln von denen aus kräftige Äste in das Zimmer ragten, die jedoch nach oben zeigten. So konnten sich Menschen und Götter nicht die Köpfe stossen. Auf den Ästen entdeckte Amélie schliesslich Reste von Ratten, Mäusen und einige Pommes frites. ,Cool! Ich komme mir vor wie Alice im Wunderland!' Amélie musste grinsen. ,,Wer ist Äliss?'' fragte Amset. Amélie erklärte es ihm und nahm sich vor das Buch für ihn zu besorgen.
 
Duamutef hatte sich ein Fell vom Baumstamm gezogen und es sich darauf auf dem moosigen Grasboden gemütlich gemacht. Er schnarchte leise. Hapi hing mit dem Schwanz am Landebalken und baumelte müssig hin und her. Immer wenn er in die Nähe der Pommes frites schwang, schnappte er sich eine und stopfte sie sich ins Maul.

,Sag dem Affen, er soll mein Essen nicht anfassen!' kam es schwach aus dem Nest. Kebi sah nicht gut aus, er lag auf dem Rücken und streckte seine Beine in die Luft. Auf seinem Bauch ruhte ein Ankhschlüssel aus Silber, besetzt mit Rubinen, grünem Türkis und Lapislazuli.  Neben seinem Nest stand Serket, seine Schutzgöttin. Sie hatte eine blaue Fayencebecher in der Hand, aus dem ein türkis-grüner Dampf aufstieg. Serket tropfte dem Falken mit einem zierlichen, goldenen Löffel die grünlich-golden schimmernde Flüssigkeit in den Schnabel. Sobald die Flüssigkeit darin verschwunden war, stieg eine Rauchfaden aus dem Schnabel auf. Kebi stöhnte hingebungsvoll.

Amélie stand auf und trat leise neben die Göttin, die sie nicht ablenken wollte. Die tropfte geduldig einen Tropfen nach dem anderen in den spitzen Schnabel und beobachtete dann ihren Patienten. ,,Er ist so klein, da braucht es nicht viel vom Lebenselexier!'' flüsterte sie Amélie zu. Amélie beobachtete wie  sich am Kopf des Falken ein goldenes Netz bildete und sich dann über Hals und Flügel wie eine Flüssigkeit ausbreitete. ,,Schau, die Lebenskraft kommt zurück!'' sagte Serket und lächelte, als auch die letzte Kralle des Falken an der Spitze golden aufblitzte.

Kebi drehte sich um, wobei der Ankh von ihm runterrutschte und hockte sich hin. Er schüttelte sein Gefieder. Serket strich ihm sanft an der Kehle entlang und er reckte den Schnabel, damit sie seinen Hals gut erreichen konnte.

Schliesslich setzten sie sich zu Amset auf den Baumstamm. Amélie fand Serket sehe Neith sehr ähnlich. Sogar ihr Kleid waren aus dem gleichen roten Leinenstoff gefertigt. Sie trug zusätzlich eine wollenes, weisses Tuch über den Schultern. ,Aber sie fühlt sich völlig anders an, als Neith!' dachte Amélie, während sie die Göttin scheu musterte. ,,Lasse dich nicht täuschen, Amélie,'' sagte Amset. ,,Serket und Neith haben sehr unterschiedliche Aufgaben, aber kämpfen können beide!'' ,Jawohl, Serket ist die stärkste\dots' fügte Kebi hinzu. ,,Hört auf, Jungs! ich weiss es zu schätzten, aber wir wollen doch keinen Streit, oder? Vergesst nie, auch Göttinnen sind eitel!''\footnote{Davon können z.B. auch die Bewohner des ehemaligen Troja ein Lied singen! Mindestens, vermutlich zwei oder drei\dots}

,,Kebi, was ist passiert?'' fragte Amset. ,,Dich haut doch sonst nichts so schnell um?'' ,Ich weiss es nicht, Amsi! Es war, als würde eine dunkle Wolke aus Norden auf mich zu rasen. Ich wurde hin und her geschleudert, obwohl ich, sobald ich die Wolke gesehen hatte, umdrehte.' ,,Irgendetwas hat dir fast deine ganze Lebenskraft abgesaugt.'' sagte Serket. ,,Das kann auch für Götter gefährlich werden. Du hast Glück gehabt. Aber viel hätte nicht gefehlt und es wären nur noch drei Horussöhne übrig!'' 

Hapi liess vor Schreck den Ast los und fiel prompt auf den Kopf. Duamutef sprang hoch und setzte sich sogleich wieder und jaulte auf. ,Was meinst du damit?' Kebi flatterte aus dem Nest und setzte sich auf Serkets Schoss. ,Wie! Nur noch drei Horussöhne?' Serket sprach sehr leise und die Horussöhne und Amélie rutschten eng zusammen. Duamutef legte seine Schnauze auf Amélies Beine und sie legte ihm die Hand auf den Kopf. ,,Es gibt sehr, sehr seltene Ausnahmen, bei denen auch Götter sterben können!'' flüsterte Serket. ,,Und was immer das gestern war, war eine solche Ausnahme!'' Sie schauten sich gegenseitig an und blickten dann auf Kebi, der in ihrer Mitte sass. Es schnürte Amélie die Brust zusammen, als sie in die dunklen, glänzenden Augen des Falkengottes sah. Zum ersten mal, seit sie mit den Horussöhnen zusammen war, spürte Amélie ein neues, unbehagliches Gefühl. Es dauerte kurz, bis sie merkte, was es war: Angst!

Amélie stand auf und ging an das Fenster. Sie bückte sich unter dem Landebalken durch. Es war schon der erste Dämmerschein im Licht. Die nächste aufregende Nacht brach bald an. 

Auf dem Dach des blauen Hauses lag ein unregelmässiger Schatten. Und als Amélie an das Fenster trat, rutschte der Schatten über das Dach auf das Fenster zu. Er schwappte wie eine Flüssigkeit über den Rand der Dachrinne und fiel. Ein Tropfen des Schattens fiel auf Amélies Handrücken.

Erschrocken zog Amélie die Hände zurück. Obwohl sie nichts erkennen konnte, wische sie die Hand an ihrem Kleid ab. ,,Was ist?'' fragte Amset. Er wollte seinen Arm um sie legen, aber Amélie wich aus. ,,Nichts! Nichts ist!'' Sie wusste ja selbst nicht, wieso sie plötzlich wie gelähmt war.

\chapter*{4. Nacht}
\addcontentsline{toc}{chapter}{4. Nacht}


\begin{quotation}

\emph{IV Pater eius est Sol, mater eius Luna. Portavit illud ventus in ventre suo. Nutrix eius terra est. \\4. Die Sonne ist sein Vater, der Mond seine Mutter. Der Wind hat es in seinem Bauche getragen. Die Erde ist seine Ernährerin.\\Tabula Smaragdina}

\end{quotation}



\section*{1}
\addcontentsline{toc}{section}{1}

Auf dem Hof war es dunkel. Das leise plätschern des Brunnens, der Sobeks Teich mit frischem Wasser fütterte, tönte harmlos. Die Palmen und Büsche bewegten sich sachte in lauem Wind. Die Gruppe der Abenteurer sammelte sich. Als letzter kam Thot durch die Tür, er trug eine Tasche bei sich. ,,Ich habe hier etwas, für den äussersten Notfall. Ich hoffe sehr, keiner von uns muss das Gerät einsetzten!'' Amélie reckte neugierig den Hals. Thot griff wie der Weihnachtsmann in seine Tasche und- holte kleine, runde Taschenspiegel raus! ,,Wie gesagt, setzt ihn nur im Notfall ein!'' wiederholte er und drückte jedem einen Spiegel in die Hand. 

Da auch Götter bis zu einem gewissen Grad lernfähig sind, hatten sie sich Kleidung angezogen und Gummistiefel! \footnote{Nicht, dass wir uns hier falsch verstehen! Prinzipiell sind alle Götter immer lernfähig! Allerdings sind sie, was jeder, der eine Zeit lang die Schule besuchte, verstehen wird, nicht immer lernwillig! Und im Gegensatz zum Menschen, steht bei den Göttern Lernkompetenz nicht auf der Liste der wichtigsten Soft Skills.} Wenn sie gehofft hatten mit den Kleidern weniger aufzufallen, so klappte der Plan nur halb. Die Menschen, die ihnen in der Dämmerung begegneten, schauten verblüfft bis ängstlich drein, was vor allem an Hans und Horus liegen mochte.

Die beiden riesigen, kräftigen Götter hielten nichts von lästigen Stoff und da es die Nacht turbulent werden würde, hatte sich Hans für eine rotbraune und Horus für eine schwarze Lederhose entschieden. Sie trugen dazu riesige, gelbe Gummistiefel, wie sie Schiffer benutzten. Allerdings hatten sie selbst Berta und Hathors Drohungen nicht dazu bewegen können mehr anzuziehen. 

Hans hatte darauf bestanden den Dackel mitzunehmen. Die Pfoten des Dackels steckten in Plastiktüten und mit vielen Metern Klebeband umwickelt. Auch der lange, schlanke Bauch des Hundes war mit Klebeband und Plastik bearbeitet worden. Wenn der Dackel mit dem Schwanz wedelte, raschelte es. \am musste grinsen, allerdings nur solange, bis sie blutrünstigen Blick des Dackels bemerkte, er knurrte leise. \am wusste, jemand würde die Nacht mit einer Bisswunde beenden. Sie stellte sich unauffällig neben Berta\dots

Die gebräunten und gestählten Oberkörper der beiden Götter glänzten matt im Licht der Strassenlaternen und ihre Gummistiefel  schlurfzten über das Pflaster. Die Pfoten des Dackels raschelten auf dem Boden. Horus trug zwar keinen wilden Bart wie Hans, aber seine Brust und sein Rücken waren mit einem zarten Federflaum überzogen, der die Färbung des Falkengefieders trug. Seine Haare waren grau, seine Kehle und Brust beige mit braunen Punkten und der Rücken rotbraun mit dunkelbraunen Punkten. In einem Wald wäre das eine gute Tarnung gewesen, auch als Mensch, in der Stadt jedoch \dots

Thot hatte seinen grauen Wollmantel wieder angezogen und trug seine massgeschneiderte Hose. Die Füsse steckten in schwarzen, schmalen Gummistiefeln, die an Reitstiefel erinnerten. Berta hatte ihr schwarzes, knöchellangen Kleid, gerafft, indem sie den Saum in den Gürtel ihres Lederschurzes steckte. So hatte man freie Sicht auf ihre kräftigen Waden und die rot-weiss karierte Gummistiefel. Sie sah aus wie Humpdi Dumpdi auf dem Weg zum Schwingfest.

 Amélie, Maat und Amset waren in Jeans geschlüpft und trugen dicke graue Kaputzenshirts und Gummistiefel. So wirkte es auf den einen oder anderen Passanten, als seien drei Teenager mit ihrem Hund von einer Horde verrückter entführt worden, oder einige Charaktere seien aus einem Computerspiel geschlüpft. Tefs Pfoten waren wie die des Dackels in Plastiktüten verpackt und mit Klebeband umwickelt worden.

Thot, der weise Meister, führt seine Truppe durch die schmalen, versteckten Nebengässlein, von denen es in Basel, den Göttern sei Dank, viele gibt, zum Gerberberglein.

Auf dem kleinen Platz blieben sie stehen. ,,Mr gönt einzln durch d Tür!'' sagte Hans. ,,Und glii dahintr stoo bliibe! Glii im Gang stoo bliibe!'' Hans ging in die Nische in der der Brunnen eingelassen war. Amélie hörte ihn rumoren und leise fluchen. ,,Isch alles igroschtet do! Mmnnngh!'' Sie hörten es schaben und dann knirschte Stein auf Stein. Ganz leise und dumpf. ,,Berta! Gehe du als erste und nehme Maat mit,'' sagte Thot. Berta zog Maat über den kleinen Platz und verschwand mit ihr in der Nische. Maat sträubte sich, dies war ihr erster Ausgang in diesen Ferien und sie hatte keine Lust so schnell wieder an einen geheimen Ort zu verschwinden.

Als die beiden nicht mehr zu sehen waren, schickte Thot Amélie, Duamutef und Amset. Horus und er schlenderten gleich hinterher. 

Als Amélie in der Mauernische ankam, sah sie, dass der Brunnen an einer grossen, dicken Steintür angebracht war. Der Brunnen und die Tür waren nach aussen geöffnet, dadurch war die Tür nur einen engen  Spalt weit offen. Der Metallwasserhahn, der in der aufgeklappten Wandtür steckte und den Brunnen mit frischem Wasser versorgt, war fast in der Mauer verschwunden. ,Das muss der Türknopf sein!' dachte Amélie. ,man muss den Wasserhahn in die Wand drücken!' 

Als sie den Gang betrat, war es stockfinster. ,,Autsch, geh' von meinem Fuss!'' flüsterte Berta. Amélie streckte die Hände vorsichtig aus und tastete an den Reisegefährten vorwärts in den Gang. Als sie bei Hans angekommen war, flüsterte er: ,,Stop! Warte!'' Schliesslich spürte Amélie Thot an sich vorbeistreifen, er roch zart nach dem After Shave mit dem Segelschiff. Amélies Bauch zog sich zusammen, denn ihr Vater benutzte dasselbe. ,Ob ich meine Eltern je wiedersehe?' dachte Amélie und für einen Moment wollte sie weinen. 

Horus betrat als letzter den Gang und Amélie hörte, wie er die schwere Steintür schloss. ,,Also!'' flüsterte Hans. ,,Keis Mücksli, will ich ghöre und untr kei Umständ s Licht gse!'' Amélie schauderte. ,,Amélie, gib mir deine Hand!'' Es war Thots Stimme. Amélie spürte wie Duamutef sich an ihrem Bein vorbeidrängelte und Amset drückte kurz ihren Arm. Er roch nach Holz, aus Kebis Zimmer und nach Sonne. Am liebsten wäre Amélie mit ihm mitgegangen.

Amélie vermutete ganz zu recht, dass sie den meisten Schutz brauchte und Thot am wenigsten Wert auf einen Kampf legte. Berta zog an ihr vorbei mit einem Hauch Bratfett und Erde. Amélie spürte das Nudelholz an ihrem Arm. Maat blieb hinter ihr. ,,Es ist so aufregend!'' flüsterte sie fröhlich. ,,Hast du keine Angst?'' fragte Amélie zurück. ,,Was? Wieso?'' fragte Maat.  ,,Vergiss es!'' sagte Amélie. Horus bildete das nervöse Schlusslicht. Der Platz war ihm sehr ungewohnt und ohne es zu merken, schob er die anderen vor sich her, indem er sich reckte und versuchte rechts und links an ihnen vorbei zu spähen, um irgendetwas in der Finsternis zu erspähen.

 \section*{2}
\addcontentsline{toc}{section}{2}

Plötzlich befanden sie sich in einer niedrigen Höhle. Obwohl es stockfinster war, konnte Amélie es am Schall ihrer Schritte hören. Ausserdem war etwas in der Höhle! Jemand, der wenig Wert auf Hygiene zu legen schien, den es stank! Amélie sog vorsichtig den Geruch ein, er erinnerte sie an etwas. ,,Das riecht wie Hühnermist!'' flüsterte sie schliesslich. ,,Gut! Nicht ganz exakt, aber gut!'' flüsterte Thot zurück.

Es war einen Augenblick erwartungsvoll still. Die Gruppe regte sich nicht und das Wesen in der Finsternis\dots es atmete leise und tief, es schlief. Die Stille streckte sich einen Moment zu lange, die Anspannung der Gruppe vibrierte. Die leisen Atemzüge stotterten und es schmatzte und gähnte. Dann war es wieder mucksmäuschenstill, allerdings lauschte ein Ohrenpaar mehr.

,,Wir müssen weiter!'' flüsterte Horus. ,,Wiee denn? Mr wüsse nit, wo s Viech isch!'' flüsterte Hans zurück.  Es raschelte leise. ,,Psst!'' flüsterte Hans. Es klatschte kurz auf, wie Leder, das auf Leder gepatscht wird. ,,Er muss links von uns sein!'' flüsterte Thot. ,,Er liegt in dr Mitte dr Höhli. Abr dr Durchgang isch uf dr rechte Siite. Mr müend uns rechts an dr Wand lang! Abr: Nüt aalange!'' flüsterte Hans. Amélie spürte wie die fremden Ohren in der tonlosen Finsternis jedes Wort aufsaugten. ,Wir könnten ein Megafon benutzten!' dachte sie. ,Was ist es? Hühnermist und Leder? Cocklan, das Hühnerbarbar?' ,,Pfff!'' kicherte Thot.

,,Kömmet!'' Amélie spürte die Bewegung Hansens mehr, als sie zu hören. Der wilde Mann konnte erstaunlich leise sein. Die Horussöhne schlichen hinterher, dann Berta und Maat. Schliesslich zog Thot an Amélies Hand. Vorsichtig setzte sie einen Schritt vor den anderen.  Mit der rechten Hand streifte sie die Wand. Diese war kalt, glitschig. Hier und da knirschte es unter einem Fuss. Amélies Nackenhaare sträubten sich. Sie spürte wie das fremde Wesen nach ihnen witterte. Bei jedem kleinsten Geräusch, das die Füsse machten, raschelte es in der Mitte der Höhle.

Die leisen Schritte der Gruppe bekam ein Echo. Tapp -Tapp\dots Taptap -Taptap\dots ,,Schnell! Er dörf niemand vo uns brüühre!'' rief Hans. Ein lautes Geraschel und Schaben begann. Amélie lief neben Thot her. Dann trat ihr Fuss auf etwas, das wegrutschte und Amélie stürzte auf den Boden. Instinktiv liess sie Thots Hand los, um den Sturz abzufangen. Ihr Finger berührten einen feuchten Knochen und sie schrie entsetzt auf. 

Thot war zwei Schritte weitergestolpert. Er hatte Amélies Hand verloren! ,,Mist!'' fluchte er. Verzweifelt tastete Thot um sich, bis das rascheln und schaben einsetzte. Es war nahe! Es war mucksmäuschenstill. Thot wagte keine Bewegung, um sich oder \am nicht zu verraten. 

Amélie kroch auf allen Vieren weiter. So leise, wie sie konnte. Ihre Hände glitten über den Boden. Er war übersät mit feuchten, stinkenden Knochen und Unrat. Amélie ekelte sich bis ins Mark. Sie schluchzte. Die Wand, die Orientierung war verschwunden. \am geriet in Panik, kroch sie dem Wesen entgegen, statt sich zu entfernen. Ihre Ohren rauschten. Sie blieb gelähmt sitzen. Ihre Hände wurden eiskalt. Sie hörte Hansens Stimme in ihrem Kopf warnen: ,,Nüt aalange!'' Sie zitterte. Sie hörte ein neues Geräusch: Es waren ihre Zähne, die klapperten\dots  

Ich bin allein! Ich, ich bin allein! Amélie spürte wie sie den Boden verlor, sie rutschte und fiel! Ich komme nicht raus! Sie fiel, bodenlos\dots Sie war verloren! Würde sie je aufschlagen? Ich bin allein! Sie werden ohne mich weitergehen! Natürlich, Du Nichtsnutziger Klops! Die Stimme kam aus ihrem Kopf, wie auch das Verächtliche Seufzen. Sie schluchzte. Eine Träne lief über ihre Wange. Amélie spürte die Feuchtigkeit. 


Dann hörte sie eine andere Stimme. ,,Amélie!'' Zu Amélies Entsetzten aus einer völlig anderen Richtung. ,,Thot!'' rief sie und verstummte sofort, als das lederne Geräusche ertönte. Das Vieh, wie Hans es genannt hatte, kam direkt auf sie zu. Instinktiv bewegte Amélie sich im Schneckentempo von dem schlurfenden Geräusch weg. Schliesslich berührte ihr rechter Fuss die Wand. Sie kauerte sich zusammen und wagte kaum zu atmen. Wo blieben sie nur, die ganzen Helden und Götter? Sie spürte für einen Moment, wie sich der Boden unter ihr wieder auftat\dots Warum sollten sie dich retten, Dummes Ding? Da bohrte sie sich die Fingernägel in den Arm bis die Haut riss.

Auf der anderen Seite der Höhle hörte sie wie die Stimmen leise murmelten. ,, Wir müssen Licht anmachen!'' ,,Nei! Nit! Bisch vrruckt! Was wenn Amélies s Viech aluegt?' ,Ich finde sie auch ohne Licht!' meinte Duamutef. ,,Mein Sohn, ich bin stolz auf dich, aber du kannst nicht alleine gehen!'' sagte Horus fröhlich. Thot seufzte, sein Freund Horus war vermutlich der einzige, der aufblühte, sobald er in Gefahr geriet.

,,Ihr dörft s Biest nit berühre! Niemals! s isch giftig, au für Göttr! Also los!'' Während sich Horus, Hans, Berta und Amset am Rand der Höhle entlang verteilten, suchte Duamutef nach Amélie: ,Amélie? Hörst du mich?' fragte er. ,Tef!' Amélie schniefte. 

,Hab dich!' Der Schakal stupste sie an und Amélie schrie. Der Schrei hallte in dem Gewölbe vielfach wider. Und das Monster blieb stehen! Die Götter, die sich rings herum im Raum verteilt hatten, schrien ihrerseits. Es hallte, tönte und schepperte! Das Monster schrie\dots Es war ein einziges Tohuwabohu

,Gut gemacht! Amélie! Halt dich an meinem Schweif fest!' meinte Duamutef. Amélie nahm zaghaft die Schweifspitze in die Hand und so kamen sie bei Thot und Maat an, die am anderen Ende der Höhle im Gang warteten. Thot pfiff, kurz und schrill. Während die vier Befreier sich zurück tasteten, fluchten sie. Das Wesen war still. Amélie dachte an ein grosses Huhn in einer Rüstung, das in der Mitte der Höhle sass und sich die Flügel auf die Ohren presste\dots 

\section*{3}
\addcontentsline{toc}{section}{3}

,,Hier müend mr uns trenne'' sagte Hans. ,,Die Söhne und ich steigen in die obere Höhle,'' sagte Horus. ,,Jo, Maat, Berta, Amélie und ich müend zur Quelle, wir gehen den Gang weiter '' meinte Hans.

Hans drückte Amélie ein dünnes Lederband in die Hand, das er an seiner Hose befestigt hatte. ,,Lass mr nit s Bändeli los!'' sagte er. Dann stapfte er in den dunklen Gang voraus. Amélie wickelte das Band mehrmals um die Hand und folgte. Die Haut an ihrem Arm brannte, wo die Fingernägel kleine, nässende Wunden hinterlassen hatten. Hinter sich hörte Amélie Bertas Stiefel schlurfen. Amélie hatte einen Klos im Hals. Sie konnte ihn nicht runterschlucken, er war sofort wieder da.

Warum war Berta so lange weg gewesen? Warum war sie von ihrer Familie getrennt worden? Zum ersten mal seit Amélie bei den Göttern war, spürte sie Angst. ,,Was ist in der Höhle passiert?'' fragte Berta plötzlich. ,,Wie meinst du das? Ich bin gestolpert!'' antwortete Amélie. ,,Ich meine danach. Ich konnte dich plötzlich nicht mehr spüren!'' antwortete Berta. Amélie fasste sich an ihren Arm. ,,Ich weiss nicht, was du meinst!'' sagte sie und war froh, dass es dunkel war und Berta ihr Gesicht nicht sehen konnte.

Sie waren alle drei in Gedanken gesunken. Keiner von ihnen hörte das leise Rascheln von Leder, das ihnen in einigem Abstand folgte.

\section*{4}
\addcontentsline{toc}{section}{4}

Der Junge war im Hotelzimmer, das sie kurz hinter Frankfurt in einem kleinen Ort genommen hatten, erschöpft eingeschlafen. Luise lag neben ihm auf dem Bett und hatte die Augen geschlossen. Wilfried vermutete, dass sie wach war, aber nicht mit ihm reden wollte und deshalb tat , als schliefe sie. Er konnte sich den Seufzer nicht verkneifen.

Leise schlich er aus dem Zimmer. Auch die alte Zimmerwirtin, die dem Jungen so freundlich eine Suppe gekocht hatte, hatte sich zurückgezogen. Der untere Stock war dunkel. In den kleinen Gastraum, der schon für ihr Frühstück eingedeckt war, schien fahles Mondlicht.

Wilfried liess sich schwer auf einen Stuhl sinken. Alles, alles, sein ganzes Leben geriet aus den Fugen\dots Seine Tochter war verschwunden, der Sohn krank, die Frau, ein keifendes Nervenbündel, aber all das, war nichts gegen die Dinge, die passierten, sobald er einschlief\dots

Sobald er einschlief zerriss ein unglaublicher Schmerz sein Herz. Er wunderte sich schon, warum er noch lebte. Genauso wie er sich wunderte, wie er all die Wochen und Monate ohne Schlaf ausgehalten hatte. Er war inzwischen wie eine Katze, die überall für einen kurzen Moment völlig entspannen konnte, ohne jedoch die Augen ganz zu schliessen. 

,,Selbst die alte Hexe weiss nicht, was machen!'' murmelte er vor sich hin. Wo war sie? Berta, die alte Hexe? Schliesslich war es ihr Job sich um die Kinder zu kümmern und der Junge war krank! Ich muss sie rauswerfen! Bisher hatte sich Luise strickt geweigert, über Bertas merkwürdiges Verhalten zu reden. Berta war nicht nur \am Amme, sondern auch ihre gewesen. ,,So jemanden wirft man nicht vor die Tür!'' hatte Luise gesagt. ,,Auch, wenn so jemand merkwürdige Dinge tut und seine Pflichten vernachlässigt?'' hatte er erwidert\dots aber Luise hatte seine Worte mit der Hand weggewischt.

Er schreckte auf, als sein Kopf auf die Brust fiel. Für einen Bruchteil einer Sekunde hatte er wieder das Bild gesehen. Er war in einem grossen tempelartigen Bau. Um die Hüfte und Schulter trug er ein Leopardenfell. Vor ihm befand sich eine Art Altar auf dem eine Gestalt unter weissen Leinen lag\dots 

Der Schmerz im Herzen holte ihn zurück in das vom Mondlicht erhellte Zimmer. Wenn es jedenfalls ein Traum wäre, dachte er bitter, denn er wusste es inzwischen besser. Es war kein Traum, es war eine Erinnerung!

Wilfried sehnte sich nach Ruhe. Aber die Schmerzen, die ihn befielen, sobald die Erinnerung auftauchte, waren eindeutig\dots Er würde sie nicht überleben\dots

\section*{5}
\addcontentsline{toc}{section}{5}

,, \eighthnote \twonotes{} Das Waaandern ist des Goooottes  Lust,  \eighthnote \twonotes{} das Waaandern ist des Gooottes Lust, das Wahandern \eighthnote \twonotes{} Ehes muss ein schlehechteher Gohott sein, \eighthnote \twonotes{} dehem niiiemals fihiel dahas Waaaandern ein\eighthnote \twonotes{} \dots '' Horus marschierte vorne weg und sang aus voller Kehle. Amset hatte es aufgegeben auf verdächtige Geräusche zu lauschen und Tef stand kurz vor einem Nervenzusammenbruch. Nur seine Körperbeherrschung und Selbstdisziplin, die er sich in den tausenden von Jahren als Grabwächter angeeignet hatte, hinderte ihn daran winselnd davon zu springen.

,,Vater! Meinst du nicht, wir sollten in diesen unbekannten Höhlen in denen bekannte Unannehmlichkeiten wie Onkel Seth warten, etwas leiser sein!'' fragte Amset. Tef tat ihm leid, ihm musste schier der Schädel platzen bei dem Gejohle ihres Vaters. ,, \eighthnote \twonotes{} Daahas Waaa\eighthnote \twonotes{} \dots '' Horus drehte sich zu seinen Söhnen um. Tef seufzte tief, aus der Richtung aus der sie gekommen war, tönte ein Echo des Seufzers. ,,Tschuldigung, mein Junge! Hab vor lauter Akustik deine zarten Öhrchen vergessen,'' murmelte Horus. Für einen Moment war es still, er schien zu lauschen. Sie gingen weiter.

Im selben Gang, einige Biegungen hinter ihnen, seufzte es ein zweites mal! Das Wesen raschelte mit den ledernen Schwingen. Es schüttelte sich leise, um das letzte Klingeln aus den Ohren zu vertreiben. Dann schabten die Krallen leise über den feuchten Boden und folgten den Göttern\dots 

,Hoffentlich sind Neith und Serket da!' meinte Tef nervös. Normalerweise machte ihm die Dunkelheit in Sokars Reich nichts aus, da er sich auf seine Nase verliess. Aber hier stank alles metallisch, faulig und ätzend nach diesem Vieh, das in der Höhle hauste. Hier war er wirklich blind.

Das Geräusch ihrer Schritte veränderte sich, der Gang öffnete sich zu einer niedrigen, runden Höhle und führte auf der anderen Seite weiter. ,,Halt! Wer da?'' hörten sie Neiths Stimme. ,,Wir sind es, Neith! Amsi, Tef und ich!'' rief Horus. ,,Ich bin froh, deine Stimme zu hören!'' sagte Neith. ,,Sobi und Nehabkau sind da! Seth fehlt noch!'' ,,Hallo, Jungs!'' rief Serket hinten aus der Höhle. Sie bewachte den Ausgang.

Sobek hatte sich für seine menschliche Variante entschieden. Wer möchte schon mit dem Bauch durch fauligen Schleim und Unrat robben? Sein Kopf war der, eines Krokodils\dots eines Krokodils mit menschlichem Körper in Lendenschurz und Gummistiefeln!
 
,,Salut, Sobi! Nehebkau! Alter Junge, hast du dir ein sicheren Platz ergattert?'' fragte Horus. ,,Ich trage ihn auf den Schultern'' antwortete Sobek. Die Schlange hob ihren Kopf und zischte leise. 

\sterne
Nehebkau war nicht nur eine riesige Schlange, sondern ein mächtiger Gott. Er war der Sohn von Geb und Serket\footnote{Siehe irgendwo die Fussnote weiter vorne, zum Thema Familienverhältnisse des altägyptischen Pantheons!}.

Als Seth die Erde betrat, setzte er die Spaltung in Gang. Die Einheit zerfiel in eine Vielheit. Seth, der Gott des Windes brachte der Erde die Luft und dem Mensch die Atmung: ein-aus, ein-aus, ein-aus, ein-aus, rechts-links, oben-unten, hell-dunkel, gut-schlecht\dots Leben-Tod! Ja, Seth brachte den Tod. Und Serket, die Göttin, ,die atmen lässt' und Geb, Gott der Vegetation, des Lebens und der Erde, zeugten einen Sohn, Nehebkau. Nehebkau die Schlange, die die Lebenskräfte von allen lebendigen Wesen, Göttern wie Menschen in sich barg und bewahrte. Nehebkau, der unter seinem  Künstlernamen heute besser bekannt ist und immer noch wirkt: der grosse Kundalini!

\sterne
 
,,Wo ist Onkel Seth?'' fragte Amset und schaute sich misstrauisch um. Obwohl er jede Nacht Seite an Seite mit seinem Onkel kämpfte, blieb ein gesunder Zweifel. Seth war kein zwiespältiger Typ, kein verhinderter Held und kein Antiheld, der mit den nötigen Therapiestunden wieder ,in Ordnung' kommen würde! Seth war böse! Seth war das Böse. Seth war der Gott des Krieges, der Wüsten und des Donners. Er war der Herr all dessen, was dem Menschen das fürchten lehrte. Und aus diesem Grund, gab es Menschen, die ihn anbeteten, weil sie als Könige, Kriegsherren und Generäle von der kriegerischen Macht des Gottes kosten wollten.\footnote{Solche Deppen finden sich überall auf der Welt. Sie meinen, wenn sie auf der Seite des Bösen stehen, sind sie auf der sicheren Seite! Aber für das Böse gibt es keine Seite! Für das Böse ist jeder ein Feind, ein Opfer, oder beides. Das Böse ist böse, weil es sich von allem getrennt fühlt, so  hat es keine Seite an die sich jemand stellen könnte. Allerdings heisst das nicht, das Böse könnte einen Deppen, der ihm in den Hintern kriecht, nicht für seine Pläne gebrauchen!}

Seth war der Herr der Materie, der Kälte und der Hitze, des Feindes und des Todes. Ein hochintelligenter Magier, der die stärksten Feinde der Götter besiegen konnte, aber eine einzige Sache vergessen hatte: die Liebe! Was ihn persönlich am A*** vorbei ging! Zitat Seth: ,,Liebe? He, Heulsuse! Frag' Nietzsche! Hahaha!!'' -Man bemerke: Zwei Ausrufezeichen! => Noch nicht, aber kurz vor dem Wahnsinn! ,,Für den Weiberkram ist mein Brüderchen Osiris zuständig. Das Weichei!'' sagte er.\footnote{Es war kompliziert. Seth hatte sich als mächtiges Geistwesen das er war, vor Urzeiten von der Einheit der Schöpfung abgespalten. Seths Tat an sich war im Einklang mit dem Rat der Götter. Er war auserwählt sich zu opfern und aus der Einheit allen Seins heraus zu treten. Was nicht nach Plan gelaufen war, war der Zeitpunkt. Seth hielt das Timing nicht ein, er handelte plötzlich eigenmächtig, -oder gehorchte er nur seiner neuen Natur? Das brachte der Erde den Tod und Seth die abgrundtiefe Einsamkeit.} 

,,Hier bin ich, mein Junge!'' flüsterte Seths Stimme in Amsets Ohr. Amset stand wie versteinert. Alle Nackenhaare waren gesträubt, er atmete flach. Eine Schweissperle löste sich und ran über seine Schläfe. Tef entblösste seine Reisszähne, verkniff sich aber jeden Mucks.

,,Haha! Seth! Immer für eine Überraschung zu haben!'' flüsterte Horus in Seths Ohr. ,,Du weisst ja, Neffe, das Böse schläft nie!'' antwortete Seth. ,,Wohl wahr!'' sagte Horus und schob sich sanft zwischen seinen Söhne und seinen Onkel. 

Den beiden Göttinnen war unwohl. Sie bewachten die zwei Zugänge der Höhle und mussten Seth den Rücken zuwenden. ,,Könnt ihr zur Sache kommen, Jungs?'' fragte Neith. 

Sie begannen das nächtliche Ritual der Erneuerung. Sobek, der Gott des Wassers und der Fruchtbarkeit, half Nehebkau die Lebenskraft von Göttern, Lebenden und Toten zu ordnen. Die Lebenskraft der frisch Verstorbenen lösten sie vom Leichnam und wiesen ihr die richtigen Platz zu. Die Lebenskraft der lebenden Menschen, die schliefen, erneuerten sie und die auch die der Götter wurde überprüft.

Seth überwachte den Vorgang, damit nicht zu viel und zu wenig Lebenskräfte das Gleichgewicht von Leben und Tod durcheinander brachten. Serket und Neith hielten Wache. Sie wachten nicht nur über die drei Götter, sondern auch, dass sie in dem sensiblen Augenblick nicht gestört wurden. 

Nehebkau, die grosse Schlange mit dem grün-golden Schuppenkleid begann sich zu recken. Die mächtige Schlange stützte sich mit ihrem Vorderteil auf die starken Schultern Sobis und reckte ihre Schwanzspitze im Uhrzeigersinn im Kreis herum nach oben. ,,Schau!'' flüsterte Amset plötzlich. Tef zuckte zusammen, dann bemerkte er das Licht. Durch die Decke der Höhle spiralte ein zarter Nebelschleier, der leuchtete. Er hatte sich mit dem Schwanz des Nehebkaus verbunden. Die Höhle war, wie sie nun sahen, hoch. Und in der Mitte stand Sobek mit Nehebkau auf den Schultern. Die Schlange bewegte nur noch die ihr Ende, das mit dem Nebellicht verbunden war.  Das Licht war schwach und die Schlange und der Gott mit dem Krokodilskopf erschienen wie ein grosser Schatten. Die Decke der Höhle schimmerte wie unter Wasser.

,, Die Leonhardskirche ist genau über der Höhle. Die Lichtsäule mit der Ka-Kraft kommt aus der Kirche direkt durch das Gestein\dots '' flüsterte Amset. -,Das ist gut!' bmerkte Tef, ,dann geht es, hoffentlich, schneller als auf Sokars Sandbank!' ,,Macht es!'' sagte Amset und zeigte auf das Schauspiel.

Der Schlangenschwanz löste sich von dem Licht. Im Uhrzeigersinn dreht sich der Schwanz in einer weiten Rotation nach unten, während der Schlangenkopf in sich in die Höhe reckte. Sobek hatte die Arme erhoben und stützte Nehebkau, damit er nicht von seiner Schulter fiel. Der Körper der Schlange reflektierte das matte Licht wie ein Regenbogen. Die beiden Götter waren ein prachtvoller Anblick.

Das Licht, das aus dem heiligen Raum der Kirche weit in die Höhle gedrungen war erlosch langsam. An Nehebkaus Stirne leuchtete nun eine einzelne Schuppe strahlend hell. die Höhlenwand warf bunte Lichtpunkte auf die Götter zurück. Nehebkau senkte den Kopf und legte seine Stirne auf Horus Steissbein, der vor ihnen stand. Amset und Tef beobachteten wie sich das Licht von Nehebkaus Stirne auf die Wirbelsäule ihres Vaters übertrug. Schliesslich hatte das Licht Horus Schädeldecke erreicht und trat oben wieder aus. Sobald das geschehen war, wurde es schlagartig dunkel.

Horus repräsentierte den Menschen. Er stellte in diesem Ritual den Prototyp, die Blaupause, den perfekte Menschen dar. Er war der erste, der die gereinigte Lebenskraft in sich aufnahm und sie prüfte, wenn er zufrieden war, dann wurde die Lebenskraft an die Kas der Toten, die Lebenden in ihren Betten und die Götter verteilt.\footnote{Das Ka, wie es die alten Ägypter nannten, war die Lebenskraft, die jeden Körper, der lebendig war, durchdrang und umgab. Sie blieb auf der Erde, wenn der Mensch gestorben ist. Die Ägypter brachten dem Ka des Verstorbenen regelmässig Speisen und Trank. Das Ka ist nicht zu verwechseln mit dem Ba. Denn der Ba des Toten war die unverwechselbare, persönliche Seelenqualität, also ein Teil der Seele.}


,,Okay! Die Energie ist muffig! Aber ich glaube, mehr liegt in dieser Höhle nicht drin. Es ist halt nicht Sokars Reich und Sokars Sand, der alles reinigt und reflektiert,''  sagte Horus. ,,Ich glaube, Nehebkau ist froh, wenn er die Höhle verlassen kann. Er ist unruhig. Irgendetwas gefällt ihm nicht!'' antwortete Sobek. ,,Wir sollten die Ka-Kraft später verteilen und hier verschwinden!''

,,Pssst, seit mal leise!'' flüsterte Neith. Sie lauschten. Neith war sich sicher. Sie hatte ein Kratzen im Gang gehört. Es war alles ruhig. ,,Es ist nichts!'' sagte Neith. Aber es war ihr nicht wohl. ,,Wir sollten dennoch gleich aufbrechen!''  

,Seth?' Tef wendete sich an seinen Onkel. ,Was ist!' auch Seth benutzte die Gedankenübertragung. ,Kannst du dich an seltsame Ereignisse zu Haremhabs Zeiten erinnern?' Tef spürte Seths Überraschung und wie sein Onkel seine Erinnerung durchforschte. ,Warum willst du das wissen?' fragte Seth. ,Nur so! Hab da etwas gehört\dots' antwortete Tef wage. ,Ist es wegen des Mädchens?' Tef spürte wie die Gedanken seines Onkels versuchten in sein Bewusstsein zu dringen. ,Nein, nein! Vergiss es! ist nicht wichtig!'

In diesem Moment raschelte es! Mitten in der Höhle, mitten unter ihnen, schabten Lederne Schwingen über den Boden. Das Wesen schlug mit den Flügeln und kreischte. Sein Atem schlug wie eine Pestpeitsche durch die Höhle und betäubte die Götter für einen Moment. Duamutef wurde ohnmächtig.

,,Schnell! Zu Serket in den Gang!'' rief Neith. ,,Schnell, schnell, kommt, kommt!'' rief Serket, sobald sie wieder Luft bekam. Sie rief unermüdlich, um den anderen mit ihrer Stimme den Weg zu weisen. Jeder der an ihr vorbeihuschte, tastete nach der Hand des nächsten. Als Neith Horus Hand ergriffen hatte, rief sie: ,,Lauft!''

Sie hasteten den Gang entlang. Erst hektisch und nach einigen Kurven und Windungen dann, wurden sie langsamer, sie gaben keine Geräusche mehr von sich. Der Gang schien sich in einem grossen Bogen nach unten zu bewegen. Ohne jedes Geräusch folgten sie ihm.

,Sind alle da?' fragte Horus. ,Ja!' ,Ja!' ,Ja!' ,Ja!',Ja!' antworteten sie ihm. Stocksteif blieben sie stehen. ,Oh, nein!' kam es von Neith. ,Das kann doch nicht sein!' Sie war kurz davor in Panik auszubrechen. ,Tef?!' fragte Amset. Es blieb still. ,Tef! Er ist noch in der Höhle?! Ich muss zurück!' Amset drehte sich um, aber Horus hielt ihn zurück. ,Wenn einer geht, dann bin ich es\dots, es fehlt nämlich noch jemand!' ,Seth!' dachte Serket. Es bedurfte keiner Antwort.

,,Seth!'' flüsterte Horus. ,,Ich bring dich um!'' ,,Nicht nötig, Neffe!'' antwortete Seths Stimme in seinem Rücken. Als Horus sich umdrehte, legte der Windgott ihm den schlaffen Leib seines Sohnes auf den Arm. ,,Dein Sohn hat heute nicht seine beste Nacht, befürchte ich!'' sagte Seth. ,Tef, mein Junge! Was  ist mit dir!' Horus drückte den Schakal an sich. Er seufzte, als er den flachen Atem unter dem Fell spürte.

,Er hat eine volle Ladung von dem Duftwolke unseres geheimen Freundes abgekriegt!' dachte Seth. ,Wie gut, dass der böse Seth nicht so eine Memme ist und ohne Verstand einfach losrennt!' Amset ballte die Fäuste. ,Bleib ruhig, mein Sohn! Nimm deinen Bruder' meinte Horus. Amset legte sich Tef über die Schulter. ,Danke, dass du meinen Sohn gerettet hast!' Horus knirschte mit den Zähnen. ,Da, nich für!' Sie alle konnten das breite Grinsen auf Seths Lippen hören. ,Aber, sage mal, liebste Neith, wolltest du nicht den Gang bewachen?' Seths Gedanke schnitt sich in ihre Köpfe. Neith schwieg. Sie konnte sich nicht erklären, wie das Wesen an ihr vorbei in die Höhle gekommen war. 

Sie schlichen weiter den Gang hinunter. Sie schwiegen, denn jeder hing seinen Gedanken nach. ,Amsi\dots ' Amset fühlte den schwachen Gedanken seines Bruders: ,Tef!?'

,Schhh, halt dich fern von Seth, er\dots ' Amset spürte wie sein Bruder erneut erschlaffte. Amset überlegte, er erinnerte sich, wie Tef Seth vorhin in der Höhle nach Haremhab gefragt hatte. Hatte Seth etwas mit ihrem Fall zu tun? Verdammt, dachte Amset, nicht auch noch Seth! Wie soll ich \am nur beschützen?

\section*{6}
\addcontentsline{toc}{section}{6}

Thot war nervös. \am war bei ihrem Sturz in der Höhle mit dem tödlichen Gift in Berührung gekommen. Es war heimtückisch, denn es wirkte mit der inneren Verfassung desjenigen, der mit ihm in Kontakt gekommen war, zusammen. War die Persönlichkeit stark und klar und sich ihrer selbst bewusst, dauerte es länger bis das Gift wirkte, als bei einer Person, die sich  schwach fühlte, oder aufgegeben hatte.

Und Thot konnte den Eindruck nicht abschütteln, dass mit \am etwas nicht stimmte. Als sie das Haus verlassen hatte, hatte er es nicht bemerkt. Aber als \am in der Höhle schluchzte, da wurde es ihm klar. Eine Dunkelheit umgab sie. Hatte sie von der Dunkelheit aus dem Norden etwas abbekommen, wie Berta und Kebi? Aber wie? Er konnte sich nicht erinnern\dots 

Maat würde es doch spüren? Sie war die Ordnung\dots

\sterne

Maat hatte die Stirne gerunzelt. Sie war verwirrt. Es fühlten sich zu viele Dinge, die vorher in Ordnung gewesen waren, plötzlich falsch an. Aber wie konnte das sein? Sie war die Ordnung. Die Ägypter massen ihr eine grosse Bedeutung zu. Das gab ihr viel Kraft die Zeit und den Raum stabil und verlässlich zu halten. Aber nun hatten sich Zeit und Raum verändert. Was vorher richtig war, schien jetzt falsch. Es gab einen Hohlraum. Ein schwarzes Loch, das Stück für Stück die Zeit und den Raum verschlang und als Chaos wieder ausspuckte. Maat seufzte. Sie hatte den seltenen, ja seltsamen Wunsch ihre Schwester Isfet um Rat zu fragen.

,,\am alles klar?'' Maat drückte \am Hand, die sie fest hielt. ,,Jaja\dots '' sagte \am leise. ,,\am muss schleunigst in die Höhle mit dem Brunnen! Sie muss das Gift abwaschen!'' flüsterte Berta. Sie tastete sich den anderen voraus, die am Lederband hielten. ,,Joh\dots '' antwortete Hans, der das Schlusslicht war. ,,\am muss alle Stellen waschen, die den Boden berührt haben. Kleider ausziehen. Gummistiefel anbehalten! Schnell!''

,,Jooh!'' antwortete der wilde Mann aus der Dunkelheit. Seine Stimme tönte dumpf. \am hörte den Dackel hecheln, er musste dem Geräusch nach auf Hansens Schulter sein.

,,Komm, \am es ist nicht mehr weit.''  Berta schlurfte schneller und Maat zog sachte an Amélies Hand. Sie spürte, wie das Mädchen zögerte. ,,Berta! Ich, ich fürchte mich! Ich will nach Hause!'' rief \am. ,,Sicher! Du machst deine Sache sehr gut! Komm! Das Gift beginnt zu wirken, wir müssen uns beeilen!''

Die vier marschierten los. Vier Paar Stiefel hinterliessen ihr Echo im Gang\dots Der Dackel zappelte auf Hans Schulter. Er knurrte und wendete den Kopf in die Richtung aus der sie gekommen waren. ,,Was hesch, Waldi?'' raunte Hans. Der Dackelt bellte kurz. Hans blieb stehen, auch Berta und \am lauschten. ,,Nichts!'' flüsterte Berta. ,,Kommt, schnell!''

Sobald ihre Gummistiefel schlurften, kratzten hinten im Gang lederne Schwingen über den Boden.

\section*{7}
\addcontentsline{toc}{section}{7}

Thot hetzte durch den Gang, der auf die Ebene des Brunnens führte. Das Höhlensystem, das sie durch den Brunnen in der Mitte betreten hatten, hatte mehrere Ebenen. Die Mittlere, war die, in der sie den Bewohner aufgescheucht hatten. Darüber war die Höhle, in der Sobek,Nehebkau, Seth und Horus das Ritual für den Ka abgehalten hatten. Der Weg war ein Labyrinth aus Gängen, die sich wie Schnecken um die verschiedenen Höhlen wandten. Sie alle hatten eine schwache Steigung. Der Wanderer spürte kaum, ob er sich horizontal, auf oder ab bewegte.

Es gab keine Kreuzungen in den Gängen. Sie gabelten sich und führten weiter im Kreis, dadurch konnte der Wanderer schon nach kurzer Zeit die Orientierung verlieren und, ohne es zu bemerken, wieder auf den gleichen Weg zurückkommen, von wo er gekommen war.

Thot hatte grossen Spass an dem Labyrinth. Aber er beeilte sich, denn sein Freund und Chef Re war mehr der gradlinige Typ\dots Kurz vor der Höhle des Monsters schlich Thot lautlos Schritt vor Schritt weiter. Er lauschte und streckte alle Fühler aus, die er hatte. Er wunderte sich. Der Höhlenbewohner war nicht da! Umso besser!

Thot öffnete von Innen den Brunnen. Die Mannschaft der Barke, war auf das nötigste beschränkt worden. Die vier Ruderer, die die Barke an einem langen Seil ziehen würdens, trugen die geschrumpfte Barke in die leere Höhle. Re in seinem Kaptänsornat mit dem Widderkopf und der Sonnenscheibe auf dem Kopf gab ein schwaches Licht ab. 

,,Wir müssen leider auf jedes Licht verzichten!'' sagte Thot. Re seufzte. Hathor, die bei keiner Barkenfahrt fehlte, holte ein kleines Tuch aus ihrem Ausschnitt und legte es vorsichtig über die goldglänzende Scheibe. Dann kletterten sie in die Barke: Re, Hathor und vier Wächtergötter\footnote{Natürlich könnten wir hier fragen, warum der stärkste Gott der altägyptischen Welt nicht auf seine Bodygards verzichtete. Das war eine Frage des Stils. Schliesslich hatte die Barke gewöhnlich die Grösse eines Ausflugsdampferss und eine Besatzung von über 100 Göttern und Toten.}. Es dauerte eine Weile. Die Barke war auf die Grösse eines Schlauchbootes geschrumpft worden und auch Götter können sich auf engem Raum besser mit etwas Licht sortieren. 

,,Wir müssen die Barke der Schlangen noch einschiffen!'' sagte Re. Die Wächter rumorten im Dunkeln. ,,Kapitän, wir brauchen Licht!''  sagte einer schliesslich. ,,Es ist zu gefährlich!'' meinte Thot.

,,Hach, Männer!'' sagte Hathor. Sie griff in ihre Schürze, deren Motiv wir nicht sehen können, und holte ein Knäuel Schnur hervor. Im Nu hatte sie mehrere Fäden am Bug der Barke befestigt. ,,So jeder nimmt sich eine der Schnüre und geht soweit er kann in die Höhle.'' befahl sie. ,,Wenn alle auf ihrem Posten lauschen, sollte sich niemand unbemerkt der Barke nähern können und die Ruderer können klar Schiff machen!''

Gesagt getan. Es hätte ihnen im Ernstfall nichts genutzt! Aber sie hatten Glück! Auch das können Götter hin und wieder brauchen\dots während Re mit seiner Sonnenscheibe ein mattes Licht scheinen liess, banden die Ruderer eine zweite kleinere Barke an die grössere. In der kleinen Barke, gross wie ein Zweipersonenschlauchboot, befanden sich drei Körbe mit Deckeln. In ihnen raschelte es. Hathor hob den Deckel des ersten Korbes: ,,Glatter? Alles Paletti? Können wir losziehen?'' Es zischte. ,,Er meint ungern, aber besser jetzt als nie!'' meinte Hathor zufrieden. Sie überprüfte persönlich den Knoten, der die Barken verband. 

Ihr Blick fiel dann auf die Treidler. Sie seufzte. ,,Also, schick ist anders!'' Die vier Männer mit den traditionellen weissen Kilts und Frisuren, trugen alle hohe, gelbe Gummistiefel\dots ,,Ich weiss, meine Liebe!'' sagte Thot. ,,Deshalb können wir ja nun, wenn alle an Bord sind, getrost das Licht wieder ausmachen. Ich stiefle dann voraus. Thot hatte sich eine von den Schnüren genommen und sich um den Bauch gebunden, das andere Ende der Schnur trug der erste Treidler.

%\foreignlanguage{russian}{Druzhba}

Die Treidler legten sich in die Riemen und sangen leise und rhythmisch: ,,''\footnote{Ej, hau ruck!
Ej, hau ruck!
Noch ein bisschen, noch einmal!
Ej, hau ruck!
Ej, hau ruck!
Noch ein bisschen, noch einmal!

Wir gehen am Ufer entlang.
Wir singen der Sonne unser Lied.
Aj-da, da aj-da!
Aj-da, da aj-da!
Wir singen der Sonne unser Lied.

Ej, Ej, zieh das Seil fester!
Wir singen der Sonne unser Lied.
Ej, hau ruck!
Ej, hau ruck!
Noch ein bisschen, noch einmal!

Ej, hau ruck!
Ej, hau ruck!
Noch ein bisschen, noch einmal!
Ej, hau ruck!
Ej, hau ruck! }

Mit einem Ruck setzten sich die Barken in Bewegung und schabte über den Boden. Durch die Ritzen der Körbe raschelte und zischte es. Als die Barken in den Gang eintauchten verstärkten sich die Geräusche zu einem kakophonischen Tumult, das in dem Hügel ein tausendfaches Echo erzeugte. 

Der Hügelbewohner, der sein Dasein sonst in heimeliger Stille verbrachte, kauerte sich im Gang zusammen und versteckte den Kopf unter den Schwingen. 

Diese Nacht würde jemand bitter bereuen, schwor sich der Basilisk. Seit seine Familie vor über 500 Jahren umgebracht worden war, hatte er keine solche miese Nacht gehabt\dots \footnote{Im Jahre 1474, steht in den Gerichtsakten der Stadt Basel, gab es eine ordentliche Gerichtsverhandlung gegen einen Hahn. Ihm wurde vorgeworfen ein Ei gelegt zu haben. Die Basler befürchteten, daraus würde ein Basilisk ausschlüpfen. Zu recht wie wir wissen. Das Ei war vernichtet und der Hahn geköpft worden. Was die tapferen Basler nicht wussten: Der Hahn hatte zwei Eier gelegt!}

\section*{8}
\addcontentsline{toc}{section}{8}

Luise wachte schweissgebadet auf. Sie hatte vorgetäuscht zu schlafen, damit sie nicht mit Wilfried reden musste und war darüber eingeschlafen. Es war dunkel in dem kleinen Hotelzimmer. Wilfried war nicht da. Der schwere, unregelmässige Atem des Jungen war das einzige Geräusch.

Verdammte \am. Das Gör brauchte immer Aufmerksamkeit. Luise hasste ihre Tochter dafür. Sie war wie ein Pickel, der an einer unerreichbaren Stelle schmerzte und eiterte. Luise wusste selbst nicht genau, warum das so war. Warum sie das eine Kind abgöttisch liebte und das andere hasste, es aber trotzdem nicht loslassen konnte und wollte.

Sie fühlte sich von ihrer Tochter beschmutzt und ausgenutzt. Sie hatte sich in ihrem Körper breit gemacht, genauso wie\dots Luise sass starr auf dem Bett. Für einen kurzen Moment tauchte sie aus ihrer Dunkelheit auf und erkannte, dass ein kleines Baby, im Bauch der Mutter nicht das gleiche war wie\dots

Sie streichelte dem Jungen sanft über die blonden Haare. Er war so anders. Er war ein Junge! Hatte es damit zu tun? Fürchtete und haste sie ihre Tochter nur, weil sie ein Mädchen war und ihr deshalb das gleiche passieren konnte\dots Hasste sie sie, weil der Schmutz der Frauen an ihr klebte.

Luise stand auf und ging in das Badezimmer. Sie machte das Licht an und erschrak. Sie hatte blutige Striemen im Gesicht! 

Sie verzog den Mund. Die Lippen, die früher schön gewesen waren, bildeten einen schmalen, verkniffenen Strich. Der blutrote Lippenstift, den sie vergessen hatte, hatte im Schlaf auf dem Kissen und in ihrem Gesicht rote Streifen hinterlassen.

Sie starrte sich an. Eine Träne löste sich aus ihrem Augenwinkel, sie wischte sie brutal von der Wange und hinterliess dabei tiefen Kratzer. Dann blinzelte sie, denn für winzigen Augenblick schaute ihr eine fremde Frau aus dem Spiegel entgegen. Sie hatte schwarze Haare, die zu einer aufwendigen ägyptischen Frisur gelegt waren. Das Gesicht war wie schön, und ebenmässig, wie aus kaffebraunem Porzellan. Es lächelte ihr zu\dots Es war kein freundliches, sondern ein hämisches Lächeln. ,,Verpiss' dich, Schlampe!'' sagte Luise. 

Sie wusch sich das Gesicht und richtete ihr blondes Haar. Sie schaute fest in den Spiegel, ihre kornblumenblauen Augen starrten sie an. Sie seufzte. Wo war Berta? 

\section*{9}
\addcontentsline{toc}{section}{9}

Hans, Berta und \am waren in der mittleren Höhle mit dem Brunnen angekommen. ,,Schnell zieh die aus!'' flüsterte Berta. ,,Wie soll ich das denn machen, mit den Gummistiefeln?'' fragte \am zurück. Berta seufzte, da hatte \am recht. ,,Ich weiss es!'' sagte sie nach kurzem Zögern. Berta streifte sich Arbeitshandschuhe aus Leder über , die sie aus ihrem Lederwams geholt hatte. ,,Zieh dich oben aus!'' ,,Berta, \dots, es ist kalt, es ist eklig\dots! '' Es war das letzte, was \am in dieser feuchten, stinkenden Höhle, in der ein Monster auftauchen konnte, wollte, nackt sein! Aber Berta fackelte nicht lange. \am spürte die borstigen Handschuhe der Erdgöttin, die energisch an ihrem Pullover zerrten und ihn über ihren Kopf zogen. Der Rest folgte bis \am mit nacktem Oberkörper da stand. Die Arme vor den Brüsten gekreuzt, schlotterte sie vor Kalte und Angst.

,,Hans! Hebe sie hoch!'' sagte Berta. ,,Excusé moi!'' sagte Hans. Amélie schrie, als der wilde Mann sie mit seinen groben Händen unter den Armen packte und wie ein Kleinkind mit vollen Windeln  hoch hob. Sie strampelte. ,,Halt still!'' rief Berta und fluchte, als Amélies Fuss sie vor der Brust traf. Sie packte Amélies Bein und zerrte den Gummistiefel von ihrem Fuss.

,,Nein! Neeein! Nein! Mama! Bertaaa! Nein! ich will nicht!'' \am Schrei gellte durch die Gänge. Der wilde Mann hielt sie wie eine Lumpenpuppe, weit von sich, aber felsenfest. Er versuchte ihren Tritten auszuweichen. Berta stürzte sich wieder und wieder auf Amélies Beine. ,,Verdammi, noeemol!'' entfuhr es Hans böse, als \am seinen Schritt nur knapp verfehlte und Waldi von seinen Schultern stürzte. Der Dackel jaulte scherzerfüllt auf, als er auf den Boden prallte.

Waldis Schmerzlaut brachte \am zu sich. ,,Waldi! Sch+++! Schuldigung!'' \am schluchzte. Sie konnte nicht aufhören zu heulen! ,,Bald hast du es geschafft!'' Berta schnaufte schwer, als sie an Amélies Hose zerrte. Was geschafft? Dachte \am bitter, nackt in einem unterirdischen Brunnen zu ersaufen, erfrieren, oder von einem Monster gefressen werden? ,,Alles zsamme!'' flüsterte Hans böse.

Waldi hörte auf zu jaulen und knurrte. Die drei Menschlichen erstarrten. Sie lauschten und in der vergrösserten Stille, hörten sie Waldi knurren und Lederne Schwingen rascheln! ,,Hans!'' kreischte Berta.

Es ritschte. Ein Streichholz leuchtete auf. Hans warf \am mit zusammengekniffenen Augen in den Brunnen. Da der Brunnen in der Mitte der Höhle war und zwei Meter entfernt, warf er sie tatsächlich. Es rumpelte, als \am gegen die Brunnenwand krachte und im Schacht verschwand. Währenddessen schrie sie sich die Kehle aus dem Leib.

Das war gut! Der Basilisk, abgelenkt durch den Schrei des begehrten Leckerbissens, achtete nicht auf Berta, die in der einen Hand das Streichholz hielt und in der anderen das Nudelholz\dots

\begin{LARGE}
Boing
\end{LARGE}

,,Autsch!'' Berta lies das Streichholz fallen. Die Dunkelheit war zurück. Hans machte die Augen wieder auf. Berta sprintete los und packte Hans am Arm. Waldi, die treue Seele, wartete am hinteren Ausgang der Höhle und bellte unentwegt, damit Hans und Berta in der Dunkelheit den Gang finden konnten. 

Danach hasteten die drei, so leise sie konnten, weiter. Berta und Hans folgten den Hundepfoten, die leise und rhythmisch über den Boden tatzten, zum tiefsten Punkt der Höhlen.

\sterne

Der Basilisk taumelte durch die Höhle. Als er an den Brunnenrand stiess, lehnte er sich darauf und tastete mit der Schwanzspitze nach der Beule an seinem Kopf. Seine dünnen Hühnerbeine fühlten sich an wie Wackelpeter. Er brauchte eine Zeit, um in seinem kleinen Hühnergehirn alles zu sortieren: 

Lecker Essen (\am ) = weg = Schlecht!

Beule am Kopf = Kopfweh = Schlecht! 

Eindringlinge = Im Zentrum seines Reiches = Schlecht! 

Frau Holle, die alte Schachtel + Nudelholz = sehr schlecht!

Leckerbissen im Brunnen = Präsentierteller = Gut! 

Frau Holle = muss schlau sein = -

Eine Weile stritten Hunger und Angst vor Berta miteinander. Hunger gewann schliesslich und der Basilisk machte sich auf den Weg\dots

\section*{10}
\addcontentsline{toc}{section}{10}

Das eiskalte Wasser schloss sich um \am ! Tausend Kältenadeln bohrten sich in die Haut und brannten wie Feuer. Kopfüber tauchte sie in die dunkle Tiefe. \am Haar schwebte wie ein Schleier hinter ihr her. Luftblasen stiegen aus ihrem vom schreien geöffneten Mund auf. Sie glitten an ihrem Bauch und Rücken vorbei nach oben. Prickelten, wenn sie an ihr platzten. 

Der Schacht schmal und ihre Arme und Hände berührten die Brunnenwände aus rauem Stein und Algen. Diese bestanden aus zarten Fäden, die mit einem schwachen, phosphoreszierendes  Licht den Schacht erleuchteten. Er schien kein Ende zu nehmen\dots

Sobald die letzten Luftblasen an ihr vorbei an die Oberfläche gestiegen waren, bemerkte \am den Strudel. Das Wasser im Brunnen bewegte sich im Uhrzeigersinn nach unten, als wäre sie in eine Teetasse gefallen in der jemand kräftig rührte. \am wurde mitgezogen.

Ihr Herz schlug bis zum Hals, dennoch konnte sie den Blick nicht abwenden. Die blassen, grünlichen Lichter der Algen zogen wie eine gestreifte Spirale an ihr vorbei. Ihr Körper, der vor Kälte steif gefroren schien, bewegte sich nicht und dennoch wurde sie bewegt. Das Wasser sprudelte leicht, als würde jemand von unten Luftbläschen mit dem Strohhalm hineinpusten. Sie drangen in ihre Haut ein und perlten und prickelten in ihrem Körper weiter. An den Stellen, die bei ihrem Sturz in der Höhle vom Gift des Basilisken berührt worden waren, lösten sich schwarze, krustige Teile. Ihre Arme und Beine, ihr Körper, sie fühlten sich erfrischt und sauber an. Bis auf eine Stelle, die Stelle auf die der schwarze Tropfen vom Dach gefallen war.

\am Lungen klagten, es wäre höchste Zeit wieder an die Luft zu kommen\dots \am begann zu wild zu rudern, die Panik war schlagartig wieder da, als sie spürte wie der Druck in ihrem Brustkorb kollabierte, dann wurde sie ohnmächtig. 

Hans legte \am vorsichtig neben Duamutef in die kleine Barke. Und Berta hüllte sie in eine dicke Decke ein. Sie betrachtete ihre Ziehtochter genau. ,,Ist sie\dots okay?'' fragte Amset leise. Er stand neben Berta und merkte wie ihm die Knie zitterten. Erst Tef und jetzt \am . Er kochte vor Verzweiflung. ,,Das schlimme ist, Amset, ich weiss es nicht!'' antwortete Berta müde. ,,Sieh her!'' Sie hob Amélies Arm hoch. Auf der Innenseite des Unterarmes sahen sie die tiefen Kratzspuren und als Berta den Arm drehte, auf dem Handrücken den dunklen Fleck, den das Wasser des Brunnens nicht abgewaschen hatte. Amset schluckte.

,,Du bewachst die beiden!'' sagte Berta bestimmt. ,,Wer weiss, was diese Nacht noch alles bereithält.'' ,,Ich helf dir dänn!'' meinte Hans und klopfte Amsi auf die Schulter. ,,Ich kha bi dm Göttrzügs eh nit hälfe!'' Dann packte Hans den kleinen dreckigen und zerzausten Dackel mit seinen kräftigen Händen. 

Sie befanden sich in der untersten Höhle, die mitten in dem See lag, der dessen Quelle den Brunnen nährte. Ein gewaltiges Dach aus Tuffstein, das auf dem See schwamm, bildete einen natürlichen Regenschirm. In der Mitte des flachen, kleinen Sees erhob sich der felsige Untergrund zu einer glatten, flachen und erstaunlich trockenen Insel.

Im Felsdach befand sich an einer Stelle am Rand eine Öffnung, der Eingang zur Höhle. Sie war nur durch das Wasser des Sees erreichbar, das sich beständig im Uhrzeigersinn drehte, angetrieben durch den Energiewirbel der Leonhardskirche weit oben drüber. Da das Dach der Höhle auf dem See schwamm, drehte es sich ebenfalls, genauso wie der Eingang der Höhle\dots Diese konnte nur in dme kurzen AUgenblick betreten werden, indem der Höhlengang, der zu dem äusseren Rand des Sees führte, genau auf der gleichen Höhe mit dem sich drehenden Höhlendach war\dots

Hans ging mit dem Dackel, der heftig strampelte, an das Ufer des Sees und tunkte ihn beherzt in das kalte Wasser. Die Strömung des Strudels brach sich an Hans Arm, es platschte und rauschte. Waldi zappelte,  fiepte, aber Hans hielt in fest. ,,Autsch!'' Hans zog seinen Arm hastig zurück. Waldi hatte sich in seiner Hand verbissen und knurrte unmissverständliche Empörung zwischen den zusammengepressten Zähnene hervor.

 ,,Heijeijei, altr Kumpel, wer wird do glii so wüetig wärde?'' Hans zog den Hund, der laut auf den Boden tropfte, von seiner Hand und wickelte ihn in eine Decke. Dann bettet er ihn zu \am und Tef in die kleine Barke. ,,Die Nacht der unfreiwilligen  Bäder!'' sagte Berta und grinste. Sie wusste ja nicht, wie recht sie behalten sollte!

\section*{11}
\addcontentsline{toc}{section}{11}

Die anderen Götter, die Gruppe um Horus aus der obersten Höhle und die Barkenfahrer um Re, nahmen keine Notiz von den Badeabenteuern, denn sie mussten die Zeremonie durchführen.

Ein schwieriges Ritual, das selbst in normalen, ägyptischen Nächten viel Konzentration erforderte und von dem sie nicht wussten, ob es ihnen an diesem fremden, feindseligen Ort gelingen würde- -,Es muss!' dachte Thot angespannt. 

Als Götter waren sie für den Ablauf der Zeit verantwortlich. Damit die Zeit am nächsten Tag weiter zur Verfügung stand und in der richtigen Menge und Geschwindigkeit zur ihren Lauf nahm, brauchte es in jeder Nacht mehrere Zeremonien. Osiris, als Gott der Fruchtbarkeit über den Neubeginn genauso die Herrschaft führte wie über das Totenreich, leitete die heikle Mission. Damit der Herr der Unterwelt jedoch den Ort betreten konnte, musste er gesichert werden.

Horus und Seth, die beiden erfahrenen Kämpfer, hielten Wache in den zwei Gängen, die zu dem Rand des Sees führten und immer dann, wenn sich die Höhlenkuppel mit der Öffnung auf ihrer Höhe befand, den Durchgang gewährten. Ihnen zur Seite standen Selket und Neith, die je tiefer in die Gänge vorgedrungen waren, um zu warnen, sobald sich der Basilisk anschleichen wollte.

Die vier Treidler und Wächter hatten sich ringsherum im Kreis auf der Felseninsel verteilt. Die kleine Barke stand am Rand. Die drei Schlangenkörbe, die mit der kleinen Barke transportiert worden waren, waren in einem Kreis in der Mitte der Höhle, direkt unter der Spirale, die bis über die Leonhardskirche hinaus in den Himmel und bis tief, tief in die Erde führte, aufgestellt. Die Schlangen hatten ihre Köpfe aus den Körben gereckt. Ihre mattgoldenen Schuppen schimmerten und warfen ein warmes, schummriges Licht. 

Die grosse Barke, die noch immer mit der kleinen verbunden war, falls ein schneller Aufbruch nötig würde, stand mit dem Bug zu dem Kreis. Re und Hathor rückten ihren Kopfschmuck zurecht und suchten sich die beste Position im Bug. Wie Schauspieler vor der Premiere zupften sie nervös ihre Kleider zurecht. Maat, und Thot nahmen die zwei freien Plätze zwischen den Schlangen ein und schlossen den Kreis. Maat war nicht wieder zu erkennen. ihr Gesicht war nun das der Göttin, zeitlos. Und ihr Körper, immer noch schlank und klein, war der einer Frau. Die Feder mit der die Herzen der Verstorbenen aufgewogen wurde, ragte hoch auf ihrem Haupt auf und schien, wie die Schlangen in einem matten Schein. ,,Schade, dass unsere Maat die Gummistiefel anbehalten muss!'' sagte Hathor und seufzte. ,,Es ist schon nicht dasselbe so!''

Thot hatte nur seinen Kopf umgezogen. Für diese Stunde der Nacht, für diesen Moment brauchte er seinen Paviankopf. Seine graue, kräftige Mähne stand wie ein Kranz um seine Schnauze. ,,Sieht denn schon' s bitzeli komisch aus\dots, so'n Pavian mit Mantel und Gummistiefeln'' flüsterte Berta Hans zu. Sie standen zwischen den Wächtern und wachten mit und wunderten sich und harten der Dinge, die die ägyptischen Gäste geplant hatten.


\section*{12}
\addcontentsline{toc}{section}{12}

Thot, Maat und Re begannen die uralten, magischen Formeln zu rezitieren, die den Herrn der Unterwelt, Osiris, riefen. Jeder von ihnen hatte unterschiedliche Gesten zu machen, unterstützt wurden sie durch einen schrillen, trillernden Gesang Hathors.

Wäre die Kraftspirale sichtbar gewesen, hätten wir gesehen wie sie weit in den Himmel wächst und sich gleichzeitig tiefer in die Erde schraubt. Sie dreht sich schneller und schneller und bildet im Inneren eine Röhre. 

In dieser Röhre erscheint Osiris in der untersten Höhle. Er ist in sein Mumiengewand gehüllt, die Beine umwickelt. Er trägt den bunten Kragen der Götter. Seine Hände sind vor der Brust gekreuzt und halten die Nechacha und den Krummstab.

Damit Osiris in dieser Weise erscheinen kann, braucht es unglaubliche, wahre Götterkräfte. Die acht, die den Kreis bildeten, schwitzten und stöhnen, während sie die Formeln murmeln. Die Energie des Spirale ist in der Höhle gewaltig. Sie ist zwar unsichtbar, aber Osiris Gestalt verwischt wieder und wieder vor den Augen und kleine Blitze entladen sich. Ihnen allen stehen die Haare zu Berge. Thots Pavianmähne bildete einen eindrucksvollen Kranz aus elektrischen Entladungen wie ein Faradayischer Käfig, ebenso wie die Spitzen Res Widderhörner. 

Die drei Schlangen drehen sich um sich selbst,sie beschworen den ,Sich Bewegenden', die dreiköpfige Schlange. Wie die Zeit vereint die gewaltige Schlange ihre drei Köpfe, die Vergangenheit, Gegenwart und Zukunft repräsentieren, in einem einzigen Körper. Der ,sich Bewegende' dreht sich langsam, Kopf voran über Osiris. Der Schwanz, der riesigen Schlange verschwindet im Dach der Höhle. 

Osiris murmelt Zauberformeln, die magischen Hieroglyphen steigen auf und bilden unter den rotierenden Schlangenköpfen einen Gitterkorb. Sobald der Klangkorb stark genug ist, ist ein dunkler, runder Schatten darin zu sehen. Mit jeder Drehung des ,sich Bewegenden' bekommt der runde Schatten einen Lichtschimmer, der wächst.

,,Schau wie schön der Mond wächst!'' sagte Berta zu Hans leise. Die beiden nordischen Götter waren völlig vertieft in das fremde und imposante Schauspiel. Weder sie, noch einer der Treidler, hörten das Gepolter über ihren Köpfen\dots

\section*{13}
\addcontentsline{toc}{section}{13}

Wäre der Basilisk doch nur ein Momentchen früher in den Brunnenschacht gehüpft\dots Doch er wählte just den Zeitpunkt, als die Beschwörungen für Osiris die Kraftspirale zu einer Art Tornado machten. Als er sich vom Rand in den Brunnen fallen lies, wurde er anfangs von dem sanften, natürlichen Strudel erfasst und dann erlebte er sein blaues Wunder! Die Kraftsäule erfasste ihn wie eine manische Waschmaschine und schleuderte ihn mit Höchstgeschwindigkeit im Schacht herum. 

Nur durch die Schwerkraft sank er langsam, in Zeitlupentempo nach unten, während er kopfüber, kopfunter im Kreis gewirbelt wurde. Er rollte sich ein so gut er konnte und rumste schliesslich wie ein Gummiball auf das Dach der unteren Höhle. Dort zog der Strudel ihn hoch und runter. Als gegen Ende der Zeremonie, dem Minimond im Beschröungskoeb unter der dreiköpfigen Schlange war fehlten noch zwei Umdrehungen um voll zu werden, die Spirale abbremste. Fiel der Basilisk auf das Dach.

Dieses war spitz, zwar abgerundet aber durch die ständige Drehbewegung spitz und der Basilisk rutschte in die Spalte zwischen Dach und Felsenwand. Das Dach, das sich vom Wasser angetrieben drehte, war noch ordentlich in Schwung\dots Wenn wir das Bild von der Waschmaschine vor Augen behalten, dann folgte nun der Tumblergang\dots

Die Zeremonie war fast beendet, Das Höhlensystem kam zur Ruhe und der Basilisk fiel in den See.

\sterne

Dann passierten viele Dinge gleichzeitig\dots

Der Basilisk, dem schwindelig und speiübel war, rollte durch den Eingang in die Höhle. Er kroch auf die kleine Insel und übergab sich, während unter den Göttern ein grosses Geschrei entstand. Die Treidler rannten in den Kreis um am Bug die Treidelseile zu schnappen und zerrten an der Barke, die sich mit einem Sprung in den Kreis bewegte. 

Osiris und der ,sich Bewegende' flüchteten, indem sie sich von der Spirale nach oben durch das Dach der Höhle treiben liessen. Die drei Schlangen, verschwanden in ihren Körben und es wurde dunkel. Allerdings nicht ganz, denn der Mond, den Osiris beschworen hatte, der die Zeit in Ebbe und Flut einteilte, schien mitten in der Höhle ein schwaches Licht.

Thot packte Maat und warf sie auf Re, der zusammen mit Hathor am Boden des Bugs lag. Der Ruck hatte sie aus dem Gleichgewicht gebracht. ,,Lauft, lauft!'' Herrschte Thot die Treidler an und dann brüllte er ,,Horus! Horus! Halt das ver***te Dach an, wir kommen raus!''

Währendessen taumelte der Basilisk sich um sich selbst drehend durch die Höhle. Es gelang ihm nicht den Blick zu fokussieren, was als Hahn schon schwierig genug war\dots Dennoch traf er einen der Wächter, der mit einem Schrei zu Stein erstarrte. Es knirschte und knackte, als würden die Knochen zermahlen-

Hans schob das Heck der kleinen Barke an und steuerte es durch den Ausgang der Höhle. Bevor der beschworene Mond erloscht, spürte der Basilisk zum zweitenmal in dieser Nacht einen tüchtigen Schlag auf den Kopf. 

Das letzte, was er hörte, waren eilige Schritte und hektische Stimmen, die verstummten, als sich das Dach der Höhle weiter drehte und dann mit einem Gummistiefel verkeilt wurde.

