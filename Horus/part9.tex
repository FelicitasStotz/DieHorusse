\part*{Neunte Stunde\\"`Anbetende, die ihren Herrn schützt"'}
\addcontentsline{toc}{part}{Sechste Stunde}

\chapter*{9. Tag,}
\addcontentsline{toc}{chapter}{1. Januar, }

\section*{1}
\addcontentsline{toc}{section}{1}

\section*{2}
\addcontentsline{toc}{section}{2}

\section*{3}
\addcontentsline{toc}{section}{3}

\section*{4}
\addcontentsline{toc}{section}{4}

\section*{5}
\addcontentsline{toc}{section}{5}

\section*{6}
\addcontentsline{toc}{section}{6}

\section*{7}
\addcontentsline{toc}{section}{7}

\section*{8}
\addcontentsline{toc}{section}{8}

\section*{9}
\addcontentsline{toc}{section}{9}

\section*{10}
\addcontentsline{toc}{section}{10}

\section*{11}
\addcontentsline{toc}{section}{11}

\section*{12}
\addcontentsline{toc}{section}{12}

\section*{13}
\addcontentsline{toc}{section}{13}

\section*{14}
\addcontentsline{toc}{section}{14}

\chapter*{9. Nacht}
\addcontentsline{toc}{chapter}{9. Nacht}

\begin{quotation}

\emph{IX Sic habebis gloriam totius mundi. Ideo fugiet a te omnis obscuritas. Hic est totius fortitudinis fortitudo fortis; quia vincet omnem rem subtilem, omnemque solidam penetrabit.  \\Tabula Smaragdina}

\end{quotation}

\section*{1}
\addcontentsline{toc}{section}{1}

\section*{2}
\addcontentsline{toc}{section}{2}

\section*{3}
\addcontentsline{toc}{section}{3}

\section*{4}
\addcontentsline{toc}{section}{4}

\section*{5}
\addcontentsline{toc}{section}{5}

\section*{6}
\addcontentsline{toc}{section}{6}

\section*{7}
\addcontentsline{toc}{section}{7}

\section*{8}
\addcontentsline{toc}{section}{8}

\section*{9}
\addcontentsline{toc}{section}{9}

\section*{10}
\addcontentsline{toc}{section}{10}

\section*{11}
\addcontentsline{toc}{section}{11}

\section*{12}
\addcontentsline{toc}{section}{12}

\section*{13}
\addcontentsline{toc}{section}{13}

\section*{14}
\addcontentsline{toc}{section}{14}
