\part*{Fünfte Stunde\\"`Die Geleitende, die inmitten ihrer Barke ist"'}
\addcontentsline{toc}{part}{Fünfte Stunde}

\chapter*{5. Tag, Tag der unschuldigen Kinder}
\addcontentsline{toc}{chapter}{28. Dezember, Tag der unschuldigen Kinder}

\section*{1}
\addcontentsline{toc}{section}{1}

,,Verda***t wir haben nur Glück gehabt!'' rief Thot und schlug mit seiner schlanken Faust auf die Sessellehne seines Ohrensessels. ,,Nun regt dich nicht so auf! Glück gehört dazu!'' sagte Re und sog an seiner Pfeife. ,,Re, es war um Haaresbreite! Und das ist auch übertrieben, denn wir haben einen Mann verloren letzte Nacht!'' antwortete Thot heftig. ,,Ein Wächter, der nicht mit der Möglichkeit rechnet im Dienst zu sterben, ist kein Wächter!'' meinte Horus. Thot schnaubte. ,,Es ist ganz einfach! Es hätte schief gehen können!'' Thot liess nicht locker. ,,Zum Glück haben gehört dazu, dass es schief gehen kann. Das ist die Bedeutung von Glück haben\dots '' antwortete Re.

,,Ach ja, und wenn wir keines gehabt hätten? Dann hätten wir Osiris verloren, oder \am, oder Maat!'' fragte Thot bissig. Re schwieg und kaut an seiner Pfeife und auch Horus, der unruhig im Zimmer hin und her getigert war, blieb am Fenster stehen. 

,,Versteht ihr, worauf ich hinaus will?'' fragte Thot. ,,Du meinst wir haben die Kontrolle verloren!'' sagte Re und schaute den Gerichtsschreiber ernst an. ,,Ja, Re, genau das meine ich!'' sagte Thot leise. ,,Dann holen wir sie uns eben wieder!'' rief Horus und schlug eine Delle in das Holz der Fensterbank. ,,Uns bleibt nichts anderes übrig!'' antwortete Thot unwirsch. 

,,Und was schlägst du vor?'' fragte Re. Er hatte keine Mühe damit Thot um Rat zu fragen, wie es sich für einen guten Chef gehörte, wusste er das Potential seines Teams zu nutzen. Im Gegensatz zu Horus, dem Thot mit seinem klugen Geschwätz manchmal so etwas von auf die Schwanzfedern ging\dots Horus staunte, als Thot antwortete: ,,Ich weiss es nicht! Es sind zu viele Unbekannte in dieser Gleichung. Der fremde Ort mit seinen Besonderheiten, die Sache mit Osiris und das Wichtigste ist, wir müssen rausfinden, was mit diesem Mädchen los ist!'' ,,Jaa!'' sagte Re. ,,Wenn Menschen im Spiel sind, dann wird es unberechenbar!'' ,,Nichts ist unlösbar!'' antwortete Thot.

,,Vor allem, wenn man die richtigen Leute fragt!'' Berta stand in der Tür und stemmte ihre Fäuste in die Taille. ,,Ich glaube es wird Zeit für Maidlipower!'' ,,Meidipauer?'' fragte Re. ,,Maidlipower!'' antwortete Berta. ,,Ich denke, es ist an der Zeit mal eine Nacht auswärts zu verbringen! \am geht es schlecht, ich weiss nicht, was ihr fehlt. Und Isis meinte, Osiris und du, Re, ihr seit heute Nacht auf besonderen Schutz angewiesen?''

Die drei männlichen Götter schwiegen eine Zeitlang. Sie brauchten einen Moment, um das Konzept von ,Maidlipower' zu begreifen und zu verdauen\footnote{Natürlich waren die ägyptischen Göttinnen ebenso stark wie Berta. Sie waren aber konservativ patriarchalisch eingestellt, was die Rollenverteilung anging. So wie die Menschen ihres Landes, das bis auf wenige Ausnahmen von Pharaonen, statt Pharaoinnen regiert und gelenkt wurde. Wobei wir nicht vergessen sollten, dass Berta genauso konservativ matriachalisch war, so wie es ihrer Kultur entsprach. Als ehemalige ,grosse Muttergöttin' reichte ihre Geschichte bis weit in die Steinzeit zurück. Eine Zeit in der Frauen, dicke Frauen eine ihrer Fülle angemessene Rolle spielten. Berta genoss auf diese Weise ein breite, kulturelle Palette, die sie vergnüglich benutzte.}. Der eine oder andere Blick wanderte dabei über Bertas gewaltigen Busen und Bauch und über die strahlendweisse Schürze\dots Am Rand des Saumes war ein Nudelholz gestickt. Re, der von den ägyptischen Göttern der älteste und Stammvater war, grinste schliesslich. Er war ähnlich alt wie Berta und konnte sich durchaus an die eine oder andere Muttergöttin erinnern, und nach vielen tausend Jahren Ehe konnten Genderfragen einen nicht wirklich erschrecken.

,,Was habt ihr Weiber denn vor?'' fragte Horus, der Praktiker, der sich von Worten umzingelt fühlte und ungeduldig wurde. ,,Ich und die Mädels schlagen vor, wir gehen heute Nacht eine Freundin in Frankreich besuchen. Mademoiselle Sante Odile freut sich auf unseren Besuch auf ihrem Puddingberg!'' 

,, Fliehen? In ein anderes Land? Hört auf mit dem Gerede, sonst lese ich Euch die Leviten!'' sagte Horus brummig. ,,Esst mal einen Keks!''  rief Horus, dann war die Schwanzspitze des Falkengottes durch das offene Fenster verschwunden. Sie sahen sich an und lachten.

Thot ging an ein kleines Schränkchen, das in einem der Regale stand und holte eine hübsche Porzellanschale mit knusprigen Keksen heraus. ,,Mit?'' fragte Re. ,,Mit!'' antwortete Thot. Wie durch ein Wunder kam Wibrandis in diesem Moment ins Zimmer geschwebt. Sie trug ein Tablett, Sie reichte Re einen Becher aus dem der herbe Duft von Ceylon aufstieg und Thot eine Tasse mit feinem Mokka. Berta, die es sich auf dem Sofa gemütlich gemacht hatte, bekam schliesslich einen Becher mit Sahnehäubchen und Kaffeeduft, der mit rauchigem Whiskyaroma gewürzt war. Die Götter schnupperten entzückt an ihren Getränken. ,,Siehst du, Thot, es geschehen Zeichen und Wunder!'' sagte Re. ,,Bei Gott, dem Allmächtigen, sind alle Dinge möglich!'' meinte Wibrandis mit erhobenem Zeigefinger. ,,Ich bevorzuge den Pharisäer, der ist mit Sahne!'' meinte Berta und senkte vorsichtig ihre Lippen in den Sahneberg. Wibrandis Finger blieb in der Luft hängen.

,,Oh! Uups! Entschuldigung! ich wollte die Herren Göttern nicht beleidigen! Herrjeh!'' rief Wibrandis. ,,Es ist gut, meine Liebe!'' antwortete Re ernst und gemessen. ,,Wir haben es genau verstanden!'' Wibrandis wurde rot vor Scham, knickste hastig und war verschwunden. ,,Bei Gott, dem Allmächtigen,\dots soso!'' sagte Re. Dann hob er seinen Becher wie zum Gruss und wartete bis Thot und Berta es ihm gleich machten. ,,Auf das Glück!'' ,,Und auf Gott!'' antwortete Thot. Und die Pharisäer!'' sagte Berta.

\section*{2}
\addcontentsline{toc}{section}{2}

Seth ächzte. Er krallte sich mit den Fingern und den nackten Zehen in die Rillen der Hauswand. Diese sollten das Hauswand auf hübsche Art und Weise unterteilen. Seth fand es äusserst zuvorkommend. Letztlich waren die Rillen für ihn wie eine Leiter mit sehr engen Tritten. Er bedauerte es, was selten geschah, keine brauchbare Tiergestalt annehmen zu können. Kurz hatte er überlegt sich in eine Schlange zu verwandeln und über den Gang in das Zimmer seines verhassten Bruders zu schleichen, aber er befürchtete dort auf Isis Wächterschlangen zu treffen.

Er war ein Kämpfer und sein menschlicher Körper trainiert\dots da kam ein kleines Mäuerchen gerade recht. Oben angekommen suchte er Halt an dem schmalen Fenstersims. Das Fenster war verschlossen. Kurzentschlossen ballte Seth die Faust und schlug die Scheibe ein. -,Von wegen Überraschen', dachte er grimmig. er zog sich weiter hoch und krallte sich mit den Zehen in die Spalte. Mit einem Arm langte er durch das Loch in der Scheibe und schaffte es das Fenster zu öffnen, bevor seine Zehen den Halt verloren.

Für einen Augenblick blieb er auf dem Fenstersims mit dem Bauch liegen. Sein Hand blutete. Sein Blut war dunkle blau. Er schob sich üweiter vor und lies sich kopfüber in das Zimmer hinein abrollen. Er gelangte sofort auf die Füsse.

,,Hallo, Bruderherz!'' Seth sah sich um. Das Zimmer sah aus wie ein Tropenhaus. Es war voller saftiger, grüner Pflanzen, die in unzähliger Blütenpracht blühten. ,,Hübsch hast du es hier!'' sagte Seth und verzog den Mund. Er berührte mit der Fingerspitze eine Blüte, deren Kelch orange leuchtete und aus dem grazil die Blütenstempel ragten. Die Blüte welkte augenblicklich und fiel zu Boden.

Gelangweilt schaute Seth auf seine Füsse um die herum die Pflanzen zu welken begannen. Wie ein Stein das Wasser ringförmig bewegt, lies seine Anwesenheit die Vegetation vertrocknen. Seth beobachtete das Sterben sorgfältig. Aber die Pflanzen waren alle gleichzeitig verdorrt. Dabei hatte er gehofft auf diese Weise das Versteck seines Bruders zu finden. 

Seth verstand nicht, warum sich das Zimmer leer anfühlte. Er, als Geist des Todes, spürte seinen Bruder den Vegetationsgott wie Feuer das Wasser spürt, oder Kälte die Wärme. Er hatte nicht viel Zeit. Seth war erstaunt genug, das Zimmer seines Bruders leer vorzufinden. 

,,Dann eben nicht, Bruder! Oder besser, ein anderes mal!'' Seth ging zur Tür und öffnete sie. im Augenwinkel meinte er eine Bewegung zu bemerken\dots , doch, es blieb leblos und still. Seth schlich auf den Gang und schloss die Tür leise. Er hörte Schritte! Er hastete so schnell er konnte bis  an die Ecke zu Thots Bureau. Er hörte Stimmen. Bevor er jedoch sein Ohr an die Tür legen konnte, zog ihn jemand am Ohr von der Tür fort.,, Autsch!'' entfuhr es ihm.

,,Wenn ich dich noch einmal so nahe bei Osiris erwische, dann wirst du es bereuen unsterblich zu sein, lieber Bruder!'' flüsterte Isis in das Ohr, das sie zwischen ihren Fingern zerquetschte. Seth riss sich los und lief an Isis vorbei und die durch das Treppenhaus abwärts.

\sterne

,,Oh, nein!'' flüsterte Isis, als sie Osiris im wahrsten Sinne des Wortes verwüstetes Zimmer sah. ,,Osiris!'' Sie lief an das zerbrochene, offene Fenster. Eisige Kälte wehte ihr ins Gesicht. Wie betäubt schloss sie es. Der Wind blies weiter Schneeflocken hinein. sie bückte sich und hob die verwelkte, orange Blüte auf. ,,Osiris?'' rief sie wieder. Dann hörte sie ein leises kratzen.

-,Hier!' Anubis kam rückwärts unter dem Bett vorgekrochen. Auf seinen Rücken war der Imiut geschnallt. Isis riss ihn von seinen Schultern und öffnete ihn. Osiris schwebte durchsichtig heraus und schlüpfte wieder unter das Bett. -,Helft mir!' dachte er matt. Isis und Anubis zogen und zerrten vorsichtig an dem zerbrechlichen Körper. ,,Wie?'' fragte Isis. -,Anubis und ich spielten die morgendliche Runde Schach, wie immer, wenn du in der Küche mein Essen zubereitest, da hörten wir ein Geräusch. Etwas schabte an der Hauswand und wurde lauter.'

-,Da Seth im Haus war, beschlossen wir auf Nummer sicher zu gehen,' fügte Anubis hinzu. -,Osiris lies sich aus dem Bett auf den Boden gleiten und rollte unter das Bett und ich versteckt sein Ba und Ka im Imiut, damit Seth sie nicht so schnell spüren konnte.' ,,Ein dummer Plan!'' sagte Isis. -,Der beste, den wir hatten!' Osiris klang erschöpft. Sie hatte ihn in das Bett gehoben und die dicke Daunendecke über ihn gelegt. es wurde immer kälter im Zimmer.

,,Ich muss schnell zu Thot. Re und Berta sind bei ihm, sie alle müssen davon erfahren. Du musst hier raus, hier bist Du nicht sicher.'' -,Es ist etwas kalt!' antwortete Osiris. Anubis legte sich vor das Bett, Isis legte ihm eine Wolldecke über den schlanken Körper.

Dann stapfte sie schnell direkt in Thots Bureau. ,,Seth ist bei Osiris eingebrochen!'' rief sie. ,,Durch das Fenster!'' ,,Also, Jungs, worauf wartet ihr noch? Auf zum Monte de Sante Odile würde ich vorschlagen.!'' ,,und zwar schnell, bevor Osiris sich den Tod holt in der eisigen Kammer!'' fügte Isis an. Thot rieb sich die Hände. ,,Tja, dann kommen wir wohl um eine Autofahrt nicht drumrum!'' Es glitzerte in seinen Augen. ,,Auto?'' fragte Re. ,,Es wird dir gefallen!'' antwortete Thot.

\section*{3}
\addcontentsline{toc}{section}{3}

Seine Hände umklammerten das Lenkrad fester. Heute war wieder ein Tag an dem ihm die Augen zufallen wollten\dots ,,Pass doch auf!'' rief Luise. Der Reifen auf dem Seitenstreifen heulte. Wilfried riss am Steuer und der Wagen schlingerte gefährlich auf die Überholspur, auf der mit Warnhupe ein BMW von hinten aus dem Nichts auftauchte und an ihnen vorbei schoss. Der Fahrer schüttelte die Faust gegen Wilfried. 

,,Papa, ich hab Angst!'' sagte der Junge leise. ,,Brauchst du nicht, ein Schatz! Papa wird sich jetzt zusammenreissen und vernünftige Fahren, nicht wahr?'' Luise warf ihrem Sohn ein Lächeln über die Schulter und fixierte Wilfried Böse. ,,Was ist los?'' fauchte sie. ,,Nichts, nichts! ich habe nicht gut geschlafen!'' meinte Wilfried ruhig. ,,Du hast garnicht geschlafen! Du warst nicht einmal im Bett!'' sagte Luise. ,,Mir ist egal, was du nachts machst. Aber fahr gefälligst vorsichtig!''

Wilfrieds Hände fasten das Steuer fester, seine Fingerknöchel wurden Weiss. Es sah aus, als würde er die Fäuste ballen. Luise schaute aus dem Fenster. Die Landschaft glitt unablässig vorbei. ,,Müssen wir noch weit fahren, bis wir \am und Berta wieder sehen?'' fragte der Junge schüchtern. ,,nein, mein Schatz, wird sind schon an Freiburg vorbeigefahren, wir sind bald da!'' meinte Luise.

,,Wenn ich \am in die Finger kriege, kann sie was erleben!'' flüsterte Luise. ,,Wir wissen doch nicht, was los ist. Wir müssen Berta finden!'' ,,\am ist ausgerissen, das ist los! Und Berta nimmt sie in Schutz wie immer und behauptet, es sei ihre Idee gewesen!'' Wilfried schwieg. Er schwieg immer, wenn er wusste seine Antwort würde Luise noch mehr auf die Palme bringen. Auch wenn er wusste, dass sie Unrecht hatte\dots auch wenn er wusste, dass seine Tochter es ausbaden würde\dots 

Er presste die Lippen zusammen. Es war nicht mehr weit. In Basel würden sie eine Unterkunft suchen und dann nach \am und Berta suchen. Er hatte keine Ahnung wie sie sie finden sollten in der grossen Stadt. Luise würde sich sehr ärgern\dots 

Der erste Hinweis auf die schweizer Grenze glitt an ihnen vorbei. ,,Wir sind da!'' sagte er erleichtert. ,,Wilfried?'' Zu seinem Erstaunen klang Luises Stimme klein. ,,Ich fürchte mich! ich weiss nicht wieso, aber ich fürchte mich davor \am und Berta zu finden!'' Sie waren in der kleinen Schlange vor dem Grenzübergang stehen geblieben. Wilfried schaute erstaunt seine Frau genauer an. Sie hatte dunkle Ringe unter den Augen. Es schien ihm, als wäre sie von einem Schatten umgeben.

Um 11:24 Uhr rollten sie über die Grenze.

\sterne

Ein Donner grollte über dem blauen Haus und ein Stapel Töpfe rutschte in der Küche zu Boden. Es polterte und schepperte. Die erschreckte Wibrandis, die mit Hans und einigen anderen Göttern zurückgeblieben war, schaute auf in den grauen, dunklen Winterhimmel. Sie fröstelte. ,,S isch khei gwöönlichr Donnr, wenn du mich würdsch froge!'' sagte Hans. ,,Was denn?'' ,,S isch dr Fluch!'' antwortete Hans düster. Wibrandis lief ein Schauer über den Rücken, sie faltete die hände zu einem kurzen Stossgebet.

\section*{4}
\addcontentsline{toc}{section}{4}

Um 11:23 Uhr rollten drei ,Göttinnen' über die burgfelder Grenze nach Frankreich\footnote{ds = Wortklang im französischen ähnlich wie ,la dèesse = die Göttin, was dem Citroen ds 21 in Ausrüstung, Technik und Stil angemessen war}. Die beiden Zollbeamten am Übergang, die in dicke Parker eingehüllt waren, staunten.

 Die erste ,Göttin' war golden und glänzte, als wäre der himmel nicht mit dunklen Wolken verhangen, sondern als würde die Sommersonne darauf scheinen. Der Fahrer war ein grosser, blonder Mann mit Sonnenbrille, die fehl am Platz wirkte. Auf dem Beifahrersitz sass ein junges Mädchen, das einen Pagenschnitt trug. Es schaute sich neugierig und selbstbewusst um, während es sich mit einer Straussenfeder unter dem Kinn kitzelte.  Auf dem hinteren Sitz sass wohl die Ehefrau. Sie trug einen Pelzmantel aus Kuhfell. Ihre grauen Haare hatten schwarze Tupfen. Und sie hatte ebenfalls eine Sonnenbrille auf der Nase. Die andere Frau auf dem Rücksitz klammerte sich am Vordersitz fest und war grünlich im Gesicht. Sobald der Wagen hinter dem Zollhaus stand, öffnete sie die Tür und übergab sich lauthals auf die Strasse. 
 
 ,,Wibrandis, Liebe, verstehe mich nicht falsch,'' sagte die Dame mit dem Kuhfellmantel. ,,Aber möchtest du sicher bis zur Odilia mitfahren?'' Die Frau, die eine seltsam altertümliche Haube und ein dickes, graues Cape aus rauer Wolle trug, wischte sich mit einem Tuch den Mund ab. ,,Natürlich, Hathor!'' antwortete Wibrandis. ,,Ich schaff das schon, ich muss mich an die Raserei erst gewöhnen!'' ,,Was sie wohl macht, wenn sie auf die Autobahn fahren!'' flüsterte der eine Zollbeamte dem anderen zu. Sie grinsten. Vielleicht hätten sie weniger gegrinst, wenn sie von dem blinden Passagier im Kofferraum geahnt hätten\dots
 
Die zweite DS, die hinter der goldenen stehen geblieben war, war schwarz. Auch sie glänzte. Die Fahrerin konnte kaum über das Steuerrad schauen, füllte aber dafür, soweit die Zollbeamten es sehen konnten den unteren Teil ihres Platzes aus. Ihre Beifahrerin war nicht viel grösser als sie, hätte aber zweimal auf dem Sitz Platz gefunden.

Auf der Rückbank sassen zwei Jugendliche, ein Junge und ein Mädchen, die als einzige normale Kleidung trugen. Dafür teilten sie sich ihren Platz mit einem Hund und einem Affen. Die Beamten fragten sich, stumm, jeder für sich, sie wollten schliesslich keine schlummernden Kollegen wecken, ob die Einfuhr von Affen als Wildtierexport gelten müsste. ,,Er sitzt ja ganz manierlich auf seinem Platz,'' meinte der eine schliesslich. Der andere Beamte, der sofort wusste, was sein Kollege damit ausdrücken wollte, beeilte sich zu antworten: ,,Genau! Und wenn er da auf der Rückbank\dots , so friedlich\dots , ich mein, dann ist es ja wohl ein Haustier, oder?'' ,,Eigentlich könnte man ihn mit einem kleinen Kind verwechseln\dots ?' dachte der andere laut. ,,Genau!''
 
 In dem Sinne wussten die Beamten nicht, wie gut sie daran taten nicht in den Kofferraum zu schauen, denn der war gefüllt mit Körben aus Bast in denen erstaunlich grosse Schlangen schliefen. und vermutlich waren sie überglücklich, nicht zu wissen, dass der Hund kein Hund, sondern ein Schakal war\dots
 
Die kleine Beifahrerin stieg aus dem Wagen, während sich die Fahrerin einen dicken Stumpen in den Mund schob und ihn anzündete. Im Nu war das Auto voller Qualm. Es hustete vierfach. Die Frau wendete sich an die dritte DS. Diese war grösser als die anderen, es war ein Leichenwagen. Die zierliche Person hatte nur ein weisses Leinenkleid an, wie es im alten Ägypten getragen wurde und darüber eine Küchenschürze, die mit Augen bestickt war. Schuhe hatte sie keine.

Mit offenen Mündern beobachteten die Beamten, wie sie zum Leichenwagen ging und die Heckklappe aufmachte. So unauffallig wie sie konnten, reckten sie die Köpfe. In dem Wagen lagen zwei Sarkophage, die wie Figuren geformt waren und kunstvoll in altägyptischem Stil bemalt waren. Der eine hatte eine merkwürdige Form, als würde der Besitzer ein sehr langes Gesicht haben, wie eine Krokodil\dots Die beiden Zollbeamten schauten sich an und taten sich leid, immer hatten sie Pech\dots

Affen und Hunde, gut, das war eine Sache, aber historische Artefakte, das konnte später Ärger geben. ,,Bonjour, Madame!'' riefen sie und näherten sich dem Wagen. Die Frau hatte ein langes Zepter aus dem Wagen geholt. Bevor die beamten weiter sprechen konnten, hielt die Frau ihnen den Zepter unter die Nase wie einen Speer und zwischen den beiden Särgen tauchte ein grosser, schwarzer Hund auf und fletschte die Zähne.

Die Beamten blieben stehen und die Frau blieb an der Grenze stehen. Der Fahrer der goldenen DS stieg aus, und der Fahrer des Leichenwagens. Der Bestatter hatte einen grauen Mantel an und schwarze Hosen. Die Beamten atmeten auf. Sie hatten auf dem Beifahrersitz des Leichenwagens nämlich eine Art Herkules bemerkt, der trotz Winter nur ein T-Shirt mit einem Falken an hatte. 

,,Boujour, Messieurs! Qu'est-ce, qu'il y a?'' fragte der Bestatter und ging freundlich auf die Beamten zu, die hatten jedoch nur einen Blick für den Herkules, der kurz antäuschte, er wolle aussteigen. ,,Rien de rien!'' rief der eine und winkte hektisch ab. ,,Rien du tout!'' der andere. Sie drehten auf der Stelle und gingen auf ihr Zollhaus zu. In das sie die letzten Meter hinein rannten und dann die Tür zu knallten. ,,Oh? Oui! tant pis!'' meinte Thot und zuckte mit den Schultern. ,,Was die wohl für ein Problem hatten?'' sagte er, als er wieder einstieg. Horus zuckte ebenfalls mit seinen muskulösen Schultern: ,,Keine Ahnung!'' 

 Währenddessen hatte Isis das Zepter in komplizierten Bewegungen durch die Luft bewegt. Wenn sie die Luft über der Grenze zwischen der Schweiz und Frankreich berührte, verschwamm die Strasse dahinter einen Moment, wie als hätte sie eine glatte Wasserfläche mit ihrem Stab berührt. Auf diese Weise blieben die magischen Muster einen Augenblick in der Luft sichtbar. 
 
,,Verrückt! Wie kann sie in der Luft Zeichen machen?'' fragte \am . ,,Sie ritzt mit dem Zepter den Schutz in die Grenze ein, damit uns die Feinde nicht auf unserer Spur folgen können.'' Erklärte Amset. ,,Aber die Grenze gibt es doch nicht!'' meinte \am . ,,Also schon, aber nur in den Köpfen der Menschen! Dachte ich\dots '' ,,Da hast du falsch gedacht!'' meinte Amset trocken. ,,Naja, ganz unrecht hast du nicht.'' sagte er dann. ,,Es dauert eine ganze Weile bis die Grenze auch im Geist entsteht. Aber wenn genug Menschen für lange Zeit eine Grenze denken, dann gibt es sie auch!'' ,,Aber bleibt die dann ewig zwischen den Ländern?'' fragte \am . ,,Nöh! Sie ist eh nur für den da, der an sie glaubt!'' ,,Hä?'' fragte \am verwirrt. ,,Wir können die Grenze doch gut gebrauchen, um einen magischen Schutz an einem bestimmten Ort zu installieren, deshalb ,glauben' wir heute an die Grenze. In einigen Tagenlöst sich die Magie auf. Und wenn wir den Schutz nicht mehr brauchen und nicht an die grenze glauben, dann ist sie weg. Dann ist es egal!'' ,,So einfach ist das?'' fragte \am misstrauisch. ,,So einfach ist das!'' antwortete Berta vergnügt. Sie paffte an ihrer Zigarre und ein synchrones vierfaches Husten echote von der Rückbank.

,,Los, weiter!'' Isis sprang auf ihren Sitz. Sie legte die Finger auf das Radio und schien zu lauschen, dann machte sie den Rock'n Roll-Sender an. ,,Lets fets!'' meinte sie und lehnte sich zurück in den Autositz. Ein kurzer Augenblick, dann war sie eingeschlafen\dots

\section*{5}
\addcontentsline{toc}{section}{5}

\am hatte die stirn an die Scheibe gelehnt und schaute auf die vorbeigleitende Landschaft. Die Hügelkette der Vogesen bildete in der flachen Rheinebene ein dunkles Band. Wenn die Strasse sich der Bergkette näherte wurden Burgen und Wald sichtbar. Jetzt im Winter waren die Hügel grünbräunlich gefärbt.

Ich möchte heim! Dachte \am . Wie es wohl dem Bruder geht? Warum hat Berta mich von zuhause weggebracht? unsicher schaute sie auf Bertas Profile. \am war unwohl, es waren so viele Gefühle und Gedanken in ihrem Kopf und  Bauch und sie wusste nicht, wieviel die Wesen, Götter oder was auch immer, davon mitbekamen. Sie wusste nicht, ob sie nur ein Werkzeug war, oder ob es darum ging ihr zu helfen. Bei was wollten sie ihr helfen? 

\am erinnerte sich an die Träume: Sie war durch einen findesteren Gang gekrochen, in dem nur um sie herum etwas Licht leuchtete. Der Gang war befand sich in gelblichem, beigem Sand. Die Luft war staubig und trocken, es roch leicht nach muffiger Sand.

sie bekam keine Luft. Etwas presste ihren Oberkörper zusammen. Es lag nicht an der Luft im Gang, sondern ihr Brustkorb war wie ein Steinklotz geworden und wollte sich nicht mehr bewegen. Weder zum Atmen, noch wollte er dem Herzen genug Platz lassen zum Schlagen. Das Herz? 

\am hatte keins! Anstelle des Herzens befand sich einfach ein Loch. Ein schwarzes Loch. Wenn sie dies im Traum bemerkte, dann bekam sie Todesangst. Sie lag bewegungsunfähig in dem Gang und wusste sie würde ewig dort liegen und Angst ausstehen\dots denn ohne Herz konnte sie nicht sterben!

Die Rettung nahte dann. Ein schlanker, gelb-schwarzer Hund kam und brachte ihr ein Bündel. Es war ein runder Gegenstand, der in eine schmutzige, staubige, graubraune Binde, die an verschiedenen Stellen verschliessen und in Fäden gerissen war, gewickelt war. \am wickelte den Gegenstand aus. Es war ein vertrocknetes, schwarzes, verschrumpeltes Herz. Es war völlig vertrocknet, leicht, geruchslos und tot. Es war das gruseligste, was \am je in der Hand gehalten hatte. 

 Der Hund schnappte das Herz und stiess es in das Loch in Amélies Brust. Der Hund konnte nicht sprechen, aber laut denken. ,Das Herz ist dir geliehen!' teilte er ihr mit. ,Du musst es dir erst wieder verdienen!' ,,Was habe ich den Böses getan?'' fragte \am . ,Du hast nichts Böses getan, du bist verflucht!'
 
 Eine Welle von Hass, Angst und Grauen wälzte sich durch den Gang auf sie zu. In dem Wirbelsturm aus Hass, tauchte eine Gestalt auf. Das Gesicht zu einer Fratze verzerrt. Obwohl die Person ein ägyptisches Äusseres hatte, erkannte \am sie sie sofort an der schrillen, wütenden Stimme. \am war wie gelähmt.
 
 Der Hund zerrte an ihr und drängte \am zur Flucht. Sie rannten durch den gelbgrauen Gang, während die Welle des Fluches dichter und dichter den Gang niederwalzte, ausfüllte, vernichtete\dots
 
\am schrie!

Tef jaulte und Hapi machte einen Satz nach vorn und klammerte sich an Isis. \am atmete schwer. Sie spürte die Anstrengung des Traumes noch. Wie sie mit dem Hund um ihr Leben durch den Gang rannte mit dem krüppeligen, verschrumpelten, ekligen Mumiending in ihrem Körper, das ihr geliehen worden war. \am legte die Hand auf die Brust und spürte das hektische Schlagen. Sie spürte Abscheu und Angst.

Tef zappelte auf dem Sitz und versuchte ihr das Gesicht zu lecken. Er zitterte. ,,Du bist der Hund!'' \am strich Tef liebevoll über den das Rückenfell, ,,Danke!'' Tef schaffte es \am über die Wange zu schlecken. Langsam beruhigte er sich. ,,Ein Fluch?'' fragte \am in die Stille des Wagens. 

,,So sieht es wohl aus!'' antwortete Berta. ,,Warum hast du nichts gesagt?'' fragte \am wütend. ,,Weil ich mir nicht sicher war. ich wollte dich nicht unnötig erschrecken!'' antwortete Berta. ,,Aha! Hat nicht geklappt!'' meinte \am . ,,Berta, ich habe jemanden gesehen im Traum!'' \am stockte, sie wollte es nicht aussprechen, weil sie nicht wusste, wie sie damit umgehen sollte. Berta musterte \am kurz über die Schulter, bevor sie wieder auf die Strasse schaute. Sie hatte die Tränen in ihren Augen wohl gesehen. 

,,Ja, jetzt weisst du auch, warum du hier bist, nicht wahr?'' fragte sie. ,,Ja!'' antwortete \am . ,,Aber warum?'' ,,Bei Odilia werden wir es hoffentlich ungestört rausfinden, antwortete Amset. 

Sie schwiegen, jeder hing seinen Gedanken nach. Einer Frage konnte \am sich nicht entziehen, wem konnte sie trauen? Der Kloss im Hals, der nach dem Traum aufgetaucht war, würgte sie. Das Gefühl, das sie in der Basiliskenhöhle angefallen hatte, packte sie erneut: Ich bin alleine! Ich bin mutterseelenallein.

\section*{6}
\addcontentsline{toc}{section}{6}

Schweiss gebadet sass Luise im Bett. Sie waren gleich nachdem sie die Grenze passiert hatten zu ihrem Hotel gefahren. Während Wilfried sich mit dem Jungen die Beine in der Altstadt vertrat und ohne es zu wissen die gleichen Wege benutzte wie seine Tochter in den Nächten zuvor, hatte Luise geschlafen.

Sie griff nach dem Wasserglas auf dem Nachttisch. Doch ihre Hand zitterte und sie verschüttete die Hälfte auf sich und das Bett. ,,Mist!'' Sie wischte sich eine Träne weg, die den Weg über ihre Wange gefunden hatte. Sie stand auf und ging unter die Dusche. Sie hoffte Wilfried und der Junge würden bald wieder zurück kommen.

Sie dachte mit dem warmen Wasser der Dusche würden die finsteren Traumbilder verschwinden, aber sie taten es nicht. da war Luise sich sicher, \am war in Gefahr. Aber was sollte sie Wilfried sagen?

Der Traum indem \am von einer unheimlichen, hasserfüllten Macht durch einen irdenen Tunnel verfolgt wird, den hatte sie in der letzten Zeit häufiger geträumt. Er war sehr beängstigend. Schliesslich hatte Luise Berta um Rat gefragt. Sie hatte sie nach einem Schlafmittel gefragt. Berta hatte dieses und jenes gefragt und dann hatte Luise von dem Traum erzählt. Berta hatte ihr Baldriantee mit Schafgarbe und Melisse  zubereitet und ihn mit Akazienhonig gesüsst. Nach dem ersten Schluck hatte Luise sich freundlich bedankt und den Aufguss, nachdem Berta das Zimmer verlassen hatte in ihrem Bad weggekippt.

Als Luise das Gespräch mit Berta wieder in den Sinn kam, fiel ihr wieder ein, wie sehr sie sich über Bertas Reaktion gewundert hatte. Das Gespräch war noch nicht lange her\dots Sie erinnerte sich, \dots es war der Nikolaustag gewesen. Sie hatte mit Berta am Abned die geputzten Stiefel der Kinder mit Schokolade und kleinen Geschenken gefüllt. 

Es war einer der seltenen Momente bei denen sie sich mit ihrer alten Amme wieder verbunden gefühlt hatte. Einer der seltenen Momente, wo sie sich nicht dem Groll hingab und Berta im Inneren vorwarf sie verraten zu haben\dots

Sie hatten gelacht und gescherzt und fröhlich heimlich in der Küche gewerkelt. Sie hatten die Stiefel vor die Tür gestellt und waren dann in das Wohnzimmer gesessen. Luise hatte sich ein Glas Wein eingeschenkt und Berta hatte ihr Glas mit Whisky aus Wilfrieds Hausbar gefüllt. In der Stimmung des Nikolausgeheimnisses und in der Vorfreude auf die leuchtenden kinderaugen am Morgen, hatte Luise Berta von dem Traum erzählt. Und Berta hatte nachdenklich zugehört\dots 

Und das war merkwürdig\dots Nicht, dass Berta zugehört hatte, nein, Berta hörte jedes Flüstern auch wenn sie so tat als wäre sie eine schwerhörige Alte. Nein, es war merkwürdig, weil Berta die Wohnstube, wie es sich für Bedienstete gehörte, praktisch nur bei der Arbeit betrat. Jetzt fiel es Luise auf, Berta hatte weder vorher noch nachher je auf dem Sofa gesessen. Und- Berta war an dem Abend spät verschwunden. Sie hatte Luise den Kräutertee gebracht, den diese wiederum im Waschbecken entsorgte und war dann verschwunden.

Berta war erst am Vormittag wieder aufgetaucht, als der Junge und \am schon in Schule und Kindergarten waren\dots

Hatte Berta mit \am verschwinden doch viel mehr zu tun, als Luise glauben wollte? Hatte Berta damals an dem Abend vom Nikolaus schon herausgefunden, was Luise heute durch den schrecklichen Traum erfahren hatte? Luise schüttelte verwirrt den Kopf, woher hätte Berta das erfahren sollen?

Luise hatte den Traum zwar mehrmals geträumt. Den Traum indem ihre Tochter durch einen irdenen Gang um ihr Leben lief und ihr Herz einem Fetzen Stoff eingehüllt war. Aber heute, heute hatte sie zum ersten mal von dem Verfolger geträumt. Sie selbst war es, die \am in diesem Traum zu Tode hetzen wollte\dots

Luise trat in das weisse, schlichte, aber elegante Zimmer und lies den blick über die Altstadt gleiten. Und dann, sie wusste nicht woher, war sie sicher, Berta hatte es gewusst. Was sollte sie Wilfried sagen? Berta hat unsere Tochter unter fadenscheinigen Vorwänden entführt, um sie vor mir zu schützen?

Das war nicht das schlimmste\dots Das Schlimmste war, dass Luise seid sie aus diesem Traum erwacht war, wusste, wie recht Berta daran tat \am vor ihr zu verstecken! Für einen Augenblick fühlte sie Trauer und Tränen, die dann von brennendem Hass verdrängt wurden und ein intensives, gieriges Lächeln in ihr Gesicht drängte.

Wilfried und der Junge kamen herein. Der Junge lief zu Luise und sie nahm ihn auf den Arm. ,,Na, mein Schatz, hast du alle Häuser von basel angeschaut?'' Wilfried musterte Luise, während der Kleine von all den spannenden Dingen berichtete. ,, Du siehst erholt aus!'' meinte Wilfried. Luise lächelte ihn an. Sie schien von Innen heraus zu leuchten. ,,Ja, ich habe geschlafen und geduscht! Jetzt bin ich munter!'' Wilfried lief ein Schauer über den Rücken, er hatte schon öfter Hass in den Augen seiner Frau gesehen. Aber noch nie so viel\dots

\section*{7}
\addcontentsline{toc}{section}{7}

Thot und Anubis standen am Geländer, am Rand des Plateaus. Ihr Blick reicht weit, weit über die ganze Rheinebene bis zum Schwarzwald, der als sich dunkler, welliger Strich mit den Dämmerungswolken vermischte. Es war still. 

-,Heute Nacht ist die schwerste Prüfung für mich!' dachte Osiris. ,,Ja, wir sind gut vorbereitet,'' sagte Thot. -,Und wir haben unsere Routine von vielen tausend Jahren. Für irgendetwas muss die auch gut sein!' dachte Anubis.

Anubis trug den Imiut, auf diese Weise konnte sein Halbbruder Osiris auch die Aussicht in sich aufnehmen. -,Ein schöner, friedlicher Ort, voller Geschichten,' dachte der Herr der Unterwelt. -,Ja, edler Herr, das ist er!' Eine Frau von majestätischer Gestalt, bescheiden in ein Äbtissinen Gewand gehüllt, war an ihre Seite getreten. Thot deutete eine Verbeugung an und reichte Odilia die Hände. Auch Anubis senkte kurz den feinen Hundekopf. 

,,Ich glaube la Chapelle des Larmes und la Chapelle des Anges werden Euch gute Dienste leisten. Der andere Teil des Kloster wird von zu vielen Menschen aufgesucht, auch in der Nacht. Aber dafür wird euch der Raum des ewigen Gebetes zusätzlich helfen und schützen. Seit Willkommen!'' ,,Danke, hohe Frau, auch im Namen des obersten Gottes zu dessen Ehren wir hierher gekommen sind,'' antwortete Thot ernst. Sie schauten wie die Ebene im Dämmerlicht verschwand.

,,Es ist nicht mehr viel Zeit! Ihr solltet noch speisen und ich will den alten Herrn der Sonne begrüssen!'' Odilia winkte mit den Armen, als wollte sie eine Schar Hühner in den Stall treiben und Toth und Anubis begaben sich in den Speisesaal des Hotels, wo sie die restliche Mann- und Frauenschaft beim Essen trafen.


\chapter*{5. Nacht}
\addcontentsline{toc}{chapter}{5. Nacht}

\begin{quotation}

\emph{V Pater omnis telesmi totius mundi est hic.\\5. Dies ist der Vater alles Vollbrachten der ganzen Welt.  \\Tabula Smaragdina}

\end{quotation}

\section*{1}
\addcontentsline{toc}{section}{1}

Auf dem Mont St Odile herrschte Aufregung im Speisesaal. Vor der Tür hing ein Schild ,,Wegen Bauarbeiten geschlossen'', damit keine, oder wie sie so sind, fast keine Menschen herein schauen würden. Die Schwestern vom heiligen Kreuz, die das Kloster betreuten, wunderten sich, hatten aber genug anderes zu tun.

Die Göttinnen hatten eine lange Tafel aufgebaut, an dem alle nicht nur einen Platz, sondern auch ihre Lieblingsspeisen vorfanden. Odilia füllte sich ein kleines Schälchen mit Schokoladencreme und übergoss diese mit Vanillesauce. ,,Das wollte ich schon immer mal probieren, seid es heisst der Odilienberg wäre ein Steinpudding!'' sagte sie. Sie ass langsam und genoss jeden Happen. Nur durch die Anwesenheit der Götter war ihr Ka kräftig genug, um einem Körper Gestalt zu geben und Pudding zu schlürfen.

Mit der Zeit materialisierten sich Wächtergötter und das ,Kollegium, das die Opferspeisen in der Höhle verteilt' und nun ersteinmal selbst tüchtig bei Brot und Bier zulangte, der Lieblingsspeise des gewöhnlichen ägyptischen Gottes. Es wurde kaum gesprochen. \am spürte ein unheimliches Knistern in der Luft. Die Götter waren angespannt. 

,,Berta? Warum sind alle so still? Ist was passiert?'' fragte sie leise. ,,Nein, \am , heute Nacht müssen wir alle viele gefährliche Rituale durchführen. Es ist eine Art Lampenfieber!'' \am schluckte ängstlich. ,,Du musst keine Angst haben!'' sagte Berta, was \am aufhorchen liess, denn Berta sagte sonst nie, was sie ,nicht' machen sollte.\footnote{Bertas Pädagogiktrick war einfach. ,Sage nicht, was du nicht willst und bringe das Kind auf die Idee. Sondern sage, was du willst! Wie bei einfachen Dingen üblich, fällt dies unter die Kategorie schwierig, weil wir selten wissen, was wir wollen und häufig bemerken, oder befürchten, was wir nicht wollen} ,,Berta, jetzt hast du mir Angst gemacht!'' sagte \am promt.

Berta seufzte. ,,Heute Nacht musst du dich erinnern, \am ! Und Erinnerungen können sehr unangenehm werden!'' Berta wendete sich \am zu und packte sie unter dem Kinn. ,,Aber! Vergiss nicht, vor allem nicht in dieser Nacht: Erinnerungen können dir nichts tun! Sie werden Gefühle auslösen, gute und Schlechte. Aber die Erinnerung selbst kann dir nicht weh tun!'' ,,ich verstehe nicht, was du meinst!'' sagte \am und spürte Trotz in sich aufstampfen! Berta musterte sie intensiv: ,,Gut so!''

\am drehte den Kopf, sie hatte keine Lust auf Bertas mystischen Erklärungen. Ihr Blick blieb in Amsets hängen. Ihr Herz machte einen kurzen Satz bevor es sich verschloss. Amset lächelte. Er strahlte sie an, als könnte er nicht anders. \am senkte schnell den Blick. Die Schamesröte kroch von unten hoch und hatte ihre Wangen in Sekunden erreicht. Blödmann! Dachte sie. Denn Rest des Abendessens sass \am schweigsam und mürrisch da und starrte die Tischplatte an.

\section*{2}
\addcontentsline{toc}{section}{2}

Die Götter schlenderten im letzten Dämmerungsschein durch den kleinen Garten auf die beiden Kapellen zu. Sie standen direkt am steilen Abhang des Mont St Odile. Hinter der Kapelle der Tränen befanden sich die Gräber, die direkt in den Felsen gehauen waren. ,,Ist das das Grab für \am ?''fragte Re. ,,Ja! Ich hoffe, sie verkraftet es\dots '' meinte Thot. ,,Du solltest mal wieder auf Göttlich umschalteten, Thot! Also wirklich! Seit wir in der physischen Welt reisen, machst du dir wieder ständig Sorgen!'' meinte Horus. Er legte seine Pranke auf Thots Schulter: ,,Hey, du und ich, was haben wir zwei schon alles erlebt! oder? Und Re ist dabei und Anubis, Sobek und die Mädels!'' ,,Und wir haben Verstärkung!'' sagte Re. ,,Denke an all die Hilfe die wir bekommen haben!''

Thots Blick glitt über die Dunkelheit in der jetzt die Lichter der Zivilisation brannten. Die Städte und Dörfer der Ebene, die mit leuchtenden Bändern, den Strassen verbunden waren, glitzerten nah und fern, soweit das Auge reichte. ,,Ich weiss ja, es ist des Menschen Aufgabe zu zweifeln, aber übertreibe es nicht,'' meinte Re.

Re und Horus gingen in die Kapelle der Engel. Thot verharrte einen kurzen Moment. Für einen Augenblick fühlte er sich zurückversetzt. Er stand auf einer ähnlichen Anhöhe. Einem Tempelbezirk auf einem Berg und starrte auf das Land unter ihm. Doch in seiner Erinnerung wurden die Lichter der Häuser von einer riesigen Welle Wasser ausgelöscht. Bevor das Wasser den Tempel überspülte, hatte er sich mit seinen liebsten und besten Schülern in eine runde Kapsel aus durchsichtigem Kristall zurückgezogen. das Wasser kam und die Kapsel wurde mitgerissen. 

Thot lächelte in die Dunkelheit. ,Ich denke, Horus irrt sich. Nicht zweifeln ist menschlich, sondern den zweifel überwinden!' dachte er. Er eilte zur Kapelle der Engel.

\section*{3}
\addcontentsline{toc}{section}{3}

Die Frauen und Göttinnen hatten sich vor dem Eingangstor getroffen. Wibrandis, Berta, Odilia und \am trugen dicke, wollene, graue Umhänge. Hathor hatte ihren Kuhpelz umgelegt. Nur Isis hatte ihr weisses Leinenkleid und ihren Schmuck an, als wäre sie am Nil. Ihre Füsse steckten in Sandalen. Aus Respekt vor der Gastgeberin hatten alle Göttinnen und Wibrandis Schürzen angezogen, die mit Augen bestickt waren.\footnote{Der Legende nach war Odilia blind zur Welt gekommen und bei ihrer Taufe sehend geworden. Sie war nicht nur die Schutzheilige für da Elsass, sondern auch für das Augenlichts. Sie trägt deshalb gerne ein Buch und Augäpfel mit sich.} Im Gänsemarsch, angeführt von Odilia, machten sie sich auf den Pfad, der durch den bewaldeten Hang abwärts führte zu der heiligen Quelle la Source de St Odile.

Wäre Maat bei ihnen gewesen, hätte sie ihre Schwester Isfet, die hinter den Frauen her huschte, bemerkt. Aber Maat musste Re helfen die schwierige Zeremonie der Wiederbelebung vorzubereiten. So blieb Isfet unbemerkt. Bis sie unten an der Quelle ankamen. 

Die Grotte war mit einem schwarzen Eisentor verschlossen. Nur ein kleines Metallrohr ragte neben dem Eingang kurz über dem Boden aus dem Felsen und liess das Quellwasser in ein winziges rundes Steinbecken plätschern. So konnten die Pilger von dem gesegneten Wasser trinken und sich die Augen reinigen, ohne die Grotte betreten zu können. 

Odilia nahm ein Schlüsselbund von ihrem Gürtel, der ihr Äbtissinnengewand hielt und öffnete die Eisenpforte. Jede der Frauen nahm aus dem Rohr einen Schluck Wasser und wusch sich dann das Gesicht und die Hände schweigsam.

Als \am nach Berta an die Reihe kam sog sie scharf die Winterluft ein. Das Wasser war eisig kalt, aber wie durch ein Wunder nicht gefroren. Im Gegensatz zum feuchten Boden der Quellgrotte. Es war rutschig. Wibrandis plumpste prompt auf den gut gepolsterten Hintern. ,,Huch!'' rief sie. Ihre Hand berührte am Boden einen Fuss, der verborgen in der rechten Ecke der Grotte versteckt war. Diesmal begnügte sie sich nicht mit einem sanften ,,Huch!''. Wibrandis Scheckensschrei halte von der Grottenwand zurück und purzelte durch das Tal.

Isis war am schnellsten sie hatte Wasser geschöpft um es zu trinken und leerte ihre Hände in die dunkle Ecke. ,,He! Isis, du Doof!'' rief es empört. Isfet trat in das matte Licht, das der bewölkte Himmel von denn Städten zurückspiegelte.

,,Oh, nein! Wie bist denn du daher gekommen!'' Hathor zog Isfet am Ohr aus der Grotte. ,,He!'' rief die wieder. Hathor liess los. Die Frauen und Göttinnen scharten sich um das Chaos und beäugten es misstrauisch. ,,So ein Auto hat viele Plätze!''

,,Cool!'' sagte Amélie.

,,Ob es cool ist, wissen wir erst hinterher!'' meinte Berta trocken. Isfet streckte ihr die Zunge raus. Amélie spürte die Kälte wieder, die sich auf ihr Gesicht und die klammen Finger gelegt hatte.

,,Amélie, komm zu mir!'' sagte Odilia. ,,Die Nacht kann für uns nicht genug Stunden haben\dots '' Amélie machte vorsichtig einige Schritte in die Grotte zu Odilia. Die kniete vor dem kleinen Rinnsal, das in ein vereistes und bemostes Becken ran. Amélie hockte sich hin und Odilia, die ein feines Tuch aus ihrem Umhang gezogen hatte, wusch ihr damit sorgfältig die Augen mit dem kalten Wasser aus. Dabei murmelte sie unablässig vor sich hin. ,Sie betet!' dachte Amélie.

Als die Gebete und die Waschung beendet waren, nahm Odilia ein weiteres Tuch und verband Amélie damit die Augen. ,,He! Was soll das?'' ,,Deine Augen sollen sich auf eine innere Schau vorbereiten!'' antwortete Odilia vergnügt und führte Amélie aus der Grotte.

Berta hackte sich bei Amélie unter und führte sie sanft, aber zügig den schmalen Pfad durch den bewaldeten Hang aufwärts zum Kloster. Amélie stolperte nicht, obwohl der Weg rutschig vom Kies und holprig von den Baumwurzeln war. Es war, als würden ihre Füsse Sensoren haben, die im Voraus wussten, wo sie am sichersten stehen konnten.

Odilia führte die Frauen an, hinter ihr ging Wibrandis, dann Berta und Amélie, hinterher liefen Isis, Isfet und Hathor. Sie brauchten kein Licht. Im Vergleich zu Sokars Sand, oder der Basiliskenhöhle, war es hell um sie.

,,Isfet?'' flüsterte Isis. ,,Was?'' ,,Heute ist die entscheidende Nacht! Wenn es schief läuft, können wir morgen nach Hause fahren. Und Amélies Herz können wir Ammit als Snack in die Unterwelt mitbringen!'' sagte Isis. ,,Ich weiss!'' antwortete Isfet. ,,Dann versau' es nicht!'' sagte Isis. Isfet schluckte. Isis war sonst keine, die zweifelte. Aber dann grinste sie. Wenn es um Menschen ging, dann waren ihre Kräfte nicht weit\dots

\section*{4}
\addcontentsline{toc}{section}{4}

Während die Göttinnen zu der Quelle gepilgert waren, hatten die Götter die Zeremonie der Wiederkehr vorbereitet. 

Re hatte sich in den Sarkophag gelegt, indem Sobi seine Autoreise im Leichenwagen überstanden hatte. Er hatte einen weissen Schurz angelegt und einen wunderschönen Kragen aus blauen, roten und goldenen Perlen. Auf seiner Brust ruhte ein Abbild seiner morgendlichen Gestalt, Cheprie, der geflügelte Skarabäus. 

,,Bist du bereit, Vater?'' fragte Horus, der sich über den Sarg gebeugt hatte. Sein Kopf hatte das Antlitz des Falken angenommen. Er würde sowohl der Wächter in der Chapelle des Anges sein, als auch der Menschheitsrepräsentant, der die Wiederkehr des Sonnengottes aus dem ,Ei, das Leben enthält' würdigte und einen Teil der magischen Worte der Zeremonie sprach.

Anubis legte die Vorderläufe an den Rand des Sarges und schaute besorgt hinein. Er war es, der als Gott der Bestattung, ,des Kastens' wie es hiess, die Aufgabe hatte, dafür zu Sorgen, dass der oberste Gott in einem Stück die Erneuerung vollziehen konnte. Wenn er Fehler machte, würde am nächsten Tag nicht mehr dieselbe Sonne aufgehen!

Sobek stand als Wächter unter den drei Fenstern. Es würden noch einige wilde Gottheiten und Wächter in dieser Nacht auftauchen und da Horus Teil des Ritus war, war Sobek zum obersten Sicherheitschef bestimmt worden. Seine mächtige Krokodilschnauze ragte in die Dunkelheit. Der Krokodilsgott hatte seine halbmenschliche gestalt angenommen.\footnote{Schliesslich war es von Vorteil, wenn der SIcherheitschef neben einem Krokodilsgebiss auch Arme und Beine zur Verfügung hatte, die zupacken und treten konnten!}

Als letzter trat Thot an das Fussende. Er lächelte seinem Freund und Meister zu. ,,Ich bin bereit, Jungs! Macht die Schotten dicht! Osiris wartet sicher schon auf mich!'' Horus packte den Sargdeckel und schwang in auf die Kiste. Anubis konnte im letzten Moment die Pfoten zurückziehen.

Acht Gottheiten hatten sich im Kreis in der kleinen Kapelle versammelt. Von den blaue-roten Mosaiken, die die Götter vor dem Abendessen bewundert hatten, war nichts zu sehen. Durch die vier bemosten Oberlichter drang zu wenig Licht. Nur die Silhouetten der drei Fenster mit Rundbogen und Sobeks Schatten zeichneten sich schwach ab.

Sobald Thot die Tür von Aussen verschlossen und verriegelt hatte, begannen die Götter ihre magischen Formeln und Gebete der fünften Stunde zu sprechen und Anubis verwandelte sich. Er brauchte seinen menschlichen Körper, der jedoch nie ohne Hundekopf zu sehen war. Anubis Körper war, wie es sich für den ägyptischen, obersten Totenpriester gehörte, in einen weissen Lendenschurz und das Fell eines Leoparden gekleidet.\footnote{Anubis war das uneheliche Kind von Nephtys und Osiris. Aus diesem Grund wurde er als Kind vor Seth, Nephtys Ehemann und Bruder, versteckt. Isis wurde seine Ziehmutter. Anubis hatte sich angewöhnt ausschliesslich seinen Hundekopf zu tragen. Wobei er während der Arbeit seinen menschlichen Körper dem des Hundes vorzog. Seine Überlegung war simpel. Er kannte seinen Onkel Seth gut genug, um zu wissen, dass er nie ganz sicher sein würde vor ihm. Sein menschliches Gesicht, das er wie Thot, Horus und alle anderen Götter hatte, wollte er sich für diesen Augenblick ,aufsparen', um es als sicheres Versteck nutzen zu können! Da er diesen Entschluss schon als kleiner Bub, äh Gott getroffen hatte, wusste selbst er nicht, wie sein menschliches, erwachsenes Gesicht aussah.}

Anubis stand am Fussende des Sarges, Horus am Kopfende. Beide hatten ihre Arme über dem Ruhenden ausgestreckt und sprachen mit dem Götterchor zusammen die Einleitung des Ritus der Wiedergeburt.


\section*{5}
\addcontentsline{toc}{section}{5}

Bevor Thot in die Chapelle des Larmes huschte, um die magische Leitung zu übernehmen, ging er auf die Rückseite des kleinen Häuschens. Es stand dicht am Abhang, jedoch befand sich ein schmaler, begehbarer Streifen dahinter, bevor es senkrecht in die Tiefe ging. Auf der drei Meter breiten Felsplatte befanden sich die Merowingergräber, die direkt in den Stein geschlagen worden waren. Das vordere Grab hatte eine menschliche Form. Für den Kopf war eine Rundung ausgehöhlt worden um sich dann zu erweitern, damit die Schultern Platz finden konnten. zu den Füssen wurde das Grab schmaler. Wie ein klassischer Sarg mit einem Extraplatz für den Kopf.

Das Grab dahinter war rechteckig. Dafür hatte es einen Rand, indem ein Deckel eingefügt werden konnte. Dieser fehlt bei dem anderen Grab.

Als Thot vor dem Geländer stehen blieb, waren die Frauen sich bei den Gräbern niedergelassen. Berta hatte sich in das offene Grab hineingelegt, ihr schwarzes Kleid und die Schürze plüschten sich auf und bildeten einen kleinen Stoffhügel.

Odilia und Isis halfen Amélie in das zweite Grab zu steigen. Amélie hatte die Augenbinde noch an und tastete sich unsicher vorwärts. Die Göttinnen waren insgeheim froh, hatte Odilia auf die Augenbinde bestanden. Sie waren sich nicht sicher, ob Amélie in ihrer jetzigen Verfassung freiwillig und dicht am Abgrund in das Grab geklettert wäre, wenn sie es gesehen hätte.

,,Uuih!'' rief Amélie. Odilia hielt ihre Hand und Isis stand im Grab und zog vorsichtig an Amélies Hüfte, damit sie den Schritt hinein wagen konnte. ,,Sachte!'' sagte Odilia. ,,Seit ihr denn soweit!'' sagte Berta, der es im ersten Grab schon eng geworden war. Für sie wäre eine runde Höhlung für den Kopf mit einer anschliessenden quadratischen Kiste passender gewesen. Es war viel Platz frei unter ihren Füssen, den sie gerne auf Hüfthöhe gehabt hätte.

Wibrandis glättete Bertas Kleid und breitete die Schürze ordentlich aus. Sie wendete den Blick nach oben. ,,Oh, beta, schau! Die Wolken sind abgezogen. Dort ist der grosse Bär und die Zwillinge!'' Berta schnaufte aus den Tiefen des Grabes.

Derweil hatten Odilia und Isis Amélie in das Grab bugsiert. ,,Nun lege dich hin,'' sagte Odilia. Amélie kniete sich vorsichtig auf den kalten Boden. ,,Sagt mal, ist das hier ein Grab, oder was?'' fragte sie, als ihre Hände um die Kannten glitten. Die Göttinnen schwiegen betreten. Sie hatten kein Problem mit Gräbern, Menschen meistens schon. Odilia blickte hilflos zu Isis. Und selbst die grosse Zauberin wusste nicht, was sie sagen sollte, damit Amélie nicht schreiend aus dem Grab hüpfte und womöglich in ernste Schwierigkeiten geriet, indem sie den Abhang hinunter stürzte\dots

,,Schätzchen, nun hör mir mal gut zu!'' Bevor eine der Göttinnen es verhindern konnte, hatte Isfet sich vor Amélie gekniet. ,,Du wirst in ein Jahrhunderte altes Grab gesteckt, damit genau in dieser Nacht und in der nächsten Stunde möglichst viele deiner Erinnerungen zum Vorschein kommen!'' ,,Ich will aber nicht!'' Amélie zerrte an der Augenbinde und als sie sie nicht lösen konnte, wollte sie aus dem Grab steigen. Aber Isfet stiess sie zurück. ,,He! Lass mich raus!'' Isfet hatte Amélie gepackt und die zappelte und wand sich.

,,Schau! Du musst dich erinnern! Wir müssen wissen, was passiert ist!'' ,,Wie, was passiert ist? Wozu ist das wichtig?'' antwortete Amélie böse. ,,Weil wir wissen müssen, ob dein Herz mit Maats Feder aufgewogen werden kann!'' antwortete Thot an Isfets Stelle, der abgewartet hatte, ob die Göttinnen ihren Ritus begonnen hatten. ,,Genau! Und wenn dein Herz zu leicht ist, dann frisst dich meine Freundin Ammit!'' sagte Isfet höhnisch. ,,Ist das ein Scherz, oder was?'' ,,Nein! Kein Scherz! Entweder wirst du vom Fluch gefressen, oder wir schaffen es rechtzeitig die Totenriten, die damals schief gelaufen sind, nachzuholen!'' sagte Berta aus der Tiefe ihres Grabes.

Amélie erschlaffte verwirrt in Isfets Griff. ,,Totenriten? Ich lebe doch!'' ,,Jetzt hast du eines der wesentlichen Probleme erkannt!'' meinte Thot. Er sah auf die Uhr. ,,Wir müssen!'' sagte er nur und wendete sich dem Eingang der Chappelle des Larmes zu.

,,Alle an ihre Plätz!'' sagte Hathor in einem Ton, der seit jahrtausenden Gehorsam gewöhnt war. Wie eine unsichtbare Hand drückte er Amélie nieder. ,,Es ist verflucht kalt in so einem Grab!'' maulte sie. ,,Der Deckel kommt gleich auf dein Grab! Ich werde darauf Wache halten! Denk dran, Erinnerungen können nur deine Gefühle wecken, sie können dich nicht verletzten!'' sagte Isfet. 

Die drei ägyptischen Göttinnen beschworen eine Felsplatte, die genau auf dem Grab Platz hatte und verschlossen es. Amélie schrie, bis der Deckel sich schloss\dots

,,Ich gehe in die Kirche zum Altar um das ewige Gebet zu unterstützen. Wir werden die erste Schutzmauer bilden!'' sagte Odilia. Hathor erhob sich und sagte: ,,Ich werde dich begleiten und von dort kann ich die drei Riten gleichzeitig im Auge behalten und den Eingang!''. Die beiden huschten über das Geländer und eilte auf dem schmalen Weg an den Gebäuden vorbei zu ihrer Kirche.

Wibrandis setzte sich zwischen die Gräber, fest in ihren Umhang umwickelt und begann ihr Gebet. Sie würde regelmässig Bertas Atem kontrollieren.

Berta verband sich über den Boden mit Amélie. Sie liess einen Goldfaden von ihrem Herzen durch den Boden bis zu Amélies Hand geleiten, auf diese Weise konnte sie wie mit einem Verstärker fühlen, was Amélie fühlte.

Isfet hockte sich tatsächlich auf Amélies Grab. Sie wusste selbst nicht, welche Rolle sie spielte, denn für sie gab es ,zu Hause' in dem altägyptischen Ritus der fünften Stunde keine Aufgabe. Allerdings nahmen daran auch keine lebenden Menschen teil. Isfets untrügliches Göttin-des-Chaos-Gefühl sagte ihr einfach, sie sollte hier sein und warten. Sie griff in die Tasche ihres plüschigen Kapuzenpoullovers und holte eine Packung Zigaretten raus. Sie zündete sich eine an und machte einen Rauchring. 

Hathor hatte sich neben Berta niedergelassen.

,,Gut! Oder auch nicht!'' sagte Isis zu Hathor und warf einen seltsamen Blick auf ihre Urgrosstante auf dem Grab. ,,Ich muss!'' Sie griff nach einem Stab, der verborgen im Dunkel an der Kapelle gelehnt hatte und kletterte über das Geländer. Sie begab sich auf den Platz zwischen den Kapellen, der sich nach Osten über die Ebene öffnete. Über ihr im Süden hatte sich Orion, das Sternbild des Osiris erhoben. Sirius, ihr Stern leuchtete darunter kräftig in die Nacht. Isis atmete tief durch und liess dann einen tiefen gleichmässigen Ton erklingen. Dieser breitete sich über ihren Körper aus und wanderte durch den Boden weiter zu den beiden Kapellen.

\section*{6}
\addcontentsline{toc}{section}{6}

Thot war in die Chapelle des Larmes eingekehrt. Die Horussöhne hatten nach dem Abendessen, für ihren Grossvater Osiris die Kapelle für die nächtliche Zeremonie vorbereitet. Die Schlangenkörbe waren je vor einer Wand platziert worden und die Schlangen lagen wie schützende Bänder um die unterste, breite Stufe vor dem kleinen Altar, in der sich der Tränenstein befand. 

Der Gott der Unterwelt lag mit dem Kopf nach Osten und den Füssen, die gegen den Eingang zeigten, nach Westen. Der Sarg stand auf dem goldfarbenen Gitterkreuz unter dem sich die tiefen Dellen befanden, die Odilias Knie während der Gebete für ihren Vater, im Boden hinterlassen hatten.\footnote{Odilia: ,,Heijeijei, also die Kapellen  heutzutage, ich meine, anno dazumal, halten auch nichts mehr aus. Kaum eine richtiges Sühnegebet hat diese durchgehalten. Es hat gleich Löcher in den Boden gemacht.}

Nachdem er alles überprüft hatte, trat Thot nach draussen und verriegelte die Tür. Was hier in dieser Kapelle und in der Duat in der fünften Stunde im ,Ei, das Leben enthält' passierte, war eines der grössten Geheimnisse der ägyptischen Götter.

Die Verwandlung des toten Gottes in den Lebensspender. Tief in der Erde, in Sokars Reich, fand diese Verwandlung statt. Niemand, selbst Thot nicht, hatte die Verwandlung die Osiris jeden Nacht vollzog, je direkt erlebt. 

Thot stellte sich unter die kahle Linde, die neben dem Eingang zur Chapelle des Anges stand. Er hatte so Isis und auch die Chapelle des Larmes, die von den Horussöhnenen bewacht wurde im Blick.

\sterne

Odilia und Hathor hatten ganz vorne, direkt vor dem Altar in der Kirche Platz genommen. Jede war mit sich beschäftigt. Odilia betete aus tiefem Herzen für den Schutz der Reisenden. Ihr Herz wendete sich ganz dem Christus zu, wie sie es gewohnt war. Dabei hatte sie stets das Gefühl gehabt, ihre Teil von ihr würde sich lösen und in den aufsteigen. 

Dieser Teil sank jedoch zu ihrem erstaunen in die Erde. Odilia wunderte sich, sie schlug ein Kreuz vor der Brust und wurde für einen Moment unsicher. Tat sie das richtige? Schliesslich waren die ägyptischen Götter Heiden? Sie konzentrierte sich intensiver auf ihre Fürbitte, dennoch wanderte ihr Gebet nicht in den Himmel. Es wanderte in die Erde und bewegte sich dort schnell vorwärts. Zu ihrer Überraschung stieg dieser Teil, wie sie deutlich spürte durch die Tränenöffnung im Stein in die Kapelle der Tränen ein. Ihr Geist bewegte sich in der Kapelle. Der Sarkophag auf dem Altar erschreckte sie zuerst. 

Sie spürte wie das Wesen darin, der Gott mit dem Tod kämpfte. Eine grosse Trauer wuchs in Odilia und wanderte zu Osiris. Und je mehr Trauer und Mitleid durch die Öffnung im Boden aufstieg, umso mehr spürte Odilia, wie das Schwarze, verhärtete, tote, das sich bis an den Rand des physischen Sein zurückgezogen hatte, ausstülpte und zu wachsen begann. Der Tod fiel ab, wie eine Kruste und ein zartes, durchscheinendes Wesen entfaltete sich. Es leuchtete in grün und gold.

Odilia, die mit ihrem Geist in der Kapelle schwebte, bemerkte ein Lichtnetz, das erst den Sarkophag überzog, den Altar und dann von dort sich wie ein Stern ausbreitete und im Boden verschwand. So dicht sie konnte näherte sich Odilia dem Sarg.  Als sie ihn berührte, war die Wärme überwältigend stark und sie wurde zurück in die Kirche geschleudert.

\section*{7}
\addcontentsline{toc}{section}{7}

Tef sass an der Rückseite der Chapelle des Larmes, denn es war seine Aufgabe mit den Brüdern zusammen Osiris in der Nacht zu bewachen. Jeder war für eine bestimmte Stunde zuständig, in der er die Wache leitete. Jeder von ihnen wachte über eine Himmelsrichtung. Tef sass nur wenige Meter neben Amélies Grab. Isfets Silhouette hob sich schwach gegen den Sternenhimmel ab. Tef entdeckte den Sirius, den Hundsstern. Es tröstete ihn, die vertrauten Sternbilder zu sehen. Den Orion, das Sternbild seines Grossvaters und den Sirius, den hellsten Stern des Firmamentes im Grossen Hund\dots

Tef war nicht wohl bei dem Gedanken, wie es Amélie dort in dem Grab gehen mochte und was ihre Erinnerung zu Tage fördern würde. Hatte er vielleicht doch einen Fehler gemacht, damals\dots 

Hapi war der Herr des Nordens und sass in einer Linde, die auf dem Platz zwischen den Kapelle wuchs. In der konnte er selbst ungesehen Ausschau halten. Gegen den Wind und Kälte schützte ihn eine Wolldecke, die ihm Odilia gebracht hatte und er hatte sich aus dem Obstkorb einige Bananen mitgebracht, schliesslich konnte eine Nacht lang werden.

Er durfte seinen Posten nicht verlassen. Denn er hatte wachte auch über Isis, die zwischen den beiden Kapellen die beiden mächtigen Götter Re, als Herrn der Oberwelt und Osiris, den Herrn der Unterwelt mit ihrer Magie verband. Sie durfte unter keinen Umständen gestört werden.

Kebi hatte auf die Autofahrt verzichtet. Er fühlte sich gesund genug selbst zu fliegen. Früh am Morgen war er aufgebrochen. Dabei hatten ihn seine Brüder vorsorglich an die französische Grenze gebracht. 

Amset es sich leichter vorgestellt, das Autofahren. Die kleine Schramme am Kotflügel der schwarzen DS war entstanden, als er versuchte die schmale Auffahrt zum Tor hinaufzufahren. Da auf dem Hof den blauen Hauses nicht genug Platz war, mussten die Götter auf dem des weissen Hauses parken.

Den Rest der Fahrt, fand Amset, lief doch ganz gut. \footnote{Dabei war es reine Glücksache gewesen. Auf dem Autobahnkreuz von Basel hatte sich zur gleichen Zeit ein Zusammenstoss mehrerer schlaftrunkener Pendler ereignet, die die leichte Glätte ignoriert hatten. Da die Polizei dort voll beschäftigt war, konnten die Horussöhne unbehelligt viele Dutzend Verkehrssünden begehen, ohne aufzufallen.}

Kebi jedenfalls genoss den Flug entlang der Vogensenkette. Er hatte genug Zeit sich seine Morgen- und Mittagsmaus zu fangen. Und kam rechtzeitig zum Abendbrot auf dem Mont Sante Odile an. Dort machte er sich über ein Stück Fisch her und fühlte sich wieder rundum Kräftig und zufrieden.

Nun hockte er mit wachsamen Augen auf der Dachrinne der Kapelle über dem Eingang. Er war der Herr und Wächter des Westens, dabei hatte er sämtliche Zugange auf den hinteren Platz der Klosteranlage im Blick. Niemand konnte sich ungesehen nähern. Weder aus dem Gebäude von der Seite her, um die Anlage herum.

Amset, als Herr und Wächter des Südens, hatte ebenfalls den schmalen Weg um die Klosteranlage herum im Blick. Diese Seite der Kapelle hatte wie die anderen ein Fenster mit rundem Bogen und einem Gitter. Sie war dem kleinen Garten zugewandt und Amset hatte sich als Posten eine einige Stufen ausgesucht, auf die er sich mit einem Kissen setzte. Er war unglücklich, denn er würde am wenigsten von allen sehen und hören, was auf der anderen Seite und hinter der Kapelle geschah. Aber er war Wächter. Er seufzte.

Seitdem er Amélies Schrei gehört hatte, fühlte er sich noch schlechter. Sie war ihm den ganzen Tag aus dem Weg gegangen, obwohl das in dem engen Auto kaum möglich gewesen war. Als sie das Kloster erreicht hatten, war sie ständig von den Frauen umgeben gewesen. 

Er wollte ihr gerne helfen\dots aber er wusste nicht, was eine Menschenfrau hilfreich fand. Göttinnen waren zufrieden, wenn sie genug Bier und Essen hatten und die richtigen magischen Werkzeuge und starken Kämpfer an ihrer Seite, wenn es nötig war. Aber Menschenfrauen? Trotz der kurzen Zeit, die Amset mit Amélie verbracht hatte, war er sich sicher, kein Schimmer davon zu haben, was in ihrem Kopf oder Herzen passierte. \footnote{Dabei stand Amset die grösste Überraschung noch bevor! Nämlich herauszufinden, was Million Jahre Evolution mit den Dingen zwischen den Beinen mit den Menschen anstellen konnten. Wie wir ja bereits wissen, sind Götter in dem Bereich der Fortpflanzung wesentlich pragmatischer als Menschen. Pardon, die ägyptischen Götter.} 

 Amset wusste, wenn er erlaubte um die Ecke einen Blick auf die Frauen zu werfen, dann wäre es mit seiner Wachsamkeit vorbei. Er musste seinem Bruder Tef vertrauen, der würde ihm alles berichten, schliesslich verbrachte er die Nacht bei Amélie.\dots Amset spürte einen Stich in der Brust. Ein kleiner Gedanke fragte sich, ob er seinem Bruder trauen konnte? Was wussten sie schon, was mit Amélie passiert war?

Als Amset sein Gedankendämon vor die Augen flatterte, zuckte er zusammen. Der kleine Kerl war giftgrün vor Eifersucht und ihm  tropfte schwarzes Misstrauen wie Teer aus dem Maul. Er war pickelig und hässlich\dots Amset wischte ihn weg, wie  eine Fliege. Der Gedanke klatschte an die Wand und rutschte daran runter. Amset rechnete damit, dass der Dämon sich auflösen würde, aber er tat es nicht\dots

\section*{8}
\addcontentsline{toc}{section}{8}

Amélie schrie. Es war stockfinster und ihre Stimme prallte an die Grabplatte über ihrem Gesicht. Der Schrei kroch direkt von ihren Stimmbändern in die Ohren und liess sie implodieren\dots Amélie atmete ein, aber es kam keine Luft in ihren Lungen. Immer schneller schnappte sie nach Luft. ,,Hilfe!'' sie kratzte mit den Fingern über den Stein der Platte und versuchte sich gegen die Platte zu stemmen, dann schnürte sich ihre Kehle zu\dots

Sie schnappte und schnappte und japste und japste, rang um Atem\dots Amélies Augen rollten nach hinten und sie wurde Ohnmächtig. Ihre Seele löste sich von ihrem Körper, aber auch die konnte das Grab nicht verlassen. Isfet verhinderte es, versperrte den Weg. Und so schlüpfte Amélies Seele in die Erinnerung.

\sterne

Amélie erwachte durch einen unglaublichen Schmerz. Er befand sich in ihrem Zentrum, in ihrer Brust. Das einzige, was sie neben dem wahnsinnigen Schmerz spürte, war das rasende Schlagen ihres Herzens. Verrückter Weise konnte sie es für einen kurzen Augenblick nicht nur schlagen fühlen, sondern sehen. Für einen Sekundenbruchteil nahmen Amélies Augen wahr, wie ihr schlagendes Herz von einer starken Hand gehalten wurde. Die Hand hatte das Herz aus einer Öffnung im Brustkorb gehoben. Ein weiterer Augenblick: Die Augen eines Priesters, die ihr in ihrer Erinnerung eine Erinnerung weckten, die erklären könnte, warum sie in dem Grab lag, wenn der Moment nicht vorüber wäre und Amélie erneut in Ohnmacht fiele\dots

In dem Moment in Ohnmacht fiel, erlöst durch den Schock, der sich im Körper ausbreitete als ein weiteres Paar Hände begann an dem Herzen aus dem aufgebrochenen Brustkorb zu reissen.

Zwei Paar Hände, die ein Herz aus dem wehrlosen, betäubten Körper rissen und es in die Kanope des Duamutef warfen, während es ein letztes mal das Blut bewegte und über die Hände spritzte. Stimmen, die sofort magische Formeln sprachen, zweistimmig, während die blutigen Hände die Kanope des Duamutef ohne die Leber, aber mit dem sterbenden Herzen versiegelten. 

Zwei Stimmen von denen eine hysterisch Lachte und die andere rau schluchzte.

\sterne

Gebannt starrten Duamutef und Isfet auf die Erinnerungsschnipsel, die Isfet durch die Grabplatte gezogen hatte. Da Amélie sich an einen Mord erinnerte, bei dem sie umgebracht worden war, blieb die Erinnerung in dem Augenblick stehen, in der das Herz aufgehört hatte zu schlagen. 

Auf Isfets ausgestreckter, flacher Hand stand die Kanope. ,,Sie waren unglaublich schnell!'' sagte die Göttin. Tef leckte sich aufgeregt die Lefzen, -,Sie haben das Herz in die meine Kanope geworfen, als es den letzten Schlag gemacht hat!' ,,Und dann haben sie es sofort gebannt und verflucht dadurch wurde es praktisch unsichtbar.'' -,Unglaublich! Wie konnte ein ausgebildeter Totenpriester das tun? Er muss doch wissen, dass er sich selbst auf ewig an dieses Verbrechen bindet?' ,,Er muss von allen guten Göttern verlassen gewesen sein!'' meinte Isfet. ,,Aber er war nicht allein! Leider konnte ich die zweite Person nicht erkennen. Aber schliesslich war sie es, die an den Armen des Priesters gezerrt hat und den Mord begangen hat!''

Tefs Nase leuchtet hell in der Nacht, so blass war sie geworden. Isfet hatte sich auf die Grabplatte gekniet und hielt ein Ohr daran. ,,Schnell! Wir müssen Amélie raus holen! Ich spüre sie nicht!'' Tef sprang über das Geländer, das die neugierigen Touristen davor bewahren sollte in die Gräber zu plumpsen, oder gleich den Abgrund hinunter zu fallen.

Er lief an die Ecke der Chapelle des Larmes und rief aufgeregt nach Amset. Dieser fackelte nicht lange und sprang von der Aussichtsplattform in den Bereich der Gräber hinunter. 

Wibrandis, kreidebleich im Gesicht, half Berta auf. Die hatte einige Mühe sich aus dem zu engen Grab zu befreien. Schliesslich gelang es Wibrandis Berta mit einem Plopp aus dem Grab zu ziehen. ,,Blödes Design!'' Schniefte Berta und rieb sich das Blut in die tauben Arme und Beine.

Isfet und Amset hatten die Platte gelöst. Amset stieg vorsichtig in das Grab und hob Amélie sachte an den Schultern hoch. Er horchte an ihrer Brust. Sie hatte aufgehört zu atmen\dots

Entsetzt sahen sie sich an. in diesem Moment hörten sie einen Fetzen von Isis Zauberformel, die sie auf dem Platz zwischen den Kapellen sprach. ,,Schnell! Zurück mit ihr in die Kiste!'' schrie Isfet. Amset sah sie entgeistert an. ,,Was soll das? nein!'' rief er. ,,Mach, was sie sagt!'' fiel Berta ihm ins Wort, sie hatte begriffen, was Isfet plante. 

Amset legte Amélie zurück und Berta und Isfet deckten sie mit der Platte wieder zu. ,,Hast du eine Ahnung, wie es gehen könnte?'' fragte Berta. ,,Nö!'' antwortete Isfet. ,,Aber schlimmer kann es nicht werden, als durch den selben Fluch zweimal zu sterben!'' ,,Hier stirbt niemand!'' sagte Berta grimmig und krempelte sich, trotz der empfindlichen Kälte, die Ärmel hoch. Dann gab sie ihre Anweisungen an Isfet und Wibrandis.

,,Und ihr Buben verschwindet! Ihr bringt mir nur das Züüg durcheinander.'' Amset war verwirrt. Er hörte Isis Stimme wieder\dots Er hatte seinen Wachtposten verlassen\dots Er lief zurück an die Südseite. er spürte eine ungewohnte Kälte auf den Wangen und als er mit der Hand darüber wischte, bemerkte er die Nässe der Tränen. 

,,Isfet, wir haben nur eine Chance, wir müssen Amélie an die Erneuerungszeremonie anschliessen!'' sagte Berta. Isfets Augen blitzten schwarz. ,,Wenn Du es versaust, dann wirst du dir wünschen sterblich zu sein!'' flüsterte Berta. Die beiden Göttinnen starrten sich für kurze Zeit an. ,,ich bin das Chaos und nicht die Wohlfahrt!'' flüsterte Isfet. ,,Nimm, was du musst, aber nur soviel du tragen kannst!'' antwortete Berta. ,,Bitte, liebe Isfet, sonst ist doch alles umsonst gewesen!'' sagte Wibrandis und rang die Hände.

Die beiden Göttinnen hatten sich die Brüste entblösst und die Schuhe ausgezogen. Sie trugen beide Ringelsocken mit denen sie kraftvoll auf den Boden stampften und um das Grab tanzten. Aus ihren Kehlen kam ein rauer, schriller Gesang. Der in irrem Gelächter aufbranndete und in leises Röhren überging. Anschwoll und abebbte, die Geräusche der Natur, des Lebens nachahmend. Wibrandis sträubten sich die Haare, aber sie tat wie geheissen. Sie sass auf dem Grab und betete zu allen Göttern, die anwesend waren und den einen an den sie glaubte.

\section*{9}
\addcontentsline{toc}{section}{9}

Es ging auf Mitternacht zu. Die Zeremonie steuerte auf ihren Höhepunkt zu. Der Sarkophag im Inneren der Tränenkapelle glühte. Kleine, elektrische Blitz entluden sich auf die goldenen Mosaiken und sprangen gleichzeitig von diesen über. Die Wächter sahen in den jeweiligen Fenstern, die sich zu ihren Himmelsrichtungen hin öffneten, die Entladungen.

Isis intonierte mit gewaltiger Stimme die magischen Formeln, die den Boden vorbereiteten. Mutter Erde, der Schoss allen Seins öffnete sich durch ihre Kraft und gab den geflügelten Schlangen den Weg frei. Diese verteilten die neu erwachte Lebenskraft an die gesamte Natur. Eine der Geflügelten kroch unter den dem Steinboden zur Kapelle der Engel. Wie die Wurzeln eines schnell wachsenden Baumes, hoben sich die Steinplatten, als die Schlangen, geleitet von Isis Zauberstab, unter der Chapelle des Anges verschwanden.

Sobald die geflügelten Schlangen schimmerte ihre Spur weiterhin golden in der Nacht. Die Spur führte unter der winzigen Kapelle, die auf dem kleinen Vorsprung throne, und verschwand. Im selben Augenblick, da die Schlangen unter der Chapelle des Anges auftauchten, schien ein Gewitter in die Kapelle eingesperrt worden zu sein. 

Dann hörten sie Sobek brüllen. Der Krokodilsgott brüllte aus tiefer Kehle, dabei vibrierte der Felsen. in Isis nackte Füsse, weiter zu Amset. Die Chapelle des Larmes zitterte und die Blitze erschienen wieder in den Fenstern, vor denen die Hörussöhne wachten. 

Wie ein Erdbeben, rüttelte der Bauch und die Platten des Kokodils den Felsen durch. Der Klang der auferstandenen Sonne. Der Klang Chepris. Er breitete sich aus wie die Wellen eines schwimmenden Krokodils.

Die beiden Göttinnen, die um Amélies Grab tanzten, stampften und liessen ihren kehligen, rauen Gesang anschwellen und eins werden mit dem Sonnenklang. Auf allen Vieren, mit den Händen und den Füssen stampften sie das die Töne des Lebens um und in die Grabstelle.

Wibrandis sprang auf, als der Klang sie schüttelte. Sie stolperte und rollte von der Platte, dann kam sie auf die Beine. Sie schloss sich den beiden Göttinnen an. Sie stampfte und klatschte und brüllte und jauchzte. 

\sterne

,,Schhh! Es ist so weit!'' flüsterte Hathor Odilia zu. Sie beiden erhoben sich leise und vorsichtig aus der Kirchenbank, um den alten Mann, der vor einer Stunde seinen Freund zum ewigen Gebet abgelöst hatte, nicht zu stören.
 Sie eilten durch das Kirchenschiff und über den schmalen Terrassenweg zum hinteren Teil des Klostergeländes. 
 
Erstaunt blieb Odilia stehen. Ihr ganzer Berg bebte und brummte, als wären Millionen von Bienen darin gefangen, riesige Bienen. Aus den Fenstern der Kapellen erhellten Blitze die Nacht. Der Stein schien zu leuchten, wie Wetterleuchten webten grünliche Bänder über den Boden.

Dann hörte sie die Schreie der Göttinnen, die um das Grab sprangen. Odilia bemerkte Hathor, die weiter gerannt war und bei Isfet und Berta angekommen war. Auch sie rief etwas. Odilia schien es, als wollte Hathor die beiden stoppen.

Als Odilia den Gräberplatz erreichte, hatte sich das Spektakel beruhigt. Die Bienen im Berg waren geschrumpft und die Blitze in den Fenstern, waren von einem warmen Schimmer abgelöst worden. Hin und wieder huschte ein kleines Leuchten über den Felsen.

,,Was habt ihr euch dabei gedacht!'' schrie Hathor. ,,Beruhig dich, Mutter!'' ,,Wie soll ich mich beruhigen, wenn du das Leben deines Vaters und das deines Urururneffen in Gefahr bringst?'' Odilia und Wibrandis hatten sich vor Schreck in den Schatten der Chapelle des Larmes zurückgezogen, wo auch Tef ängstlich auf seine Ururururoma starrte. Isis trat an das Geländer. Funken stoben von ihrem Zauberstab. Tef warf sich flach wie eine Flunder auf den Boden, mucksmäuschenstill\dots

Isfet stand lässig da und starrte Isis an. Berta hatte die Fäuste in die Taille gestemmt und stand breitbeinig da.\footnote{Bertas Figur hatte die Form einer Sanduhr. Einer Sanduhr, die nicht all zu dünn in der Mitte und umso kugelförmiger oben und unten war. Aus diesem Grund eignete sie sich, die Figur, hervorragend, um die Fäuste in die Taille zu stemmen. Jeder, der eine kleine, energische Grossmutter mit der Figur einer Venus von Willensdorf hat, weiss wovon ich rede.} Nach einem kurzen Moment der Stille, in dem sich Berta und Isis taxiert hatten, atmeten beide tief durch.

Berta ging zu Isis hinüber und Wibrandis und Odilia sahen erstaunt, wie die eine kleine aber runde Göttin sich durch das Geländer zwängte und die kleine, zierliche Göttin vorsichtig in ihre Arme nahm und tröstete bis diese aufhörte zu beben und zu schluchzen.

\section*{10}
\addcontentsline{toc}{section}{10}

Thot löste sich von der Linde, gegen die er sich gelehnt hatte. Er lief über den Platz, der nun still und dunkel da lag.

Die Göttinnen hatten sich um Isis und Berta versammelt. Ihre müden, blassen Gesichter zeichneten sich von der Dunkelheit ab. ,,Meine Damen, uns bleibt nun nichts weiter übrig, als das erste Morgenlicht abzuwarten!'' 

,,Ja, aber Herr, soll den die Amélie im Grab bleiben? Wird sie nicht erfrieren?'' fragte Wibrandis. Berta und Hathor warfen sich einen Blick zu, sie hatten es beide bemerkt: Wibrandis rechnete damit, dass Amélie am Leben war.

Hathor nahm Wibrandis Hand, um die Frau zu beruhigen: ,,Du musst dich nicht sorgen, Wibrandis! Amélie als erster Mensch Teil der Erneuerungszeremonie gewesen. Der Same, den Osiris Kraft in sie gelegt hat, muss bis Sonnenaufgang reifen. Erst wenn Cheprie, der heilige Skarabäus die Sonneenkugel über den östlichen Horizont schiebt, ist die Zeremonie abgeschlossen!''

Wibrandis warf einen ungläubigen Blick auf die Grabplatte. ,,Sie hat den göttlichen Funken in sich, sie wird nicht frieren,'' beantwortete Thot ihre unausgesprochene Frage.

Berta nahm Isis bei der Hand und auch Isfet schien erschöpft. Sie alle wendeten sich Odilia zu. ,,Kommet! Der Kamin ist schon eingeheizt und die heisse Trinkschokolade steht auf dem Herd.

,,Mit Chili?''

,,Mit Rahm?''

,,Mit Ingwer, Kurkuma und Rosenblüten?''

,,Mit Mokka?''

,,Mit Whisky?''

,,Schokolade?''

Die Anspannung der letzten Stunde löste sich in Gelächter. Odilia breitete die Arme aus: ,,Mit allem!'' rief sie. Sie gingen in den Frühstücksaal und setzten sich an den Kamin.

Wibrandis liess es sich nicht nehmen, nachdem sie vorsichtig einen Schluck heisse Schokolade, ,,bitte mit Honig!'' probiert hatte, den Horussöhnen drei Becher voll zu bringen.

,,Einen mit Banane!''

,,Einen mit Milch und Honig!''

,,Einen mit Vanille!''

und ein Tellerchen mit Pouletresten vom Abendbuffet.

\section*{11}
\addcontentsline{toc}{section}{11}

Wilfried war für wenige Minuten eingenickt. Er hatte sich in der Hotelbar in einen Clubsessel zurück gezogen. Er wollte Luise und den Jungen nicht stören. Die innere Unruhe, die ihn seit Jahren nicht schlafen lies, war noch stärker. Er war stundenlang durch Basel gejoggt, ohne die Stadt zu sehen, nur rennen, rennen\dots

In den kurzen Traum sah er eine junge Frau unter sich liegen. Sie konnte sich nicht bewegen. Ihr Brustkorb war aufgerissen und er hatte das schlagende Herz zwischen seinen Händen. Die Frau, die betäubt gewesen war, wachte durch den wahnsinnigen Schmerz kurz auf und starrte ihn an. Er konnte den Blick nicht wenden, war wie gelähmt. Da erschien eine Gestalt neben ihm und packte seine Hände und zerrte daran. 

Sie rissen zusammen das Herz aus der Brust der Frau. Und dann warfen sie es, immer noch zuckend, in einen Tontopf mit einem Deckel, der wie ein Hundekopf geformt war.

Wilfried stand senkrecht vor dem Sessel. Sein eigenes Herz hämmerte gegen seine Rippen. Er keuchte, er bekam keine Luft. er fasste sich mit der Hand an den Hals. Etwas störte ihn an seiner Hand\dots er schaute sie an. Ein rotbraunes Mal hatte sich darauf ausgebreitet. 

Er streckte die andere Hand aus. Auch sie war gezeichnet. An dem erstaunten Barkeeper vorbei stürzte Wilfried auf die Toilette und wusch und schrubbte die Hände bis sie nicht nur aussahen als waren sie in Blut getaucht, sondern wirklich bluteten. 


Schliesslich sank er erschöpft auf den Boden der Herrentoilette und schluchzte. Wie sollte er das Luise erklären? Und wie erst dem Kleinen?

Als er sich endlich in das Hotelzimmer wagte, traf er auf Luise. Ihre weisse Haut war noch durchsichtiger geworden. Und- sie trug Handschuhe! Er selbst behielt seine Hände in den Hosentaschen vergraben.

