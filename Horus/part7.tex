\part*{Siebte Stunde\\"`Welche die Bande des Seth abwehrt"'}
\addcontentsline{toc}{part}{Sechste Stunde}

\chapter*{7. Tag,}
\addcontentsline{toc}{chapter}{30. Dezember, }

\section*{1}
\addcontentsline{toc}{section}{1}

Amélie schlug die Augen auf. Sie lag in ihrem Bett im Blauen Haus. Sie konnte sich nicht mehr erinnern wie sie zurückgekehrt waren aus Frankreich.

,,Wie geht es dir?'' fragte Odilia und legte ihr kühle Hand auf Amélies Stirn. ,,Ich weiss nicht. Was ist passiert?''

,,Du bist in den See gefallen und Sobek hat dich gerettet! Du wärest fast ertrunken, denn während der Verwandlung des Sonnengottes konnte niemand sich bewegen und dir zur Hilfe kommen.'' ,,Oh, jeh! Die Sonnenbrille! Jetzt fällt es mir wieder ein! Oh,nein! Thot hat mich gewarnt und gesagt ich muss, uss die SOnnenbrille tragen, wenn Sobek zwischen die beiden Barken schwimmt!'' Amélie verbarg das Gesicht in den Händen.

Ein weiterer Erinnerungsfetzen schwebte in ihrem Kopf, \dots

,,Und der Kelch?'' rief sie. ,,Verloren!'' sagte Odilia. Amélie setzte sich auf. Eine Träne lief über ihre Wange. ,,Das bedeutet, es gibt kein Goldwasser? Und der Kelch ist im See versunken?'' ,,Ja!'' antwortete Odilia. Sie nahm Amélies Hand in die ihre und hielt ihren Blick, der ausweichen wollte, fest.

,,Du bist eine mutige Frau, Amélie! Und Fehler passieren! Es wird einen weiteren Weg geben. Und wenn nicht, dann ist genau das dein Schicksal!'' ,,Ich war zu dumm! Ich hätte nur die Sonnenbrille anziehen sollen. Das ist doch nicht schwer! Ich kann das nicht! Ich kann diese Aufgaben nicht übernehmen! Odilia, du musst ihnen sagen, dass ich es nicht kann!'' Amélie hatte die Äbtissen am Ärmel ihrer Kutte gepackt.

,,Unsinn!'' Odilia erhob sich. ,,Du musst dich sputen, denn in kurzer Zeit werden sich alle in Osiris Zimmer versammeln und beraten.'' Odilia ging leise aus dem Zimmer. sie drehte sich nicht mehr um.

Amélie blickte auf die Kommode auf der die heiligen Insignien lagen. Der Platz des Kelches war leer. Stab, Schwert und die Münze, die sie dank Wibrandis Hilfe wiedergefunden hatte, lagen auf der weissen Leinendecke.

Amset! Amset, er war nicht hier heute Morgen. Er wird mich hassen! Dachte Amélie. Sie suchte sich schlichte, schwarze Kleider aus der Kommode, streifte den Bademantel um und schlich unbemerkt in das Bad der Frauen. Ein dicker Klos wuchs in ihrem Hasl, als sie an der Küche vorbeischlich und das leise Gemurel der Götter hörte, das gedämpft in den Gang floss.

-Sie reden sicher über mich, was ich für eine Versagerin bin, schoss es ihr in den Sinn. WIeder musste sie an Amset denken und ihr Herz zog sich schmerhaft zusammen. Sie würde ihn vergessen! Jetzt, wo er all die Strapazen auf sich genommen hatte und sie versagt hatte, würde er sich von ihr abwenden. -Soll er doch! Ich schaffe es auch allein! Amélie spürte, wie sich alles in ihr zusamenzog und sie eine Gänsehaut bekam. Sie schluckte, aber ihr Rachen war wie gelähmt\dots

Versteinert betrat sie das leere Bad und stellte sich unter die kalte Dusche. Eine wie ich hat das warme Bad nicht verdient!

\section*{2}
\addcontentsline{toc}{section}{2}

In Osiris Zimmer herrschte betretenes Schweigen. Bis auf Isfet, deren MP3-Player von dem niemand wusste, woher sie ihn hatte, ein bassiges \dots umz\dots umz\dots umz im Raum verteilte. Osiris lag von der Zeremonie der Nacht belebt in seinem Bett und schimmert grünlich. Isis hatte sich auf das Fussende gesetzt und an die Wand gelehnt. Ihre Schlange hat sich auf ihrem Schoss eingeringelt und schnarchte leise.

Geb lag mit dem Schnatterer auf dem Boden des Zimmers. Die Anwesenheit von ihm und seinem Sohn Osiris, die beide Vegetationsgötter waren, hatte das Zimmer mit Waldboden gefüllt. Die Wände des alten Hauses, die aus Lehm und Holz gebaut waren, warne lebendig geworden und hatten sich begrünt. So wirkte das Zimmer wie das verwunschene Zimmer eines Baumelfen und wie ein winziger Wald gleichzeitig.

Für Re hatte sie den Sessel geholt, ebenso für Thot. Die beiden sassen und tranken Tee und Kaffee mit ernsten Gesichtern. Selbst der Sonnengott schaut still. Hathor brachte zusammen mit Berta und Wibrandis Servierwagen mit verschiedenen kleinen Leckerein. 

Die Horussöhne liessen sich nicht lange bitten und schnappten sich leise die Leckerbissen. Hapi verzog sich mit zwei Bananen auf einen dicht belaubten Ast, der aus der Wand gewachsen war. Kebi liess sich von Wibrandis drei kleine, frische Mäuse auf denselben Ast legen. Für  Tef hatte Wibrandis eine tönerne Schüssel mit Hundemuster und gefüllt mit leckeren Innereien in die hintere Ecke des Zimmers gestellt. Der Schackal liess es sich schmecken.

Nur Amset fehlte.

Anubis lag zu Füssen von Thot in einem riesigen Hundekorb, der mit feinem königsblauem Brokat ausgekleidet war\footnote{Der hervorragend zu seinem schwarzen, glänzenden Fell passte. Gerade für einen Bestattungsgott ist es von Vorteil, wenn er über guten Geschmack und Stil verfügt.}

Für die Göttinnen und Frauen waren gemütliche und da alle Göttinnen klein waren, niedrige Sessel gebarcht worden. Hathor und Berta hatte ihre Füsse dennoch auf einen Schemel gelegt. Hathors Sessel war mit Stoff von  Kuhfellmuster und Bertas mit einem aus Schneeflocken bezogen. Berta hatte Waldi auf den Schoss genommen.

In einer Hängematte aus weissem, golddurchwirktem Leinen lag Maat und schaukelte sacht. Die uralte Göttin in mädchenhafter Gestalt sah merklich mitgenommen aus. Einzig Horus konnte nicht still halten. Er trabte im Zimmer auf und ab. Manchmal fluchte er.

Neith, die Beschützerin von Tef und Muuter Sobeks hatte sich auf einen Baumstamm gesetzt, der mit Moos überzogen im Raum lag. SIe hatte ihren Pfeil und Bogen mitgebracht.

Odilia betrat den Raum und ihr schien ein Lichthauch zu folgen. Sie setzte sichauf einen Schemel vor Osiris Bett. Leise fragte sie den grünen Gott nach seinem Wohlbefinden. Und flüsterte mit Isis.

Hin und wieder durchbrach Waldis seufzen und weinen die Stille. Hathor zog einen grossen, mit Fleisch behafteten Knochen aus ihrer Schürze und hielt ihn dem Dackel sacht vor die Nase. Aber Waldi leckte ihr kurz über die Hand und versteckte seine Schnauze in Bertas Armbeuge. Die Göttinnen sahen sich traurig an und Hathor legte den Knochen neben Bertas Sessel.


\section*{3}
\addcontentsline{toc}{section}{3}

Luise erwachte im Hotelzimmer. Sie konnte sich nicht erinnern, was letzte Nacht geschehen war, aus an einige seltsame Bilder. Waren es Träume. SIe schaute auf ihre Hände, \dots sie waren blass und weiss, makellos\dots

Sie drehte ihren Kopf zu Wilfried, der sich neben ihr regte. Es war das erste mal seit langer Zeit, dass sie im gleichen Bett aufwachten! Wilfried schien ihr, war genauso irritiert wie sie. Und auch er hob seine Hände vor sein Gesicht und betrachtete sie. Sie waren wie immer\dots

,,Was ist geschehen?'' fragte Luise. Wilfried starrte an die Zimmerdecke. Er bedachte sich bevor er antwortete. ,,Ich weiss es nicht! Ich erinnere mich an wirre Träume. Sonst nichts!''

,,Wir müssen Amélie finden! Sie gehört nach Hause!'' rief Luise und schwang die Beine aus dem Bett. Dann hielt sie inne, eine Welle von Hass spülte über sie hinweg. Und sie wusste wieder, was ihre Aufgabe war. Keine Ruhe würde sie haben, solange Amélie lebte und Berta, die mächtige Hexe ihr auf den Fersen war. Sie würde nicht locker lassen.

Luise drhete sich zu Wilfried um. ,,Beweg' Dich!'' Wilfried zuckte unter ihem kalten Blick zusammen. ,,Ja, mein Schatz? Hast du gut geschlafen? Komm, ich helfe dir beim Anziehen, dann können wir fühstücken!'' Luise nahm die Hand ihres Sohnes, der zu ihnen gekommen war und führte ihn schnell in das Badezimmer.

Wilfried lag gelähmt. Der Vorhang des Vergessens riss und offenbarte die Ereignisse der Nacht. Wilfired würgte es. Er rannte an das Fenster, riss es auf und übergab sich. 

Als Luise mit dem Jungen aus dem Bad kam. Beide gewaschen und angekleidet, hatte WIlfried sich weider gefasst. Er ging stumm an ihnen in das Badezimmer und machte sich bereit. Er war froh, als er hörte wie sich die Zimmertür hinter Luise und dem Kind schloss.

Wilfried stand in der Mitte des Zimmers. Wie eine Spirale bewegten sich in Ihm Bilder, schreckliche Bilder. Sie tauchten auf und verschwanden, tauchten auf und verschwanden und nahmen an Intensität und Kraft zu.

Da war dieser Mann, der in der Höhle gelegen hatte. Was war mit ihm geschehen? Was war aus diesem Fremden geworden, dem sie völlig blauäugig gefolgt waren? Dieser Fremde\dots ein Schauer des Grauens rann Wilfried den Rücken hinunter. Eine panische Angst folgte\dots

Er stand erstarrt da, einem Alptraum gleich. Ein Schmerz fuhr im in die Brust und das zuckende Herz in seiner, seiner Hand, das er über den zerbrochenen Brustbein einer Frau hielt. Der letzte Blutschwall, der aus dem Organ in den leeren Platz in der noch zuckenden, warmen Leiche tropfte\dots

Wilfried schrie, doch es kam kein Laut mehr aus seiner Kehle, das Enstetzten war zu gross.

Langsam kam Bewegung in seine Beine, Leben!

Er wollte aus dem Hotelzimmer stürmen\dots

Doch wohin? Mehr als ein verächtliches Lachen konnte er von Luise nicht erwarten. Berta, ja Berta hätte gewusst, was zu tun wäre, aber Berta war nicht da. Amélie, ihre Gegenwardt würde ihn trösten, sie war nicht da. Sein Sohn? War es sein Sohn? Er war soweit weg, eingehüllt in Luises Haltlosigkeit, eingehüllt in ihren Sog aus Liebe und Verzeweiflung.

Wilfried spürte keinen Hunger, er spürte keinen Durst. Das Blut des WIlden Mannes hatte ihn durchdrungen. Er spürte die fremde, machtvolle Kraft, die in ihm floss. Sie rann wie ein Feuer durch seine Adern. Ein Feuer, das ihn verbrannte. Leise, leise, unaufhörlich frass es sich ins Zentrum, langsam\dots

Wilfried flüchtete in den Speisesaal des Hotels. Luise und der Junge sassen an einem Tisch und assen Gipfeli. Der Keline hatte Kakao vor sich stehen und knapperte zufrieden, spärlich an einer Schale Müsli. Luise hatte einen Kaffee vor sich stehen. SIe schien keinen Appetit zu haben.

,,Wir müssen Amélie finden!'' sagte sie. Ihre Stimme war erstaunlich sanft. Sehr sanft. Wilfried war froh. ,,Ja! Wenn wir bloss wüssten wo? Basel ist gross!'' Sie tranken schluckweise von ihrem Kaffee und schwiegen.

Wilfrieds Smartphone summte. Eine Nachricht! ,,Um 12:00 Uhr im Kreuzgang des Münsters!'' Eine eiskalte Hand packte Wilfried an der Kehle. Er hustete. Er reichte Luise stumme das Telephon. 

,,Sieht so aus, als wollte uns jemand helfen!'' sagte Luise. Sie lächelte und Wilfried starrte auf ihre entblössten Zähne.


\section*{4}
\addcontentsline{toc}{section}{4}

Amélie betrat Osiris Zimmer. Amset fehlt! dachte sie. Warum? Sie spürte Tränen in den Augen. ,,Komm, setzt DIch zu uns!'' Berta klopfte auf den Sessel neben sich. Amélie starrte sie an. Sie spürte keine Wärme mehr für die alte Frau. Sie hatte genug von ihrer Bevormundung, von ihrer Herrschsucht. 

Amélie blickte in die Runde. Sowohl Thot als auch Isfet hatten sie fest im Blick. Isfets Gesichtszüge hatten sich verändert. Bisher hatte die Chaosschwester der Maat wie ein ungepflegter Teenager ausgesehen. Nun schien es Amélie als wären die Gesichtszüge verschoben, in Unordnung wie Haar und Kleider. 

Amélie fürchtete sich. Isfets Gesicht wurde nicht nur einem Tier ähnlicher, es sah hungrig aus. Amélie spürte eine Aufmerksamkeit und erleichtert bemerkte sie Thots wachsamen Augen. Auf sich und auf Isfet. Isfet versteckte ihre Verwandlung nicht.

,,Guten Morgen! Nehmt Kaffee, nehmt Tee, esst\dots Ihr werdet es brauchen!'' begrüsste der Sonnengott seine Familie und Freunde. ,,Es war eine\dots schwierige Nacht und die heutige wird ebenso eine Herausforderung sein!'' Er nahm einen Schluck von seinem Tee. Die Sonnenbrille verdeckte seine AUgen. Odilia hatte sie gut bewacht und der Sonnengott hatte sie sich wieder eingesetzt.

,,Der Zeitsprung, den wir gemacht haben ist gross! Es haben sich Dinge verändert! Es haben sich DInge verändert von denen wir in den letzten tausend JAhren in der Duat nicht wussten, die wir, ja doch, ein wenig verträumten?'' Es war mucksmäuschen Still unter den Göttern, unter den toten und dem lebenden Menschen. Konnte das sein?

,,Auch für uns ist die Ganzheit zu gross!'' sprach Thot sanft. ,,Wir wissen vieles, wir leiten und beeinflussen vieles und doch ist es nur eine Kleinigkeit für die Gesamtheit der Dinge! Das ist das Schöne, das ist Leben!'' Thot schwieg. Nahm mit den Fingerspitzen vorsichtig ein Stück Mandelschokolade, betrachtete es kurz liebe- und lustvoll und schob es in den Mund. Er schloss den Mund und eine Welle von zarten Schokoladenschmelz  breitete sich im Raum aus. Berührte jeden einzelnen und hinterliess, ausser bei Tef und Anubis, ein seeliges, kurzes Lächeln!

,,Ich habe Cagliostro gebeten uns auch heute zu bearten und zu begleiten. Er sollte gleich da sein!'' Schon klopfte es an der Tür und der kleine, kugelige Alchimist rauschte in das Zimmer. Er verbeugte sich und strahlte. ,,Buongiorno, mie dee, miei dei, donne e uomini! Stehts zu Euren Diensten!'' 

Wibrandis sprang entzückt auf und bot dem Gast ihren Sessel an. Dieser zierte sich, gab ihr einen formvollendeten Handkuss und nahm neben Thot Platz. WIbrandis brachte ihm eine schöne, wohlgeformte vergoldete Tasse mit Schokolade. Der Alchimist seufzte entzückte. Er hielt sich die Tasse vor die Nase und sog geräuschvoll den Duft ein. ,,oh lala, condito con peperoncino e cardamomo, mmmh buono, delizioso Il cibo degli dei, caucauatl!'' Wibrandis errötete.

,,Kannst DU uns weiterhelfen, Alessandro?'' fragte Thot. ,,Wir müssen uns gestehen, dass wir nicht genau verstehen, was gerade passiert. Es scheint vieles, was undenkar gewesen wäre zu unserer Zeit, plötzlich möglich und andererseits klappt die einfachste Magie nicht!'' Cagliostro wurde ernst. ,,Es hat eine interessante Verschiebung gegeben im Bweusstsein des Menschen. Es zeichnete sich schon zu meiner Zeit ab, aber wie weit es gehen würde, ahnte selbst ich nicht! Es scheint als wäre mit dem Intellekt, der jetzt seine Zeit zum Reifen hatte, die Fähigkeit verschwunden Wirklichkeit und Vorstellung auseinander zu halten. ich verstehe es nicht ganz. Aber mir scheint als würden die Menschen, wenn ich sie feinstofflich betrachte, nur noch aus dünnen Beinchen und riesigen Köpfen bestehen.''

,,Kopffüssler! Ja, Alles Kopffüssler!'' knurrte Berta. ,,Ich kann Alessandro nur recht geben. Ich habe jetzt immer bei und mit den Menschen gelebt. Sehr viele haben sich weit von ihrem Dasein auf der Erde, der Wirklichkeit abgewandt und leben einzig in dem, was ihr Gehirn an Bildern produziert. Wahnsinn! Völliger Wahnsinn!'' Berta schnaupte! Isfet lachte. 

,,Freunde, selbst ich weiss nichts genaueres. Aber ich konnte zwei Freunde gewinnen uns zu helfen. Wir haben Glück! Sie haben beide hier gelebt, gewirkt und können einfach hierher zurückkommen. Ich werde sie holen!'' Cagliostro, der bisher mit spitzer Zunge in den Kakao getappt und getippt hatte, schlürfte die Tasse leer, \footnote{Thot erschauderte über diese Grobheit.} sprang auf und lief hinaus.

Amélie erwachte aus ihrer Lethagie. Die wirren Gedanken in ihrem Kopf beruhigten sich kurz. Sie war gespannt, wen Cagliostro mitbringen würde.

Schon nach kurzer Zeit kam der Alchimist zurück, gefolgt von zwei Männern. ,,Voilà! Ich möchte Euch hier Carl Gustav und Albert vorstellen!''

\section*{5}
\addcontentsline{toc}{section}{5}

\section*{6}
\addcontentsline{toc}{section}{6}

\section*{7}
\addcontentsline{toc}{section}{7}

\section*{8}
\addcontentsline{toc}{section}{8}

\section*{9}
\addcontentsline{toc}{section}{9}

\section*{10}
\addcontentsline{toc}{section}{10}

\section*{11}
\addcontentsline{toc}{section}{11}

\section*{12}
\addcontentsline{toc}{section}{12}

\section*{13}
\addcontentsline{toc}{section}{13}

\section*{14}
\addcontentsline{toc}{section}{14}



\chapter*{7. Nacht}
\addcontentsline{toc}{chapter}{7. Nacht}



\begin{quotation}

\emph{VII Separabis terram ab igne, subtile a spisso, suaviter cum magno ingenio.\\7. Du wirst die Erde vom Feuer trennen, das Feine vom Dichten, lieblich mit großer Entschlossenheit.  \\Tabula Smaragdina}

\end{quotation}


\section*{1}
\addcontentsline{toc}{section}{1}

\section*{2}
\addcontentsline{toc}{section}{2}

\section*{3}
\addcontentsline{toc}{section}{3}

\section*{4}
\addcontentsline{toc}{section}{4}

\section*{5}
\addcontentsline{toc}{section}{5}

\section*{6}
\addcontentsline{toc}{section}{6}

\section*{7}
\addcontentsline{toc}{section}{7}

\section*{8}
\addcontentsline{toc}{section}{8}

\section*{9}
\addcontentsline{toc}{section}{9}

\section*{10}
\addcontentsline{toc}{section}{10}

\section*{11}
\addcontentsline{toc}{section}{11}

\section*{12}
\addcontentsline{toc}{section}{12}

\section*{13}
\addcontentsline{toc}{section}{13}

\section*{14}
\addcontentsline{toc}{section}{14}
