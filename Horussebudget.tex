3.2.8.
Wachtmeister Moser kratze sich nachdenklich am Kopf und fuhr mit seinem Protokoll fort: 22:20 Die Polizisten Moser und Meyer fahren langsam den Marktplatz Richtung BArfüsserplatze hinauf. Sie können die verdächtigen Personen jedoch  nicht sehen und daher beschliessen die Beamten Per Pedes die Gerbergasse zu inspizieren.

In der Nähe des gerbergässleins auf dem kleinen Platz treffen die beiden Polizisten auf die beiden Verdächtigen Personen. Sie haben sich vor dem Gerberbrunnen eingefunden. Dort liegen sie am Boden und reden laut und aufgeregt miteinander. Die Gans, oder der Ganter, watschelt auf dem Platz und schnattert. In den Wohnung über dem Bastelladen ist ein fenster aufgegangen und ein Bürger beschwert sich über die Ruhestörung. Er beruhigt sich, als er dsieht, dass wir Polizisten bereits vor Ort sind. In der zwischenzeit haben sich weitere fenster geöffnet. Die Anwohner beschweren sich, die poliszei solle endlich eingreifen.

Sobald sich die Polizisten nähern greift der Gänsevogel an und ein kleiner Hund, eine ART dackel kommt plötzlich aus der Ecke des Brunnens und fletscht die Zähne. dem Beamten bleibt nichts anderes übrig als die beiden Männer mit gehörigem Absatnd anzusprechen. Das ist schwierid, da sie nach wie vor am Boden liegen und die Beamten nicht bemerkt haben.

22:40: Die beiden Poizisten haben inzwischen zwei weitere Kollegen informiert, da sie sich den beiden Männern nicht nähern können, ohne von der Gans, dem Hahn und dem Dackel angegriffen zu werden. Der BEamte Moser hat einige tiefe Kratzer im Gesicht erhalten, die von den Sporen des Hahnes satmmen, während die Kollegin Meyer mehrfach von dem Dackel in die Wasde gezwisĉkt wurde. Die Gans greifdt abwechselnd beide Polizisten an.

Die Kollegen Schmied und Hofer treffen ein. Sie haben Fangnetze mitgebracht. Die Kollegen Meyer, Hofer und Moser versuchen die Tiere mit den Netzen einzufangen. Dies entpuppt sich als schwierig. Die Gans ist zu gross für das Netz, da damit in der Regel Katzen gefangen werden und sie sich daraus befreien kann sobald sie mit den Flügeln schlägt. Dabei trifft sie den Polizist Hofer am Schienbein, worauf dieser zu Boden geht und von den Kollegen Meyer und Moser aus der gefahrenzone gebracht werden muss. Dabei hat sich der Dackel bei Meyer an der Hose festgebissen und der Hahn sich beim  Moser auf dem Kopf in den Haaren festgekrallt.

Moser ist nicht sicher, ob das alles in dieser Ausführlichkeit in das Protokoll gehört. Aber seine Frau hat ihm den ganzen Kopf dick mit Salbe eingeschmiert  und mit einer guten Lage Binden umwickelt. Kollegin Meyer, deren Verband an den Waden man nicht sehen konnte, hatte gut lachen gehabt heute morgen. ,Von hinten siehste aus, wie ein mumifiziertes Ei' hatte sie gesagt. Er hatte gezwungenermassen mitgelacht, schliesslich wollte er nicht wie eine beleidigte Leberwurst dastehen. Aber leichtefallen ist es ihm nicht, schliesslich ist er froh über jedes Haar auf seinem Kopf. Warum musste der Hahn ausgerechnet ihn zerfleddern? Der Kollege Hofer hätte mit seinen dichten Locken ein viel lohnenderes Objekt abgegegen.

22:45: Dank dem wagemutigen und intensiven Einsatz der drei Polizisten Meyer, Hofer und Moser, kann sich Poliziste Schmied den beiden Männern nähern. Es wird höchste Zeit, den die ersten Anwohner und spüteren Passanten sibd stehen geblieben und beginnen mit ihren Handy Fotos zu schiessen und filmchen zu drehen. Die beiden Herren reagieren unfreundlich auf den Polizisten Schmied, der sie aufgefordert hat ihre Papiere zu zeigen. Während der eine der beiden den toten Hundebalg vom Rücken nimmt und keinerlei Reaktion zeigt und in einer fremden Sprache zu sprechen scheint. Fragt der andere nackte Mann auf allemanisch, ob der Polizist Schmied ihm vielleicht sagen könne, wo zum Henker, er ohne Taschen Papiere unterbringen soll und ob er zu dusselig sei, den Wappenherold der Ehrengesellschaft zur HHären zu erkennen.

Polizist Schmied rief daraufhin über Funk zwei weitere Streifen und vorsichtshalber eine Sanität in die Gerbergasse. Da sich die Menschenmenge weiter vergrössert hatte, brauchte es Beamten, die diese auf Abstand hielten. Polizist Schmied hat sich zu den drei anderen Polizisten begeben, um sich zu beraten. Dies ist sehr schweirig, da die Menschen immer näher an die Tiere und die beiden Männer zusammenrücken und sowohl die Verdächtigen, als auch die Polizisten fest umringt sind. 

Am Anfang hatten wir ja geglaubt, dass die verdächtigen genausowenig flüchten könnten, wie wir, aber \dots

22:50: Die zwei weitren Streifen treffen ein. DIe vier Kollegen können nach einigen Minuten die Passanten zum weitergehen bewegen. Sofort versucht der Poliszist Schmied erneut Kontakt zu den verdächtigen Personen aufzunehmen. Die Poliszisten Hofer, Meyer und Moser werden erneut von den Tieren beträngt, sobald sich Polizist Schmied den beiden Männern nähert.

Der Eine nackte nimmt den Tierbalg von seinem Rücken, in diesem bewegt sich etwas. Der zweite nackte Mann, der sich als Wappenherold bezeichnet hat, versucht nun seinerseits den Kollegen in ein Gespräch zu verwickeln und vom ersten Nackten abzuhalten. Er sag, es sei eine christliche Mission im Rahmen der Weihnachtsfeierlichkeit und sie dürften nicht unterbrochen werden. Er bat den Poizist Schmied weiter dafür zu sorgen, dass nicht soviele Passanten herumstünden. Er entschuldigt sich für die Umtriebe, das wäre nicht geplant gewesen, normalerweise würden sie beide, der andere Herr Geb und er unsichtbar operieren und nicht soviel Volk anlocken. ANdererseits fände er es schon toll, dass die Basler offensichtlich ein Gespür für die Rauhnächte hätten.

Der Mann wollte sich trotz mehrmaliger Aufforderung des Polizsten Schmied weder abführen lassen, noch den Ort verlassen. Polizist Schmied forderte den Mann weiterhin auf zu gehen. Da das nicht fruchtete wollte er ihn in gewahrsam nehmen und die beiden auf die Wache bringen. Sobald er jedoch den nackten Mann berühren wollte, flog plötzlich und unerwartet ein riesiger Reiher aus dem Tierbalg des anderen Mannes. 

Nun musste Moser eine Pause machen, denn ab da war es kompliziert geworden. Als der riesige Vogel in dem engen Mauervorsprung des Gerberbrunnens auftauchte, geriet der Kollege SChmied, der schon sehr unter STrom stand, völlig aus der Fassung. Er zog seine Pistole und richtete sie auf den Vogel und die beiden Männer abwechselnd und brüllte: ,Ergeben Sie sich! das ist Ihre letzte Chance!' Leider hatten sich in der Zwischezeit wieder neue Zuschauer eingefunden und sobald Schmied die Pistole gezogen hatte, gab es von allen Seiten ein Blitzlichtgewitter. Der Schmied brüllte dann, es sollten sich alle verpissen\dots leider waren unter den Gaffern einige, die filmten. Einige waren aufgeregt und kamen näher. Da hatten sie allerdings die Rechnung ohne die übrigen Tiere gemacht.

Die Tiere\dots Moser schüttelte den Kopf. er war bei der Kaffeemaschine angelangt und wartete auf den schlabbrigen Kaffe, der in seinen Becher blubberte. Er kannte sich mit Gänsen und Hühnern, bzw. Hähnen nicht aus (ausser, wenn sie lecker knusprig gebraten auf seinem Teller lagen, sowie gestern am ersten Festtag.) ABer es kam ihm nicht richtig vor, wie schlau die Tiere es angestellt hatten und zu dritt unermüdlich schafften, was sieben Polizisten nicht gelingen wollte, nämlich den Plebs auf Abstand halten. Moser musste schmunzeln, er nahm einen Schluck des Wassers, dem die Kaffebohnen kurz zugewunken hatten. Er war nicht der einzige, der heute wegen dem dämlichen Hahn einen Verband tragen musste\dots und der Hofer, wie die Gans ihn gleich zweimal mit den Flügeln zu Boden gerissen hatte\dots Moser wusste, es war ungerecht, aber er mochte den Hofer nicht. Wenn er genauer darüber nachgedacht hätte, hätte er sich eingestehen müssen, dass es an dem Hofer seiner Jungend lag und an sonst nichts.

Es war schliesslich ein Schuss gefallen. Gott sei Dank, hatte Schmied jedoch in die Luft geschossen. Daraufhin hatte der nackte, der mit Schmied gesprochen hatte schrill gepfiffen und der Hund war sofort zu ihm gelaufen. Er hatte ihn geschnappt. Der grosse Vogel flog auf und die beiden Männer sprangenüber das Geländer des MAuervorsprunges aufden Platz und jeder griff sich ein Bein des Reihers, der sie dann beide in die Höhe zog. Gans und Hahn waren ebenfalls in die Lift gegangen.

-Das glaubt uns doch kein Schwein, dachte Moser. Er machte sich betrübt auf den WEg zurück in sein Bureau. Ihm fielen die vielen Passanten ein, deren Filmchen sicher schon die Einemillonklickmarke geknackt hatten. Er kratzte sich am Kopf. Aber es war unbefriedigend, weil es unter dem dicken Verband angefangen hatte zu jucken.

Er setzte sich auf seinen Drehstuhl und streckte die Beine aus und verschränkte die Hände über dem Bauch: Der riesige Vogel war unglaublich hoch geflogen und hatte dabei geschriehen. Sie waren in die Wagen gesprungen und versuchten ihm zu folgen. erst schien es als würde er Richtung Norden fliegen.  

  

